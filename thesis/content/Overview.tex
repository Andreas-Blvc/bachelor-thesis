\chapter{Overview}

The problem of autonomous driving can be divided into four main components, each representing a crucial aspect of the system.

\begin{figure}[h!]
	\centering
	\begin{tikzpicture}[
			node distance=1cm,
			auto,
			thick,
			box/.style={rectangle, draw, text width=2cm, align=center, rounded corners, minimum height=1.5cm},
			arrow/.style={-Latex, thick},
			label/.style={font=\small, text width=2cm, align=center},
			highlighted/.style={fill=blue!30, font=\bfseries}
		]
		% Nodes
		\node[box] (route) {Route Planning};
		\node[box] (behavior) [right=of route] {Behavioral Layer};
		\node[box] (motion) [right=of behavior, highlighted] {Motion Planning Layer};
		\node[box] (local) [right=of motion] {Local Feedback};

		% Initial and final arrows
		\draw[arrow] (-2, 0) -- (route.west);
		\draw[arrow] (route.east) -- (behavior.west);
		\draw[arrow] (behavior.east) -- (motion.west);
		\draw[arrow] (motion.east) -- (local.west);
		\draw[arrow] (local.east) -- (11.8, 0);

		% Text labels below arrows
		% \node[label] at (-1.2, -1.25) {Destination};
		% \node[label] at (11.2, -1.25) {Control Commands};

	\end{tikzpicture}
	\caption{Overview of Autonomous Driving Problem Decomposition}
	\label{fig:autonomous_driving_overview}
\end{figure}

The process begins with the user providing a travel destination, which serves as the input to the first component, Route Planning.
In this phase, the system generates a sequence of waypoints through a predefined road network.

Next, the Behavioral Layer refines the waypoints by considering environmental factors such as other vehicles, obstacles, and road signs.
This layer defines the driving behavior at any given time, ensuring the vehicle adapts to dynamic traffic conditions.

With a behavior strategy in place, the Motion Planning Layer generates a trajectory that adheres to strict physical and safety constraints, ensuring
feasibility and compliance with rules of the road.

Finally, the Local Feedback component executes the plan by generating precise control commands—steering, throttle, and brake inputs—based on
real-time vehicle and environmental feedback.

\section{Motion Planning}

Finding an exact solution to the motion planning problem is computationally intractable in most cases.
Therefore, numerical approaches are employed.
These methods fall into three main categories:

\begin{enumerate}
	\item Graph-Based Algorithms: These algorithms discretize the vehicle's possible states
	      and connect valid state transitions with edges.
	      A graph search algorithm can then be used to find an optimal trajectory.

	\item Incremental Tree Approaches: This category involves generating branches of feasible trajectories by randomly applying control commands and simulating
	      the resulting states.
	      The tree expands incrementally, exploring potential paths until a suitable trajectory is found.

	\item Optimization-Based Methods: These methods formulate the problem as an optimization task over a function space.
	      Optimization techniques aim to minimize an objective function (e.g., minimizing travel time or maximizing safety) while respecting constraints.
	      This is the approach we focus on.
\end{enumerate}

Our objective is to design a motion planner that consistently provides near real-time solutions.
We achieve this by leveraging modern, reliable solvers that can efficiently handle the optimization problem.

\section{Convex Optimization}

A set $K\subset \mathbb{R}^n$ is called convex if, for all $x, y\in K$ and $\lambda\in [0, 1]$, the following condition holds

\begin{equation}
	\lambda x + (1-\lambda) y \in K
	\label{eq:convex_set_criteria}
\end{equation}

A real-valued function $f$ defined over a convex subset $X$ of a vector space is called convex if, for all $x,y\in X$ and $\lambda\in [0,1]$, the inequality below is satisfied:

\begin{equation}
	f(\lambda x + (1-\lambda) y) \leq \lambda f(x) + (1-\lambda) f(y)
	\label{eq:convex_function_criteria}
\end{equation}

A function $f$ is said to be concave if $-f$ is convex.

An optimization problem is defined by a feasible set $X\subset \mathbb{R}^n$ and an objective function $f:X\to \mathbb{R}$.
The goal is to find:

\[ min_{x\in X}f(x) \]

\section{Disciplined Convex Programming (DCP)}

For an optimization problem to be efficiently and reliably solvable, it must adhere to the principles of Disciplined Convex
Programming (DCP).
These rules can be summarized as follows:

An optimization problem $(X,f)$ satisfies the DCP rules if the feasible set $X$ is defined by a series of equality and inequality constraints, where each constraint conforms to one of the following patterns:
\begin{itemize}
	\item $affine = affine$
	\item $convex\leq concave$
	\item $concave\geq convex$
\end{itemize}
Additionally, the objective function $f$ must be convex.
By adhering to these rules, we ensure that state-of-the-art solvers can efficiently find optimal solutions.

To be more precise, each constraint in a convex optimization problem consists of a left-hand side (LHS) and a right-hand side (RHS), both of which
can be expressed as functions $f:\mathbb{R}^n\to \mathbb{R}$ and $g:\mathbb{R}^n\to \mathbb{R}$.
The DCP rules can be interpreted as follows:
\begin{itemize}
	\item $affine = affine$ as $f, g$ are real-valued affine functions
	\item $convex\leq concave$ as $f$ is convex over $\mathbb{R}^n$ and $g$ is concave over $\mathbb{R}^n$
	\item $concave\geq convex$ as $f$ is concave over $\mathbb{R}^n$ and $g$ is convex over $\mathbb{R}^n$
\end{itemize}

If a problem adheres to these rules, it is considered a convex optimization problem.
However, these rules are not comprehensive.
It is possible to create valid convex expressions and models that fall outside the scope of these rules.
In such cases, additional analysis may be needed to verify convexity.

Advanced solvers, such as 'CVX', apply these rules to systematically determine whether an expression is affine, convex, or concave.
For further details on how these solvers handle disciplined convex programming, refer to their documentation:
https://web.cvxr.com/cvx/beta/doc/dcp.html.
