\section{Challenges and Limitations}

This section discusses the various challenges encountered during the development and implementation of the control and planning algorithms for the
vehicle.
It highlights specific issues related to replanning, dynamics lag, and the application of McCormick relaxations.
Each subsection provides a detailed explanation of the problem, the approach taken to address it, and the results obtained.
The purpose of this section is to provide insight into the practical difficulties faced, and the solutions devised to overcome them, thereby
contributing to the overall understanding and improvement of the system's performance.

\subsection{Replanning Challenges for Point Mass}

During the replanning phase for the point mass, we encountered a significant challenge.
Our control layer introduced some inaccuracies, resulting in the vehicle reaching a higher velocity than anticipated at the time of replanning.
This discrepancy led to infeasibility issues during the replanning process.

To address this problem, we introduced soft constraints.
These soft constraints allowed for a more flexible approach to handling the velocity overshoot, ensuring that the replanning process could still
generate feasible trajectories despite the initial inaccuracies.
By incorporating these soft constraints, we were able to mitigate the impact of the velocity overshoot and maintain the overall feasibility of the
replanning process.
\pagebreak
\subsection{Lag in Dynamics}
We experienced a lag caused by using the forward Euler method on the vehicle model's dynamics and adding a feedback loop to adjust the vehicle
orientation.
The issue arose because a change in the steering was accounted for in the orientation two additional steps later, leading to overcorrection and
resulting in oscillation, illustrated by Figure \ref{fig:lag_orientation}.

\begin{figure}[h]
	\centering
	\resizebox{0.5\textwidth}{!}{%% Creator: Matplotlib, PGF backend
%%
%% To include the figure in your LaTeX document, write
%%   \input{<filename>.pgf}
%%
%% Make sure the required packages are loaded in your preamble
%%   \usepackage{pgf}
%%
%% Also ensure that all the required font packages are loaded; for instance,
%% the lmodern package is sometimes necessary when using math font.
%%   \usepackage{lmodern}
%%
%% Figures using additional raster images can only be included by \input if
%% they are in the same directory as the main LaTeX file. For loading figures
%% from other directories you can use the `import` package
%%   \usepackage{import}
%%
%% and then include the figures with
%%   \import{<path to file>}{<filename>.pgf}
%%
%% Matplotlib used the following preamble
%%   \def\mathdefault#1{#1}
%%   \everymath=\expandafter{\the\everymath\displaystyle}
%%   
%%   \ifdefined\pdftexversion\else  % non-pdftex case.
%%     \usepackage{fontspec}
%%   \fi
%%   \makeatletter\@ifpackageloaded{underscore}{}{\usepackage[strings]{underscore}}\makeatother
%%
\begingroup%
\makeatletter%
\begin{pgfpicture}%
\pgfpathrectangle{\pgfpointorigin}{\pgfqpoint{5.485616in}{4.295074in}}%
\pgfusepath{use as bounding box, clip}%
\begin{pgfscope}%
\pgfsetbuttcap%
\pgfsetmiterjoin%
\definecolor{currentfill}{rgb}{1.000000,1.000000,1.000000}%
\pgfsetfillcolor{currentfill}%
\pgfsetlinewidth{0.000000pt}%
\definecolor{currentstroke}{rgb}{1.000000,1.000000,1.000000}%
\pgfsetstrokecolor{currentstroke}%
\pgfsetdash{}{0pt}%
\pgfpathmoveto{\pgfqpoint{0.000000in}{0.000000in}}%
\pgfpathlineto{\pgfqpoint{5.485616in}{0.000000in}}%
\pgfpathlineto{\pgfqpoint{5.485616in}{4.295074in}}%
\pgfpathlineto{\pgfqpoint{0.000000in}{4.295074in}}%
\pgfpathlineto{\pgfqpoint{0.000000in}{0.000000in}}%
\pgfpathclose%
\pgfusepath{fill}%
\end{pgfscope}%
\begin{pgfscope}%
\pgfsetbuttcap%
\pgfsetmiterjoin%
\definecolor{currentfill}{rgb}{1.000000,1.000000,1.000000}%
\pgfsetfillcolor{currentfill}%
\pgfsetlinewidth{0.000000pt}%
\definecolor{currentstroke}{rgb}{0.000000,0.000000,0.000000}%
\pgfsetstrokecolor{currentstroke}%
\pgfsetstrokeopacity{0.000000}%
\pgfsetdash{}{0pt}%
\pgfpathmoveto{\pgfqpoint{0.425616in}{0.499074in}}%
\pgfpathlineto{\pgfqpoint{5.385616in}{0.499074in}}%
\pgfpathlineto{\pgfqpoint{5.385616in}{4.195074in}}%
\pgfpathlineto{\pgfqpoint{0.425616in}{4.195074in}}%
\pgfpathlineto{\pgfqpoint{0.425616in}{0.499074in}}%
\pgfpathclose%
\pgfusepath{fill}%
\end{pgfscope}%
\begin{pgfscope}%
\pgfpathrectangle{\pgfqpoint{0.425616in}{0.499074in}}{\pgfqpoint{4.960000in}{3.696000in}}%
\pgfusepath{clip}%
\pgfsetbuttcap%
\pgfsetroundjoin%
\pgfsetlinewidth{0.803000pt}%
\definecolor{currentstroke}{rgb}{0.501961,0.501961,0.501961}%
\pgfsetstrokecolor{currentstroke}%
\pgfsetstrokeopacity{0.700000}%
\pgfsetdash{{0.800000pt}{1.320000pt}}{0.000000pt}%
\pgfpathmoveto{\pgfqpoint{0.651070in}{0.499074in}}%
\pgfpathlineto{\pgfqpoint{0.651070in}{4.195074in}}%
\pgfusepath{stroke}%
\end{pgfscope}%
\begin{pgfscope}%
\pgfsetbuttcap%
\pgfsetroundjoin%
\definecolor{currentfill}{rgb}{0.000000,0.000000,0.000000}%
\pgfsetfillcolor{currentfill}%
\pgfsetlinewidth{0.803000pt}%
\definecolor{currentstroke}{rgb}{0.000000,0.000000,0.000000}%
\pgfsetstrokecolor{currentstroke}%
\pgfsetdash{}{0pt}%
\pgfsys@defobject{currentmarker}{\pgfqpoint{0.000000in}{-0.048611in}}{\pgfqpoint{0.000000in}{0.000000in}}{%
\pgfpathmoveto{\pgfqpoint{0.000000in}{0.000000in}}%
\pgfpathlineto{\pgfqpoint{0.000000in}{-0.048611in}}%
\pgfusepath{stroke,fill}%
}%
\begin{pgfscope}%
\pgfsys@transformshift{0.651070in}{0.499074in}%
\pgfsys@useobject{currentmarker}{}%
\end{pgfscope}%
\end{pgfscope}%
\begin{pgfscope}%
\definecolor{textcolor}{rgb}{0.000000,0.000000,0.000000}%
\pgfsetstrokecolor{textcolor}%
\pgfsetfillcolor{textcolor}%
\pgftext[x=0.651070in,y=0.401852in,,top]{\color{textcolor}{\rmfamily\fontsize{9.000000}{10.800000}\selectfont\catcode`\^=\active\def^{\ifmmode\sp\else\^{}\fi}\catcode`\%=\active\def%{\%}$\mathdefault{0}$}}%
\end{pgfscope}%
\begin{pgfscope}%
\pgfpathrectangle{\pgfqpoint{0.425616in}{0.499074in}}{\pgfqpoint{4.960000in}{3.696000in}}%
\pgfusepath{clip}%
\pgfsetbuttcap%
\pgfsetroundjoin%
\pgfsetlinewidth{0.803000pt}%
\definecolor{currentstroke}{rgb}{0.501961,0.501961,0.501961}%
\pgfsetstrokecolor{currentstroke}%
\pgfsetstrokeopacity{0.700000}%
\pgfsetdash{{0.800000pt}{1.320000pt}}{0.000000pt}%
\pgfpathmoveto{\pgfqpoint{1.553791in}{0.499074in}}%
\pgfpathlineto{\pgfqpoint{1.553791in}{4.195074in}}%
\pgfusepath{stroke}%
\end{pgfscope}%
\begin{pgfscope}%
\pgfsetbuttcap%
\pgfsetroundjoin%
\definecolor{currentfill}{rgb}{0.000000,0.000000,0.000000}%
\pgfsetfillcolor{currentfill}%
\pgfsetlinewidth{0.803000pt}%
\definecolor{currentstroke}{rgb}{0.000000,0.000000,0.000000}%
\pgfsetstrokecolor{currentstroke}%
\pgfsetdash{}{0pt}%
\pgfsys@defobject{currentmarker}{\pgfqpoint{0.000000in}{-0.048611in}}{\pgfqpoint{0.000000in}{0.000000in}}{%
\pgfpathmoveto{\pgfqpoint{0.000000in}{0.000000in}}%
\pgfpathlineto{\pgfqpoint{0.000000in}{-0.048611in}}%
\pgfusepath{stroke,fill}%
}%
\begin{pgfscope}%
\pgfsys@transformshift{1.553791in}{0.499074in}%
\pgfsys@useobject{currentmarker}{}%
\end{pgfscope}%
\end{pgfscope}%
\begin{pgfscope}%
\definecolor{textcolor}{rgb}{0.000000,0.000000,0.000000}%
\pgfsetstrokecolor{textcolor}%
\pgfsetfillcolor{textcolor}%
\pgftext[x=1.553791in,y=0.401852in,,top]{\color{textcolor}{\rmfamily\fontsize{9.000000}{10.800000}\selectfont\catcode`\^=\active\def^{\ifmmode\sp\else\^{}\fi}\catcode`\%=\active\def%{\%}$\mathdefault{200}$}}%
\end{pgfscope}%
\begin{pgfscope}%
\pgfpathrectangle{\pgfqpoint{0.425616in}{0.499074in}}{\pgfqpoint{4.960000in}{3.696000in}}%
\pgfusepath{clip}%
\pgfsetbuttcap%
\pgfsetroundjoin%
\pgfsetlinewidth{0.803000pt}%
\definecolor{currentstroke}{rgb}{0.501961,0.501961,0.501961}%
\pgfsetstrokecolor{currentstroke}%
\pgfsetstrokeopacity{0.700000}%
\pgfsetdash{{0.800000pt}{1.320000pt}}{0.000000pt}%
\pgfpathmoveto{\pgfqpoint{2.456512in}{0.499074in}}%
\pgfpathlineto{\pgfqpoint{2.456512in}{4.195074in}}%
\pgfusepath{stroke}%
\end{pgfscope}%
\begin{pgfscope}%
\pgfsetbuttcap%
\pgfsetroundjoin%
\definecolor{currentfill}{rgb}{0.000000,0.000000,0.000000}%
\pgfsetfillcolor{currentfill}%
\pgfsetlinewidth{0.803000pt}%
\definecolor{currentstroke}{rgb}{0.000000,0.000000,0.000000}%
\pgfsetstrokecolor{currentstroke}%
\pgfsetdash{}{0pt}%
\pgfsys@defobject{currentmarker}{\pgfqpoint{0.000000in}{-0.048611in}}{\pgfqpoint{0.000000in}{0.000000in}}{%
\pgfpathmoveto{\pgfqpoint{0.000000in}{0.000000in}}%
\pgfpathlineto{\pgfqpoint{0.000000in}{-0.048611in}}%
\pgfusepath{stroke,fill}%
}%
\begin{pgfscope}%
\pgfsys@transformshift{2.456512in}{0.499074in}%
\pgfsys@useobject{currentmarker}{}%
\end{pgfscope}%
\end{pgfscope}%
\begin{pgfscope}%
\definecolor{textcolor}{rgb}{0.000000,0.000000,0.000000}%
\pgfsetstrokecolor{textcolor}%
\pgfsetfillcolor{textcolor}%
\pgftext[x=2.456512in,y=0.401852in,,top]{\color{textcolor}{\rmfamily\fontsize{9.000000}{10.800000}\selectfont\catcode`\^=\active\def^{\ifmmode\sp\else\^{}\fi}\catcode`\%=\active\def%{\%}$\mathdefault{400}$}}%
\end{pgfscope}%
\begin{pgfscope}%
\pgfpathrectangle{\pgfqpoint{0.425616in}{0.499074in}}{\pgfqpoint{4.960000in}{3.696000in}}%
\pgfusepath{clip}%
\pgfsetbuttcap%
\pgfsetroundjoin%
\pgfsetlinewidth{0.803000pt}%
\definecolor{currentstroke}{rgb}{0.501961,0.501961,0.501961}%
\pgfsetstrokecolor{currentstroke}%
\pgfsetstrokeopacity{0.700000}%
\pgfsetdash{{0.800000pt}{1.320000pt}}{0.000000pt}%
\pgfpathmoveto{\pgfqpoint{3.359233in}{0.499074in}}%
\pgfpathlineto{\pgfqpoint{3.359233in}{4.195074in}}%
\pgfusepath{stroke}%
\end{pgfscope}%
\begin{pgfscope}%
\pgfsetbuttcap%
\pgfsetroundjoin%
\definecolor{currentfill}{rgb}{0.000000,0.000000,0.000000}%
\pgfsetfillcolor{currentfill}%
\pgfsetlinewidth{0.803000pt}%
\definecolor{currentstroke}{rgb}{0.000000,0.000000,0.000000}%
\pgfsetstrokecolor{currentstroke}%
\pgfsetdash{}{0pt}%
\pgfsys@defobject{currentmarker}{\pgfqpoint{0.000000in}{-0.048611in}}{\pgfqpoint{0.000000in}{0.000000in}}{%
\pgfpathmoveto{\pgfqpoint{0.000000in}{0.000000in}}%
\pgfpathlineto{\pgfqpoint{0.000000in}{-0.048611in}}%
\pgfusepath{stroke,fill}%
}%
\begin{pgfscope}%
\pgfsys@transformshift{3.359233in}{0.499074in}%
\pgfsys@useobject{currentmarker}{}%
\end{pgfscope}%
\end{pgfscope}%
\begin{pgfscope}%
\definecolor{textcolor}{rgb}{0.000000,0.000000,0.000000}%
\pgfsetstrokecolor{textcolor}%
\pgfsetfillcolor{textcolor}%
\pgftext[x=3.359233in,y=0.401852in,,top]{\color{textcolor}{\rmfamily\fontsize{9.000000}{10.800000}\selectfont\catcode`\^=\active\def^{\ifmmode\sp\else\^{}\fi}\catcode`\%=\active\def%{\%}$\mathdefault{600}$}}%
\end{pgfscope}%
\begin{pgfscope}%
\pgfpathrectangle{\pgfqpoint{0.425616in}{0.499074in}}{\pgfqpoint{4.960000in}{3.696000in}}%
\pgfusepath{clip}%
\pgfsetbuttcap%
\pgfsetroundjoin%
\pgfsetlinewidth{0.803000pt}%
\definecolor{currentstroke}{rgb}{0.501961,0.501961,0.501961}%
\pgfsetstrokecolor{currentstroke}%
\pgfsetstrokeopacity{0.700000}%
\pgfsetdash{{0.800000pt}{1.320000pt}}{0.000000pt}%
\pgfpathmoveto{\pgfqpoint{4.261954in}{0.499074in}}%
\pgfpathlineto{\pgfqpoint{4.261954in}{4.195074in}}%
\pgfusepath{stroke}%
\end{pgfscope}%
\begin{pgfscope}%
\pgfsetbuttcap%
\pgfsetroundjoin%
\definecolor{currentfill}{rgb}{0.000000,0.000000,0.000000}%
\pgfsetfillcolor{currentfill}%
\pgfsetlinewidth{0.803000pt}%
\definecolor{currentstroke}{rgb}{0.000000,0.000000,0.000000}%
\pgfsetstrokecolor{currentstroke}%
\pgfsetdash{}{0pt}%
\pgfsys@defobject{currentmarker}{\pgfqpoint{0.000000in}{-0.048611in}}{\pgfqpoint{0.000000in}{0.000000in}}{%
\pgfpathmoveto{\pgfqpoint{0.000000in}{0.000000in}}%
\pgfpathlineto{\pgfqpoint{0.000000in}{-0.048611in}}%
\pgfusepath{stroke,fill}%
}%
\begin{pgfscope}%
\pgfsys@transformshift{4.261954in}{0.499074in}%
\pgfsys@useobject{currentmarker}{}%
\end{pgfscope}%
\end{pgfscope}%
\begin{pgfscope}%
\definecolor{textcolor}{rgb}{0.000000,0.000000,0.000000}%
\pgfsetstrokecolor{textcolor}%
\pgfsetfillcolor{textcolor}%
\pgftext[x=4.261954in,y=0.401852in,,top]{\color{textcolor}{\rmfamily\fontsize{9.000000}{10.800000}\selectfont\catcode`\^=\active\def^{\ifmmode\sp\else\^{}\fi}\catcode`\%=\active\def%{\%}$\mathdefault{800}$}}%
\end{pgfscope}%
\begin{pgfscope}%
\pgfpathrectangle{\pgfqpoint{0.425616in}{0.499074in}}{\pgfqpoint{4.960000in}{3.696000in}}%
\pgfusepath{clip}%
\pgfsetbuttcap%
\pgfsetroundjoin%
\pgfsetlinewidth{0.803000pt}%
\definecolor{currentstroke}{rgb}{0.501961,0.501961,0.501961}%
\pgfsetstrokecolor{currentstroke}%
\pgfsetstrokeopacity{0.700000}%
\pgfsetdash{{0.800000pt}{1.320000pt}}{0.000000pt}%
\pgfpathmoveto{\pgfqpoint{5.164675in}{0.499074in}}%
\pgfpathlineto{\pgfqpoint{5.164675in}{4.195074in}}%
\pgfusepath{stroke}%
\end{pgfscope}%
\begin{pgfscope}%
\pgfsetbuttcap%
\pgfsetroundjoin%
\definecolor{currentfill}{rgb}{0.000000,0.000000,0.000000}%
\pgfsetfillcolor{currentfill}%
\pgfsetlinewidth{0.803000pt}%
\definecolor{currentstroke}{rgb}{0.000000,0.000000,0.000000}%
\pgfsetstrokecolor{currentstroke}%
\pgfsetdash{}{0pt}%
\pgfsys@defobject{currentmarker}{\pgfqpoint{0.000000in}{-0.048611in}}{\pgfqpoint{0.000000in}{0.000000in}}{%
\pgfpathmoveto{\pgfqpoint{0.000000in}{0.000000in}}%
\pgfpathlineto{\pgfqpoint{0.000000in}{-0.048611in}}%
\pgfusepath{stroke,fill}%
}%
\begin{pgfscope}%
\pgfsys@transformshift{5.164675in}{0.499074in}%
\pgfsys@useobject{currentmarker}{}%
\end{pgfscope}%
\end{pgfscope}%
\begin{pgfscope}%
\definecolor{textcolor}{rgb}{0.000000,0.000000,0.000000}%
\pgfsetstrokecolor{textcolor}%
\pgfsetfillcolor{textcolor}%
\pgftext[x=5.164675in,y=0.401852in,,top]{\color{textcolor}{\rmfamily\fontsize{9.000000}{10.800000}\selectfont\catcode`\^=\active\def^{\ifmmode\sp\else\^{}\fi}\catcode`\%=\active\def%{\%}$\mathdefault{1000}$}}%
\end{pgfscope}%
\begin{pgfscope}%
\definecolor{textcolor}{rgb}{0.000000,0.000000,0.000000}%
\pgfsetstrokecolor{textcolor}%
\pgfsetfillcolor{textcolor}%
\pgftext[x=2.905616in,y=0.235185in,,top]{\color{textcolor}{\rmfamily\fontsize{11.000000}{13.200000}\selectfont\catcode`\^=\active\def^{\ifmmode\sp\else\^{}\fi}\catcode`\%=\active\def%{\%}X-axis}}%
\end{pgfscope}%
\begin{pgfscope}%
\pgfpathrectangle{\pgfqpoint{0.425616in}{0.499074in}}{\pgfqpoint{4.960000in}{3.696000in}}%
\pgfusepath{clip}%
\pgfsetbuttcap%
\pgfsetroundjoin%
\pgfsetlinewidth{0.803000pt}%
\definecolor{currentstroke}{rgb}{0.501961,0.501961,0.501961}%
\pgfsetstrokecolor{currentstroke}%
\pgfsetstrokeopacity{0.700000}%
\pgfsetdash{{0.800000pt}{1.320000pt}}{0.000000pt}%
\pgfpathmoveto{\pgfqpoint{0.425616in}{1.128970in}}%
\pgfpathlineto{\pgfqpoint{5.385616in}{1.128970in}}%
\pgfusepath{stroke}%
\end{pgfscope}%
\begin{pgfscope}%
\pgfsetbuttcap%
\pgfsetroundjoin%
\definecolor{currentfill}{rgb}{0.000000,0.000000,0.000000}%
\pgfsetfillcolor{currentfill}%
\pgfsetlinewidth{0.803000pt}%
\definecolor{currentstroke}{rgb}{0.000000,0.000000,0.000000}%
\pgfsetstrokecolor{currentstroke}%
\pgfsetdash{}{0pt}%
\pgfsys@defobject{currentmarker}{\pgfqpoint{-0.048611in}{0.000000in}}{\pgfqpoint{-0.000000in}{0.000000in}}{%
\pgfpathmoveto{\pgfqpoint{-0.000000in}{0.000000in}}%
\pgfpathlineto{\pgfqpoint{-0.048611in}{0.000000in}}%
\pgfusepath{stroke,fill}%
}%
\begin{pgfscope}%
\pgfsys@transformshift{0.425616in}{1.128970in}%
\pgfsys@useobject{currentmarker}{}%
\end{pgfscope}%
\end{pgfscope}%
\begin{pgfscope}%
\definecolor{textcolor}{rgb}{0.000000,0.000000,0.000000}%
\pgfsetstrokecolor{textcolor}%
\pgfsetfillcolor{textcolor}%
\pgftext[x=0.100000in, y=1.085567in, left, base]{\color{textcolor}{\rmfamily\fontsize{9.000000}{10.800000}\selectfont\catcode`\^=\active\def^{\ifmmode\sp\else\^{}\fi}\catcode`\%=\active\def%{\%}$\mathdefault{0.00}$}}%
\end{pgfscope}%
\begin{pgfscope}%
\pgfpathrectangle{\pgfqpoint{0.425616in}{0.499074in}}{\pgfqpoint{4.960000in}{3.696000in}}%
\pgfusepath{clip}%
\pgfsetbuttcap%
\pgfsetroundjoin%
\pgfsetlinewidth{0.803000pt}%
\definecolor{currentstroke}{rgb}{0.501961,0.501961,0.501961}%
\pgfsetstrokecolor{currentstroke}%
\pgfsetstrokeopacity{0.700000}%
\pgfsetdash{{0.800000pt}{1.320000pt}}{0.000000pt}%
\pgfpathmoveto{\pgfqpoint{0.425616in}{1.871655in}}%
\pgfpathlineto{\pgfqpoint{5.385616in}{1.871655in}}%
\pgfusepath{stroke}%
\end{pgfscope}%
\begin{pgfscope}%
\pgfsetbuttcap%
\pgfsetroundjoin%
\definecolor{currentfill}{rgb}{0.000000,0.000000,0.000000}%
\pgfsetfillcolor{currentfill}%
\pgfsetlinewidth{0.803000pt}%
\definecolor{currentstroke}{rgb}{0.000000,0.000000,0.000000}%
\pgfsetstrokecolor{currentstroke}%
\pgfsetdash{}{0pt}%
\pgfsys@defobject{currentmarker}{\pgfqpoint{-0.048611in}{0.000000in}}{\pgfqpoint{-0.000000in}{0.000000in}}{%
\pgfpathmoveto{\pgfqpoint{-0.000000in}{0.000000in}}%
\pgfpathlineto{\pgfqpoint{-0.048611in}{0.000000in}}%
\pgfusepath{stroke,fill}%
}%
\begin{pgfscope}%
\pgfsys@transformshift{0.425616in}{1.871655in}%
\pgfsys@useobject{currentmarker}{}%
\end{pgfscope}%
\end{pgfscope}%
\begin{pgfscope}%
\definecolor{textcolor}{rgb}{0.000000,0.000000,0.000000}%
\pgfsetstrokecolor{textcolor}%
\pgfsetfillcolor{textcolor}%
\pgftext[x=0.100000in, y=1.828252in, left, base]{\color{textcolor}{\rmfamily\fontsize{9.000000}{10.800000}\selectfont\catcode`\^=\active\def^{\ifmmode\sp\else\^{}\fi}\catcode`\%=\active\def%{\%}$\mathdefault{0.05}$}}%
\end{pgfscope}%
\begin{pgfscope}%
\pgfpathrectangle{\pgfqpoint{0.425616in}{0.499074in}}{\pgfqpoint{4.960000in}{3.696000in}}%
\pgfusepath{clip}%
\pgfsetbuttcap%
\pgfsetroundjoin%
\pgfsetlinewidth{0.803000pt}%
\definecolor{currentstroke}{rgb}{0.501961,0.501961,0.501961}%
\pgfsetstrokecolor{currentstroke}%
\pgfsetstrokeopacity{0.700000}%
\pgfsetdash{{0.800000pt}{1.320000pt}}{0.000000pt}%
\pgfpathmoveto{\pgfqpoint{0.425616in}{2.614340in}}%
\pgfpathlineto{\pgfqpoint{5.385616in}{2.614340in}}%
\pgfusepath{stroke}%
\end{pgfscope}%
\begin{pgfscope}%
\pgfsetbuttcap%
\pgfsetroundjoin%
\definecolor{currentfill}{rgb}{0.000000,0.000000,0.000000}%
\pgfsetfillcolor{currentfill}%
\pgfsetlinewidth{0.803000pt}%
\definecolor{currentstroke}{rgb}{0.000000,0.000000,0.000000}%
\pgfsetstrokecolor{currentstroke}%
\pgfsetdash{}{0pt}%
\pgfsys@defobject{currentmarker}{\pgfqpoint{-0.048611in}{0.000000in}}{\pgfqpoint{-0.000000in}{0.000000in}}{%
\pgfpathmoveto{\pgfqpoint{-0.000000in}{0.000000in}}%
\pgfpathlineto{\pgfqpoint{-0.048611in}{0.000000in}}%
\pgfusepath{stroke,fill}%
}%
\begin{pgfscope}%
\pgfsys@transformshift{0.425616in}{2.614340in}%
\pgfsys@useobject{currentmarker}{}%
\end{pgfscope}%
\end{pgfscope}%
\begin{pgfscope}%
\definecolor{textcolor}{rgb}{0.000000,0.000000,0.000000}%
\pgfsetstrokecolor{textcolor}%
\pgfsetfillcolor{textcolor}%
\pgftext[x=0.100000in, y=2.570937in, left, base]{\color{textcolor}{\rmfamily\fontsize{9.000000}{10.800000}\selectfont\catcode`\^=\active\def^{\ifmmode\sp\else\^{}\fi}\catcode`\%=\active\def%{\%}$\mathdefault{0.10}$}}%
\end{pgfscope}%
\begin{pgfscope}%
\pgfpathrectangle{\pgfqpoint{0.425616in}{0.499074in}}{\pgfqpoint{4.960000in}{3.696000in}}%
\pgfusepath{clip}%
\pgfsetbuttcap%
\pgfsetroundjoin%
\pgfsetlinewidth{0.803000pt}%
\definecolor{currentstroke}{rgb}{0.501961,0.501961,0.501961}%
\pgfsetstrokecolor{currentstroke}%
\pgfsetstrokeopacity{0.700000}%
\pgfsetdash{{0.800000pt}{1.320000pt}}{0.000000pt}%
\pgfpathmoveto{\pgfqpoint{0.425616in}{3.357024in}}%
\pgfpathlineto{\pgfqpoint{5.385616in}{3.357024in}}%
\pgfusepath{stroke}%
\end{pgfscope}%
\begin{pgfscope}%
\pgfsetbuttcap%
\pgfsetroundjoin%
\definecolor{currentfill}{rgb}{0.000000,0.000000,0.000000}%
\pgfsetfillcolor{currentfill}%
\pgfsetlinewidth{0.803000pt}%
\definecolor{currentstroke}{rgb}{0.000000,0.000000,0.000000}%
\pgfsetstrokecolor{currentstroke}%
\pgfsetdash{}{0pt}%
\pgfsys@defobject{currentmarker}{\pgfqpoint{-0.048611in}{0.000000in}}{\pgfqpoint{-0.000000in}{0.000000in}}{%
\pgfpathmoveto{\pgfqpoint{-0.000000in}{0.000000in}}%
\pgfpathlineto{\pgfqpoint{-0.048611in}{0.000000in}}%
\pgfusepath{stroke,fill}%
}%
\begin{pgfscope}%
\pgfsys@transformshift{0.425616in}{3.357024in}%
\pgfsys@useobject{currentmarker}{}%
\end{pgfscope}%
\end{pgfscope}%
\begin{pgfscope}%
\definecolor{textcolor}{rgb}{0.000000,0.000000,0.000000}%
\pgfsetstrokecolor{textcolor}%
\pgfsetfillcolor{textcolor}%
\pgftext[x=0.100000in, y=3.313621in, left, base]{\color{textcolor}{\rmfamily\fontsize{9.000000}{10.800000}\selectfont\catcode`\^=\active\def^{\ifmmode\sp\else\^{}\fi}\catcode`\%=\active\def%{\%}$\mathdefault{0.15}$}}%
\end{pgfscope}%
\begin{pgfscope}%
\pgfpathrectangle{\pgfqpoint{0.425616in}{0.499074in}}{\pgfqpoint{4.960000in}{3.696000in}}%
\pgfusepath{clip}%
\pgfsetbuttcap%
\pgfsetroundjoin%
\pgfsetlinewidth{0.803000pt}%
\definecolor{currentstroke}{rgb}{0.501961,0.501961,0.501961}%
\pgfsetstrokecolor{currentstroke}%
\pgfsetstrokeopacity{0.700000}%
\pgfsetdash{{0.800000pt}{1.320000pt}}{0.000000pt}%
\pgfpathmoveto{\pgfqpoint{0.425616in}{4.099709in}}%
\pgfpathlineto{\pgfqpoint{5.385616in}{4.099709in}}%
\pgfusepath{stroke}%
\end{pgfscope}%
\begin{pgfscope}%
\pgfsetbuttcap%
\pgfsetroundjoin%
\definecolor{currentfill}{rgb}{0.000000,0.000000,0.000000}%
\pgfsetfillcolor{currentfill}%
\pgfsetlinewidth{0.803000pt}%
\definecolor{currentstroke}{rgb}{0.000000,0.000000,0.000000}%
\pgfsetstrokecolor{currentstroke}%
\pgfsetdash{}{0pt}%
\pgfsys@defobject{currentmarker}{\pgfqpoint{-0.048611in}{0.000000in}}{\pgfqpoint{-0.000000in}{0.000000in}}{%
\pgfpathmoveto{\pgfqpoint{-0.000000in}{0.000000in}}%
\pgfpathlineto{\pgfqpoint{-0.048611in}{0.000000in}}%
\pgfusepath{stroke,fill}%
}%
\begin{pgfscope}%
\pgfsys@transformshift{0.425616in}{4.099709in}%
\pgfsys@useobject{currentmarker}{}%
\end{pgfscope}%
\end{pgfscope}%
\begin{pgfscope}%
\definecolor{textcolor}{rgb}{0.000000,0.000000,0.000000}%
\pgfsetstrokecolor{textcolor}%
\pgfsetfillcolor{textcolor}%
\pgftext[x=0.100000in, y=4.056306in, left, base]{\color{textcolor}{\rmfamily\fontsize{9.000000}{10.800000}\selectfont\catcode`\^=\active\def^{\ifmmode\sp\else\^{}\fi}\catcode`\%=\active\def%{\%}$\mathdefault{0.20}$}}%
\end{pgfscope}%
\begin{pgfscope}%
\pgfpathrectangle{\pgfqpoint{0.425616in}{0.499074in}}{\pgfqpoint{4.960000in}{3.696000in}}%
\pgfusepath{clip}%
\pgfsetrectcap%
\pgfsetroundjoin%
\pgfsetlinewidth{1.505625pt}%
\definecolor{currentstroke}{rgb}{0.121569,0.466667,0.705882}%
\pgfsetstrokecolor{currentstroke}%
\pgfsetdash{}{0pt}%
\pgfpathmoveto{\pgfqpoint{0.651070in}{1.128970in}}%
\pgfpathlineto{\pgfqpoint{5.160161in}{3.107482in}}%
\pgfpathlineto{\pgfqpoint{5.160161in}{3.107482in}}%
\pgfusepath{stroke}%
\end{pgfscope}%
\begin{pgfscope}%
\pgfpathrectangle{\pgfqpoint{0.425616in}{0.499074in}}{\pgfqpoint{4.960000in}{3.696000in}}%
\pgfusepath{clip}%
\pgfsetrectcap%
\pgfsetroundjoin%
\pgfsetlinewidth{1.505625pt}%
\definecolor{currentstroke}{rgb}{1.000000,0.498039,0.054902}%
\pgfsetstrokecolor{currentstroke}%
\pgfsetdash{}{0pt}%
\pgfpathmoveto{\pgfqpoint{0.651070in}{1.128970in}}%
\pgfpathlineto{\pgfqpoint{0.664611in}{1.128970in}}%
\pgfpathlineto{\pgfqpoint{0.669125in}{1.131226in}}%
\pgfpathlineto{\pgfqpoint{0.673638in}{1.135998in}}%
\pgfpathlineto{\pgfqpoint{0.678152in}{1.143250in}}%
\pgfpathlineto{\pgfqpoint{0.696206in}{1.181105in}}%
\pgfpathlineto{\pgfqpoint{0.700720in}{1.185354in}}%
\pgfpathlineto{\pgfqpoint{0.705234in}{1.187119in}}%
\pgfpathlineto{\pgfqpoint{0.709747in}{1.186399in}}%
\pgfpathlineto{\pgfqpoint{0.714261in}{1.183195in}}%
\pgfpathlineto{\pgfqpoint{0.718774in}{1.177505in}}%
\pgfpathlineto{\pgfqpoint{0.723288in}{1.169331in}}%
\pgfpathlineto{\pgfqpoint{0.732315in}{1.145527in}}%
\pgfpathlineto{\pgfqpoint{0.741342in}{1.117444in}}%
\pgfpathlineto{\pgfqpoint{0.750370in}{1.099812in}}%
\pgfpathlineto{\pgfqpoint{0.754883in}{1.094728in}}%
\pgfpathlineto{\pgfqpoint{0.759397in}{1.092129in}}%
\pgfpathlineto{\pgfqpoint{0.763910in}{1.092016in}}%
\pgfpathlineto{\pgfqpoint{0.768424in}{1.094387in}}%
\pgfpathlineto{\pgfqpoint{0.772938in}{1.099243in}}%
\pgfpathlineto{\pgfqpoint{0.777451in}{1.106585in}}%
\pgfpathlineto{\pgfqpoint{0.786479in}{1.128722in}}%
\pgfpathlineto{\pgfqpoint{0.795506in}{1.160799in}}%
\pgfpathlineto{\pgfqpoint{0.804533in}{1.202815in}}%
\pgfpathlineto{\pgfqpoint{0.822587in}{1.300473in}}%
\pgfpathlineto{\pgfqpoint{0.831615in}{1.336307in}}%
\pgfpathlineto{\pgfqpoint{0.840642in}{1.362201in}}%
\pgfpathlineto{\pgfqpoint{0.849669in}{1.378157in}}%
\pgfpathlineto{\pgfqpoint{0.854183in}{1.382407in}}%
\pgfpathlineto{\pgfqpoint{0.858696in}{1.384172in}}%
\pgfpathlineto{\pgfqpoint{0.863210in}{1.383452in}}%
\pgfpathlineto{\pgfqpoint{0.867723in}{1.380248in}}%
\pgfpathlineto{\pgfqpoint{0.872237in}{1.374558in}}%
\pgfpathlineto{\pgfqpoint{0.876751in}{1.366384in}}%
\pgfpathlineto{\pgfqpoint{0.885778in}{1.342580in}}%
\pgfpathlineto{\pgfqpoint{0.894805in}{1.308836in}}%
\pgfpathlineto{\pgfqpoint{0.903832in}{1.265153in}}%
\pgfpathlineto{\pgfqpoint{0.912859in}{1.211530in}}%
\pgfpathlineto{\pgfqpoint{0.921887in}{1.153387in}}%
\pgfpathlineto{\pgfqpoint{0.930914in}{1.105518in}}%
\pgfpathlineto{\pgfqpoint{0.939941in}{1.067595in}}%
\pgfpathlineto{\pgfqpoint{0.948968in}{1.039611in}}%
\pgfpathlineto{\pgfqpoint{0.957995in}{1.021568in}}%
\pgfpathlineto{\pgfqpoint{0.962509in}{1.016273in}}%
\pgfpathlineto{\pgfqpoint{0.967023in}{1.013464in}}%
\pgfpathlineto{\pgfqpoint{0.971536in}{1.013140in}}%
\pgfpathlineto{\pgfqpoint{0.976050in}{1.015300in}}%
\pgfpathlineto{\pgfqpoint{0.980564in}{1.019945in}}%
\pgfpathlineto{\pgfqpoint{0.985077in}{1.027076in}}%
\pgfpathlineto{\pgfqpoint{0.994104in}{1.048791in}}%
\pgfpathlineto{\pgfqpoint{1.003132in}{1.080446in}}%
\pgfpathlineto{\pgfqpoint{1.012159in}{1.122041in}}%
\pgfpathlineto{\pgfqpoint{1.021186in}{1.173576in}}%
\pgfpathlineto{\pgfqpoint{1.030213in}{1.235050in}}%
\pgfpathlineto{\pgfqpoint{1.043754in}{1.345898in}}%
\pgfpathlineto{\pgfqpoint{1.057295in}{1.462917in}}%
\pgfpathlineto{\pgfqpoint{1.070836in}{1.557036in}}%
\pgfpathlineto{\pgfqpoint{1.079863in}{1.607359in}}%
\pgfpathlineto{\pgfqpoint{1.088890in}{1.647742in}}%
\pgfpathlineto{\pgfqpoint{1.097917in}{1.678184in}}%
\pgfpathlineto{\pgfqpoint{1.106944in}{1.698688in}}%
\pgfpathlineto{\pgfqpoint{1.111458in}{1.705212in}}%
\pgfpathlineto{\pgfqpoint{1.115972in}{1.709251in}}%
\pgfpathlineto{\pgfqpoint{1.120485in}{1.710806in}}%
\pgfpathlineto{\pgfqpoint{1.124999in}{1.709875in}}%
\pgfpathlineto{\pgfqpoint{1.129512in}{1.706459in}}%
\pgfpathlineto{\pgfqpoint{1.134026in}{1.700559in}}%
\pgfpathlineto{\pgfqpoint{1.138540in}{1.692174in}}%
\pgfpathlineto{\pgfqpoint{1.147567in}{1.667948in}}%
\pgfpathlineto{\pgfqpoint{1.156594in}{1.633783in}}%
\pgfpathlineto{\pgfqpoint{1.165621in}{1.589678in}}%
\pgfpathlineto{\pgfqpoint{1.174649in}{1.535633in}}%
\pgfpathlineto{\pgfqpoint{1.188189in}{1.435929in}}%
\pgfpathlineto{\pgfqpoint{1.201730in}{1.313860in}}%
\pgfpathlineto{\pgfqpoint{1.215271in}{1.185621in}}%
\pgfpathlineto{\pgfqpoint{1.228812in}{1.080282in}}%
\pgfpathlineto{\pgfqpoint{1.242353in}{0.997306in}}%
\pgfpathlineto{\pgfqpoint{1.251380in}{0.954413in}}%
\pgfpathlineto{\pgfqpoint{1.260407in}{0.921460in}}%
\pgfpathlineto{\pgfqpoint{1.269434in}{0.898446in}}%
\pgfpathlineto{\pgfqpoint{1.273948in}{0.890667in}}%
\pgfpathlineto{\pgfqpoint{1.278461in}{0.885373in}}%
\pgfpathlineto{\pgfqpoint{1.282975in}{0.882563in}}%
\pgfpathlineto{\pgfqpoint{1.287489in}{0.882239in}}%
\pgfpathlineto{\pgfqpoint{1.292002in}{0.884399in}}%
\pgfpathlineto{\pgfqpoint{1.296516in}{0.889044in}}%
\pgfpathlineto{\pgfqpoint{1.301029in}{0.896175in}}%
\pgfpathlineto{\pgfqpoint{1.310057in}{0.917890in}}%
\pgfpathlineto{\pgfqpoint{1.319084in}{0.949545in}}%
\pgfpathlineto{\pgfqpoint{1.328111in}{0.991140in}}%
\pgfpathlineto{\pgfqpoint{1.337138in}{1.042675in}}%
\pgfpathlineto{\pgfqpoint{1.346165in}{1.104149in}}%
\pgfpathlineto{\pgfqpoint{1.359706in}{1.214998in}}%
\pgfpathlineto{\pgfqpoint{1.373247in}{1.348210in}}%
\pgfpathlineto{\pgfqpoint{1.386788in}{1.503788in}}%
\pgfpathlineto{\pgfqpoint{1.404842in}{1.723204in}}%
\pgfpathlineto{\pgfqpoint{1.418383in}{1.863152in}}%
\pgfpathlineto{\pgfqpoint{1.431924in}{1.980736in}}%
\pgfpathlineto{\pgfqpoint{1.445465in}{2.075955in}}%
\pgfpathlineto{\pgfqpoint{1.454492in}{2.127010in}}%
\pgfpathlineto{\pgfqpoint{1.463519in}{2.168125in}}%
\pgfpathlineto{\pgfqpoint{1.472546in}{2.199301in}}%
\pgfpathlineto{\pgfqpoint{1.481574in}{2.220537in}}%
\pgfpathlineto{\pgfqpoint{1.486087in}{2.227427in}}%
\pgfpathlineto{\pgfqpoint{1.490601in}{2.231833in}}%
\pgfpathlineto{\pgfqpoint{1.495114in}{2.233753in}}%
\pgfpathlineto{\pgfqpoint{1.499628in}{2.233189in}}%
\pgfpathlineto{\pgfqpoint{1.504142in}{2.230140in}}%
\pgfpathlineto{\pgfqpoint{1.508655in}{2.224606in}}%
\pgfpathlineto{\pgfqpoint{1.513169in}{2.216587in}}%
\pgfpathlineto{\pgfqpoint{1.522196in}{2.193093in}}%
\pgfpathlineto{\pgfqpoint{1.531223in}{2.159661in}}%
\pgfpathlineto{\pgfqpoint{1.540250in}{2.116288in}}%
\pgfpathlineto{\pgfqpoint{1.549278in}{2.062976in}}%
\pgfpathlineto{\pgfqpoint{1.562818in}{1.964371in}}%
\pgfpathlineto{\pgfqpoint{1.576359in}{1.843401in}}%
\pgfpathlineto{\pgfqpoint{1.589900in}{1.700067in}}%
\pgfpathlineto{\pgfqpoint{1.603441in}{1.534368in}}%
\pgfpathlineto{\pgfqpoint{1.626009in}{1.240217in}}%
\pgfpathlineto{\pgfqpoint{1.639550in}{1.091246in}}%
\pgfpathlineto{\pgfqpoint{1.653091in}{0.964640in}}%
\pgfpathlineto{\pgfqpoint{1.666631in}{0.860398in}}%
\pgfpathlineto{\pgfqpoint{1.675659in}{0.803328in}}%
\pgfpathlineto{\pgfqpoint{1.684686in}{0.756197in}}%
\pgfpathlineto{\pgfqpoint{1.693713in}{0.719007in}}%
\pgfpathlineto{\pgfqpoint{1.702740in}{0.691756in}}%
\pgfpathlineto{\pgfqpoint{1.711767in}{0.674445in}}%
\pgfpathlineto{\pgfqpoint{1.716281in}{0.669517in}}%
\pgfpathlineto{\pgfqpoint{1.720795in}{0.667074in}}%
\pgfpathlineto{\pgfqpoint{1.725308in}{0.667116in}}%
\pgfpathlineto{\pgfqpoint{1.729822in}{0.669643in}}%
\pgfpathlineto{\pgfqpoint{1.734335in}{0.674654in}}%
\pgfpathlineto{\pgfqpoint{1.738849in}{0.682151in}}%
\pgfpathlineto{\pgfqpoint{1.747876in}{0.704599in}}%
\pgfpathlineto{\pgfqpoint{1.756903in}{0.736986in}}%
\pgfpathlineto{\pgfqpoint{1.765931in}{0.779314in}}%
\pgfpathlineto{\pgfqpoint{1.774958in}{0.831581in}}%
\pgfpathlineto{\pgfqpoint{1.783985in}{0.893788in}}%
\pgfpathlineto{\pgfqpoint{1.797526in}{1.005735in}}%
\pgfpathlineto{\pgfqpoint{1.811067in}{1.140047in}}%
\pgfpathlineto{\pgfqpoint{1.824608in}{1.296723in}}%
\pgfpathlineto{\pgfqpoint{1.838148in}{1.475764in}}%
\pgfpathlineto{\pgfqpoint{1.878771in}{2.049237in}}%
\pgfpathlineto{\pgfqpoint{1.892312in}{2.204095in}}%
\pgfpathlineto{\pgfqpoint{1.905852in}{2.336588in}}%
\pgfpathlineto{\pgfqpoint{1.919393in}{2.446718in}}%
\pgfpathlineto{\pgfqpoint{1.932934in}{2.534482in}}%
\pgfpathlineto{\pgfqpoint{1.941961in}{2.580567in}}%
\pgfpathlineto{\pgfqpoint{1.950988in}{2.616713in}}%
\pgfpathlineto{\pgfqpoint{1.960016in}{2.642918in}}%
\pgfpathlineto{\pgfqpoint{1.969043in}{2.659184in}}%
\pgfpathlineto{\pgfqpoint{1.973557in}{2.663590in}}%
\pgfpathlineto{\pgfqpoint{1.978070in}{2.665511in}}%
\pgfpathlineto{\pgfqpoint{1.982584in}{2.664946in}}%
\pgfpathlineto{\pgfqpoint{1.987097in}{2.661897in}}%
\pgfpathlineto{\pgfqpoint{1.991611in}{2.656363in}}%
\pgfpathlineto{\pgfqpoint{1.996125in}{2.648344in}}%
\pgfpathlineto{\pgfqpoint{2.005152in}{2.624851in}}%
\pgfpathlineto{\pgfqpoint{2.014179in}{2.591418in}}%
\pgfpathlineto{\pgfqpoint{2.023206in}{2.548045in}}%
\pgfpathlineto{\pgfqpoint{2.032233in}{2.494733in}}%
\pgfpathlineto{\pgfqpoint{2.045774in}{2.396128in}}%
\pgfpathlineto{\pgfqpoint{2.059315in}{2.275158in}}%
\pgfpathlineto{\pgfqpoint{2.072856in}{2.131824in}}%
\pgfpathlineto{\pgfqpoint{2.086397in}{1.966125in}}%
\pgfpathlineto{\pgfqpoint{2.099937in}{1.778063in}}%
\pgfpathlineto{\pgfqpoint{2.122505in}{1.449121in}}%
\pgfpathlineto{\pgfqpoint{2.136046in}{1.277787in}}%
\pgfpathlineto{\pgfqpoint{2.149587in}{1.128816in}}%
\pgfpathlineto{\pgfqpoint{2.163128in}{1.002210in}}%
\pgfpathlineto{\pgfqpoint{2.176669in}{0.897968in}}%
\pgfpathlineto{\pgfqpoint{2.185696in}{0.840898in}}%
\pgfpathlineto{\pgfqpoint{2.194723in}{0.793768in}}%
\pgfpathlineto{\pgfqpoint{2.203750in}{0.756577in}}%
\pgfpathlineto{\pgfqpoint{2.212778in}{0.729326in}}%
\pgfpathlineto{\pgfqpoint{2.221805in}{0.712015in}}%
\pgfpathlineto{\pgfqpoint{2.226318in}{0.707087in}}%
\pgfpathlineto{\pgfqpoint{2.230832in}{0.704644in}}%
\pgfpathlineto{\pgfqpoint{2.235346in}{0.704686in}}%
\pgfpathlineto{\pgfqpoint{2.239859in}{0.707213in}}%
\pgfpathlineto{\pgfqpoint{2.244373in}{0.712224in}}%
\pgfpathlineto{\pgfqpoint{2.248886in}{0.719721in}}%
\pgfpathlineto{\pgfqpoint{2.257914in}{0.742169in}}%
\pgfpathlineto{\pgfqpoint{2.266941in}{0.774556in}}%
\pgfpathlineto{\pgfqpoint{2.275968in}{0.816884in}}%
\pgfpathlineto{\pgfqpoint{2.284995in}{0.869151in}}%
\pgfpathlineto{\pgfqpoint{2.294022in}{0.931358in}}%
\pgfpathlineto{\pgfqpoint{2.307563in}{1.043305in}}%
\pgfpathlineto{\pgfqpoint{2.321104in}{1.177617in}}%
\pgfpathlineto{\pgfqpoint{2.334645in}{1.334293in}}%
\pgfpathlineto{\pgfqpoint{2.348186in}{1.513334in}}%
\pgfpathlineto{\pgfqpoint{2.406863in}{2.341407in}}%
\pgfpathlineto{\pgfqpoint{2.420403in}{2.488810in}}%
\pgfpathlineto{\pgfqpoint{2.433944in}{2.613849in}}%
\pgfpathlineto{\pgfqpoint{2.447485in}{2.716523in}}%
\pgfpathlineto{\pgfqpoint{2.456512in}{2.772548in}}%
\pgfpathlineto{\pgfqpoint{2.465539in}{2.818633in}}%
\pgfpathlineto{\pgfqpoint{2.474567in}{2.854779in}}%
\pgfpathlineto{\pgfqpoint{2.483594in}{2.880984in}}%
\pgfpathlineto{\pgfqpoint{2.492621in}{2.897250in}}%
\pgfpathlineto{\pgfqpoint{2.497135in}{2.901656in}}%
\pgfpathlineto{\pgfqpoint{2.501648in}{2.903576in}}%
\pgfpathlineto{\pgfqpoint{2.506162in}{2.903012in}}%
\pgfpathlineto{\pgfqpoint{2.510675in}{2.899963in}}%
\pgfpathlineto{\pgfqpoint{2.515189in}{2.894429in}}%
\pgfpathlineto{\pgfqpoint{2.519703in}{2.886410in}}%
\pgfpathlineto{\pgfqpoint{2.528730in}{2.862917in}}%
\pgfpathlineto{\pgfqpoint{2.537757in}{2.829484in}}%
\pgfpathlineto{\pgfqpoint{2.546784in}{2.786111in}}%
\pgfpathlineto{\pgfqpoint{2.555811in}{2.732799in}}%
\pgfpathlineto{\pgfqpoint{2.569352in}{2.634194in}}%
\pgfpathlineto{\pgfqpoint{2.582893in}{2.513224in}}%
\pgfpathlineto{\pgfqpoint{2.596434in}{2.369890in}}%
\pgfpathlineto{\pgfqpoint{2.609975in}{2.204191in}}%
\pgfpathlineto{\pgfqpoint{2.623516in}{2.016128in}}%
\pgfpathlineto{\pgfqpoint{2.646084in}{1.687187in}}%
\pgfpathlineto{\pgfqpoint{2.659624in}{1.515853in}}%
\pgfpathlineto{\pgfqpoint{2.673165in}{1.366882in}}%
\pgfpathlineto{\pgfqpoint{2.686706in}{1.240276in}}%
\pgfpathlineto{\pgfqpoint{2.700247in}{1.136034in}}%
\pgfpathlineto{\pgfqpoint{2.709274in}{1.078964in}}%
\pgfpathlineto{\pgfqpoint{2.718301in}{1.031833in}}%
\pgfpathlineto{\pgfqpoint{2.727328in}{0.994643in}}%
\pgfpathlineto{\pgfqpoint{2.736356in}{0.967392in}}%
\pgfpathlineto{\pgfqpoint{2.745383in}{0.950081in}}%
\pgfpathlineto{\pgfqpoint{2.749896in}{0.945153in}}%
\pgfpathlineto{\pgfqpoint{2.754410in}{0.942710in}}%
\pgfpathlineto{\pgfqpoint{2.758924in}{0.942752in}}%
\pgfpathlineto{\pgfqpoint{2.763437in}{0.945278in}}%
\pgfpathlineto{\pgfqpoint{2.767951in}{0.950290in}}%
\pgfpathlineto{\pgfqpoint{2.772465in}{0.957787in}}%
\pgfpathlineto{\pgfqpoint{2.781492in}{0.980235in}}%
\pgfpathlineto{\pgfqpoint{2.790519in}{1.012622in}}%
\pgfpathlineto{\pgfqpoint{2.799546in}{1.054950in}}%
\pgfpathlineto{\pgfqpoint{2.808573in}{1.107217in}}%
\pgfpathlineto{\pgfqpoint{2.817601in}{1.169424in}}%
\pgfpathlineto{\pgfqpoint{2.831141in}{1.281371in}}%
\pgfpathlineto{\pgfqpoint{2.844682in}{1.415683in}}%
\pgfpathlineto{\pgfqpoint{2.858223in}{1.572359in}}%
\pgfpathlineto{\pgfqpoint{2.871764in}{1.751400in}}%
\pgfpathlineto{\pgfqpoint{2.930441in}{2.579473in}}%
\pgfpathlineto{\pgfqpoint{2.943981in}{2.726876in}}%
\pgfpathlineto{\pgfqpoint{2.957522in}{2.851915in}}%
\pgfpathlineto{\pgfqpoint{2.971063in}{2.954589in}}%
\pgfpathlineto{\pgfqpoint{2.980090in}{3.010614in}}%
\pgfpathlineto{\pgfqpoint{2.989118in}{3.056699in}}%
\pgfpathlineto{\pgfqpoint{2.998145in}{3.092845in}}%
\pgfpathlineto{\pgfqpoint{3.007172in}{3.119050in}}%
\pgfpathlineto{\pgfqpoint{3.016199in}{3.135316in}}%
\pgfpathlineto{\pgfqpoint{3.020713in}{3.139722in}}%
\pgfpathlineto{\pgfqpoint{3.025226in}{3.141642in}}%
\pgfpathlineto{\pgfqpoint{3.029740in}{3.141078in}}%
\pgfpathlineto{\pgfqpoint{3.034254in}{3.138029in}}%
\pgfpathlineto{\pgfqpoint{3.038767in}{3.132495in}}%
\pgfpathlineto{\pgfqpoint{3.043281in}{3.124476in}}%
\pgfpathlineto{\pgfqpoint{3.052308in}{3.100983in}}%
\pgfpathlineto{\pgfqpoint{3.061335in}{3.067550in}}%
\pgfpathlineto{\pgfqpoint{3.070362in}{3.024177in}}%
\pgfpathlineto{\pgfqpoint{3.079390in}{2.970865in}}%
\pgfpathlineto{\pgfqpoint{3.092930in}{2.872260in}}%
\pgfpathlineto{\pgfqpoint{3.106471in}{2.751290in}}%
\pgfpathlineto{\pgfqpoint{3.120012in}{2.607956in}}%
\pgfpathlineto{\pgfqpoint{3.133553in}{2.442257in}}%
\pgfpathlineto{\pgfqpoint{3.147094in}{2.254194in}}%
\pgfpathlineto{\pgfqpoint{3.169662in}{1.925253in}}%
\pgfpathlineto{\pgfqpoint{3.183203in}{1.753918in}}%
\pgfpathlineto{\pgfqpoint{3.196743in}{1.604948in}}%
\pgfpathlineto{\pgfqpoint{3.210284in}{1.478341in}}%
\pgfpathlineto{\pgfqpoint{3.223825in}{1.374100in}}%
\pgfpathlineto{\pgfqpoint{3.232852in}{1.317030in}}%
\pgfpathlineto{\pgfqpoint{3.241879in}{1.269899in}}%
\pgfpathlineto{\pgfqpoint{3.250907in}{1.232709in}}%
\pgfpathlineto{\pgfqpoint{3.259934in}{1.205458in}}%
\pgfpathlineto{\pgfqpoint{3.268961in}{1.188147in}}%
\pgfpathlineto{\pgfqpoint{3.273475in}{1.183219in}}%
\pgfpathlineto{\pgfqpoint{3.277988in}{1.180776in}}%
\pgfpathlineto{\pgfqpoint{3.282502in}{1.180818in}}%
\pgfpathlineto{\pgfqpoint{3.287015in}{1.183344in}}%
\pgfpathlineto{\pgfqpoint{3.291529in}{1.188356in}}%
\pgfpathlineto{\pgfqpoint{3.296043in}{1.195853in}}%
\pgfpathlineto{\pgfqpoint{3.305070in}{1.218301in}}%
\pgfpathlineto{\pgfqpoint{3.314097in}{1.250688in}}%
\pgfpathlineto{\pgfqpoint{3.323124in}{1.293016in}}%
\pgfpathlineto{\pgfqpoint{3.332151in}{1.345283in}}%
\pgfpathlineto{\pgfqpoint{3.341179in}{1.407490in}}%
\pgfpathlineto{\pgfqpoint{3.354719in}{1.519437in}}%
\pgfpathlineto{\pgfqpoint{3.368260in}{1.653749in}}%
\pgfpathlineto{\pgfqpoint{3.381801in}{1.810425in}}%
\pgfpathlineto{\pgfqpoint{3.395342in}{1.989465in}}%
\pgfpathlineto{\pgfqpoint{3.454019in}{2.817539in}}%
\pgfpathlineto{\pgfqpoint{3.467560in}{2.964942in}}%
\pgfpathlineto{\pgfqpoint{3.481100in}{3.089981in}}%
\pgfpathlineto{\pgfqpoint{3.494641in}{3.192655in}}%
\pgfpathlineto{\pgfqpoint{3.503668in}{3.248680in}}%
\pgfpathlineto{\pgfqpoint{3.512696in}{3.294765in}}%
\pgfpathlineto{\pgfqpoint{3.521723in}{3.330910in}}%
\pgfpathlineto{\pgfqpoint{3.530750in}{3.357116in}}%
\pgfpathlineto{\pgfqpoint{3.539777in}{3.373382in}}%
\pgfpathlineto{\pgfqpoint{3.544291in}{3.377788in}}%
\pgfpathlineto{\pgfqpoint{3.548804in}{3.379708in}}%
\pgfpathlineto{\pgfqpoint{3.553318in}{3.379144in}}%
\pgfpathlineto{\pgfqpoint{3.557832in}{3.376095in}}%
\pgfpathlineto{\pgfqpoint{3.562345in}{3.370561in}}%
\pgfpathlineto{\pgfqpoint{3.566859in}{3.362541in}}%
\pgfpathlineto{\pgfqpoint{3.575886in}{3.339048in}}%
\pgfpathlineto{\pgfqpoint{3.584913in}{3.305616in}}%
\pgfpathlineto{\pgfqpoint{3.593941in}{3.262243in}}%
\pgfpathlineto{\pgfqpoint{3.602968in}{3.208931in}}%
\pgfpathlineto{\pgfqpoint{3.616509in}{3.110326in}}%
\pgfpathlineto{\pgfqpoint{3.630049in}{2.989356in}}%
\pgfpathlineto{\pgfqpoint{3.643590in}{2.846022in}}%
\pgfpathlineto{\pgfqpoint{3.657131in}{2.680323in}}%
\pgfpathlineto{\pgfqpoint{3.670672in}{2.492260in}}%
\pgfpathlineto{\pgfqpoint{3.693240in}{2.163319in}}%
\pgfpathlineto{\pgfqpoint{3.706781in}{1.991984in}}%
\pgfpathlineto{\pgfqpoint{3.720321in}{1.843014in}}%
\pgfpathlineto{\pgfqpoint{3.733862in}{1.716407in}}%
\pgfpathlineto{\pgfqpoint{3.747403in}{1.612165in}}%
\pgfpathlineto{\pgfqpoint{3.756430in}{1.555095in}}%
\pgfpathlineto{\pgfqpoint{3.765457in}{1.507965in}}%
\pgfpathlineto{\pgfqpoint{3.774485in}{1.470775in}}%
\pgfpathlineto{\pgfqpoint{3.783512in}{1.443524in}}%
\pgfpathlineto{\pgfqpoint{3.792539in}{1.426213in}}%
\pgfpathlineto{\pgfqpoint{3.797053in}{1.421285in}}%
\pgfpathlineto{\pgfqpoint{3.801566in}{1.418842in}}%
\pgfpathlineto{\pgfqpoint{3.806080in}{1.418884in}}%
\pgfpathlineto{\pgfqpoint{3.810594in}{1.421410in}}%
\pgfpathlineto{\pgfqpoint{3.815107in}{1.426422in}}%
\pgfpathlineto{\pgfqpoint{3.819621in}{1.433918in}}%
\pgfpathlineto{\pgfqpoint{3.828648in}{1.456366in}}%
\pgfpathlineto{\pgfqpoint{3.837675in}{1.488754in}}%
\pgfpathlineto{\pgfqpoint{3.846702in}{1.531082in}}%
\pgfpathlineto{\pgfqpoint{3.855730in}{1.583349in}}%
\pgfpathlineto{\pgfqpoint{3.864757in}{1.645556in}}%
\pgfpathlineto{\pgfqpoint{3.878298in}{1.757503in}}%
\pgfpathlineto{\pgfqpoint{3.891838in}{1.891815in}}%
\pgfpathlineto{\pgfqpoint{3.905379in}{2.048491in}}%
\pgfpathlineto{\pgfqpoint{3.918920in}{2.227531in}}%
\pgfpathlineto{\pgfqpoint{3.977597in}{3.055605in}}%
\pgfpathlineto{\pgfqpoint{3.991138in}{3.203008in}}%
\pgfpathlineto{\pgfqpoint{4.004679in}{3.328047in}}%
\pgfpathlineto{\pgfqpoint{4.018219in}{3.430721in}}%
\pgfpathlineto{\pgfqpoint{4.027247in}{3.486746in}}%
\pgfpathlineto{\pgfqpoint{4.036274in}{3.532831in}}%
\pgfpathlineto{\pgfqpoint{4.045301in}{3.568976in}}%
\pgfpathlineto{\pgfqpoint{4.054328in}{3.595182in}}%
\pgfpathlineto{\pgfqpoint{4.063355in}{3.611448in}}%
\pgfpathlineto{\pgfqpoint{4.067869in}{3.615854in}}%
\pgfpathlineto{\pgfqpoint{4.072383in}{3.617774in}}%
\pgfpathlineto{\pgfqpoint{4.076896in}{3.617210in}}%
\pgfpathlineto{\pgfqpoint{4.081410in}{3.614161in}}%
\pgfpathlineto{\pgfqpoint{4.085923in}{3.608626in}}%
\pgfpathlineto{\pgfqpoint{4.090437in}{3.600607in}}%
\pgfpathlineto{\pgfqpoint{4.099464in}{3.577114in}}%
\pgfpathlineto{\pgfqpoint{4.108491in}{3.543682in}}%
\pgfpathlineto{\pgfqpoint{4.117519in}{3.500309in}}%
\pgfpathlineto{\pgfqpoint{4.126546in}{3.446997in}}%
\pgfpathlineto{\pgfqpoint{4.140087in}{3.348392in}}%
\pgfpathlineto{\pgfqpoint{4.153627in}{3.227422in}}%
\pgfpathlineto{\pgfqpoint{4.167168in}{3.084088in}}%
\pgfpathlineto{\pgfqpoint{4.180709in}{2.918389in}}%
\pgfpathlineto{\pgfqpoint{4.194250in}{2.730326in}}%
\pgfpathlineto{\pgfqpoint{4.216818in}{2.401385in}}%
\pgfpathlineto{\pgfqpoint{4.230359in}{2.230050in}}%
\pgfpathlineto{\pgfqpoint{4.243900in}{2.081080in}}%
\pgfpathlineto{\pgfqpoint{4.257440in}{1.954473in}}%
\pgfpathlineto{\pgfqpoint{4.270981in}{1.850231in}}%
\pgfpathlineto{\pgfqpoint{4.280008in}{1.793161in}}%
\pgfpathlineto{\pgfqpoint{4.289036in}{1.746031in}}%
\pgfpathlineto{\pgfqpoint{4.298063in}{1.708841in}}%
\pgfpathlineto{\pgfqpoint{4.307090in}{1.681590in}}%
\pgfpathlineto{\pgfqpoint{4.316117in}{1.664279in}}%
\pgfpathlineto{\pgfqpoint{4.320631in}{1.659351in}}%
\pgfpathlineto{\pgfqpoint{4.325144in}{1.656908in}}%
\pgfpathlineto{\pgfqpoint{4.329658in}{1.656949in}}%
\pgfpathlineto{\pgfqpoint{4.334172in}{1.659476in}}%
\pgfpathlineto{\pgfqpoint{4.338685in}{1.664488in}}%
\pgfpathlineto{\pgfqpoint{4.343199in}{1.671984in}}%
\pgfpathlineto{\pgfqpoint{4.352226in}{1.694432in}}%
\pgfpathlineto{\pgfqpoint{4.361253in}{1.726820in}}%
\pgfpathlineto{\pgfqpoint{4.370280in}{1.769147in}}%
\pgfpathlineto{\pgfqpoint{4.379308in}{1.821415in}}%
\pgfpathlineto{\pgfqpoint{4.388335in}{1.883621in}}%
\pgfpathlineto{\pgfqpoint{4.401876in}{1.995569in}}%
\pgfpathlineto{\pgfqpoint{4.415417in}{2.129881in}}%
\pgfpathlineto{\pgfqpoint{4.428957in}{2.286557in}}%
\pgfpathlineto{\pgfqpoint{4.442498in}{2.465597in}}%
\pgfpathlineto{\pgfqpoint{4.501175in}{3.293671in}}%
\pgfpathlineto{\pgfqpoint{4.514716in}{3.441074in}}%
\pgfpathlineto{\pgfqpoint{4.528257in}{3.566113in}}%
\pgfpathlineto{\pgfqpoint{4.541797in}{3.668787in}}%
\pgfpathlineto{\pgfqpoint{4.550825in}{3.724812in}}%
\pgfpathlineto{\pgfqpoint{4.559852in}{3.770897in}}%
\pgfpathlineto{\pgfqpoint{4.568879in}{3.807042in}}%
\pgfpathlineto{\pgfqpoint{4.577906in}{3.833248in}}%
\pgfpathlineto{\pgfqpoint{4.586934in}{3.849514in}}%
\pgfpathlineto{\pgfqpoint{4.591447in}{3.853919in}}%
\pgfpathlineto{\pgfqpoint{4.595961in}{3.855840in}}%
\pgfpathlineto{\pgfqpoint{4.600474in}{3.855276in}}%
\pgfpathlineto{\pgfqpoint{4.604988in}{3.852227in}}%
\pgfpathlineto{\pgfqpoint{4.609502in}{3.846692in}}%
\pgfpathlineto{\pgfqpoint{4.614015in}{3.838673in}}%
\pgfpathlineto{\pgfqpoint{4.623042in}{3.815180in}}%
\pgfpathlineto{\pgfqpoint{4.632070in}{3.781747in}}%
\pgfpathlineto{\pgfqpoint{4.641097in}{3.738375in}}%
\pgfpathlineto{\pgfqpoint{4.650124in}{3.685063in}}%
\pgfpathlineto{\pgfqpoint{4.663665in}{3.586457in}}%
\pgfpathlineto{\pgfqpoint{4.677206in}{3.465488in}}%
\pgfpathlineto{\pgfqpoint{4.690746in}{3.322154in}}%
\pgfpathlineto{\pgfqpoint{4.704287in}{3.156455in}}%
\pgfpathlineto{\pgfqpoint{4.717828in}{2.968392in}}%
\pgfpathlineto{\pgfqpoint{4.740396in}{2.639451in}}%
\pgfpathlineto{\pgfqpoint{4.753937in}{2.468116in}}%
\pgfpathlineto{\pgfqpoint{4.767478in}{2.319145in}}%
\pgfpathlineto{\pgfqpoint{4.781019in}{2.192539in}}%
\pgfpathlineto{\pgfqpoint{4.794559in}{2.088297in}}%
\pgfpathlineto{\pgfqpoint{4.803587in}{2.031227in}}%
\pgfpathlineto{\pgfqpoint{4.812614in}{1.984097in}}%
\pgfpathlineto{\pgfqpoint{4.821641in}{1.946907in}}%
\pgfpathlineto{\pgfqpoint{4.830668in}{1.919656in}}%
\pgfpathlineto{\pgfqpoint{4.839695in}{1.902345in}}%
\pgfpathlineto{\pgfqpoint{4.844209in}{1.897417in}}%
\pgfpathlineto{\pgfqpoint{4.848723in}{1.894974in}}%
\pgfpathlineto{\pgfqpoint{4.853236in}{1.895015in}}%
\pgfpathlineto{\pgfqpoint{4.857750in}{1.897542in}}%
\pgfpathlineto{\pgfqpoint{4.862263in}{1.902554in}}%
\pgfpathlineto{\pgfqpoint{4.866777in}{1.910050in}}%
\pgfpathlineto{\pgfqpoint{4.875804in}{1.932498in}}%
\pgfpathlineto{\pgfqpoint{4.884831in}{1.964886in}}%
\pgfpathlineto{\pgfqpoint{4.893859in}{2.007213in}}%
\pgfpathlineto{\pgfqpoint{4.902886in}{2.059480in}}%
\pgfpathlineto{\pgfqpoint{4.911913in}{2.121687in}}%
\pgfpathlineto{\pgfqpoint{4.925454in}{2.233635in}}%
\pgfpathlineto{\pgfqpoint{4.938995in}{2.367946in}}%
\pgfpathlineto{\pgfqpoint{4.952535in}{2.524623in}}%
\pgfpathlineto{\pgfqpoint{4.966076in}{2.703663in}}%
\pgfpathlineto{\pgfqpoint{5.020240in}{3.464905in}}%
\pgfpathlineto{\pgfqpoint{5.033780in}{3.612308in}}%
\pgfpathlineto{\pgfqpoint{5.047321in}{3.737346in}}%
\pgfpathlineto{\pgfqpoint{5.060862in}{3.840021in}}%
\pgfpathlineto{\pgfqpoint{5.069889in}{3.896046in}}%
\pgfpathlineto{\pgfqpoint{5.078916in}{3.942131in}}%
\pgfpathlineto{\pgfqpoint{5.087944in}{3.978276in}}%
\pgfpathlineto{\pgfqpoint{5.096971in}{4.004482in}}%
\pgfpathlineto{\pgfqpoint{5.105998in}{4.020748in}}%
\pgfpathlineto{\pgfqpoint{5.110512in}{4.025153in}}%
\pgfpathlineto{\pgfqpoint{5.115025in}{4.027074in}}%
\pgfpathlineto{\pgfqpoint{5.119539in}{4.026510in}}%
\pgfpathlineto{\pgfqpoint{5.124052in}{4.023460in}}%
\pgfpathlineto{\pgfqpoint{5.128566in}{4.017926in}}%
\pgfpathlineto{\pgfqpoint{5.133080in}{4.009907in}}%
\pgfpathlineto{\pgfqpoint{5.142107in}{3.986414in}}%
\pgfpathlineto{\pgfqpoint{5.151134in}{3.952981in}}%
\pgfpathlineto{\pgfqpoint{5.160161in}{3.909609in}}%
\pgfpathlineto{\pgfqpoint{5.160161in}{3.909609in}}%
\pgfusepath{stroke}%
\end{pgfscope}%
\begin{pgfscope}%
\pgfsetrectcap%
\pgfsetmiterjoin%
\pgfsetlinewidth{0.803000pt}%
\definecolor{currentstroke}{rgb}{0.000000,0.000000,0.000000}%
\pgfsetstrokecolor{currentstroke}%
\pgfsetdash{}{0pt}%
\pgfpathmoveto{\pgfqpoint{0.425616in}{0.499074in}}%
\pgfpathlineto{\pgfqpoint{0.425616in}{4.195074in}}%
\pgfusepath{stroke}%
\end{pgfscope}%
\begin{pgfscope}%
\pgfsetrectcap%
\pgfsetmiterjoin%
\pgfsetlinewidth{0.803000pt}%
\definecolor{currentstroke}{rgb}{0.000000,0.000000,0.000000}%
\pgfsetstrokecolor{currentstroke}%
\pgfsetdash{}{0pt}%
\pgfpathmoveto{\pgfqpoint{5.385616in}{0.499074in}}%
\pgfpathlineto{\pgfqpoint{5.385616in}{4.195074in}}%
\pgfusepath{stroke}%
\end{pgfscope}%
\begin{pgfscope}%
\pgfsetrectcap%
\pgfsetmiterjoin%
\pgfsetlinewidth{0.803000pt}%
\definecolor{currentstroke}{rgb}{0.000000,0.000000,0.000000}%
\pgfsetstrokecolor{currentstroke}%
\pgfsetdash{}{0pt}%
\pgfpathmoveto{\pgfqpoint{0.425616in}{0.499074in}}%
\pgfpathlineto{\pgfqpoint{5.385616in}{0.499074in}}%
\pgfusepath{stroke}%
\end{pgfscope}%
\begin{pgfscope}%
\pgfsetrectcap%
\pgfsetmiterjoin%
\pgfsetlinewidth{0.803000pt}%
\definecolor{currentstroke}{rgb}{0.000000,0.000000,0.000000}%
\pgfsetstrokecolor{currentstroke}%
\pgfsetdash{}{0pt}%
\pgfpathmoveto{\pgfqpoint{0.425616in}{4.195074in}}%
\pgfpathlineto{\pgfqpoint{5.385616in}{4.195074in}}%
\pgfusepath{stroke}%
\end{pgfscope}%
\begin{pgfscope}%
\pgfsetbuttcap%
\pgfsetmiterjoin%
\definecolor{currentfill}{rgb}{1.000000,1.000000,1.000000}%
\pgfsetfillcolor{currentfill}%
\pgfsetfillopacity{0.800000}%
\pgfsetlinewidth{1.003750pt}%
\definecolor{currentstroke}{rgb}{0.800000,0.800000,0.800000}%
\pgfsetstrokecolor{currentstroke}%
\pgfsetstrokeopacity{0.800000}%
\pgfsetdash{}{0pt}%
\pgfpathmoveto{\pgfqpoint{0.522838in}{3.696617in}}%
\pgfpathlineto{\pgfqpoint{2.340742in}{3.696617in}}%
\pgfpathquadraticcurveto{\pgfqpoint{2.368520in}{3.696617in}}{\pgfqpoint{2.368520in}{3.724395in}}%
\pgfpathlineto{\pgfqpoint{2.368520in}{4.097852in}}%
\pgfpathquadraticcurveto{\pgfqpoint{2.368520in}{4.125630in}}{\pgfqpoint{2.340742in}{4.125630in}}%
\pgfpathlineto{\pgfqpoint{0.522838in}{4.125630in}}%
\pgfpathquadraticcurveto{\pgfqpoint{0.495060in}{4.125630in}}{\pgfqpoint{0.495060in}{4.097852in}}%
\pgfpathlineto{\pgfqpoint{0.495060in}{3.724395in}}%
\pgfpathquadraticcurveto{\pgfqpoint{0.495060in}{3.696617in}}{\pgfqpoint{0.522838in}{3.696617in}}%
\pgfpathlineto{\pgfqpoint{0.522838in}{3.696617in}}%
\pgfpathclose%
\pgfusepath{stroke,fill}%
\end{pgfscope}%
\begin{pgfscope}%
\pgfsetrectcap%
\pgfsetroundjoin%
\pgfsetlinewidth{1.505625pt}%
\definecolor{currentstroke}{rgb}{0.121569,0.466667,0.705882}%
\pgfsetstrokecolor{currentstroke}%
\pgfsetdash{}{0pt}%
\pgfpathmoveto{\pgfqpoint{0.550616in}{4.021463in}}%
\pgfpathlineto{\pgfqpoint{0.689505in}{4.021463in}}%
\pgfpathlineto{\pgfqpoint{0.828394in}{4.021463in}}%
\pgfusepath{stroke}%
\end{pgfscope}%
\begin{pgfscope}%
\definecolor{textcolor}{rgb}{0.000000,0.000000,0.000000}%
\pgfsetstrokecolor{textcolor}%
\pgfsetfillcolor{textcolor}%
\pgftext[x=0.939505in,y=3.972852in,left,base]{\color{textcolor}{\rmfamily\fontsize{10.000000}{12.000000}\selectfont\catcode`\^=\active\def^{\ifmmode\sp\else\^{}\fi}\catcode`\%=\active\def%{\%}Point mass orientation}}%
\end{pgfscope}%
\begin{pgfscope}%
\pgfsetrectcap%
\pgfsetroundjoin%
\pgfsetlinewidth{1.505625pt}%
\definecolor{currentstroke}{rgb}{1.000000,0.498039,0.054902}%
\pgfsetstrokecolor{currentstroke}%
\pgfsetdash{}{0pt}%
\pgfpathmoveto{\pgfqpoint{0.550616in}{3.827790in}}%
\pgfpathlineto{\pgfqpoint{0.689505in}{3.827790in}}%
\pgfpathlineto{\pgfqpoint{0.828394in}{3.827790in}}%
\pgfusepath{stroke}%
\end{pgfscope}%
\begin{pgfscope}%
\definecolor{textcolor}{rgb}{0.000000,0.000000,0.000000}%
\pgfsetstrokecolor{textcolor}%
\pgfsetfillcolor{textcolor}%
\pgftext[x=0.939505in,y=3.779179in,left,base]{\color{textcolor}{\rmfamily\fontsize{10.000000}{12.000000}\selectfont\catcode`\^=\active\def^{\ifmmode\sp\else\^{}\fi}\catcode`\%=\active\def%{\%}Car Orientation}}%
\end{pgfscope}%
\end{pgfpicture}%
\makeatother%
\endgroup%
}
	\caption{Oscillation in vehicle orientation due to lag in dynamics.}
	\label{fig:lag_orientation}
\end{figure}

To address this problem, we employed a better discretization scheme and accounted for the lag in the feedback loop.
By doing so, we were able to mitigate the overcorrection and eliminate the oscillation, leading to a more stable and accurate control of the
vehicle's orientation.

\subsection{McCormick Relaxation}

To illustrate the application of these relaxations in practice, consider a path-planning scenario with \( v_{min} = 1 \), \( v_{max} = 5 \), and \(
v_{start} = 1 \).
In this scenario, the bilinear term \( v\xi \), which appears in the equation of motion for \(\dot{n} = v \sin{\xi} \approx v\xi\), is approximated
using McCormick relaxations.

\begin{figure}[h]
	\centering
	\resizebox{1\textwidth}{!}{%% Creator: Matplotlib, PGF backend
%%
%% To include the figure in your LaTeX document, write
%%   \input{<filename>.pgf}
%%
%% Make sure the required packages are loaded in your preamble
%%   \usepackage{pgf}
%%
%% Also ensure that all the required font packages are loaded; for instance,
%% the lmodern package is sometimes necessary when using math font.
%%   \usepackage{lmodern}
%%
%% Figures using additional raster images can only be included by \input if
%% they are in the same directory as the main LaTeX file. For loading figures
%% from other directories you can use the `import` package
%%   \usepackage{import}
%%
%% and then include the figures with
%%   \import{<path to file>}{<filename>.pgf}
%%
%% Matplotlib used the following preamble
%%   \def\mathdefault#1{#1}
%%   \everymath=\expandafter{\the\everymath\displaystyle}
%%   
%%   \ifdefined\pdftexversion\else  % non-pdftex case.
%%     \usepackage{fontspec}
%%   \fi
%%   \makeatletter\@ifpackageloaded{underscore}{}{\usepackage[strings]{underscore}}\makeatother
%%
\begingroup%
\makeatletter%
\begin{pgfpicture}%
\pgfpathrectangle{\pgfpointorigin}{\pgfqpoint{9.855329in}{1.757149in}}%
\pgfusepath{use as bounding box, clip}%
\begin{pgfscope}%
\pgfsetbuttcap%
\pgfsetmiterjoin%
\definecolor{currentfill}{rgb}{1.000000,1.000000,1.000000}%
\pgfsetfillcolor{currentfill}%
\pgfsetlinewidth{0.000000pt}%
\definecolor{currentstroke}{rgb}{1.000000,1.000000,1.000000}%
\pgfsetstrokecolor{currentstroke}%
\pgfsetdash{}{0pt}%
\pgfpathmoveto{\pgfqpoint{0.000000in}{0.000000in}}%
\pgfpathlineto{\pgfqpoint{9.855329in}{0.000000in}}%
\pgfpathlineto{\pgfqpoint{9.855329in}{1.757149in}}%
\pgfpathlineto{\pgfqpoint{0.000000in}{1.757149in}}%
\pgfpathlineto{\pgfqpoint{0.000000in}{0.000000in}}%
\pgfpathclose%
\pgfusepath{fill}%
\end{pgfscope}%
\begin{pgfscope}%
\pgfsetbuttcap%
\pgfsetmiterjoin%
\definecolor{currentfill}{rgb}{1.000000,1.000000,1.000000}%
\pgfsetfillcolor{currentfill}%
\pgfsetlinewidth{0.000000pt}%
\definecolor{currentstroke}{rgb}{0.000000,0.000000,0.000000}%
\pgfsetstrokecolor{currentstroke}%
\pgfsetstrokeopacity{0.000000}%
\pgfsetdash{}{0pt}%
\pgfpathmoveto{\pgfqpoint{0.452199in}{0.515972in}}%
\pgfpathlineto{\pgfqpoint{9.755329in}{0.515972in}}%
\pgfpathlineto{\pgfqpoint{9.755329in}{1.657149in}}%
\pgfpathlineto{\pgfqpoint{0.452199in}{1.657149in}}%
\pgfpathlineto{\pgfqpoint{0.452199in}{0.515972in}}%
\pgfpathclose%
\pgfusepath{fill}%
\end{pgfscope}%
\begin{pgfscope}%
\pgfsetbuttcap%
\pgfsetroundjoin%
\definecolor{currentfill}{rgb}{0.000000,0.000000,0.000000}%
\pgfsetfillcolor{currentfill}%
\pgfsetlinewidth{0.803000pt}%
\definecolor{currentstroke}{rgb}{0.000000,0.000000,0.000000}%
\pgfsetstrokecolor{currentstroke}%
\pgfsetdash{}{0pt}%
\pgfsys@defobject{currentmarker}{\pgfqpoint{0.000000in}{-0.048611in}}{\pgfqpoint{0.000000in}{0.000000in}}{%
\pgfpathmoveto{\pgfqpoint{0.000000in}{0.000000in}}%
\pgfpathlineto{\pgfqpoint{0.000000in}{-0.048611in}}%
\pgfusepath{stroke,fill}%
}%
\begin{pgfscope}%
\pgfsys@transformshift{0.875068in}{0.515972in}%
\pgfsys@useobject{currentmarker}{}%
\end{pgfscope}%
\end{pgfscope}%
\begin{pgfscope}%
\definecolor{textcolor}{rgb}{0.000000,0.000000,0.000000}%
\pgfsetstrokecolor{textcolor}%
\pgfsetfillcolor{textcolor}%
\pgftext[x=0.875068in,y=0.418750in,,top]{\color{textcolor}{\rmfamily\fontsize{9.000000}{10.800000}\selectfont\catcode`\^=\active\def^{\ifmmode\sp\else\^{}\fi}\catcode`\%=\active\def%{\%}$\mathdefault{0.0}$}}%
\end{pgfscope}%
\begin{pgfscope}%
\pgfsetbuttcap%
\pgfsetroundjoin%
\definecolor{currentfill}{rgb}{0.000000,0.000000,0.000000}%
\pgfsetfillcolor{currentfill}%
\pgfsetlinewidth{0.803000pt}%
\definecolor{currentstroke}{rgb}{0.000000,0.000000,0.000000}%
\pgfsetstrokecolor{currentstroke}%
\pgfsetdash{}{0pt}%
\pgfsys@defobject{currentmarker}{\pgfqpoint{0.000000in}{-0.048611in}}{\pgfqpoint{0.000000in}{0.000000in}}{%
\pgfpathmoveto{\pgfqpoint{0.000000in}{0.000000in}}%
\pgfpathlineto{\pgfqpoint{0.000000in}{-0.048611in}}%
\pgfusepath{stroke,fill}%
}%
\begin{pgfscope}%
\pgfsys@transformshift{1.932242in}{0.515972in}%
\pgfsys@useobject{currentmarker}{}%
\end{pgfscope}%
\end{pgfscope}%
\begin{pgfscope}%
\definecolor{textcolor}{rgb}{0.000000,0.000000,0.000000}%
\pgfsetstrokecolor{textcolor}%
\pgfsetfillcolor{textcolor}%
\pgftext[x=1.932242in,y=0.418750in,,top]{\color{textcolor}{\rmfamily\fontsize{9.000000}{10.800000}\selectfont\catcode`\^=\active\def^{\ifmmode\sp\else\^{}\fi}\catcode`\%=\active\def%{\%}$\mathdefault{0.5}$}}%
\end{pgfscope}%
\begin{pgfscope}%
\pgfsetbuttcap%
\pgfsetroundjoin%
\definecolor{currentfill}{rgb}{0.000000,0.000000,0.000000}%
\pgfsetfillcolor{currentfill}%
\pgfsetlinewidth{0.803000pt}%
\definecolor{currentstroke}{rgb}{0.000000,0.000000,0.000000}%
\pgfsetstrokecolor{currentstroke}%
\pgfsetdash{}{0pt}%
\pgfsys@defobject{currentmarker}{\pgfqpoint{0.000000in}{-0.048611in}}{\pgfqpoint{0.000000in}{0.000000in}}{%
\pgfpathmoveto{\pgfqpoint{0.000000in}{0.000000in}}%
\pgfpathlineto{\pgfqpoint{0.000000in}{-0.048611in}}%
\pgfusepath{stroke,fill}%
}%
\begin{pgfscope}%
\pgfsys@transformshift{2.989416in}{0.515972in}%
\pgfsys@useobject{currentmarker}{}%
\end{pgfscope}%
\end{pgfscope}%
\begin{pgfscope}%
\definecolor{textcolor}{rgb}{0.000000,0.000000,0.000000}%
\pgfsetstrokecolor{textcolor}%
\pgfsetfillcolor{textcolor}%
\pgftext[x=2.989416in,y=0.418750in,,top]{\color{textcolor}{\rmfamily\fontsize{9.000000}{10.800000}\selectfont\catcode`\^=\active\def^{\ifmmode\sp\else\^{}\fi}\catcode`\%=\active\def%{\%}$\mathdefault{1.0}$}}%
\end{pgfscope}%
\begin{pgfscope}%
\pgfsetbuttcap%
\pgfsetroundjoin%
\definecolor{currentfill}{rgb}{0.000000,0.000000,0.000000}%
\pgfsetfillcolor{currentfill}%
\pgfsetlinewidth{0.803000pt}%
\definecolor{currentstroke}{rgb}{0.000000,0.000000,0.000000}%
\pgfsetstrokecolor{currentstroke}%
\pgfsetdash{}{0pt}%
\pgfsys@defobject{currentmarker}{\pgfqpoint{0.000000in}{-0.048611in}}{\pgfqpoint{0.000000in}{0.000000in}}{%
\pgfpathmoveto{\pgfqpoint{0.000000in}{0.000000in}}%
\pgfpathlineto{\pgfqpoint{0.000000in}{-0.048611in}}%
\pgfusepath{stroke,fill}%
}%
\begin{pgfscope}%
\pgfsys@transformshift{4.046590in}{0.515972in}%
\pgfsys@useobject{currentmarker}{}%
\end{pgfscope}%
\end{pgfscope}%
\begin{pgfscope}%
\definecolor{textcolor}{rgb}{0.000000,0.000000,0.000000}%
\pgfsetstrokecolor{textcolor}%
\pgfsetfillcolor{textcolor}%
\pgftext[x=4.046590in,y=0.418750in,,top]{\color{textcolor}{\rmfamily\fontsize{9.000000}{10.800000}\selectfont\catcode`\^=\active\def^{\ifmmode\sp\else\^{}\fi}\catcode`\%=\active\def%{\%}$\mathdefault{1.5}$}}%
\end{pgfscope}%
\begin{pgfscope}%
\pgfsetbuttcap%
\pgfsetroundjoin%
\definecolor{currentfill}{rgb}{0.000000,0.000000,0.000000}%
\pgfsetfillcolor{currentfill}%
\pgfsetlinewidth{0.803000pt}%
\definecolor{currentstroke}{rgb}{0.000000,0.000000,0.000000}%
\pgfsetstrokecolor{currentstroke}%
\pgfsetdash{}{0pt}%
\pgfsys@defobject{currentmarker}{\pgfqpoint{0.000000in}{-0.048611in}}{\pgfqpoint{0.000000in}{0.000000in}}{%
\pgfpathmoveto{\pgfqpoint{0.000000in}{0.000000in}}%
\pgfpathlineto{\pgfqpoint{0.000000in}{-0.048611in}}%
\pgfusepath{stroke,fill}%
}%
\begin{pgfscope}%
\pgfsys@transformshift{5.103764in}{0.515972in}%
\pgfsys@useobject{currentmarker}{}%
\end{pgfscope}%
\end{pgfscope}%
\begin{pgfscope}%
\definecolor{textcolor}{rgb}{0.000000,0.000000,0.000000}%
\pgfsetstrokecolor{textcolor}%
\pgfsetfillcolor{textcolor}%
\pgftext[x=5.103764in,y=0.418750in,,top]{\color{textcolor}{\rmfamily\fontsize{9.000000}{10.800000}\selectfont\catcode`\^=\active\def^{\ifmmode\sp\else\^{}\fi}\catcode`\%=\active\def%{\%}$\mathdefault{2.0}$}}%
\end{pgfscope}%
\begin{pgfscope}%
\pgfsetbuttcap%
\pgfsetroundjoin%
\definecolor{currentfill}{rgb}{0.000000,0.000000,0.000000}%
\pgfsetfillcolor{currentfill}%
\pgfsetlinewidth{0.803000pt}%
\definecolor{currentstroke}{rgb}{0.000000,0.000000,0.000000}%
\pgfsetstrokecolor{currentstroke}%
\pgfsetdash{}{0pt}%
\pgfsys@defobject{currentmarker}{\pgfqpoint{0.000000in}{-0.048611in}}{\pgfqpoint{0.000000in}{0.000000in}}{%
\pgfpathmoveto{\pgfqpoint{0.000000in}{0.000000in}}%
\pgfpathlineto{\pgfqpoint{0.000000in}{-0.048611in}}%
\pgfusepath{stroke,fill}%
}%
\begin{pgfscope}%
\pgfsys@transformshift{6.160938in}{0.515972in}%
\pgfsys@useobject{currentmarker}{}%
\end{pgfscope}%
\end{pgfscope}%
\begin{pgfscope}%
\definecolor{textcolor}{rgb}{0.000000,0.000000,0.000000}%
\pgfsetstrokecolor{textcolor}%
\pgfsetfillcolor{textcolor}%
\pgftext[x=6.160938in,y=0.418750in,,top]{\color{textcolor}{\rmfamily\fontsize{9.000000}{10.800000}\selectfont\catcode`\^=\active\def^{\ifmmode\sp\else\^{}\fi}\catcode`\%=\active\def%{\%}$\mathdefault{2.5}$}}%
\end{pgfscope}%
\begin{pgfscope}%
\pgfsetbuttcap%
\pgfsetroundjoin%
\definecolor{currentfill}{rgb}{0.000000,0.000000,0.000000}%
\pgfsetfillcolor{currentfill}%
\pgfsetlinewidth{0.803000pt}%
\definecolor{currentstroke}{rgb}{0.000000,0.000000,0.000000}%
\pgfsetstrokecolor{currentstroke}%
\pgfsetdash{}{0pt}%
\pgfsys@defobject{currentmarker}{\pgfqpoint{0.000000in}{-0.048611in}}{\pgfqpoint{0.000000in}{0.000000in}}{%
\pgfpathmoveto{\pgfqpoint{0.000000in}{0.000000in}}%
\pgfpathlineto{\pgfqpoint{0.000000in}{-0.048611in}}%
\pgfusepath{stroke,fill}%
}%
\begin{pgfscope}%
\pgfsys@transformshift{7.218112in}{0.515972in}%
\pgfsys@useobject{currentmarker}{}%
\end{pgfscope}%
\end{pgfscope}%
\begin{pgfscope}%
\definecolor{textcolor}{rgb}{0.000000,0.000000,0.000000}%
\pgfsetstrokecolor{textcolor}%
\pgfsetfillcolor{textcolor}%
\pgftext[x=7.218112in,y=0.418750in,,top]{\color{textcolor}{\rmfamily\fontsize{9.000000}{10.800000}\selectfont\catcode`\^=\active\def^{\ifmmode\sp\else\^{}\fi}\catcode`\%=\active\def%{\%}$\mathdefault{3.0}$}}%
\end{pgfscope}%
\begin{pgfscope}%
\pgfsetbuttcap%
\pgfsetroundjoin%
\definecolor{currentfill}{rgb}{0.000000,0.000000,0.000000}%
\pgfsetfillcolor{currentfill}%
\pgfsetlinewidth{0.803000pt}%
\definecolor{currentstroke}{rgb}{0.000000,0.000000,0.000000}%
\pgfsetstrokecolor{currentstroke}%
\pgfsetdash{}{0pt}%
\pgfsys@defobject{currentmarker}{\pgfqpoint{0.000000in}{-0.048611in}}{\pgfqpoint{0.000000in}{0.000000in}}{%
\pgfpathmoveto{\pgfqpoint{0.000000in}{0.000000in}}%
\pgfpathlineto{\pgfqpoint{0.000000in}{-0.048611in}}%
\pgfusepath{stroke,fill}%
}%
\begin{pgfscope}%
\pgfsys@transformshift{8.275286in}{0.515972in}%
\pgfsys@useobject{currentmarker}{}%
\end{pgfscope}%
\end{pgfscope}%
\begin{pgfscope}%
\definecolor{textcolor}{rgb}{0.000000,0.000000,0.000000}%
\pgfsetstrokecolor{textcolor}%
\pgfsetfillcolor{textcolor}%
\pgftext[x=8.275286in,y=0.418750in,,top]{\color{textcolor}{\rmfamily\fontsize{9.000000}{10.800000}\selectfont\catcode`\^=\active\def^{\ifmmode\sp\else\^{}\fi}\catcode`\%=\active\def%{\%}$\mathdefault{3.5}$}}%
\end{pgfscope}%
\begin{pgfscope}%
\pgfsetbuttcap%
\pgfsetroundjoin%
\definecolor{currentfill}{rgb}{0.000000,0.000000,0.000000}%
\pgfsetfillcolor{currentfill}%
\pgfsetlinewidth{0.803000pt}%
\definecolor{currentstroke}{rgb}{0.000000,0.000000,0.000000}%
\pgfsetstrokecolor{currentstroke}%
\pgfsetdash{}{0pt}%
\pgfsys@defobject{currentmarker}{\pgfqpoint{0.000000in}{-0.048611in}}{\pgfqpoint{0.000000in}{0.000000in}}{%
\pgfpathmoveto{\pgfqpoint{0.000000in}{0.000000in}}%
\pgfpathlineto{\pgfqpoint{0.000000in}{-0.048611in}}%
\pgfusepath{stroke,fill}%
}%
\begin{pgfscope}%
\pgfsys@transformshift{9.332460in}{0.515972in}%
\pgfsys@useobject{currentmarker}{}%
\end{pgfscope}%
\end{pgfscope}%
\begin{pgfscope}%
\definecolor{textcolor}{rgb}{0.000000,0.000000,0.000000}%
\pgfsetstrokecolor{textcolor}%
\pgfsetfillcolor{textcolor}%
\pgftext[x=9.332460in,y=0.418750in,,top]{\color{textcolor}{\rmfamily\fontsize{9.000000}{10.800000}\selectfont\catcode`\^=\active\def^{\ifmmode\sp\else\^{}\fi}\catcode`\%=\active\def%{\%}$\mathdefault{4.0}$}}%
\end{pgfscope}%
\begin{pgfscope}%
\definecolor{textcolor}{rgb}{0.000000,0.000000,0.000000}%
\pgfsetstrokecolor{textcolor}%
\pgfsetfillcolor{textcolor}%
\pgftext[x=5.103764in,y=0.252083in,,top]{\color{textcolor}{\rmfamily\fontsize{11.000000}{13.200000}\selectfont\catcode`\^=\active\def^{\ifmmode\sp\else\^{}\fi}\catcode`\%=\active\def%{\%}Time [s]}}%
\end{pgfscope}%
\begin{pgfscope}%
\pgfsetbuttcap%
\pgfsetroundjoin%
\definecolor{currentfill}{rgb}{0.000000,0.000000,0.000000}%
\pgfsetfillcolor{currentfill}%
\pgfsetlinewidth{0.803000pt}%
\definecolor{currentstroke}{rgb}{0.000000,0.000000,0.000000}%
\pgfsetstrokecolor{currentstroke}%
\pgfsetdash{}{0pt}%
\pgfsys@defobject{currentmarker}{\pgfqpoint{-0.048611in}{0.000000in}}{\pgfqpoint{-0.000000in}{0.000000in}}{%
\pgfpathmoveto{\pgfqpoint{-0.000000in}{0.000000in}}%
\pgfpathlineto{\pgfqpoint{-0.048611in}{0.000000in}}%
\pgfusepath{stroke,fill}%
}%
\begin{pgfscope}%
\pgfsys@transformshift{0.452199in}{0.567844in}%
\pgfsys@useobject{currentmarker}{}%
\end{pgfscope}%
\end{pgfscope}%
\begin{pgfscope}%
\definecolor{textcolor}{rgb}{0.000000,0.000000,0.000000}%
\pgfsetstrokecolor{textcolor}%
\pgfsetfillcolor{textcolor}%
\pgftext[x=0.290741in, y=0.524441in, left, base]{\color{textcolor}{\rmfamily\fontsize{9.000000}{10.800000}\selectfont\catcode`\^=\active\def^{\ifmmode\sp\else\^{}\fi}\catcode`\%=\active\def%{\%}$\mathdefault{1}$}}%
\end{pgfscope}%
\begin{pgfscope}%
\pgfsetbuttcap%
\pgfsetroundjoin%
\definecolor{currentfill}{rgb}{0.000000,0.000000,0.000000}%
\pgfsetfillcolor{currentfill}%
\pgfsetlinewidth{0.803000pt}%
\definecolor{currentstroke}{rgb}{0.000000,0.000000,0.000000}%
\pgfsetstrokecolor{currentstroke}%
\pgfsetdash{}{0pt}%
\pgfsys@defobject{currentmarker}{\pgfqpoint{-0.048611in}{0.000000in}}{\pgfqpoint{-0.000000in}{0.000000in}}{%
\pgfpathmoveto{\pgfqpoint{-0.000000in}{0.000000in}}%
\pgfpathlineto{\pgfqpoint{-0.048611in}{0.000000in}}%
\pgfusepath{stroke,fill}%
}%
\begin{pgfscope}%
\pgfsys@transformshift{0.452199in}{0.913655in}%
\pgfsys@useobject{currentmarker}{}%
\end{pgfscope}%
\end{pgfscope}%
\begin{pgfscope}%
\definecolor{textcolor}{rgb}{0.000000,0.000000,0.000000}%
\pgfsetstrokecolor{textcolor}%
\pgfsetfillcolor{textcolor}%
\pgftext[x=0.290741in, y=0.870252in, left, base]{\color{textcolor}{\rmfamily\fontsize{9.000000}{10.800000}\selectfont\catcode`\^=\active\def^{\ifmmode\sp\else\^{}\fi}\catcode`\%=\active\def%{\%}$\mathdefault{2}$}}%
\end{pgfscope}%
\begin{pgfscope}%
\pgfsetbuttcap%
\pgfsetroundjoin%
\definecolor{currentfill}{rgb}{0.000000,0.000000,0.000000}%
\pgfsetfillcolor{currentfill}%
\pgfsetlinewidth{0.803000pt}%
\definecolor{currentstroke}{rgb}{0.000000,0.000000,0.000000}%
\pgfsetstrokecolor{currentstroke}%
\pgfsetdash{}{0pt}%
\pgfsys@defobject{currentmarker}{\pgfqpoint{-0.048611in}{0.000000in}}{\pgfqpoint{-0.000000in}{0.000000in}}{%
\pgfpathmoveto{\pgfqpoint{-0.000000in}{0.000000in}}%
\pgfpathlineto{\pgfqpoint{-0.048611in}{0.000000in}}%
\pgfusepath{stroke,fill}%
}%
\begin{pgfscope}%
\pgfsys@transformshift{0.452199in}{1.259466in}%
\pgfsys@useobject{currentmarker}{}%
\end{pgfscope}%
\end{pgfscope}%
\begin{pgfscope}%
\definecolor{textcolor}{rgb}{0.000000,0.000000,0.000000}%
\pgfsetstrokecolor{textcolor}%
\pgfsetfillcolor{textcolor}%
\pgftext[x=0.290741in, y=1.216063in, left, base]{\color{textcolor}{\rmfamily\fontsize{9.000000}{10.800000}\selectfont\catcode`\^=\active\def^{\ifmmode\sp\else\^{}\fi}\catcode`\%=\active\def%{\%}$\mathdefault{3}$}}%
\end{pgfscope}%
\begin{pgfscope}%
\pgfsetbuttcap%
\pgfsetroundjoin%
\definecolor{currentfill}{rgb}{0.000000,0.000000,0.000000}%
\pgfsetfillcolor{currentfill}%
\pgfsetlinewidth{0.803000pt}%
\definecolor{currentstroke}{rgb}{0.000000,0.000000,0.000000}%
\pgfsetstrokecolor{currentstroke}%
\pgfsetdash{}{0pt}%
\pgfsys@defobject{currentmarker}{\pgfqpoint{-0.048611in}{0.000000in}}{\pgfqpoint{-0.000000in}{0.000000in}}{%
\pgfpathmoveto{\pgfqpoint{-0.000000in}{0.000000in}}%
\pgfpathlineto{\pgfqpoint{-0.048611in}{0.000000in}}%
\pgfusepath{stroke,fill}%
}%
\begin{pgfscope}%
\pgfsys@transformshift{0.452199in}{1.605277in}%
\pgfsys@useobject{currentmarker}{}%
\end{pgfscope}%
\end{pgfscope}%
\begin{pgfscope}%
\definecolor{textcolor}{rgb}{0.000000,0.000000,0.000000}%
\pgfsetstrokecolor{textcolor}%
\pgfsetfillcolor{textcolor}%
\pgftext[x=0.290741in, y=1.561874in, left, base]{\color{textcolor}{\rmfamily\fontsize{9.000000}{10.800000}\selectfont\catcode`\^=\active\def^{\ifmmode\sp\else\^{}\fi}\catcode`\%=\active\def%{\%}$\mathdefault{4}$}}%
\end{pgfscope}%
\begin{pgfscope}%
\definecolor{textcolor}{rgb}{0.000000,0.000000,0.000000}%
\pgfsetstrokecolor{textcolor}%
\pgfsetfillcolor{textcolor}%
\pgftext[x=0.235185in,y=1.086561in,,bottom,rotate=90.000000]{\color{textcolor}{\rmfamily\fontsize{11.000000}{13.200000}\selectfont\catcode`\^=\active\def^{\ifmmode\sp\else\^{}\fi}\catcode`\%=\active\def%{\%}Velocity}}%
\end{pgfscope}%
\begin{pgfscope}%
\pgfpathrectangle{\pgfqpoint{0.452199in}{0.515972in}}{\pgfqpoint{9.303131in}{1.141177in}}%
\pgfusepath{clip}%
\pgfsetrectcap%
\pgfsetroundjoin%
\pgfsetlinewidth{1.505625pt}%
\definecolor{currentstroke}{rgb}{0.000000,0.000000,1.000000}%
\pgfsetstrokecolor{currentstroke}%
\pgfsetdash{}{0pt}%
\pgfpathmoveto{\pgfqpoint{0.875068in}{0.567844in}}%
\pgfpathlineto{\pgfqpoint{0.945546in}{0.590898in}}%
\pgfpathlineto{\pgfqpoint{1.016025in}{0.613952in}}%
\pgfpathlineto{\pgfqpoint{1.086503in}{0.637006in}}%
\pgfpathlineto{\pgfqpoint{1.156981in}{0.660060in}}%
\pgfpathlineto{\pgfqpoint{1.227460in}{0.683114in}}%
\pgfpathlineto{\pgfqpoint{1.297938in}{0.706168in}}%
\pgfpathlineto{\pgfqpoint{1.368416in}{0.729222in}}%
\pgfpathlineto{\pgfqpoint{1.438894in}{0.752276in}}%
\pgfpathlineto{\pgfqpoint{1.509373in}{0.775330in}}%
\pgfpathlineto{\pgfqpoint{1.579851in}{0.798385in}}%
\pgfpathlineto{\pgfqpoint{1.650329in}{0.821439in}}%
\pgfpathlineto{\pgfqpoint{1.720807in}{0.844493in}}%
\pgfpathlineto{\pgfqpoint{1.791286in}{0.867547in}}%
\pgfpathlineto{\pgfqpoint{1.861764in}{0.890601in}}%
\pgfpathlineto{\pgfqpoint{1.932242in}{0.913655in}}%
\pgfpathlineto{\pgfqpoint{2.002720in}{0.936709in}}%
\pgfpathlineto{\pgfqpoint{2.073199in}{0.959763in}}%
\pgfpathlineto{\pgfqpoint{2.143677in}{0.982817in}}%
\pgfpathlineto{\pgfqpoint{2.214155in}{1.005871in}}%
\pgfpathlineto{\pgfqpoint{2.284633in}{1.028925in}}%
\pgfpathlineto{\pgfqpoint{2.355112in}{1.051979in}}%
\pgfpathlineto{\pgfqpoint{2.425590in}{1.075033in}}%
\pgfpathlineto{\pgfqpoint{2.496068in}{1.098088in}}%
\pgfpathlineto{\pgfqpoint{2.566547in}{1.121142in}}%
\pgfpathlineto{\pgfqpoint{2.637025in}{1.144196in}}%
\pgfpathlineto{\pgfqpoint{2.707503in}{1.167250in}}%
\pgfpathlineto{\pgfqpoint{2.777981in}{1.190304in}}%
\pgfpathlineto{\pgfqpoint{2.848460in}{1.213358in}}%
\pgfpathlineto{\pgfqpoint{2.918938in}{1.236412in}}%
\pgfpathlineto{\pgfqpoint{2.989416in}{1.259466in}}%
\pgfpathlineto{\pgfqpoint{3.059894in}{1.282520in}}%
\pgfpathlineto{\pgfqpoint{3.130373in}{1.305574in}}%
\pgfpathlineto{\pgfqpoint{3.200851in}{1.328628in}}%
\pgfpathlineto{\pgfqpoint{3.271329in}{1.351682in}}%
\pgfpathlineto{\pgfqpoint{3.341807in}{1.374736in}}%
\pgfpathlineto{\pgfqpoint{3.412286in}{1.397791in}}%
\pgfpathlineto{\pgfqpoint{3.482764in}{1.420845in}}%
\pgfpathlineto{\pgfqpoint{3.553242in}{1.443899in}}%
\pgfpathlineto{\pgfqpoint{3.623721in}{1.466953in}}%
\pgfpathlineto{\pgfqpoint{3.694199in}{1.490007in}}%
\pgfpathlineto{\pgfqpoint{3.764677in}{1.513061in}}%
\pgfpathlineto{\pgfqpoint{3.835155in}{1.536115in}}%
\pgfpathlineto{\pgfqpoint{3.905634in}{1.559169in}}%
\pgfpathlineto{\pgfqpoint{3.976112in}{1.582223in}}%
\pgfpathlineto{\pgfqpoint{4.046590in}{1.605277in}}%
\pgfpathlineto{\pgfqpoint{4.117068in}{1.605277in}}%
\pgfpathlineto{\pgfqpoint{4.187547in}{1.605277in}}%
\pgfpathlineto{\pgfqpoint{4.258025in}{1.605277in}}%
\pgfpathlineto{\pgfqpoint{4.328503in}{1.605277in}}%
\pgfpathlineto{\pgfqpoint{4.398981in}{1.605277in}}%
\pgfpathlineto{\pgfqpoint{4.469460in}{1.605277in}}%
\pgfpathlineto{\pgfqpoint{4.539938in}{1.605277in}}%
\pgfpathlineto{\pgfqpoint{4.610416in}{1.605277in}}%
\pgfpathlineto{\pgfqpoint{4.680894in}{1.605277in}}%
\pgfpathlineto{\pgfqpoint{4.751373in}{1.605277in}}%
\pgfpathlineto{\pgfqpoint{4.821851in}{1.605277in}}%
\pgfpathlineto{\pgfqpoint{4.892329in}{1.605277in}}%
\pgfpathlineto{\pgfqpoint{4.962808in}{1.605277in}}%
\pgfpathlineto{\pgfqpoint{5.033286in}{1.605277in}}%
\pgfpathlineto{\pgfqpoint{5.103764in}{1.605277in}}%
\pgfpathlineto{\pgfqpoint{5.174242in}{1.605277in}}%
\pgfpathlineto{\pgfqpoint{5.244721in}{1.605277in}}%
\pgfpathlineto{\pgfqpoint{5.315199in}{1.605277in}}%
\pgfpathlineto{\pgfqpoint{5.385677in}{1.605277in}}%
\pgfpathlineto{\pgfqpoint{5.456155in}{1.605277in}}%
\pgfpathlineto{\pgfqpoint{5.526634in}{1.605277in}}%
\pgfpathlineto{\pgfqpoint{5.597112in}{1.605277in}}%
\pgfpathlineto{\pgfqpoint{5.667590in}{1.605277in}}%
\pgfpathlineto{\pgfqpoint{5.738068in}{1.605277in}}%
\pgfpathlineto{\pgfqpoint{5.808547in}{1.605277in}}%
\pgfpathlineto{\pgfqpoint{5.879025in}{1.605277in}}%
\pgfpathlineto{\pgfqpoint{5.949503in}{1.605277in}}%
\pgfpathlineto{\pgfqpoint{6.019981in}{1.605277in}}%
\pgfpathlineto{\pgfqpoint{6.090460in}{1.605277in}}%
\pgfpathlineto{\pgfqpoint{6.160938in}{1.605277in}}%
\pgfpathlineto{\pgfqpoint{6.231416in}{1.605277in}}%
\pgfpathlineto{\pgfqpoint{6.301895in}{1.605277in}}%
\pgfpathlineto{\pgfqpoint{6.372373in}{1.605277in}}%
\pgfpathlineto{\pgfqpoint{6.442851in}{1.605277in}}%
\pgfpathlineto{\pgfqpoint{6.513329in}{1.605277in}}%
\pgfpathlineto{\pgfqpoint{6.583808in}{1.605277in}}%
\pgfpathlineto{\pgfqpoint{6.654286in}{1.605277in}}%
\pgfpathlineto{\pgfqpoint{6.724764in}{1.605277in}}%
\pgfpathlineto{\pgfqpoint{6.795242in}{1.605277in}}%
\pgfpathlineto{\pgfqpoint{6.865721in}{1.605277in}}%
\pgfpathlineto{\pgfqpoint{6.936199in}{1.605277in}}%
\pgfpathlineto{\pgfqpoint{7.006677in}{1.605277in}}%
\pgfpathlineto{\pgfqpoint{7.077155in}{1.605277in}}%
\pgfpathlineto{\pgfqpoint{7.147634in}{1.605277in}}%
\pgfpathlineto{\pgfqpoint{7.218112in}{1.605277in}}%
\pgfpathlineto{\pgfqpoint{7.288590in}{1.605277in}}%
\pgfpathlineto{\pgfqpoint{7.359069in}{1.605277in}}%
\pgfpathlineto{\pgfqpoint{7.429547in}{1.605277in}}%
\pgfpathlineto{\pgfqpoint{7.500025in}{1.605277in}}%
\pgfpathlineto{\pgfqpoint{7.570503in}{1.605277in}}%
\pgfpathlineto{\pgfqpoint{7.640982in}{1.605277in}}%
\pgfpathlineto{\pgfqpoint{7.711460in}{1.605277in}}%
\pgfpathlineto{\pgfqpoint{7.781938in}{1.605277in}}%
\pgfpathlineto{\pgfqpoint{7.852416in}{1.605277in}}%
\pgfpathlineto{\pgfqpoint{7.922895in}{1.605277in}}%
\pgfpathlineto{\pgfqpoint{7.993373in}{1.605277in}}%
\pgfpathlineto{\pgfqpoint{8.063851in}{1.605277in}}%
\pgfpathlineto{\pgfqpoint{8.134329in}{1.605277in}}%
\pgfpathlineto{\pgfqpoint{8.204808in}{1.605277in}}%
\pgfpathlineto{\pgfqpoint{8.275286in}{1.605277in}}%
\pgfpathlineto{\pgfqpoint{8.345764in}{1.605277in}}%
\pgfpathlineto{\pgfqpoint{8.416242in}{1.605277in}}%
\pgfpathlineto{\pgfqpoint{8.486721in}{1.605277in}}%
\pgfpathlineto{\pgfqpoint{8.557199in}{1.605277in}}%
\pgfpathlineto{\pgfqpoint{8.627677in}{1.605277in}}%
\pgfpathlineto{\pgfqpoint{8.698156in}{1.605277in}}%
\pgfpathlineto{\pgfqpoint{8.768634in}{1.605277in}}%
\pgfpathlineto{\pgfqpoint{8.839112in}{1.605277in}}%
\pgfpathlineto{\pgfqpoint{8.909590in}{1.605277in}}%
\pgfpathlineto{\pgfqpoint{8.980069in}{1.605277in}}%
\pgfpathlineto{\pgfqpoint{9.050547in}{1.605277in}}%
\pgfpathlineto{\pgfqpoint{9.121025in}{1.605277in}}%
\pgfpathlineto{\pgfqpoint{9.191503in}{1.605277in}}%
\pgfpathlineto{\pgfqpoint{9.261982in}{1.605277in}}%
\pgfpathlineto{\pgfqpoint{9.332460in}{1.605277in}}%
\pgfusepath{stroke}%
\end{pgfscope}%
\begin{pgfscope}%
\pgfpathrectangle{\pgfqpoint{0.452199in}{0.515972in}}{\pgfqpoint{9.303131in}{1.141177in}}%
\pgfusepath{clip}%
\pgfsetrectcap%
\pgfsetroundjoin%
\pgfsetlinewidth{1.505625pt}%
\definecolor{currentstroke}{rgb}{1.000000,0.000000,0.000000}%
\pgfsetstrokecolor{currentstroke}%
\pgfsetdash{}{0pt}%
\pgfusepath{stroke}%
\end{pgfscope}%
\begin{pgfscope}%
\pgfpathrectangle{\pgfqpoint{0.452199in}{0.515972in}}{\pgfqpoint{9.303131in}{1.141177in}}%
\pgfusepath{clip}%
\pgfsetrectcap%
\pgfsetroundjoin%
\pgfsetlinewidth{1.505625pt}%
\definecolor{currentstroke}{rgb}{0.000000,0.501961,0.000000}%
\pgfsetstrokecolor{currentstroke}%
\pgfsetdash{}{0pt}%
\pgfusepath{stroke}%
\end{pgfscope}%
\begin{pgfscope}%
\pgfsetrectcap%
\pgfsetmiterjoin%
\pgfsetlinewidth{0.803000pt}%
\definecolor{currentstroke}{rgb}{0.000000,0.000000,0.000000}%
\pgfsetstrokecolor{currentstroke}%
\pgfsetdash{}{0pt}%
\pgfpathmoveto{\pgfqpoint{0.452199in}{0.515972in}}%
\pgfpathlineto{\pgfqpoint{0.452199in}{1.657149in}}%
\pgfusepath{stroke}%
\end{pgfscope}%
\begin{pgfscope}%
\pgfsetrectcap%
\pgfsetmiterjoin%
\pgfsetlinewidth{0.803000pt}%
\definecolor{currentstroke}{rgb}{0.000000,0.000000,0.000000}%
\pgfsetstrokecolor{currentstroke}%
\pgfsetdash{}{0pt}%
\pgfpathmoveto{\pgfqpoint{9.755329in}{0.515972in}}%
\pgfpathlineto{\pgfqpoint{9.755329in}{1.657149in}}%
\pgfusepath{stroke}%
\end{pgfscope}%
\begin{pgfscope}%
\pgfsetrectcap%
\pgfsetmiterjoin%
\pgfsetlinewidth{0.803000pt}%
\definecolor{currentstroke}{rgb}{0.000000,0.000000,0.000000}%
\pgfsetstrokecolor{currentstroke}%
\pgfsetdash{}{0pt}%
\pgfpathmoveto{\pgfqpoint{0.452199in}{0.515972in}}%
\pgfpathlineto{\pgfqpoint{9.755329in}{0.515972in}}%
\pgfusepath{stroke}%
\end{pgfscope}%
\begin{pgfscope}%
\pgfsetrectcap%
\pgfsetmiterjoin%
\pgfsetlinewidth{0.803000pt}%
\definecolor{currentstroke}{rgb}{0.000000,0.000000,0.000000}%
\pgfsetstrokecolor{currentstroke}%
\pgfsetdash{}{0pt}%
\pgfpathmoveto{\pgfqpoint{0.452199in}{1.657149in}}%
\pgfpathlineto{\pgfqpoint{9.755329in}{1.657149in}}%
\pgfusepath{stroke}%
\end{pgfscope}%
\begin{pgfscope}%
\pgfsetbuttcap%
\pgfsetmiterjoin%
\definecolor{currentfill}{rgb}{1.000000,1.000000,1.000000}%
\pgfsetfillcolor{currentfill}%
\pgfsetfillopacity{0.800000}%
\pgfsetlinewidth{1.003750pt}%
\definecolor{currentstroke}{rgb}{0.800000,0.800000,0.800000}%
\pgfsetstrokecolor{currentstroke}%
\pgfsetstrokeopacity{0.800000}%
\pgfsetdash{}{0pt}%
\pgfpathmoveto{\pgfqpoint{8.040821in}{0.921038in}}%
\pgfpathlineto{\pgfqpoint{9.658107in}{0.921038in}}%
\pgfpathquadraticcurveto{\pgfqpoint{9.685885in}{0.921038in}}{\pgfqpoint{9.685885in}{0.948816in}}%
\pgfpathlineto{\pgfqpoint{9.685885in}{1.559927in}}%
\pgfpathquadraticcurveto{\pgfqpoint{9.685885in}{1.587704in}}{\pgfqpoint{9.658107in}{1.587704in}}%
\pgfpathlineto{\pgfqpoint{8.040821in}{1.587704in}}%
\pgfpathquadraticcurveto{\pgfqpoint{8.013043in}{1.587704in}}{\pgfqpoint{8.013043in}{1.559927in}}%
\pgfpathlineto{\pgfqpoint{8.013043in}{0.948816in}}%
\pgfpathquadraticcurveto{\pgfqpoint{8.013043in}{0.921038in}}{\pgfqpoint{8.040821in}{0.921038in}}%
\pgfpathlineto{\pgfqpoint{8.040821in}{0.921038in}}%
\pgfpathclose%
\pgfusepath{stroke,fill}%
\end{pgfscope}%
\begin{pgfscope}%
\pgfsetrectcap%
\pgfsetroundjoin%
\pgfsetlinewidth{1.505625pt}%
\definecolor{currentstroke}{rgb}{0.000000,0.000000,1.000000}%
\pgfsetstrokecolor{currentstroke}%
\pgfsetdash{}{0pt}%
\pgfpathmoveto{\pgfqpoint{8.068599in}{1.476593in}}%
\pgfpathlineto{\pgfqpoint{8.207488in}{1.476593in}}%
\pgfpathlineto{\pgfqpoint{8.346377in}{1.476593in}}%
\pgfusepath{stroke}%
\end{pgfscope}%
\begin{pgfscope}%
\definecolor{textcolor}{rgb}{0.000000,0.000000,0.000000}%
\pgfsetstrokecolor{textcolor}%
\pgfsetfillcolor{textcolor}%
\pgftext[x=8.457488in,y=1.427982in,left,base]{\color{textcolor}{\rmfamily\fontsize{10.000000}{12.000000}\selectfont\catcode`\^=\active\def^{\ifmmode\sp\else\^{}\fi}\catcode`\%=\active\def%{\%}Velocity (Positive)}}%
\end{pgfscope}%
\begin{pgfscope}%
\pgfsetrectcap%
\pgfsetroundjoin%
\pgfsetlinewidth{1.505625pt}%
\definecolor{currentstroke}{rgb}{1.000000,0.000000,0.000000}%
\pgfsetstrokecolor{currentstroke}%
\pgfsetdash{}{0pt}%
\pgfpathmoveto{\pgfqpoint{8.068599in}{1.268260in}}%
\pgfpathlineto{\pgfqpoint{8.207488in}{1.268260in}}%
\pgfpathlineto{\pgfqpoint{8.346377in}{1.268260in}}%
\pgfusepath{stroke}%
\end{pgfscope}%
\begin{pgfscope}%
\definecolor{textcolor}{rgb}{0.000000,0.000000,0.000000}%
\pgfsetstrokecolor{textcolor}%
\pgfsetfillcolor{textcolor}%
\pgftext[x=8.457488in,y=1.219649in,left,base]{\color{textcolor}{\rmfamily\fontsize{10.000000}{12.000000}\selectfont\catcode`\^=\active\def^{\ifmmode\sp\else\^{}\fi}\catcode`\%=\active\def%{\%}Velocity (Negative)}}%
\end{pgfscope}%
\begin{pgfscope}%
\pgfsetrectcap%
\pgfsetroundjoin%
\pgfsetlinewidth{1.505625pt}%
\definecolor{currentstroke}{rgb}{0.000000,0.501961,0.000000}%
\pgfsetstrokecolor{currentstroke}%
\pgfsetdash{}{0pt}%
\pgfpathmoveto{\pgfqpoint{8.068599in}{1.059927in}}%
\pgfpathlineto{\pgfqpoint{8.207488in}{1.059927in}}%
\pgfpathlineto{\pgfqpoint{8.346377in}{1.059927in}}%
\pgfusepath{stroke}%
\end{pgfscope}%
\begin{pgfscope}%
\definecolor{textcolor}{rgb}{0.000000,0.000000,0.000000}%
\pgfsetstrokecolor{textcolor}%
\pgfsetfillcolor{textcolor}%
\pgftext[x=8.457488in,y=1.011316in,left,base]{\color{textcolor}{\rmfamily\fontsize{10.000000}{12.000000}\selectfont\catcode`\^=\active\def^{\ifmmode\sp\else\^{}\fi}\catcode`\%=\active\def%{\%}Velocity (Zero)}}%
\end{pgfscope}%
\end{pgfpicture}%
\makeatother%
\endgroup%
}
	\caption{Planned velocity profile.}
	\label{fig:velocity}
\end{figure}

Figure \ref{fig:velocity} shows the planned velocity profile, which quickly reaches its upper limit.
The alignment error $\xi$ is close to zero.
Here, \( \xi \) is bounded within \(-45^{\circ} \leq \xi \leq 45^{\circ} \).
It is noteworthy that \( \xi \) does not reach these bounds.

\begin{figure}[h]
	\centering
	\resizebox{0.6\textwidth}{!}{%% Creator: Matplotlib, PGF backend
%%
%% To include the figure in your LaTeX document, write
%%   \input{<filename>.pgf}
%%
%% Make sure the required packages are loaded in your preamble
%%   \usepackage{pgf}
%%
%% Also ensure that all the required font packages are loaded; for instance,
%% the lmodern package is sometimes necessary when using math font.
%%   \usepackage{lmodern}
%%
%% Figures using additional raster images can only be included by \input if
%% they are in the same directory as the main LaTeX file. For loading figures
%% from other directories you can use the `import` package
%%   \usepackage{import}
%%
%% and then include the figures with
%%   \import{<path to file>}{<filename>.pgf}
%%
%% Matplotlib used the following preamble
%%   \def\mathdefault#1{#1}
%%   \everymath=\expandafter{\the\everymath\displaystyle}
%%   
%%   \ifdefined\pdftexversion\else  % non-pdftex case.
%%     \usepackage{fontspec}
%%   \fi
%%   \makeatletter\@ifpackageloaded{underscore}{}{\usepackage[strings]{underscore}}\makeatother
%%
\begingroup%
\makeatletter%
\begin{pgfpicture}%
\pgfpathrectangle{\pgfpointorigin}{\pgfqpoint{5.712043in}{4.295074in}}%
\pgfusepath{use as bounding box, clip}%
\begin{pgfscope}%
\pgfsetbuttcap%
\pgfsetmiterjoin%
\definecolor{currentfill}{rgb}{1.000000,1.000000,1.000000}%
\pgfsetfillcolor{currentfill}%
\pgfsetlinewidth{0.000000pt}%
\definecolor{currentstroke}{rgb}{1.000000,1.000000,1.000000}%
\pgfsetstrokecolor{currentstroke}%
\pgfsetdash{}{0pt}%
\pgfpathmoveto{\pgfqpoint{0.000000in}{0.000000in}}%
\pgfpathlineto{\pgfqpoint{5.712043in}{0.000000in}}%
\pgfpathlineto{\pgfqpoint{5.712043in}{4.295074in}}%
\pgfpathlineto{\pgfqpoint{0.000000in}{4.295074in}}%
\pgfpathlineto{\pgfqpoint{0.000000in}{0.000000in}}%
\pgfpathclose%
\pgfusepath{fill}%
\end{pgfscope}%
\begin{pgfscope}%
\pgfsetbuttcap%
\pgfsetmiterjoin%
\definecolor{currentfill}{rgb}{1.000000,1.000000,1.000000}%
\pgfsetfillcolor{currentfill}%
\pgfsetlinewidth{0.000000pt}%
\definecolor{currentstroke}{rgb}{0.000000,0.000000,0.000000}%
\pgfsetstrokecolor{currentstroke}%
\pgfsetstrokeopacity{0.000000}%
\pgfsetdash{}{0pt}%
\pgfpathmoveto{\pgfqpoint{0.652043in}{0.499074in}}%
\pgfpathlineto{\pgfqpoint{5.612043in}{0.499074in}}%
\pgfpathlineto{\pgfqpoint{5.612043in}{4.195074in}}%
\pgfpathlineto{\pgfqpoint{0.652043in}{4.195074in}}%
\pgfpathlineto{\pgfqpoint{0.652043in}{0.499074in}}%
\pgfpathclose%
\pgfusepath{fill}%
\end{pgfscope}%
\begin{pgfscope}%
\pgfpathrectangle{\pgfqpoint{0.652043in}{0.499074in}}{\pgfqpoint{4.960000in}{3.696000in}}%
\pgfusepath{clip}%
\pgfsetbuttcap%
\pgfsetroundjoin%
\definecolor{currentfill}{rgb}{0.000000,0.501961,0.000000}%
\pgfsetfillcolor{currentfill}%
\pgfsetfillopacity{0.200000}%
\pgfsetlinewidth{1.003750pt}%
\definecolor{currentstroke}{rgb}{0.000000,0.501961,0.000000}%
\pgfsetstrokecolor{currentstroke}%
\pgfsetstrokeopacity{0.200000}%
\pgfsetdash{}{0pt}%
\pgfsys@defobject{currentmarker}{\pgfqpoint{0.877497in}{0.667074in}}{\pgfqpoint{5.386588in}{4.027074in}}{%
\pgfpathmoveto{\pgfqpoint{0.877497in}{2.348687in}}%
\pgfpathlineto{\pgfqpoint{0.877497in}{2.348687in}}%
\pgfpathlineto{\pgfqpoint{0.896285in}{2.306660in}}%
\pgfpathlineto{\pgfqpoint{0.915073in}{2.264630in}}%
\pgfpathlineto{\pgfqpoint{0.933861in}{2.222597in}}%
\pgfpathlineto{\pgfqpoint{0.952649in}{2.180565in}}%
\pgfpathlineto{\pgfqpoint{0.971437in}{2.138534in}}%
\pgfpathlineto{\pgfqpoint{0.990225in}{2.096505in}}%
\pgfpathlineto{\pgfqpoint{1.009013in}{2.054478in}}%
\pgfpathlineto{\pgfqpoint{1.027800in}{2.012452in}}%
\pgfpathlineto{\pgfqpoint{1.046588in}{1.970428in}}%
\pgfpathlineto{\pgfqpoint{1.065376in}{1.928405in}}%
\pgfpathlineto{\pgfqpoint{1.084164in}{1.886383in}}%
\pgfpathlineto{\pgfqpoint{1.102952in}{1.844360in}}%
\pgfpathlineto{\pgfqpoint{1.121740in}{1.802337in}}%
\pgfpathlineto{\pgfqpoint{1.140528in}{1.760314in}}%
\pgfpathlineto{\pgfqpoint{1.159316in}{1.718289in}}%
\pgfpathlineto{\pgfqpoint{1.178104in}{1.676263in}}%
\pgfpathlineto{\pgfqpoint{1.196891in}{1.634235in}}%
\pgfpathlineto{\pgfqpoint{1.215679in}{1.592204in}}%
\pgfpathlineto{\pgfqpoint{1.234467in}{1.550172in}}%
\pgfpathlineto{\pgfqpoint{1.253255in}{1.508136in}}%
\pgfpathlineto{\pgfqpoint{1.272043in}{1.466099in}}%
\pgfpathlineto{\pgfqpoint{1.290831in}{1.424059in}}%
\pgfpathlineto{\pgfqpoint{1.309619in}{1.382018in}}%
\pgfpathlineto{\pgfqpoint{1.328407in}{1.339974in}}%
\pgfpathlineto{\pgfqpoint{1.347194in}{1.297928in}}%
\pgfpathlineto{\pgfqpoint{1.365982in}{1.255880in}}%
\pgfpathlineto{\pgfqpoint{1.384770in}{1.213830in}}%
\pgfpathlineto{\pgfqpoint{1.403558in}{1.171778in}}%
\pgfpathlineto{\pgfqpoint{1.422346in}{1.129725in}}%
\pgfpathlineto{\pgfqpoint{1.441134in}{1.087670in}}%
\pgfpathlineto{\pgfqpoint{1.459922in}{1.045614in}}%
\pgfpathlineto{\pgfqpoint{1.478710in}{1.003557in}}%
\pgfpathlineto{\pgfqpoint{1.497497in}{0.961498in}}%
\pgfpathlineto{\pgfqpoint{1.516285in}{0.919439in}}%
\pgfpathlineto{\pgfqpoint{1.535073in}{0.877379in}}%
\pgfpathlineto{\pgfqpoint{1.553861in}{0.835318in}}%
\pgfpathlineto{\pgfqpoint{1.572649in}{0.793257in}}%
\pgfpathlineto{\pgfqpoint{1.591437in}{0.751196in}}%
\pgfpathlineto{\pgfqpoint{1.610225in}{0.709135in}}%
\pgfpathlineto{\pgfqpoint{1.629013in}{0.667074in}}%
\pgfpathlineto{\pgfqpoint{1.647800in}{0.706780in}}%
\pgfpathlineto{\pgfqpoint{1.666588in}{0.748640in}}%
\pgfpathlineto{\pgfqpoint{1.685376in}{0.790502in}}%
\pgfpathlineto{\pgfqpoint{1.704164in}{0.832368in}}%
\pgfpathlineto{\pgfqpoint{1.722952in}{0.874239in}}%
\pgfpathlineto{\pgfqpoint{1.741740in}{0.916114in}}%
\pgfpathlineto{\pgfqpoint{1.760528in}{0.957995in}}%
\pgfpathlineto{\pgfqpoint{1.779316in}{0.999881in}}%
\pgfpathlineto{\pgfqpoint{1.798104in}{1.041774in}}%
\pgfpathlineto{\pgfqpoint{1.816891in}{1.083673in}}%
\pgfpathlineto{\pgfqpoint{1.835679in}{1.125579in}}%
\pgfpathlineto{\pgfqpoint{1.854467in}{1.167492in}}%
\pgfpathlineto{\pgfqpoint{1.873255in}{1.209413in}}%
\pgfpathlineto{\pgfqpoint{1.892043in}{1.251340in}}%
\pgfpathlineto{\pgfqpoint{1.910831in}{1.293275in}}%
\pgfpathlineto{\pgfqpoint{1.929619in}{1.335218in}}%
\pgfpathlineto{\pgfqpoint{1.948407in}{1.377167in}}%
\pgfpathlineto{\pgfqpoint{1.967194in}{1.419124in}}%
\pgfpathlineto{\pgfqpoint{1.985982in}{1.461088in}}%
\pgfpathlineto{\pgfqpoint{2.004770in}{1.503059in}}%
\pgfpathlineto{\pgfqpoint{2.023558in}{1.545037in}}%
\pgfpathlineto{\pgfqpoint{2.042346in}{1.587021in}}%
\pgfpathlineto{\pgfqpoint{2.061134in}{1.629012in}}%
\pgfpathlineto{\pgfqpoint{2.079922in}{1.671008in}}%
\pgfpathlineto{\pgfqpoint{2.098710in}{1.713011in}}%
\pgfpathlineto{\pgfqpoint{2.117497in}{1.755020in}}%
\pgfpathlineto{\pgfqpoint{2.136285in}{1.797034in}}%
\pgfpathlineto{\pgfqpoint{2.155073in}{1.839054in}}%
\pgfpathlineto{\pgfqpoint{2.173861in}{1.881078in}}%
\pgfpathlineto{\pgfqpoint{2.192649in}{1.923108in}}%
\pgfpathlineto{\pgfqpoint{2.211437in}{1.965143in}}%
\pgfpathlineto{\pgfqpoint{2.230225in}{2.007182in}}%
\pgfpathlineto{\pgfqpoint{2.249013in}{2.049226in}}%
\pgfpathlineto{\pgfqpoint{2.267800in}{2.091274in}}%
\pgfpathlineto{\pgfqpoint{2.286588in}{2.133327in}}%
\pgfpathlineto{\pgfqpoint{2.305376in}{2.175384in}}%
\pgfpathlineto{\pgfqpoint{2.324164in}{2.217446in}}%
\pgfpathlineto{\pgfqpoint{2.342952in}{2.259512in}}%
\pgfpathlineto{\pgfqpoint{2.361740in}{2.301582in}}%
\pgfpathlineto{\pgfqpoint{2.380528in}{2.343656in}}%
\pgfpathlineto{\pgfqpoint{2.399316in}{2.343708in}}%
\pgfpathlineto{\pgfqpoint{2.418104in}{2.343762in}}%
\pgfpathlineto{\pgfqpoint{2.436891in}{2.343819in}}%
\pgfpathlineto{\pgfqpoint{2.455679in}{2.343877in}}%
\pgfpathlineto{\pgfqpoint{2.474467in}{2.343937in}}%
\pgfpathlineto{\pgfqpoint{2.493255in}{2.344000in}}%
\pgfpathlineto{\pgfqpoint{2.512043in}{2.344063in}}%
\pgfpathlineto{\pgfqpoint{2.530831in}{2.344128in}}%
\pgfpathlineto{\pgfqpoint{2.549619in}{2.344194in}}%
\pgfpathlineto{\pgfqpoint{2.568407in}{2.344262in}}%
\pgfpathlineto{\pgfqpoint{2.587194in}{2.344330in}}%
\pgfpathlineto{\pgfqpoint{2.605982in}{2.344399in}}%
\pgfpathlineto{\pgfqpoint{2.624770in}{2.344469in}}%
\pgfpathlineto{\pgfqpoint{2.643558in}{2.344540in}}%
\pgfpathlineto{\pgfqpoint{2.662346in}{2.344611in}}%
\pgfpathlineto{\pgfqpoint{2.681134in}{2.344683in}}%
\pgfpathlineto{\pgfqpoint{2.699922in}{2.344755in}}%
\pgfpathlineto{\pgfqpoint{2.718710in}{2.344827in}}%
\pgfpathlineto{\pgfqpoint{2.737497in}{2.344900in}}%
\pgfpathlineto{\pgfqpoint{2.756285in}{2.344972in}}%
\pgfpathlineto{\pgfqpoint{2.775073in}{2.345045in}}%
\pgfpathlineto{\pgfqpoint{2.793861in}{2.345118in}}%
\pgfpathlineto{\pgfqpoint{2.812649in}{2.345190in}}%
\pgfpathlineto{\pgfqpoint{2.831437in}{2.345262in}}%
\pgfpathlineto{\pgfqpoint{2.850225in}{2.345335in}}%
\pgfpathlineto{\pgfqpoint{2.869013in}{2.345406in}}%
\pgfpathlineto{\pgfqpoint{2.887800in}{2.345478in}}%
\pgfpathlineto{\pgfqpoint{2.906588in}{2.345549in}}%
\pgfpathlineto{\pgfqpoint{2.925376in}{2.345620in}}%
\pgfpathlineto{\pgfqpoint{2.944164in}{2.345690in}}%
\pgfpathlineto{\pgfqpoint{2.962952in}{2.345760in}}%
\pgfpathlineto{\pgfqpoint{2.981740in}{2.345829in}}%
\pgfpathlineto{\pgfqpoint{3.000528in}{2.345897in}}%
\pgfpathlineto{\pgfqpoint{3.019316in}{2.345965in}}%
\pgfpathlineto{\pgfqpoint{3.038104in}{2.346032in}}%
\pgfpathlineto{\pgfqpoint{3.056891in}{2.346099in}}%
\pgfpathlineto{\pgfqpoint{3.075679in}{2.346164in}}%
\pgfpathlineto{\pgfqpoint{3.094467in}{2.346229in}}%
\pgfpathlineto{\pgfqpoint{3.113255in}{2.346293in}}%
\pgfpathlineto{\pgfqpoint{3.132043in}{2.346357in}}%
\pgfpathlineto{\pgfqpoint{3.150831in}{2.346419in}}%
\pgfpathlineto{\pgfqpoint{3.169619in}{2.346481in}}%
\pgfpathlineto{\pgfqpoint{3.188407in}{2.346542in}}%
\pgfpathlineto{\pgfqpoint{3.207194in}{2.346602in}}%
\pgfpathlineto{\pgfqpoint{3.225982in}{2.346661in}}%
\pgfpathlineto{\pgfqpoint{3.244770in}{2.346719in}}%
\pgfpathlineto{\pgfqpoint{3.263558in}{2.346776in}}%
\pgfpathlineto{\pgfqpoint{3.282346in}{2.346832in}}%
\pgfpathlineto{\pgfqpoint{3.301134in}{2.346887in}}%
\pgfpathlineto{\pgfqpoint{3.319922in}{2.346941in}}%
\pgfpathlineto{\pgfqpoint{3.338710in}{2.346994in}}%
\pgfpathlineto{\pgfqpoint{3.357497in}{2.347047in}}%
\pgfpathlineto{\pgfqpoint{3.376285in}{2.347098in}}%
\pgfpathlineto{\pgfqpoint{3.395073in}{2.347148in}}%
\pgfpathlineto{\pgfqpoint{3.413861in}{2.347197in}}%
\pgfpathlineto{\pgfqpoint{3.432649in}{2.347245in}}%
\pgfpathlineto{\pgfqpoint{3.451437in}{2.347293in}}%
\pgfpathlineto{\pgfqpoint{3.470225in}{2.347339in}}%
\pgfpathlineto{\pgfqpoint{3.489013in}{2.347384in}}%
\pgfpathlineto{\pgfqpoint{3.507800in}{2.347428in}}%
\pgfpathlineto{\pgfqpoint{3.526588in}{2.347471in}}%
\pgfpathlineto{\pgfqpoint{3.545376in}{2.347513in}}%
\pgfpathlineto{\pgfqpoint{3.564164in}{2.347554in}}%
\pgfpathlineto{\pgfqpoint{3.582952in}{2.347594in}}%
\pgfpathlineto{\pgfqpoint{3.601740in}{2.347634in}}%
\pgfpathlineto{\pgfqpoint{3.620528in}{2.347672in}}%
\pgfpathlineto{\pgfqpoint{3.639316in}{2.347709in}}%
\pgfpathlineto{\pgfqpoint{3.658104in}{2.347745in}}%
\pgfpathlineto{\pgfqpoint{3.676891in}{2.347780in}}%
\pgfpathlineto{\pgfqpoint{3.695679in}{2.347815in}}%
\pgfpathlineto{\pgfqpoint{3.714467in}{2.347848in}}%
\pgfpathlineto{\pgfqpoint{3.733255in}{2.347881in}}%
\pgfpathlineto{\pgfqpoint{3.752043in}{2.347912in}}%
\pgfpathlineto{\pgfqpoint{3.770831in}{2.347943in}}%
\pgfpathlineto{\pgfqpoint{3.789619in}{2.347972in}}%
\pgfpathlineto{\pgfqpoint{3.808407in}{2.348001in}}%
\pgfpathlineto{\pgfqpoint{3.827194in}{2.348029in}}%
\pgfpathlineto{\pgfqpoint{3.845982in}{2.348056in}}%
\pgfpathlineto{\pgfqpoint{3.864770in}{2.348082in}}%
\pgfpathlineto{\pgfqpoint{3.883558in}{2.348108in}}%
\pgfpathlineto{\pgfqpoint{3.902346in}{2.348132in}}%
\pgfpathlineto{\pgfqpoint{3.921134in}{2.348156in}}%
\pgfpathlineto{\pgfqpoint{3.939922in}{2.348179in}}%
\pgfpathlineto{\pgfqpoint{3.958710in}{2.348201in}}%
\pgfpathlineto{\pgfqpoint{3.977497in}{2.348223in}}%
\pgfpathlineto{\pgfqpoint{3.996285in}{2.348244in}}%
\pgfpathlineto{\pgfqpoint{4.015073in}{2.348264in}}%
\pgfpathlineto{\pgfqpoint{4.033861in}{2.348283in}}%
\pgfpathlineto{\pgfqpoint{4.052649in}{2.348301in}}%
\pgfpathlineto{\pgfqpoint{4.071437in}{2.348319in}}%
\pgfpathlineto{\pgfqpoint{4.090225in}{2.348336in}}%
\pgfpathlineto{\pgfqpoint{4.109013in}{2.348353in}}%
\pgfpathlineto{\pgfqpoint{4.127800in}{2.348369in}}%
\pgfpathlineto{\pgfqpoint{4.146588in}{2.348384in}}%
\pgfpathlineto{\pgfqpoint{4.165376in}{2.348399in}}%
\pgfpathlineto{\pgfqpoint{4.184164in}{2.348413in}}%
\pgfpathlineto{\pgfqpoint{4.202952in}{2.348427in}}%
\pgfpathlineto{\pgfqpoint{4.221740in}{2.348440in}}%
\pgfpathlineto{\pgfqpoint{4.240528in}{2.348452in}}%
\pgfpathlineto{\pgfqpoint{4.259316in}{2.348464in}}%
\pgfpathlineto{\pgfqpoint{4.278104in}{2.348475in}}%
\pgfpathlineto{\pgfqpoint{4.296891in}{2.348486in}}%
\pgfpathlineto{\pgfqpoint{4.315679in}{2.348497in}}%
\pgfpathlineto{\pgfqpoint{4.334467in}{2.348507in}}%
\pgfpathlineto{\pgfqpoint{4.353255in}{2.348516in}}%
\pgfpathlineto{\pgfqpoint{4.372043in}{2.348525in}}%
\pgfpathlineto{\pgfqpoint{4.390831in}{2.348534in}}%
\pgfpathlineto{\pgfqpoint{4.409619in}{2.348542in}}%
\pgfpathlineto{\pgfqpoint{4.428407in}{2.348550in}}%
\pgfpathlineto{\pgfqpoint{4.447194in}{2.348558in}}%
\pgfpathlineto{\pgfqpoint{4.465982in}{2.348565in}}%
\pgfpathlineto{\pgfqpoint{4.484770in}{2.348572in}}%
\pgfpathlineto{\pgfqpoint{4.503558in}{2.348578in}}%
\pgfpathlineto{\pgfqpoint{4.522346in}{2.348584in}}%
\pgfpathlineto{\pgfqpoint{4.541134in}{2.348590in}}%
\pgfpathlineto{\pgfqpoint{4.559922in}{2.348595in}}%
\pgfpathlineto{\pgfqpoint{4.578710in}{2.348601in}}%
\pgfpathlineto{\pgfqpoint{4.597497in}{2.348606in}}%
\pgfpathlineto{\pgfqpoint{4.616285in}{2.348610in}}%
\pgfpathlineto{\pgfqpoint{4.635073in}{2.348615in}}%
\pgfpathlineto{\pgfqpoint{4.653861in}{2.348619in}}%
\pgfpathlineto{\pgfqpoint{4.672649in}{2.348623in}}%
\pgfpathlineto{\pgfqpoint{4.691437in}{2.348627in}}%
\pgfpathlineto{\pgfqpoint{4.710225in}{2.348631in}}%
\pgfpathlineto{\pgfqpoint{4.729013in}{2.348634in}}%
\pgfpathlineto{\pgfqpoint{4.747800in}{2.348637in}}%
\pgfpathlineto{\pgfqpoint{4.766588in}{2.348640in}}%
\pgfpathlineto{\pgfqpoint{4.785376in}{2.348643in}}%
\pgfpathlineto{\pgfqpoint{4.804164in}{2.348646in}}%
\pgfpathlineto{\pgfqpoint{4.822952in}{2.348649in}}%
\pgfpathlineto{\pgfqpoint{4.841740in}{2.348651in}}%
\pgfpathlineto{\pgfqpoint{4.860528in}{2.348653in}}%
\pgfpathlineto{\pgfqpoint{4.879316in}{2.348656in}}%
\pgfpathlineto{\pgfqpoint{4.898104in}{2.348658in}}%
\pgfpathlineto{\pgfqpoint{4.916891in}{2.348660in}}%
\pgfpathlineto{\pgfqpoint{4.935679in}{2.348661in}}%
\pgfpathlineto{\pgfqpoint{4.954467in}{2.348663in}}%
\pgfpathlineto{\pgfqpoint{4.973255in}{2.348665in}}%
\pgfpathlineto{\pgfqpoint{4.992043in}{2.348666in}}%
\pgfpathlineto{\pgfqpoint{5.010831in}{2.348668in}}%
\pgfpathlineto{\pgfqpoint{5.029619in}{2.348669in}}%
\pgfpathlineto{\pgfqpoint{5.048407in}{2.348671in}}%
\pgfpathlineto{\pgfqpoint{5.067194in}{2.348672in}}%
\pgfpathlineto{\pgfqpoint{5.085982in}{2.348673in}}%
\pgfpathlineto{\pgfqpoint{5.104770in}{2.348674in}}%
\pgfpathlineto{\pgfqpoint{5.123558in}{2.348675in}}%
\pgfpathlineto{\pgfqpoint{5.142346in}{2.348676in}}%
\pgfpathlineto{\pgfqpoint{5.161134in}{2.348677in}}%
\pgfpathlineto{\pgfqpoint{5.179922in}{2.348678in}}%
\pgfpathlineto{\pgfqpoint{5.198710in}{2.348678in}}%
\pgfpathlineto{\pgfqpoint{5.217497in}{2.348679in}}%
\pgfpathlineto{\pgfqpoint{5.236285in}{2.348680in}}%
\pgfpathlineto{\pgfqpoint{5.255073in}{2.348680in}}%
\pgfpathlineto{\pgfqpoint{5.273861in}{2.348681in}}%
\pgfpathlineto{\pgfqpoint{5.292649in}{2.348681in}}%
\pgfpathlineto{\pgfqpoint{5.311437in}{2.348681in}}%
\pgfpathlineto{\pgfqpoint{5.330225in}{2.348682in}}%
\pgfpathlineto{\pgfqpoint{5.349013in}{2.348682in}}%
\pgfpathlineto{\pgfqpoint{5.367800in}{2.348682in}}%
\pgfpathlineto{\pgfqpoint{5.386588in}{2.348683in}}%
\pgfpathlineto{\pgfqpoint{5.386588in}{2.348683in}}%
\pgfpathlineto{\pgfqpoint{5.386588in}{2.348683in}}%
\pgfpathlineto{\pgfqpoint{5.367800in}{2.348682in}}%
\pgfpathlineto{\pgfqpoint{5.349013in}{2.348682in}}%
\pgfpathlineto{\pgfqpoint{5.330225in}{2.348682in}}%
\pgfpathlineto{\pgfqpoint{5.311437in}{2.348681in}}%
\pgfpathlineto{\pgfqpoint{5.292649in}{2.348681in}}%
\pgfpathlineto{\pgfqpoint{5.273861in}{2.348681in}}%
\pgfpathlineto{\pgfqpoint{5.255073in}{2.348680in}}%
\pgfpathlineto{\pgfqpoint{5.236285in}{2.348680in}}%
\pgfpathlineto{\pgfqpoint{5.217497in}{2.348679in}}%
\pgfpathlineto{\pgfqpoint{5.198710in}{2.348678in}}%
\pgfpathlineto{\pgfqpoint{5.179922in}{2.348678in}}%
\pgfpathlineto{\pgfqpoint{5.161134in}{2.348677in}}%
\pgfpathlineto{\pgfqpoint{5.142346in}{2.348676in}}%
\pgfpathlineto{\pgfqpoint{5.123558in}{2.348675in}}%
\pgfpathlineto{\pgfqpoint{5.104770in}{2.348674in}}%
\pgfpathlineto{\pgfqpoint{5.085982in}{2.348673in}}%
\pgfpathlineto{\pgfqpoint{5.067194in}{2.348672in}}%
\pgfpathlineto{\pgfqpoint{5.048407in}{2.348671in}}%
\pgfpathlineto{\pgfqpoint{5.029619in}{2.348669in}}%
\pgfpathlineto{\pgfqpoint{5.010831in}{2.348668in}}%
\pgfpathlineto{\pgfqpoint{4.992043in}{2.348666in}}%
\pgfpathlineto{\pgfqpoint{4.973255in}{2.348665in}}%
\pgfpathlineto{\pgfqpoint{4.954467in}{2.348663in}}%
\pgfpathlineto{\pgfqpoint{4.935679in}{2.348661in}}%
\pgfpathlineto{\pgfqpoint{4.916891in}{2.348660in}}%
\pgfpathlineto{\pgfqpoint{4.898104in}{2.348658in}}%
\pgfpathlineto{\pgfqpoint{4.879316in}{2.348656in}}%
\pgfpathlineto{\pgfqpoint{4.860528in}{2.348653in}}%
\pgfpathlineto{\pgfqpoint{4.841740in}{2.348651in}}%
\pgfpathlineto{\pgfqpoint{4.822952in}{2.348649in}}%
\pgfpathlineto{\pgfqpoint{4.804164in}{2.348646in}}%
\pgfpathlineto{\pgfqpoint{4.785376in}{2.348643in}}%
\pgfpathlineto{\pgfqpoint{4.766588in}{2.348640in}}%
\pgfpathlineto{\pgfqpoint{4.747800in}{2.348637in}}%
\pgfpathlineto{\pgfqpoint{4.729013in}{2.348634in}}%
\pgfpathlineto{\pgfqpoint{4.710225in}{2.348631in}}%
\pgfpathlineto{\pgfqpoint{4.691437in}{2.348627in}}%
\pgfpathlineto{\pgfqpoint{4.672649in}{2.348623in}}%
\pgfpathlineto{\pgfqpoint{4.653861in}{2.348619in}}%
\pgfpathlineto{\pgfqpoint{4.635073in}{2.348615in}}%
\pgfpathlineto{\pgfqpoint{4.616285in}{2.348610in}}%
\pgfpathlineto{\pgfqpoint{4.597497in}{2.348606in}}%
\pgfpathlineto{\pgfqpoint{4.578710in}{2.348601in}}%
\pgfpathlineto{\pgfqpoint{4.559922in}{2.348595in}}%
\pgfpathlineto{\pgfqpoint{4.541134in}{2.348590in}}%
\pgfpathlineto{\pgfqpoint{4.522346in}{2.348584in}}%
\pgfpathlineto{\pgfqpoint{4.503558in}{2.348578in}}%
\pgfpathlineto{\pgfqpoint{4.484770in}{2.348572in}}%
\pgfpathlineto{\pgfqpoint{4.465982in}{2.348565in}}%
\pgfpathlineto{\pgfqpoint{4.447194in}{2.348558in}}%
\pgfpathlineto{\pgfqpoint{4.428407in}{2.348550in}}%
\pgfpathlineto{\pgfqpoint{4.409619in}{2.348542in}}%
\pgfpathlineto{\pgfqpoint{4.390831in}{2.348534in}}%
\pgfpathlineto{\pgfqpoint{4.372043in}{2.348525in}}%
\pgfpathlineto{\pgfqpoint{4.353255in}{2.348516in}}%
\pgfpathlineto{\pgfqpoint{4.334467in}{2.348507in}}%
\pgfpathlineto{\pgfqpoint{4.315679in}{2.348497in}}%
\pgfpathlineto{\pgfqpoint{4.296891in}{2.348486in}}%
\pgfpathlineto{\pgfqpoint{4.278104in}{2.348475in}}%
\pgfpathlineto{\pgfqpoint{4.259316in}{2.348464in}}%
\pgfpathlineto{\pgfqpoint{4.240528in}{2.348452in}}%
\pgfpathlineto{\pgfqpoint{4.221740in}{2.348440in}}%
\pgfpathlineto{\pgfqpoint{4.202952in}{2.348427in}}%
\pgfpathlineto{\pgfqpoint{4.184164in}{2.348413in}}%
\pgfpathlineto{\pgfqpoint{4.165376in}{2.348399in}}%
\pgfpathlineto{\pgfqpoint{4.146588in}{2.348384in}}%
\pgfpathlineto{\pgfqpoint{4.127800in}{2.348369in}}%
\pgfpathlineto{\pgfqpoint{4.109013in}{2.348353in}}%
\pgfpathlineto{\pgfqpoint{4.090225in}{2.348336in}}%
\pgfpathlineto{\pgfqpoint{4.071437in}{2.348319in}}%
\pgfpathlineto{\pgfqpoint{4.052649in}{2.348301in}}%
\pgfpathlineto{\pgfqpoint{4.033861in}{2.348283in}}%
\pgfpathlineto{\pgfqpoint{4.015073in}{2.348264in}}%
\pgfpathlineto{\pgfqpoint{3.996285in}{2.348244in}}%
\pgfpathlineto{\pgfqpoint{3.977497in}{2.348223in}}%
\pgfpathlineto{\pgfqpoint{3.958710in}{2.348201in}}%
\pgfpathlineto{\pgfqpoint{3.939922in}{2.348179in}}%
\pgfpathlineto{\pgfqpoint{3.921134in}{2.348156in}}%
\pgfpathlineto{\pgfqpoint{3.902346in}{2.348132in}}%
\pgfpathlineto{\pgfqpoint{3.883558in}{2.348108in}}%
\pgfpathlineto{\pgfqpoint{3.864770in}{2.348082in}}%
\pgfpathlineto{\pgfqpoint{3.845982in}{2.348056in}}%
\pgfpathlineto{\pgfqpoint{3.827194in}{2.348029in}}%
\pgfpathlineto{\pgfqpoint{3.808407in}{2.348001in}}%
\pgfpathlineto{\pgfqpoint{3.789619in}{2.347972in}}%
\pgfpathlineto{\pgfqpoint{3.770831in}{2.347943in}}%
\pgfpathlineto{\pgfqpoint{3.752043in}{2.347912in}}%
\pgfpathlineto{\pgfqpoint{3.733255in}{2.347881in}}%
\pgfpathlineto{\pgfqpoint{3.714467in}{2.347848in}}%
\pgfpathlineto{\pgfqpoint{3.695679in}{2.347815in}}%
\pgfpathlineto{\pgfqpoint{3.676891in}{2.347780in}}%
\pgfpathlineto{\pgfqpoint{3.658104in}{2.347745in}}%
\pgfpathlineto{\pgfqpoint{3.639316in}{2.347709in}}%
\pgfpathlineto{\pgfqpoint{3.620528in}{2.347672in}}%
\pgfpathlineto{\pgfqpoint{3.601740in}{2.347634in}}%
\pgfpathlineto{\pgfqpoint{3.582952in}{2.347594in}}%
\pgfpathlineto{\pgfqpoint{3.564164in}{2.347554in}}%
\pgfpathlineto{\pgfqpoint{3.545376in}{2.347513in}}%
\pgfpathlineto{\pgfqpoint{3.526588in}{2.347471in}}%
\pgfpathlineto{\pgfqpoint{3.507800in}{2.347428in}}%
\pgfpathlineto{\pgfqpoint{3.489013in}{2.347384in}}%
\pgfpathlineto{\pgfqpoint{3.470225in}{2.347339in}}%
\pgfpathlineto{\pgfqpoint{3.451437in}{2.347293in}}%
\pgfpathlineto{\pgfqpoint{3.432649in}{2.347245in}}%
\pgfpathlineto{\pgfqpoint{3.413861in}{2.347197in}}%
\pgfpathlineto{\pgfqpoint{3.395073in}{2.347148in}}%
\pgfpathlineto{\pgfqpoint{3.376285in}{2.347098in}}%
\pgfpathlineto{\pgfqpoint{3.357497in}{2.347047in}}%
\pgfpathlineto{\pgfqpoint{3.338710in}{2.346994in}}%
\pgfpathlineto{\pgfqpoint{3.319922in}{2.346941in}}%
\pgfpathlineto{\pgfqpoint{3.301134in}{2.346887in}}%
\pgfpathlineto{\pgfqpoint{3.282346in}{2.346832in}}%
\pgfpathlineto{\pgfqpoint{3.263558in}{2.346776in}}%
\pgfpathlineto{\pgfqpoint{3.244770in}{2.346719in}}%
\pgfpathlineto{\pgfqpoint{3.225982in}{2.346661in}}%
\pgfpathlineto{\pgfqpoint{3.207194in}{2.346602in}}%
\pgfpathlineto{\pgfqpoint{3.188407in}{2.346542in}}%
\pgfpathlineto{\pgfqpoint{3.169619in}{2.346481in}}%
\pgfpathlineto{\pgfqpoint{3.150831in}{2.346419in}}%
\pgfpathlineto{\pgfqpoint{3.132043in}{2.346357in}}%
\pgfpathlineto{\pgfqpoint{3.113255in}{2.346293in}}%
\pgfpathlineto{\pgfqpoint{3.094467in}{2.346229in}}%
\pgfpathlineto{\pgfqpoint{3.075679in}{2.346164in}}%
\pgfpathlineto{\pgfqpoint{3.056891in}{2.346099in}}%
\pgfpathlineto{\pgfqpoint{3.038104in}{2.346032in}}%
\pgfpathlineto{\pgfqpoint{3.019316in}{2.345965in}}%
\pgfpathlineto{\pgfqpoint{3.000528in}{2.345897in}}%
\pgfpathlineto{\pgfqpoint{2.981740in}{2.345829in}}%
\pgfpathlineto{\pgfqpoint{2.962952in}{2.345760in}}%
\pgfpathlineto{\pgfqpoint{2.944164in}{2.345690in}}%
\pgfpathlineto{\pgfqpoint{2.925376in}{2.345620in}}%
\pgfpathlineto{\pgfqpoint{2.906588in}{2.345549in}}%
\pgfpathlineto{\pgfqpoint{2.887800in}{2.345478in}}%
\pgfpathlineto{\pgfqpoint{2.869013in}{2.345406in}}%
\pgfpathlineto{\pgfqpoint{2.850225in}{2.345335in}}%
\pgfpathlineto{\pgfqpoint{2.831437in}{2.345262in}}%
\pgfpathlineto{\pgfqpoint{2.812649in}{2.345190in}}%
\pgfpathlineto{\pgfqpoint{2.793861in}{2.345118in}}%
\pgfpathlineto{\pgfqpoint{2.775073in}{2.345045in}}%
\pgfpathlineto{\pgfqpoint{2.756285in}{2.344972in}}%
\pgfpathlineto{\pgfqpoint{2.737497in}{2.344900in}}%
\pgfpathlineto{\pgfqpoint{2.718710in}{2.344827in}}%
\pgfpathlineto{\pgfqpoint{2.699922in}{2.344755in}}%
\pgfpathlineto{\pgfqpoint{2.681134in}{2.344683in}}%
\pgfpathlineto{\pgfqpoint{2.662346in}{2.344611in}}%
\pgfpathlineto{\pgfqpoint{2.643558in}{2.344540in}}%
\pgfpathlineto{\pgfqpoint{2.624770in}{2.344469in}}%
\pgfpathlineto{\pgfqpoint{2.605982in}{2.344399in}}%
\pgfpathlineto{\pgfqpoint{2.587194in}{2.344330in}}%
\pgfpathlineto{\pgfqpoint{2.568407in}{2.344262in}}%
\pgfpathlineto{\pgfqpoint{2.549619in}{2.344194in}}%
\pgfpathlineto{\pgfqpoint{2.530831in}{2.344128in}}%
\pgfpathlineto{\pgfqpoint{2.512043in}{2.344063in}}%
\pgfpathlineto{\pgfqpoint{2.493255in}{2.344000in}}%
\pgfpathlineto{\pgfqpoint{2.474467in}{2.343937in}}%
\pgfpathlineto{\pgfqpoint{2.455679in}{2.343877in}}%
\pgfpathlineto{\pgfqpoint{2.436891in}{2.343819in}}%
\pgfpathlineto{\pgfqpoint{2.418104in}{2.343762in}}%
\pgfpathlineto{\pgfqpoint{2.399316in}{2.343708in}}%
\pgfpathlineto{\pgfqpoint{2.380528in}{2.343656in}}%
\pgfpathlineto{\pgfqpoint{2.361740in}{2.385636in}}%
\pgfpathlineto{\pgfqpoint{2.342952in}{2.427619in}}%
\pgfpathlineto{\pgfqpoint{2.324164in}{2.469607in}}%
\pgfpathlineto{\pgfqpoint{2.305376in}{2.511599in}}%
\pgfpathlineto{\pgfqpoint{2.286588in}{2.553596in}}%
\pgfpathlineto{\pgfqpoint{2.267800in}{2.595597in}}%
\pgfpathlineto{\pgfqpoint{2.249013in}{2.637602in}}%
\pgfpathlineto{\pgfqpoint{2.230225in}{2.679612in}}%
\pgfpathlineto{\pgfqpoint{2.211437in}{2.721627in}}%
\pgfpathlineto{\pgfqpoint{2.192649in}{2.763646in}}%
\pgfpathlineto{\pgfqpoint{2.173861in}{2.805670in}}%
\pgfpathlineto{\pgfqpoint{2.155073in}{2.847699in}}%
\pgfpathlineto{\pgfqpoint{2.136285in}{2.889733in}}%
\pgfpathlineto{\pgfqpoint{2.117497in}{2.931773in}}%
\pgfpathlineto{\pgfqpoint{2.098710in}{2.973818in}}%
\pgfpathlineto{\pgfqpoint{2.079922in}{3.015869in}}%
\pgfpathlineto{\pgfqpoint{2.061134in}{3.057926in}}%
\pgfpathlineto{\pgfqpoint{2.042346in}{3.099989in}}%
\pgfpathlineto{\pgfqpoint{2.023558in}{3.142058in}}%
\pgfpathlineto{\pgfqpoint{2.004770in}{3.184134in}}%
\pgfpathlineto{\pgfqpoint{1.985982in}{3.226217in}}%
\pgfpathlineto{\pgfqpoint{1.967194in}{3.268307in}}%
\pgfpathlineto{\pgfqpoint{1.948407in}{3.310404in}}%
\pgfpathlineto{\pgfqpoint{1.929619in}{3.352508in}}%
\pgfpathlineto{\pgfqpoint{1.910831in}{3.394620in}}%
\pgfpathlineto{\pgfqpoint{1.892043in}{3.436738in}}%
\pgfpathlineto{\pgfqpoint{1.873255in}{3.478864in}}%
\pgfpathlineto{\pgfqpoint{1.854467in}{3.520998in}}%
\pgfpathlineto{\pgfqpoint{1.835679in}{3.563138in}}%
\pgfpathlineto{\pgfqpoint{1.816891in}{3.605286in}}%
\pgfpathlineto{\pgfqpoint{1.798104in}{3.647441in}}%
\pgfpathlineto{\pgfqpoint{1.779316in}{3.689602in}}%
\pgfpathlineto{\pgfqpoint{1.760528in}{3.731769in}}%
\pgfpathlineto{\pgfqpoint{1.741740in}{3.773942in}}%
\pgfpathlineto{\pgfqpoint{1.722952in}{3.816121in}}%
\pgfpathlineto{\pgfqpoint{1.704164in}{3.858304in}}%
\pgfpathlineto{\pgfqpoint{1.685376in}{3.900492in}}%
\pgfpathlineto{\pgfqpoint{1.666588in}{3.942683in}}%
\pgfpathlineto{\pgfqpoint{1.647800in}{3.984878in}}%
\pgfpathlineto{\pgfqpoint{1.629013in}{4.027074in}}%
\pgfpathlineto{\pgfqpoint{1.610225in}{3.987232in}}%
\pgfpathlineto{\pgfqpoint{1.591437in}{3.945240in}}%
\pgfpathlineto{\pgfqpoint{1.572649in}{3.903247in}}%
\pgfpathlineto{\pgfqpoint{1.553861in}{3.861254in}}%
\pgfpathlineto{\pgfqpoint{1.535073in}{3.819261in}}%
\pgfpathlineto{\pgfqpoint{1.516285in}{3.777267in}}%
\pgfpathlineto{\pgfqpoint{1.497497in}{3.735273in}}%
\pgfpathlineto{\pgfqpoint{1.478710in}{3.693277in}}%
\pgfpathlineto{\pgfqpoint{1.459922in}{3.651281in}}%
\pgfpathlineto{\pgfqpoint{1.441134in}{3.609283in}}%
\pgfpathlineto{\pgfqpoint{1.422346in}{3.567284in}}%
\pgfpathlineto{\pgfqpoint{1.403558in}{3.525284in}}%
\pgfpathlineto{\pgfqpoint{1.384770in}{3.483281in}}%
\pgfpathlineto{\pgfqpoint{1.365982in}{3.441278in}}%
\pgfpathlineto{\pgfqpoint{1.347194in}{3.399272in}}%
\pgfpathlineto{\pgfqpoint{1.328407in}{3.357264in}}%
\pgfpathlineto{\pgfqpoint{1.309619in}{3.315254in}}%
\pgfpathlineto{\pgfqpoint{1.290831in}{3.273242in}}%
\pgfpathlineto{\pgfqpoint{1.272043in}{3.231228in}}%
\pgfpathlineto{\pgfqpoint{1.253255in}{3.189212in}}%
\pgfpathlineto{\pgfqpoint{1.234467in}{3.147193in}}%
\pgfpathlineto{\pgfqpoint{1.215679in}{3.105172in}}%
\pgfpathlineto{\pgfqpoint{1.196891in}{3.063149in}}%
\pgfpathlineto{\pgfqpoint{1.178104in}{3.021124in}}%
\pgfpathlineto{\pgfqpoint{1.159316in}{2.979096in}}%
\pgfpathlineto{\pgfqpoint{1.140528in}{2.937067in}}%
\pgfpathlineto{\pgfqpoint{1.121740in}{2.895036in}}%
\pgfpathlineto{\pgfqpoint{1.102952in}{2.853005in}}%
\pgfpathlineto{\pgfqpoint{1.084164in}{2.810974in}}%
\pgfpathlineto{\pgfqpoint{1.065376in}{2.768943in}}%
\pgfpathlineto{\pgfqpoint{1.046588in}{2.726912in}}%
\pgfpathlineto{\pgfqpoint{1.027800in}{2.684883in}}%
\pgfpathlineto{\pgfqpoint{1.009013in}{2.642854in}}%
\pgfpathlineto{\pgfqpoint{0.990225in}{2.600827in}}%
\pgfpathlineto{\pgfqpoint{0.971437in}{2.558803in}}%
\pgfpathlineto{\pgfqpoint{0.952649in}{2.516780in}}%
\pgfpathlineto{\pgfqpoint{0.933861in}{2.474759in}}%
\pgfpathlineto{\pgfqpoint{0.915073in}{2.432737in}}%
\pgfpathlineto{\pgfqpoint{0.896285in}{2.390714in}}%
\pgfpathlineto{\pgfqpoint{0.877497in}{2.348687in}}%
\pgfpathlineto{\pgfqpoint{0.877497in}{2.348687in}}%
\pgfpathclose%
\pgfusepath{stroke,fill}%
}%
\begin{pgfscope}%
\pgfsys@transformshift{0.000000in}{0.000000in}%
\pgfsys@useobject{currentmarker}{}%
\end{pgfscope}%
\end{pgfscope}%
\begin{pgfscope}%
\pgfpathrectangle{\pgfqpoint{0.652043in}{0.499074in}}{\pgfqpoint{4.960000in}{3.696000in}}%
\pgfusepath{clip}%
\pgfsetbuttcap%
\pgfsetroundjoin%
\pgfsetlinewidth{0.803000pt}%
\definecolor{currentstroke}{rgb}{0.501961,0.501961,0.501961}%
\pgfsetstrokecolor{currentstroke}%
\pgfsetstrokeopacity{0.700000}%
\pgfsetdash{{0.800000pt}{1.320000pt}}{0.000000pt}%
\pgfpathmoveto{\pgfqpoint{0.877497in}{0.499074in}}%
\pgfpathlineto{\pgfqpoint{0.877497in}{4.195074in}}%
\pgfusepath{stroke}%
\end{pgfscope}%
\begin{pgfscope}%
\pgfsetbuttcap%
\pgfsetroundjoin%
\definecolor{currentfill}{rgb}{0.000000,0.000000,0.000000}%
\pgfsetfillcolor{currentfill}%
\pgfsetlinewidth{0.803000pt}%
\definecolor{currentstroke}{rgb}{0.000000,0.000000,0.000000}%
\pgfsetstrokecolor{currentstroke}%
\pgfsetdash{}{0pt}%
\pgfsys@defobject{currentmarker}{\pgfqpoint{0.000000in}{-0.048611in}}{\pgfqpoint{0.000000in}{0.000000in}}{%
\pgfpathmoveto{\pgfqpoint{0.000000in}{0.000000in}}%
\pgfpathlineto{\pgfqpoint{0.000000in}{-0.048611in}}%
\pgfusepath{stroke,fill}%
}%
\begin{pgfscope}%
\pgfsys@transformshift{0.877497in}{0.499074in}%
\pgfsys@useobject{currentmarker}{}%
\end{pgfscope}%
\end{pgfscope}%
\begin{pgfscope}%
\definecolor{textcolor}{rgb}{0.000000,0.000000,0.000000}%
\pgfsetstrokecolor{textcolor}%
\pgfsetfillcolor{textcolor}%
\pgftext[x=0.877497in,y=0.401852in,,top]{\color{textcolor}{\rmfamily\fontsize{9.000000}{10.800000}\selectfont\catcode`\^=\active\def^{\ifmmode\sp\else\^{}\fi}\catcode`\%=\active\def%{\%}$\mathdefault{0.0}$}}%
\end{pgfscope}%
\begin{pgfscope}%
\pgfpathrectangle{\pgfqpoint{0.652043in}{0.499074in}}{\pgfqpoint{4.960000in}{3.696000in}}%
\pgfusepath{clip}%
\pgfsetbuttcap%
\pgfsetroundjoin%
\pgfsetlinewidth{0.803000pt}%
\definecolor{currentstroke}{rgb}{0.501961,0.501961,0.501961}%
\pgfsetstrokecolor{currentstroke}%
\pgfsetstrokeopacity{0.700000}%
\pgfsetdash{{0.800000pt}{1.320000pt}}{0.000000pt}%
\pgfpathmoveto{\pgfqpoint{1.441134in}{0.499074in}}%
\pgfpathlineto{\pgfqpoint{1.441134in}{4.195074in}}%
\pgfusepath{stroke}%
\end{pgfscope}%
\begin{pgfscope}%
\pgfsetbuttcap%
\pgfsetroundjoin%
\definecolor{currentfill}{rgb}{0.000000,0.000000,0.000000}%
\pgfsetfillcolor{currentfill}%
\pgfsetlinewidth{0.803000pt}%
\definecolor{currentstroke}{rgb}{0.000000,0.000000,0.000000}%
\pgfsetstrokecolor{currentstroke}%
\pgfsetdash{}{0pt}%
\pgfsys@defobject{currentmarker}{\pgfqpoint{0.000000in}{-0.048611in}}{\pgfqpoint{0.000000in}{0.000000in}}{%
\pgfpathmoveto{\pgfqpoint{0.000000in}{0.000000in}}%
\pgfpathlineto{\pgfqpoint{0.000000in}{-0.048611in}}%
\pgfusepath{stroke,fill}%
}%
\begin{pgfscope}%
\pgfsys@transformshift{1.441134in}{0.499074in}%
\pgfsys@useobject{currentmarker}{}%
\end{pgfscope}%
\end{pgfscope}%
\begin{pgfscope}%
\definecolor{textcolor}{rgb}{0.000000,0.000000,0.000000}%
\pgfsetstrokecolor{textcolor}%
\pgfsetfillcolor{textcolor}%
\pgftext[x=1.441134in,y=0.401852in,,top]{\color{textcolor}{\rmfamily\fontsize{9.000000}{10.800000}\selectfont\catcode`\^=\active\def^{\ifmmode\sp\else\^{}\fi}\catcode`\%=\active\def%{\%}$\mathdefault{0.5}$}}%
\end{pgfscope}%
\begin{pgfscope}%
\pgfpathrectangle{\pgfqpoint{0.652043in}{0.499074in}}{\pgfqpoint{4.960000in}{3.696000in}}%
\pgfusepath{clip}%
\pgfsetbuttcap%
\pgfsetroundjoin%
\pgfsetlinewidth{0.803000pt}%
\definecolor{currentstroke}{rgb}{0.501961,0.501961,0.501961}%
\pgfsetstrokecolor{currentstroke}%
\pgfsetstrokeopacity{0.700000}%
\pgfsetdash{{0.800000pt}{1.320000pt}}{0.000000pt}%
\pgfpathmoveto{\pgfqpoint{2.004770in}{0.499074in}}%
\pgfpathlineto{\pgfqpoint{2.004770in}{4.195074in}}%
\pgfusepath{stroke}%
\end{pgfscope}%
\begin{pgfscope}%
\pgfsetbuttcap%
\pgfsetroundjoin%
\definecolor{currentfill}{rgb}{0.000000,0.000000,0.000000}%
\pgfsetfillcolor{currentfill}%
\pgfsetlinewidth{0.803000pt}%
\definecolor{currentstroke}{rgb}{0.000000,0.000000,0.000000}%
\pgfsetstrokecolor{currentstroke}%
\pgfsetdash{}{0pt}%
\pgfsys@defobject{currentmarker}{\pgfqpoint{0.000000in}{-0.048611in}}{\pgfqpoint{0.000000in}{0.000000in}}{%
\pgfpathmoveto{\pgfqpoint{0.000000in}{0.000000in}}%
\pgfpathlineto{\pgfqpoint{0.000000in}{-0.048611in}}%
\pgfusepath{stroke,fill}%
}%
\begin{pgfscope}%
\pgfsys@transformshift{2.004770in}{0.499074in}%
\pgfsys@useobject{currentmarker}{}%
\end{pgfscope}%
\end{pgfscope}%
\begin{pgfscope}%
\definecolor{textcolor}{rgb}{0.000000,0.000000,0.000000}%
\pgfsetstrokecolor{textcolor}%
\pgfsetfillcolor{textcolor}%
\pgftext[x=2.004770in,y=0.401852in,,top]{\color{textcolor}{\rmfamily\fontsize{9.000000}{10.800000}\selectfont\catcode`\^=\active\def^{\ifmmode\sp\else\^{}\fi}\catcode`\%=\active\def%{\%}$\mathdefault{1.0}$}}%
\end{pgfscope}%
\begin{pgfscope}%
\pgfpathrectangle{\pgfqpoint{0.652043in}{0.499074in}}{\pgfqpoint{4.960000in}{3.696000in}}%
\pgfusepath{clip}%
\pgfsetbuttcap%
\pgfsetroundjoin%
\pgfsetlinewidth{0.803000pt}%
\definecolor{currentstroke}{rgb}{0.501961,0.501961,0.501961}%
\pgfsetstrokecolor{currentstroke}%
\pgfsetstrokeopacity{0.700000}%
\pgfsetdash{{0.800000pt}{1.320000pt}}{0.000000pt}%
\pgfpathmoveto{\pgfqpoint{2.568407in}{0.499074in}}%
\pgfpathlineto{\pgfqpoint{2.568407in}{4.195074in}}%
\pgfusepath{stroke}%
\end{pgfscope}%
\begin{pgfscope}%
\pgfsetbuttcap%
\pgfsetroundjoin%
\definecolor{currentfill}{rgb}{0.000000,0.000000,0.000000}%
\pgfsetfillcolor{currentfill}%
\pgfsetlinewidth{0.803000pt}%
\definecolor{currentstroke}{rgb}{0.000000,0.000000,0.000000}%
\pgfsetstrokecolor{currentstroke}%
\pgfsetdash{}{0pt}%
\pgfsys@defobject{currentmarker}{\pgfqpoint{0.000000in}{-0.048611in}}{\pgfqpoint{0.000000in}{0.000000in}}{%
\pgfpathmoveto{\pgfqpoint{0.000000in}{0.000000in}}%
\pgfpathlineto{\pgfqpoint{0.000000in}{-0.048611in}}%
\pgfusepath{stroke,fill}%
}%
\begin{pgfscope}%
\pgfsys@transformshift{2.568407in}{0.499074in}%
\pgfsys@useobject{currentmarker}{}%
\end{pgfscope}%
\end{pgfscope}%
\begin{pgfscope}%
\definecolor{textcolor}{rgb}{0.000000,0.000000,0.000000}%
\pgfsetstrokecolor{textcolor}%
\pgfsetfillcolor{textcolor}%
\pgftext[x=2.568407in,y=0.401852in,,top]{\color{textcolor}{\rmfamily\fontsize{9.000000}{10.800000}\selectfont\catcode`\^=\active\def^{\ifmmode\sp\else\^{}\fi}\catcode`\%=\active\def%{\%}$\mathdefault{1.5}$}}%
\end{pgfscope}%
\begin{pgfscope}%
\pgfpathrectangle{\pgfqpoint{0.652043in}{0.499074in}}{\pgfqpoint{4.960000in}{3.696000in}}%
\pgfusepath{clip}%
\pgfsetbuttcap%
\pgfsetroundjoin%
\pgfsetlinewidth{0.803000pt}%
\definecolor{currentstroke}{rgb}{0.501961,0.501961,0.501961}%
\pgfsetstrokecolor{currentstroke}%
\pgfsetstrokeopacity{0.700000}%
\pgfsetdash{{0.800000pt}{1.320000pt}}{0.000000pt}%
\pgfpathmoveto{\pgfqpoint{3.132043in}{0.499074in}}%
\pgfpathlineto{\pgfqpoint{3.132043in}{4.195074in}}%
\pgfusepath{stroke}%
\end{pgfscope}%
\begin{pgfscope}%
\pgfsetbuttcap%
\pgfsetroundjoin%
\definecolor{currentfill}{rgb}{0.000000,0.000000,0.000000}%
\pgfsetfillcolor{currentfill}%
\pgfsetlinewidth{0.803000pt}%
\definecolor{currentstroke}{rgb}{0.000000,0.000000,0.000000}%
\pgfsetstrokecolor{currentstroke}%
\pgfsetdash{}{0pt}%
\pgfsys@defobject{currentmarker}{\pgfqpoint{0.000000in}{-0.048611in}}{\pgfqpoint{0.000000in}{0.000000in}}{%
\pgfpathmoveto{\pgfqpoint{0.000000in}{0.000000in}}%
\pgfpathlineto{\pgfqpoint{0.000000in}{-0.048611in}}%
\pgfusepath{stroke,fill}%
}%
\begin{pgfscope}%
\pgfsys@transformshift{3.132043in}{0.499074in}%
\pgfsys@useobject{currentmarker}{}%
\end{pgfscope}%
\end{pgfscope}%
\begin{pgfscope}%
\definecolor{textcolor}{rgb}{0.000000,0.000000,0.000000}%
\pgfsetstrokecolor{textcolor}%
\pgfsetfillcolor{textcolor}%
\pgftext[x=3.132043in,y=0.401852in,,top]{\color{textcolor}{\rmfamily\fontsize{9.000000}{10.800000}\selectfont\catcode`\^=\active\def^{\ifmmode\sp\else\^{}\fi}\catcode`\%=\active\def%{\%}$\mathdefault{2.0}$}}%
\end{pgfscope}%
\begin{pgfscope}%
\pgfpathrectangle{\pgfqpoint{0.652043in}{0.499074in}}{\pgfqpoint{4.960000in}{3.696000in}}%
\pgfusepath{clip}%
\pgfsetbuttcap%
\pgfsetroundjoin%
\pgfsetlinewidth{0.803000pt}%
\definecolor{currentstroke}{rgb}{0.501961,0.501961,0.501961}%
\pgfsetstrokecolor{currentstroke}%
\pgfsetstrokeopacity{0.700000}%
\pgfsetdash{{0.800000pt}{1.320000pt}}{0.000000pt}%
\pgfpathmoveto{\pgfqpoint{3.695679in}{0.499074in}}%
\pgfpathlineto{\pgfqpoint{3.695679in}{4.195074in}}%
\pgfusepath{stroke}%
\end{pgfscope}%
\begin{pgfscope}%
\pgfsetbuttcap%
\pgfsetroundjoin%
\definecolor{currentfill}{rgb}{0.000000,0.000000,0.000000}%
\pgfsetfillcolor{currentfill}%
\pgfsetlinewidth{0.803000pt}%
\definecolor{currentstroke}{rgb}{0.000000,0.000000,0.000000}%
\pgfsetstrokecolor{currentstroke}%
\pgfsetdash{}{0pt}%
\pgfsys@defobject{currentmarker}{\pgfqpoint{0.000000in}{-0.048611in}}{\pgfqpoint{0.000000in}{0.000000in}}{%
\pgfpathmoveto{\pgfqpoint{0.000000in}{0.000000in}}%
\pgfpathlineto{\pgfqpoint{0.000000in}{-0.048611in}}%
\pgfusepath{stroke,fill}%
}%
\begin{pgfscope}%
\pgfsys@transformshift{3.695679in}{0.499074in}%
\pgfsys@useobject{currentmarker}{}%
\end{pgfscope}%
\end{pgfscope}%
\begin{pgfscope}%
\definecolor{textcolor}{rgb}{0.000000,0.000000,0.000000}%
\pgfsetstrokecolor{textcolor}%
\pgfsetfillcolor{textcolor}%
\pgftext[x=3.695679in,y=0.401852in,,top]{\color{textcolor}{\rmfamily\fontsize{9.000000}{10.800000}\selectfont\catcode`\^=\active\def^{\ifmmode\sp\else\^{}\fi}\catcode`\%=\active\def%{\%}$\mathdefault{2.5}$}}%
\end{pgfscope}%
\begin{pgfscope}%
\pgfpathrectangle{\pgfqpoint{0.652043in}{0.499074in}}{\pgfqpoint{4.960000in}{3.696000in}}%
\pgfusepath{clip}%
\pgfsetbuttcap%
\pgfsetroundjoin%
\pgfsetlinewidth{0.803000pt}%
\definecolor{currentstroke}{rgb}{0.501961,0.501961,0.501961}%
\pgfsetstrokecolor{currentstroke}%
\pgfsetstrokeopacity{0.700000}%
\pgfsetdash{{0.800000pt}{1.320000pt}}{0.000000pt}%
\pgfpathmoveto{\pgfqpoint{4.259316in}{0.499074in}}%
\pgfpathlineto{\pgfqpoint{4.259316in}{4.195074in}}%
\pgfusepath{stroke}%
\end{pgfscope}%
\begin{pgfscope}%
\pgfsetbuttcap%
\pgfsetroundjoin%
\definecolor{currentfill}{rgb}{0.000000,0.000000,0.000000}%
\pgfsetfillcolor{currentfill}%
\pgfsetlinewidth{0.803000pt}%
\definecolor{currentstroke}{rgb}{0.000000,0.000000,0.000000}%
\pgfsetstrokecolor{currentstroke}%
\pgfsetdash{}{0pt}%
\pgfsys@defobject{currentmarker}{\pgfqpoint{0.000000in}{-0.048611in}}{\pgfqpoint{0.000000in}{0.000000in}}{%
\pgfpathmoveto{\pgfqpoint{0.000000in}{0.000000in}}%
\pgfpathlineto{\pgfqpoint{0.000000in}{-0.048611in}}%
\pgfusepath{stroke,fill}%
}%
\begin{pgfscope}%
\pgfsys@transformshift{4.259316in}{0.499074in}%
\pgfsys@useobject{currentmarker}{}%
\end{pgfscope}%
\end{pgfscope}%
\begin{pgfscope}%
\definecolor{textcolor}{rgb}{0.000000,0.000000,0.000000}%
\pgfsetstrokecolor{textcolor}%
\pgfsetfillcolor{textcolor}%
\pgftext[x=4.259316in,y=0.401852in,,top]{\color{textcolor}{\rmfamily\fontsize{9.000000}{10.800000}\selectfont\catcode`\^=\active\def^{\ifmmode\sp\else\^{}\fi}\catcode`\%=\active\def%{\%}$\mathdefault{3.0}$}}%
\end{pgfscope}%
\begin{pgfscope}%
\pgfpathrectangle{\pgfqpoint{0.652043in}{0.499074in}}{\pgfqpoint{4.960000in}{3.696000in}}%
\pgfusepath{clip}%
\pgfsetbuttcap%
\pgfsetroundjoin%
\pgfsetlinewidth{0.803000pt}%
\definecolor{currentstroke}{rgb}{0.501961,0.501961,0.501961}%
\pgfsetstrokecolor{currentstroke}%
\pgfsetstrokeopacity{0.700000}%
\pgfsetdash{{0.800000pt}{1.320000pt}}{0.000000pt}%
\pgfpathmoveto{\pgfqpoint{4.822952in}{0.499074in}}%
\pgfpathlineto{\pgfqpoint{4.822952in}{4.195074in}}%
\pgfusepath{stroke}%
\end{pgfscope}%
\begin{pgfscope}%
\pgfsetbuttcap%
\pgfsetroundjoin%
\definecolor{currentfill}{rgb}{0.000000,0.000000,0.000000}%
\pgfsetfillcolor{currentfill}%
\pgfsetlinewidth{0.803000pt}%
\definecolor{currentstroke}{rgb}{0.000000,0.000000,0.000000}%
\pgfsetstrokecolor{currentstroke}%
\pgfsetdash{}{0pt}%
\pgfsys@defobject{currentmarker}{\pgfqpoint{0.000000in}{-0.048611in}}{\pgfqpoint{0.000000in}{0.000000in}}{%
\pgfpathmoveto{\pgfqpoint{0.000000in}{0.000000in}}%
\pgfpathlineto{\pgfqpoint{0.000000in}{-0.048611in}}%
\pgfusepath{stroke,fill}%
}%
\begin{pgfscope}%
\pgfsys@transformshift{4.822952in}{0.499074in}%
\pgfsys@useobject{currentmarker}{}%
\end{pgfscope}%
\end{pgfscope}%
\begin{pgfscope}%
\definecolor{textcolor}{rgb}{0.000000,0.000000,0.000000}%
\pgfsetstrokecolor{textcolor}%
\pgfsetfillcolor{textcolor}%
\pgftext[x=4.822952in,y=0.401852in,,top]{\color{textcolor}{\rmfamily\fontsize{9.000000}{10.800000}\selectfont\catcode`\^=\active\def^{\ifmmode\sp\else\^{}\fi}\catcode`\%=\active\def%{\%}$\mathdefault{3.5}$}}%
\end{pgfscope}%
\begin{pgfscope}%
\pgfpathrectangle{\pgfqpoint{0.652043in}{0.499074in}}{\pgfqpoint{4.960000in}{3.696000in}}%
\pgfusepath{clip}%
\pgfsetbuttcap%
\pgfsetroundjoin%
\pgfsetlinewidth{0.803000pt}%
\definecolor{currentstroke}{rgb}{0.501961,0.501961,0.501961}%
\pgfsetstrokecolor{currentstroke}%
\pgfsetstrokeopacity{0.700000}%
\pgfsetdash{{0.800000pt}{1.320000pt}}{0.000000pt}%
\pgfpathmoveto{\pgfqpoint{5.386588in}{0.499074in}}%
\pgfpathlineto{\pgfqpoint{5.386588in}{4.195074in}}%
\pgfusepath{stroke}%
\end{pgfscope}%
\begin{pgfscope}%
\pgfsetbuttcap%
\pgfsetroundjoin%
\definecolor{currentfill}{rgb}{0.000000,0.000000,0.000000}%
\pgfsetfillcolor{currentfill}%
\pgfsetlinewidth{0.803000pt}%
\definecolor{currentstroke}{rgb}{0.000000,0.000000,0.000000}%
\pgfsetstrokecolor{currentstroke}%
\pgfsetdash{}{0pt}%
\pgfsys@defobject{currentmarker}{\pgfqpoint{0.000000in}{-0.048611in}}{\pgfqpoint{0.000000in}{0.000000in}}{%
\pgfpathmoveto{\pgfqpoint{0.000000in}{0.000000in}}%
\pgfpathlineto{\pgfqpoint{0.000000in}{-0.048611in}}%
\pgfusepath{stroke,fill}%
}%
\begin{pgfscope}%
\pgfsys@transformshift{5.386588in}{0.499074in}%
\pgfsys@useobject{currentmarker}{}%
\end{pgfscope}%
\end{pgfscope}%
\begin{pgfscope}%
\definecolor{textcolor}{rgb}{0.000000,0.000000,0.000000}%
\pgfsetstrokecolor{textcolor}%
\pgfsetfillcolor{textcolor}%
\pgftext[x=5.386588in,y=0.401852in,,top]{\color{textcolor}{\rmfamily\fontsize{9.000000}{10.800000}\selectfont\catcode`\^=\active\def^{\ifmmode\sp\else\^{}\fi}\catcode`\%=\active\def%{\%}$\mathdefault{4.0}$}}%
\end{pgfscope}%
\begin{pgfscope}%
\definecolor{textcolor}{rgb}{0.000000,0.000000,0.000000}%
\pgfsetstrokecolor{textcolor}%
\pgfsetfillcolor{textcolor}%
\pgftext[x=3.132043in,y=0.235185in,,top]{\color{textcolor}{\rmfamily\fontsize{11.000000}{13.200000}\selectfont\catcode`\^=\active\def^{\ifmmode\sp\else\^{}\fi}\catcode`\%=\active\def%{\%}Time [s]}}%
\end{pgfscope}%
\begin{pgfscope}%
\pgfpathrectangle{\pgfqpoint{0.652043in}{0.499074in}}{\pgfqpoint{4.960000in}{3.696000in}}%
\pgfusepath{clip}%
\pgfsetbuttcap%
\pgfsetroundjoin%
\pgfsetlinewidth{0.803000pt}%
\definecolor{currentstroke}{rgb}{0.501961,0.501961,0.501961}%
\pgfsetstrokecolor{currentstroke}%
\pgfsetstrokeopacity{0.700000}%
\pgfsetdash{{0.800000pt}{1.320000pt}}{0.000000pt}%
\pgfpathmoveto{\pgfqpoint{0.652043in}{0.743378in}}%
\pgfpathlineto{\pgfqpoint{5.612043in}{0.743378in}}%
\pgfusepath{stroke}%
\end{pgfscope}%
\begin{pgfscope}%
\pgfsetbuttcap%
\pgfsetroundjoin%
\definecolor{currentfill}{rgb}{0.000000,0.000000,0.000000}%
\pgfsetfillcolor{currentfill}%
\pgfsetlinewidth{0.803000pt}%
\definecolor{currentstroke}{rgb}{0.000000,0.000000,0.000000}%
\pgfsetstrokecolor{currentstroke}%
\pgfsetdash{}{0pt}%
\pgfsys@defobject{currentmarker}{\pgfqpoint{-0.048611in}{0.000000in}}{\pgfqpoint{-0.000000in}{0.000000in}}{%
\pgfpathmoveto{\pgfqpoint{-0.000000in}{0.000000in}}%
\pgfpathlineto{\pgfqpoint{-0.048611in}{0.000000in}}%
\pgfusepath{stroke,fill}%
}%
\begin{pgfscope}%
\pgfsys@transformshift{0.652043in}{0.743378in}%
\pgfsys@useobject{currentmarker}{}%
\end{pgfscope}%
\end{pgfscope}%
\begin{pgfscope}%
\definecolor{textcolor}{rgb}{0.000000,0.000000,0.000000}%
\pgfsetstrokecolor{textcolor}%
\pgfsetfillcolor{textcolor}%
\pgftext[x=0.290741in, y=0.699976in, left, base]{\color{textcolor}{\rmfamily\fontsize{9.000000}{10.800000}\selectfont\catcode`\^=\active\def^{\ifmmode\sp\else\^{}\fi}\catcode`\%=\active\def%{\%}$\mathdefault{\ensuremath{-}1.5}$}}%
\end{pgfscope}%
\begin{pgfscope}%
\pgfpathrectangle{\pgfqpoint{0.652043in}{0.499074in}}{\pgfqpoint{4.960000in}{3.696000in}}%
\pgfusepath{clip}%
\pgfsetbuttcap%
\pgfsetroundjoin%
\pgfsetlinewidth{0.803000pt}%
\definecolor{currentstroke}{rgb}{0.501961,0.501961,0.501961}%
\pgfsetstrokecolor{currentstroke}%
\pgfsetstrokeopacity{0.700000}%
\pgfsetdash{{0.800000pt}{1.320000pt}}{0.000000pt}%
\pgfpathmoveto{\pgfqpoint{0.652043in}{1.278481in}}%
\pgfpathlineto{\pgfqpoint{5.612043in}{1.278481in}}%
\pgfusepath{stroke}%
\end{pgfscope}%
\begin{pgfscope}%
\pgfsetbuttcap%
\pgfsetroundjoin%
\definecolor{currentfill}{rgb}{0.000000,0.000000,0.000000}%
\pgfsetfillcolor{currentfill}%
\pgfsetlinewidth{0.803000pt}%
\definecolor{currentstroke}{rgb}{0.000000,0.000000,0.000000}%
\pgfsetstrokecolor{currentstroke}%
\pgfsetdash{}{0pt}%
\pgfsys@defobject{currentmarker}{\pgfqpoint{-0.048611in}{0.000000in}}{\pgfqpoint{-0.000000in}{0.000000in}}{%
\pgfpathmoveto{\pgfqpoint{-0.000000in}{0.000000in}}%
\pgfpathlineto{\pgfqpoint{-0.048611in}{0.000000in}}%
\pgfusepath{stroke,fill}%
}%
\begin{pgfscope}%
\pgfsys@transformshift{0.652043in}{1.278481in}%
\pgfsys@useobject{currentmarker}{}%
\end{pgfscope}%
\end{pgfscope}%
\begin{pgfscope}%
\definecolor{textcolor}{rgb}{0.000000,0.000000,0.000000}%
\pgfsetstrokecolor{textcolor}%
\pgfsetfillcolor{textcolor}%
\pgftext[x=0.290741in, y=1.235079in, left, base]{\color{textcolor}{\rmfamily\fontsize{9.000000}{10.800000}\selectfont\catcode`\^=\active\def^{\ifmmode\sp\else\^{}\fi}\catcode`\%=\active\def%{\%}$\mathdefault{\ensuremath{-}1.0}$}}%
\end{pgfscope}%
\begin{pgfscope}%
\pgfpathrectangle{\pgfqpoint{0.652043in}{0.499074in}}{\pgfqpoint{4.960000in}{3.696000in}}%
\pgfusepath{clip}%
\pgfsetbuttcap%
\pgfsetroundjoin%
\pgfsetlinewidth{0.803000pt}%
\definecolor{currentstroke}{rgb}{0.501961,0.501961,0.501961}%
\pgfsetstrokecolor{currentstroke}%
\pgfsetstrokeopacity{0.700000}%
\pgfsetdash{{0.800000pt}{1.320000pt}}{0.000000pt}%
\pgfpathmoveto{\pgfqpoint{0.652043in}{1.813584in}}%
\pgfpathlineto{\pgfqpoint{5.612043in}{1.813584in}}%
\pgfusepath{stroke}%
\end{pgfscope}%
\begin{pgfscope}%
\pgfsetbuttcap%
\pgfsetroundjoin%
\definecolor{currentfill}{rgb}{0.000000,0.000000,0.000000}%
\pgfsetfillcolor{currentfill}%
\pgfsetlinewidth{0.803000pt}%
\definecolor{currentstroke}{rgb}{0.000000,0.000000,0.000000}%
\pgfsetstrokecolor{currentstroke}%
\pgfsetdash{}{0pt}%
\pgfsys@defobject{currentmarker}{\pgfqpoint{-0.048611in}{0.000000in}}{\pgfqpoint{-0.000000in}{0.000000in}}{%
\pgfpathmoveto{\pgfqpoint{-0.000000in}{0.000000in}}%
\pgfpathlineto{\pgfqpoint{-0.048611in}{0.000000in}}%
\pgfusepath{stroke,fill}%
}%
\begin{pgfscope}%
\pgfsys@transformshift{0.652043in}{1.813584in}%
\pgfsys@useobject{currentmarker}{}%
\end{pgfscope}%
\end{pgfscope}%
\begin{pgfscope}%
\definecolor{textcolor}{rgb}{0.000000,0.000000,0.000000}%
\pgfsetstrokecolor{textcolor}%
\pgfsetfillcolor{textcolor}%
\pgftext[x=0.290741in, y=1.770181in, left, base]{\color{textcolor}{\rmfamily\fontsize{9.000000}{10.800000}\selectfont\catcode`\^=\active\def^{\ifmmode\sp\else\^{}\fi}\catcode`\%=\active\def%{\%}$\mathdefault{\ensuremath{-}0.5}$}}%
\end{pgfscope}%
\begin{pgfscope}%
\pgfpathrectangle{\pgfqpoint{0.652043in}{0.499074in}}{\pgfqpoint{4.960000in}{3.696000in}}%
\pgfusepath{clip}%
\pgfsetbuttcap%
\pgfsetroundjoin%
\pgfsetlinewidth{0.803000pt}%
\definecolor{currentstroke}{rgb}{0.501961,0.501961,0.501961}%
\pgfsetstrokecolor{currentstroke}%
\pgfsetstrokeopacity{0.700000}%
\pgfsetdash{{0.800000pt}{1.320000pt}}{0.000000pt}%
\pgfpathmoveto{\pgfqpoint{0.652043in}{2.348687in}}%
\pgfpathlineto{\pgfqpoint{5.612043in}{2.348687in}}%
\pgfusepath{stroke}%
\end{pgfscope}%
\begin{pgfscope}%
\pgfsetbuttcap%
\pgfsetroundjoin%
\definecolor{currentfill}{rgb}{0.000000,0.000000,0.000000}%
\pgfsetfillcolor{currentfill}%
\pgfsetlinewidth{0.803000pt}%
\definecolor{currentstroke}{rgb}{0.000000,0.000000,0.000000}%
\pgfsetstrokecolor{currentstroke}%
\pgfsetdash{}{0pt}%
\pgfsys@defobject{currentmarker}{\pgfqpoint{-0.048611in}{0.000000in}}{\pgfqpoint{-0.000000in}{0.000000in}}{%
\pgfpathmoveto{\pgfqpoint{-0.000000in}{0.000000in}}%
\pgfpathlineto{\pgfqpoint{-0.048611in}{0.000000in}}%
\pgfusepath{stroke,fill}%
}%
\begin{pgfscope}%
\pgfsys@transformshift{0.652043in}{2.348687in}%
\pgfsys@useobject{currentmarker}{}%
\end{pgfscope}%
\end{pgfscope}%
\begin{pgfscope}%
\definecolor{textcolor}{rgb}{0.000000,0.000000,0.000000}%
\pgfsetstrokecolor{textcolor}%
\pgfsetfillcolor{textcolor}%
\pgftext[x=0.390663in, y=2.305284in, left, base]{\color{textcolor}{\rmfamily\fontsize{9.000000}{10.800000}\selectfont\catcode`\^=\active\def^{\ifmmode\sp\else\^{}\fi}\catcode`\%=\active\def%{\%}$\mathdefault{0.0}$}}%
\end{pgfscope}%
\begin{pgfscope}%
\pgfpathrectangle{\pgfqpoint{0.652043in}{0.499074in}}{\pgfqpoint{4.960000in}{3.696000in}}%
\pgfusepath{clip}%
\pgfsetbuttcap%
\pgfsetroundjoin%
\pgfsetlinewidth{0.803000pt}%
\definecolor{currentstroke}{rgb}{0.501961,0.501961,0.501961}%
\pgfsetstrokecolor{currentstroke}%
\pgfsetstrokeopacity{0.700000}%
\pgfsetdash{{0.800000pt}{1.320000pt}}{0.000000pt}%
\pgfpathmoveto{\pgfqpoint{0.652043in}{2.883790in}}%
\pgfpathlineto{\pgfqpoint{5.612043in}{2.883790in}}%
\pgfusepath{stroke}%
\end{pgfscope}%
\begin{pgfscope}%
\pgfsetbuttcap%
\pgfsetroundjoin%
\definecolor{currentfill}{rgb}{0.000000,0.000000,0.000000}%
\pgfsetfillcolor{currentfill}%
\pgfsetlinewidth{0.803000pt}%
\definecolor{currentstroke}{rgb}{0.000000,0.000000,0.000000}%
\pgfsetstrokecolor{currentstroke}%
\pgfsetdash{}{0pt}%
\pgfsys@defobject{currentmarker}{\pgfqpoint{-0.048611in}{0.000000in}}{\pgfqpoint{-0.000000in}{0.000000in}}{%
\pgfpathmoveto{\pgfqpoint{-0.000000in}{0.000000in}}%
\pgfpathlineto{\pgfqpoint{-0.048611in}{0.000000in}}%
\pgfusepath{stroke,fill}%
}%
\begin{pgfscope}%
\pgfsys@transformshift{0.652043in}{2.883790in}%
\pgfsys@useobject{currentmarker}{}%
\end{pgfscope}%
\end{pgfscope}%
\begin{pgfscope}%
\definecolor{textcolor}{rgb}{0.000000,0.000000,0.000000}%
\pgfsetstrokecolor{textcolor}%
\pgfsetfillcolor{textcolor}%
\pgftext[x=0.390663in, y=2.840387in, left, base]{\color{textcolor}{\rmfamily\fontsize{9.000000}{10.800000}\selectfont\catcode`\^=\active\def^{\ifmmode\sp\else\^{}\fi}\catcode`\%=\active\def%{\%}$\mathdefault{0.5}$}}%
\end{pgfscope}%
\begin{pgfscope}%
\pgfpathrectangle{\pgfqpoint{0.652043in}{0.499074in}}{\pgfqpoint{4.960000in}{3.696000in}}%
\pgfusepath{clip}%
\pgfsetbuttcap%
\pgfsetroundjoin%
\pgfsetlinewidth{0.803000pt}%
\definecolor{currentstroke}{rgb}{0.501961,0.501961,0.501961}%
\pgfsetstrokecolor{currentstroke}%
\pgfsetstrokeopacity{0.700000}%
\pgfsetdash{{0.800000pt}{1.320000pt}}{0.000000pt}%
\pgfpathmoveto{\pgfqpoint{0.652043in}{3.418893in}}%
\pgfpathlineto{\pgfqpoint{5.612043in}{3.418893in}}%
\pgfusepath{stroke}%
\end{pgfscope}%
\begin{pgfscope}%
\pgfsetbuttcap%
\pgfsetroundjoin%
\definecolor{currentfill}{rgb}{0.000000,0.000000,0.000000}%
\pgfsetfillcolor{currentfill}%
\pgfsetlinewidth{0.803000pt}%
\definecolor{currentstroke}{rgb}{0.000000,0.000000,0.000000}%
\pgfsetstrokecolor{currentstroke}%
\pgfsetdash{}{0pt}%
\pgfsys@defobject{currentmarker}{\pgfqpoint{-0.048611in}{0.000000in}}{\pgfqpoint{-0.000000in}{0.000000in}}{%
\pgfpathmoveto{\pgfqpoint{-0.000000in}{0.000000in}}%
\pgfpathlineto{\pgfqpoint{-0.048611in}{0.000000in}}%
\pgfusepath{stroke,fill}%
}%
\begin{pgfscope}%
\pgfsys@transformshift{0.652043in}{3.418893in}%
\pgfsys@useobject{currentmarker}{}%
\end{pgfscope}%
\end{pgfscope}%
\begin{pgfscope}%
\definecolor{textcolor}{rgb}{0.000000,0.000000,0.000000}%
\pgfsetstrokecolor{textcolor}%
\pgfsetfillcolor{textcolor}%
\pgftext[x=0.390663in, y=3.375490in, left, base]{\color{textcolor}{\rmfamily\fontsize{9.000000}{10.800000}\selectfont\catcode`\^=\active\def^{\ifmmode\sp\else\^{}\fi}\catcode`\%=\active\def%{\%}$\mathdefault{1.0}$}}%
\end{pgfscope}%
\begin{pgfscope}%
\pgfpathrectangle{\pgfqpoint{0.652043in}{0.499074in}}{\pgfqpoint{4.960000in}{3.696000in}}%
\pgfusepath{clip}%
\pgfsetbuttcap%
\pgfsetroundjoin%
\pgfsetlinewidth{0.803000pt}%
\definecolor{currentstroke}{rgb}{0.501961,0.501961,0.501961}%
\pgfsetstrokecolor{currentstroke}%
\pgfsetstrokeopacity{0.700000}%
\pgfsetdash{{0.800000pt}{1.320000pt}}{0.000000pt}%
\pgfpathmoveto{\pgfqpoint{0.652043in}{3.953996in}}%
\pgfpathlineto{\pgfqpoint{5.612043in}{3.953996in}}%
\pgfusepath{stroke}%
\end{pgfscope}%
\begin{pgfscope}%
\pgfsetbuttcap%
\pgfsetroundjoin%
\definecolor{currentfill}{rgb}{0.000000,0.000000,0.000000}%
\pgfsetfillcolor{currentfill}%
\pgfsetlinewidth{0.803000pt}%
\definecolor{currentstroke}{rgb}{0.000000,0.000000,0.000000}%
\pgfsetstrokecolor{currentstroke}%
\pgfsetdash{}{0pt}%
\pgfsys@defobject{currentmarker}{\pgfqpoint{-0.048611in}{0.000000in}}{\pgfqpoint{-0.000000in}{0.000000in}}{%
\pgfpathmoveto{\pgfqpoint{-0.000000in}{0.000000in}}%
\pgfpathlineto{\pgfqpoint{-0.048611in}{0.000000in}}%
\pgfusepath{stroke,fill}%
}%
\begin{pgfscope}%
\pgfsys@transformshift{0.652043in}{3.953996in}%
\pgfsys@useobject{currentmarker}{}%
\end{pgfscope}%
\end{pgfscope}%
\begin{pgfscope}%
\definecolor{textcolor}{rgb}{0.000000,0.000000,0.000000}%
\pgfsetstrokecolor{textcolor}%
\pgfsetfillcolor{textcolor}%
\pgftext[x=0.390663in, y=3.910593in, left, base]{\color{textcolor}{\rmfamily\fontsize{9.000000}{10.800000}\selectfont\catcode`\^=\active\def^{\ifmmode\sp\else\^{}\fi}\catcode`\%=\active\def%{\%}$\mathdefault{1.5}$}}%
\end{pgfscope}%
\begin{pgfscope}%
\definecolor{textcolor}{rgb}{0.000000,0.000000,0.000000}%
\pgfsetstrokecolor{textcolor}%
\pgfsetfillcolor{textcolor}%
\pgftext[x=0.235185in,y=2.347074in,,bottom,rotate=90.000000]{\color{textcolor}{\rmfamily\fontsize{11.000000}{13.200000}\selectfont\catcode`\^=\active\def^{\ifmmode\sp\else\^{}\fi}\catcode`\%=\active\def%{\%}}}%
\end{pgfscope}%
\begin{pgfscope}%
\pgfpathrectangle{\pgfqpoint{0.652043in}{0.499074in}}{\pgfqpoint{4.960000in}{3.696000in}}%
\pgfusepath{clip}%
\pgfsetbuttcap%
\pgfsetroundjoin%
\pgfsetlinewidth{1.505625pt}%
\definecolor{currentstroke}{rgb}{1.000000,0.000000,0.000000}%
\pgfsetstrokecolor{currentstroke}%
\pgfsetdash{{1.500000pt}{2.475000pt}}{0.000000pt}%
\pgfpathmoveto{\pgfqpoint{0.877497in}{2.348687in}}%
\pgfpathlineto{\pgfqpoint{1.629013in}{0.667074in}}%
\pgfpathlineto{\pgfqpoint{1.666588in}{0.748640in}}%
\pgfpathlineto{\pgfqpoint{2.230225in}{2.007182in}}%
\pgfpathlineto{\pgfqpoint{2.380528in}{2.343656in}}%
\pgfpathlineto{\pgfqpoint{4.578710in}{2.348601in}}%
\pgfpathlineto{\pgfqpoint{5.386588in}{2.348683in}}%
\pgfpathlineto{\pgfqpoint{5.386588in}{2.348683in}}%
\pgfusepath{stroke}%
\end{pgfscope}%
\begin{pgfscope}%
\pgfpathrectangle{\pgfqpoint{0.652043in}{0.499074in}}{\pgfqpoint{4.960000in}{3.696000in}}%
\pgfusepath{clip}%
\pgfsetbuttcap%
\pgfsetroundjoin%
\pgfsetlinewidth{1.505625pt}%
\definecolor{currentstroke}{rgb}{1.000000,0.000000,0.000000}%
\pgfsetstrokecolor{currentstroke}%
\pgfsetdash{{5.550000pt}{2.400000pt}}{0.000000pt}%
\pgfpathmoveto{\pgfqpoint{0.877497in}{2.348687in}}%
\pgfpathlineto{\pgfqpoint{1.629013in}{4.027074in}}%
\pgfpathlineto{\pgfqpoint{2.211437in}{2.721627in}}%
\pgfpathlineto{\pgfqpoint{2.380528in}{2.343656in}}%
\pgfpathlineto{\pgfqpoint{4.578710in}{2.348601in}}%
\pgfpathlineto{\pgfqpoint{5.386588in}{2.348683in}}%
\pgfpathlineto{\pgfqpoint{5.386588in}{2.348683in}}%
\pgfusepath{stroke}%
\end{pgfscope}%
\begin{pgfscope}%
\pgfpathrectangle{\pgfqpoint{0.652043in}{0.499074in}}{\pgfqpoint{4.960000in}{3.696000in}}%
\pgfusepath{clip}%
\pgfsetrectcap%
\pgfsetroundjoin%
\pgfsetlinewidth{1.505625pt}%
\definecolor{currentstroke}{rgb}{0.121569,0.466667,0.705882}%
\pgfsetstrokecolor{currentstroke}%
\pgfsetdash{}{0pt}%
\pgfpathmoveto{\pgfqpoint{0.877497in}{2.348687in}}%
\pgfpathlineto{\pgfqpoint{0.896285in}{2.340491in}}%
\pgfpathlineto{\pgfqpoint{0.915073in}{2.355274in}}%
\pgfpathlineto{\pgfqpoint{0.952649in}{2.361301in}}%
\pgfpathlineto{\pgfqpoint{1.009013in}{2.369957in}}%
\pgfpathlineto{\pgfqpoint{1.065376in}{2.375963in}}%
\pgfpathlineto{\pgfqpoint{1.121740in}{2.375117in}}%
\pgfpathlineto{\pgfqpoint{1.196891in}{2.370825in}}%
\pgfpathlineto{\pgfqpoint{1.328407in}{2.359829in}}%
\pgfpathlineto{\pgfqpoint{1.459922in}{2.349737in}}%
\pgfpathlineto{\pgfqpoint{1.553861in}{2.344860in}}%
\pgfpathlineto{\pgfqpoint{1.647800in}{2.342251in}}%
\pgfpathlineto{\pgfqpoint{1.760528in}{2.341458in}}%
\pgfpathlineto{\pgfqpoint{2.117497in}{2.343059in}}%
\pgfpathlineto{\pgfqpoint{3.150831in}{2.346419in}}%
\pgfpathlineto{\pgfqpoint{4.052649in}{2.348301in}}%
\pgfpathlineto{\pgfqpoint{5.386588in}{2.348683in}}%
\pgfpathlineto{\pgfqpoint{5.386588in}{2.348683in}}%
\pgfusepath{stroke}%
\end{pgfscope}%
\begin{pgfscope}%
\pgfpathrectangle{\pgfqpoint{0.652043in}{0.499074in}}{\pgfqpoint{4.960000in}{3.696000in}}%
\pgfusepath{clip}%
\pgfsetrectcap%
\pgfsetroundjoin%
\pgfsetlinewidth{1.505625pt}%
\definecolor{currentstroke}{rgb}{1.000000,0.498039,0.054902}%
\pgfsetstrokecolor{currentstroke}%
\pgfsetdash{}{0pt}%
\pgfpathmoveto{\pgfqpoint{0.877497in}{2.348687in}}%
\pgfpathlineto{\pgfqpoint{1.535073in}{2.347677in}}%
\pgfpathlineto{\pgfqpoint{2.380528in}{2.343656in}}%
\pgfpathlineto{\pgfqpoint{4.578710in}{2.348601in}}%
\pgfpathlineto{\pgfqpoint{5.386588in}{2.348683in}}%
\pgfpathlineto{\pgfqpoint{5.386588in}{2.348683in}}%
\pgfusepath{stroke}%
\end{pgfscope}%
\begin{pgfscope}%
\pgfsetrectcap%
\pgfsetmiterjoin%
\pgfsetlinewidth{0.803000pt}%
\definecolor{currentstroke}{rgb}{0.000000,0.000000,0.000000}%
\pgfsetstrokecolor{currentstroke}%
\pgfsetdash{}{0pt}%
\pgfpathmoveto{\pgfqpoint{0.652043in}{0.499074in}}%
\pgfpathlineto{\pgfqpoint{0.652043in}{4.195074in}}%
\pgfusepath{stroke}%
\end{pgfscope}%
\begin{pgfscope}%
\pgfsetrectcap%
\pgfsetmiterjoin%
\pgfsetlinewidth{0.803000pt}%
\definecolor{currentstroke}{rgb}{0.000000,0.000000,0.000000}%
\pgfsetstrokecolor{currentstroke}%
\pgfsetdash{}{0pt}%
\pgfpathmoveto{\pgfqpoint{5.612043in}{0.499074in}}%
\pgfpathlineto{\pgfqpoint{5.612043in}{4.195074in}}%
\pgfusepath{stroke}%
\end{pgfscope}%
\begin{pgfscope}%
\pgfsetrectcap%
\pgfsetmiterjoin%
\pgfsetlinewidth{0.803000pt}%
\definecolor{currentstroke}{rgb}{0.000000,0.000000,0.000000}%
\pgfsetstrokecolor{currentstroke}%
\pgfsetdash{}{0pt}%
\pgfpathmoveto{\pgfqpoint{0.652043in}{0.499074in}}%
\pgfpathlineto{\pgfqpoint{5.612043in}{0.499074in}}%
\pgfusepath{stroke}%
\end{pgfscope}%
\begin{pgfscope}%
\pgfsetrectcap%
\pgfsetmiterjoin%
\pgfsetlinewidth{0.803000pt}%
\definecolor{currentstroke}{rgb}{0.000000,0.000000,0.000000}%
\pgfsetstrokecolor{currentstroke}%
\pgfsetdash{}{0pt}%
\pgfpathmoveto{\pgfqpoint{0.652043in}{4.195074in}}%
\pgfpathlineto{\pgfqpoint{5.612043in}{4.195074in}}%
\pgfusepath{stroke}%
\end{pgfscope}%
\begin{pgfscope}%
\pgfsetbuttcap%
\pgfsetmiterjoin%
\definecolor{currentfill}{rgb}{1.000000,1.000000,1.000000}%
\pgfsetfillcolor{currentfill}%
\pgfsetfillopacity{0.800000}%
\pgfsetlinewidth{1.003750pt}%
\definecolor{currentstroke}{rgb}{0.800000,0.800000,0.800000}%
\pgfsetstrokecolor{currentstroke}%
\pgfsetstrokeopacity{0.800000}%
\pgfsetdash{}{0pt}%
\pgfpathmoveto{\pgfqpoint{3.482410in}{3.115599in}}%
\pgfpathlineto{\pgfqpoint{5.514821in}{3.115599in}}%
\pgfpathquadraticcurveto{\pgfqpoint{5.542598in}{3.115599in}}{\pgfqpoint{5.542598in}{3.143377in}}%
\pgfpathlineto{\pgfqpoint{5.542598in}{4.097852in}}%
\pgfpathquadraticcurveto{\pgfqpoint{5.542598in}{4.125630in}}{\pgfqpoint{5.514821in}{4.125630in}}%
\pgfpathlineto{\pgfqpoint{3.482410in}{4.125630in}}%
\pgfpathquadraticcurveto{\pgfqpoint{3.454632in}{4.125630in}}{\pgfqpoint{3.454632in}{4.097852in}}%
\pgfpathlineto{\pgfqpoint{3.454632in}{3.143377in}}%
\pgfpathquadraticcurveto{\pgfqpoint{3.454632in}{3.115599in}}{\pgfqpoint{3.482410in}{3.115599in}}%
\pgfpathlineto{\pgfqpoint{3.482410in}{3.115599in}}%
\pgfpathclose%
\pgfusepath{stroke,fill}%
\end{pgfscope}%
\begin{pgfscope}%
\pgfsetbuttcap%
\pgfsetmiterjoin%
\definecolor{currentfill}{rgb}{0.000000,0.501961,0.000000}%
\pgfsetfillcolor{currentfill}%
\pgfsetfillopacity{0.200000}%
\pgfsetlinewidth{1.003750pt}%
\definecolor{currentstroke}{rgb}{0.000000,0.501961,0.000000}%
\pgfsetstrokecolor{currentstroke}%
\pgfsetstrokeopacity{0.200000}%
\pgfsetdash{}{0pt}%
\pgfpathmoveto{\pgfqpoint{3.510188in}{3.972852in}}%
\pgfpathlineto{\pgfqpoint{3.787966in}{3.972852in}}%
\pgfpathlineto{\pgfqpoint{3.787966in}{4.070074in}}%
\pgfpathlineto{\pgfqpoint{3.510188in}{4.070074in}}%
\pgfpathlineto{\pgfqpoint{3.510188in}{3.972852in}}%
\pgfpathclose%
\pgfusepath{stroke,fill}%
\end{pgfscope}%
\begin{pgfscope}%
\definecolor{textcolor}{rgb}{0.000000,0.000000,0.000000}%
\pgfsetstrokecolor{textcolor}%
\pgfsetfillcolor{textcolor}%
\pgftext[x=3.899077in,y=3.972852in,left,base]{\color{textcolor}{\rmfamily\fontsize{10.000000}{12.000000}\selectfont\catcode`\^=\active\def^{\ifmmode\sp\else\^{}\fi}\catcode`\%=\active\def%{\%}Bounds}}%
\end{pgfscope}%
\begin{pgfscope}%
\pgfsetbuttcap%
\pgfsetroundjoin%
\pgfsetlinewidth{1.505625pt}%
\definecolor{currentstroke}{rgb}{1.000000,0.000000,0.000000}%
\pgfsetstrokecolor{currentstroke}%
\pgfsetdash{{1.500000pt}{2.475000pt}}{0.000000pt}%
\pgfpathmoveto{\pgfqpoint{3.510188in}{3.827790in}}%
\pgfpathlineto{\pgfqpoint{3.649077in}{3.827790in}}%
\pgfpathlineto{\pgfqpoint{3.787966in}{3.827790in}}%
\pgfusepath{stroke}%
\end{pgfscope}%
\begin{pgfscope}%
\definecolor{textcolor}{rgb}{0.000000,0.000000,0.000000}%
\pgfsetstrokecolor{textcolor}%
\pgfsetfillcolor{textcolor}%
\pgftext[x=3.899077in,y=3.779179in,left,base]{\color{textcolor}{\rmfamily\fontsize{10.000000}{12.000000}\selectfont\catcode`\^=\active\def^{\ifmmode\sp\else\^{}\fi}\catcode`\%=\active\def%{\%}Lower Bound}}%
\end{pgfscope}%
\begin{pgfscope}%
\pgfsetbuttcap%
\pgfsetroundjoin%
\pgfsetlinewidth{1.505625pt}%
\definecolor{currentstroke}{rgb}{1.000000,0.000000,0.000000}%
\pgfsetstrokecolor{currentstroke}%
\pgfsetdash{{5.550000pt}{2.400000pt}}{0.000000pt}%
\pgfpathmoveto{\pgfqpoint{3.510188in}{3.634117in}}%
\pgfpathlineto{\pgfqpoint{3.649077in}{3.634117in}}%
\pgfpathlineto{\pgfqpoint{3.787966in}{3.634117in}}%
\pgfusepath{stroke}%
\end{pgfscope}%
\begin{pgfscope}%
\definecolor{textcolor}{rgb}{0.000000,0.000000,0.000000}%
\pgfsetstrokecolor{textcolor}%
\pgfsetfillcolor{textcolor}%
\pgftext[x=3.899077in,y=3.585506in,left,base]{\color{textcolor}{\rmfamily\fontsize{10.000000}{12.000000}\selectfont\catcode`\^=\active\def^{\ifmmode\sp\else\^{}\fi}\catcode`\%=\active\def%{\%}Upper Bound}}%
\end{pgfscope}%
\begin{pgfscope}%
\pgfsetrectcap%
\pgfsetroundjoin%
\pgfsetlinewidth{1.505625pt}%
\definecolor{currentstroke}{rgb}{0.121569,0.466667,0.705882}%
\pgfsetstrokecolor{currentstroke}%
\pgfsetdash{}{0pt}%
\pgfpathmoveto{\pgfqpoint{3.510188in}{3.440445in}}%
\pgfpathlineto{\pgfqpoint{3.649077in}{3.440445in}}%
\pgfpathlineto{\pgfqpoint{3.787966in}{3.440445in}}%
\pgfusepath{stroke}%
\end{pgfscope}%
\begin{pgfscope}%
\definecolor{textcolor}{rgb}{0.000000,0.000000,0.000000}%
\pgfsetstrokecolor{textcolor}%
\pgfsetfillcolor{textcolor}%
\pgftext[x=3.899077in,y=3.391833in,left,base]{\color{textcolor}{\rmfamily\fontsize{10.000000}{12.000000}\selectfont\catcode`\^=\active\def^{\ifmmode\sp\else\^{}\fi}\catcode`\%=\active\def%{\%}Relaxation Variable Value}}%
\end{pgfscope}%
\begin{pgfscope}%
\pgfsetrectcap%
\pgfsetroundjoin%
\pgfsetlinewidth{1.505625pt}%
\definecolor{currentstroke}{rgb}{1.000000,0.498039,0.054902}%
\pgfsetstrokecolor{currentstroke}%
\pgfsetdash{}{0pt}%
\pgfpathmoveto{\pgfqpoint{3.510188in}{3.246772in}}%
\pgfpathlineto{\pgfqpoint{3.649077in}{3.246772in}}%
\pgfpathlineto{\pgfqpoint{3.787966in}{3.246772in}}%
\pgfusepath{stroke}%
\end{pgfscope}%
\begin{pgfscope}%
\definecolor{textcolor}{rgb}{0.000000,0.000000,0.000000}%
\pgfsetstrokecolor{textcolor}%
\pgfsetfillcolor{textcolor}%
\pgftext[x=3.899077in,y=3.198161in,left,base]{\color{textcolor}{\rmfamily\fontsize{10.000000}{12.000000}\selectfont\catcode`\^=\active\def^{\ifmmode\sp\else\^{}\fi}\catcode`\%=\active\def%{\%}Actual Bilinear Value}}%
\end{pgfscope}%
\end{pgfpicture}%
\makeatother%
\endgroup%
}
	\caption{McCormick Relaxation on $v\xi$.}
	\label{fig:dn-term-approx}
\end{figure}

Figure \ref{fig:dn-term-approx} compares the actual bilinear value \( v\xi \) with the relaxation variable \( w \) introduced via McCormick
envelopes.
This comparison highlights the accuracy of the relaxation approach in approximating the bilinear interaction and its effect on the state transition
of \( n \).
Once the velocity reaches its limit, the approximation becomes increasingly accurate.
\pagebreak
\subsubsection{Challenges with McCormick Relaxations}

McCormick performs poorly when the velocity is at the midpoint of its limits.
Since we decouple the lateral movement from the steering angle and velocity using McCormick relaxations, it is possible for the vehicle to move
laterally without any longitudinal movement.
To prevent this, we use an approach that allows for tighter bounds on the velocity by considering the maximum and minimum acceleration.
Figure \ref{fig:mccormick_problem} illustrates the McCormick relaxation without tighter bounds, while Figure \ref{fig:mccormick_problem_better}
demonstrates the improvement with tighter bounds.

\begin{figure}[h]
	\centering
	\begin{subfigure}[b]{0.45\textwidth}
		\centering
		\resizebox{\textwidth}{!}{%% Creator: Matplotlib, PGF backend
%%
%% To include the figure in your LaTeX document, write
%%   \input{<filename>.pgf}
%%
%% Make sure the required packages are loaded in your preamble
%%   \usepackage{pgf}
%%
%% Also ensure that all the required font packages are loaded; for instance,
%% the lmodern package is sometimes necessary when using math font.
%%   \usepackage{lmodern}
%%
%% Figures using additional raster images can only be included by \input if
%% they are in the same directory as the main LaTeX file. For loading figures
%% from other directories you can use the `import` package
%%   \usepackage{import}
%%
%% and then include the figures with
%%   \import{<path to file>}{<filename>.pgf}
%%
%% Matplotlib used the following preamble
%%   \def\mathdefault#1{#1}
%%   \everymath=\expandafter{\the\everymath\displaystyle}
%%   
%%   \ifdefined\pdftexversion\else  % non-pdftex case.
%%     \usepackage{fontspec}
%%   \fi
%%   \makeatletter\@ifpackageloaded{underscore}{}{\usepackage[strings]{underscore}}\makeatother
%%
\begingroup%
\makeatletter%
\begin{pgfpicture}%
\pgfpathrectangle{\pgfpointorigin}{\pgfqpoint{5.612121in}{4.295074in}}%
\pgfusepath{use as bounding box, clip}%
\begin{pgfscope}%
\pgfsetbuttcap%
\pgfsetmiterjoin%
\definecolor{currentfill}{rgb}{1.000000,1.000000,1.000000}%
\pgfsetfillcolor{currentfill}%
\pgfsetlinewidth{0.000000pt}%
\definecolor{currentstroke}{rgb}{1.000000,1.000000,1.000000}%
\pgfsetstrokecolor{currentstroke}%
\pgfsetdash{}{0pt}%
\pgfpathmoveto{\pgfqpoint{0.000000in}{0.000000in}}%
\pgfpathlineto{\pgfqpoint{5.612121in}{0.000000in}}%
\pgfpathlineto{\pgfqpoint{5.612121in}{4.295074in}}%
\pgfpathlineto{\pgfqpoint{0.000000in}{4.295074in}}%
\pgfpathlineto{\pgfqpoint{0.000000in}{0.000000in}}%
\pgfpathclose%
\pgfusepath{fill}%
\end{pgfscope}%
\begin{pgfscope}%
\pgfsetbuttcap%
\pgfsetmiterjoin%
\definecolor{currentfill}{rgb}{1.000000,1.000000,1.000000}%
\pgfsetfillcolor{currentfill}%
\pgfsetlinewidth{0.000000pt}%
\definecolor{currentstroke}{rgb}{0.000000,0.000000,0.000000}%
\pgfsetstrokecolor{currentstroke}%
\pgfsetstrokeopacity{0.000000}%
\pgfsetdash{}{0pt}%
\pgfpathmoveto{\pgfqpoint{0.552121in}{0.499074in}}%
\pgfpathlineto{\pgfqpoint{5.512121in}{0.499074in}}%
\pgfpathlineto{\pgfqpoint{5.512121in}{4.195074in}}%
\pgfpathlineto{\pgfqpoint{0.552121in}{4.195074in}}%
\pgfpathlineto{\pgfqpoint{0.552121in}{0.499074in}}%
\pgfpathclose%
\pgfusepath{fill}%
\end{pgfscope}%
\begin{pgfscope}%
\pgfpathrectangle{\pgfqpoint{0.552121in}{0.499074in}}{\pgfqpoint{4.960000in}{3.696000in}}%
\pgfusepath{clip}%
\pgfsetbuttcap%
\pgfsetroundjoin%
\definecolor{currentfill}{rgb}{0.000000,0.501961,0.000000}%
\pgfsetfillcolor{currentfill}%
\pgfsetfillopacity{0.200000}%
\pgfsetlinewidth{1.003750pt}%
\definecolor{currentstroke}{rgb}{0.000000,0.501961,0.000000}%
\pgfsetstrokecolor{currentstroke}%
\pgfsetstrokeopacity{0.200000}%
\pgfsetdash{}{0pt}%
\pgfsys@defobject{currentmarker}{\pgfqpoint{0.777575in}{0.667074in}}{\pgfqpoint{5.286666in}{4.027074in}}{%
\pgfpathmoveto{\pgfqpoint{0.777575in}{4.021068in}}%
\pgfpathlineto{\pgfqpoint{0.777575in}{0.667074in}}%
\pgfpathlineto{\pgfqpoint{0.852727in}{0.667078in}}%
\pgfpathlineto{\pgfqpoint{0.927878in}{0.667085in}}%
\pgfpathlineto{\pgfqpoint{1.003030in}{0.667094in}}%
\pgfpathlineto{\pgfqpoint{1.078181in}{0.667105in}}%
\pgfpathlineto{\pgfqpoint{1.153333in}{0.667117in}}%
\pgfpathlineto{\pgfqpoint{1.228484in}{0.667131in}}%
\pgfpathlineto{\pgfqpoint{1.303636in}{0.667149in}}%
\pgfpathlineto{\pgfqpoint{1.378788in}{0.667173in}}%
\pgfpathlineto{\pgfqpoint{1.453939in}{0.667207in}}%
\pgfpathlineto{\pgfqpoint{1.529091in}{0.667253in}}%
\pgfpathlineto{\pgfqpoint{1.604242in}{0.667316in}}%
\pgfpathlineto{\pgfqpoint{1.679394in}{0.667398in}}%
\pgfpathlineto{\pgfqpoint{1.754545in}{0.667502in}}%
\pgfpathlineto{\pgfqpoint{1.829697in}{0.667633in}}%
\pgfpathlineto{\pgfqpoint{1.904848in}{0.667792in}}%
\pgfpathlineto{\pgfqpoint{1.980000in}{0.667983in}}%
\pgfpathlineto{\pgfqpoint{2.055151in}{0.668197in}}%
\pgfpathlineto{\pgfqpoint{2.130303in}{0.668407in}}%
\pgfpathlineto{\pgfqpoint{2.205454in}{0.668614in}}%
\pgfpathlineto{\pgfqpoint{2.280606in}{0.668816in}}%
\pgfpathlineto{\pgfqpoint{2.355757in}{0.669015in}}%
\pgfpathlineto{\pgfqpoint{2.430909in}{0.669210in}}%
\pgfpathlineto{\pgfqpoint{2.506060in}{0.669402in}}%
\pgfpathlineto{\pgfqpoint{2.581212in}{0.669589in}}%
\pgfpathlineto{\pgfqpoint{2.656363in}{0.669773in}}%
\pgfpathlineto{\pgfqpoint{2.731515in}{0.669952in}}%
\pgfpathlineto{\pgfqpoint{2.806666in}{0.670128in}}%
\pgfpathlineto{\pgfqpoint{2.881818in}{0.670299in}}%
\pgfpathlineto{\pgfqpoint{2.956969in}{0.670467in}}%
\pgfpathlineto{\pgfqpoint{3.032121in}{0.670630in}}%
\pgfpathlineto{\pgfqpoint{3.107272in}{0.670789in}}%
\pgfpathlineto{\pgfqpoint{3.182424in}{0.670944in}}%
\pgfpathlineto{\pgfqpoint{3.257575in}{0.671094in}}%
\pgfpathlineto{\pgfqpoint{3.332727in}{0.671239in}}%
\pgfpathlineto{\pgfqpoint{3.407878in}{0.671380in}}%
\pgfpathlineto{\pgfqpoint{3.483030in}{0.671517in}}%
\pgfpathlineto{\pgfqpoint{3.558181in}{0.671648in}}%
\pgfpathlineto{\pgfqpoint{3.633333in}{0.671774in}}%
\pgfpathlineto{\pgfqpoint{3.708484in}{0.671895in}}%
\pgfpathlineto{\pgfqpoint{3.783636in}{0.672011in}}%
\pgfpathlineto{\pgfqpoint{3.858788in}{0.672122in}}%
\pgfpathlineto{\pgfqpoint{3.933939in}{0.672227in}}%
\pgfpathlineto{\pgfqpoint{4.009091in}{0.672326in}}%
\pgfpathlineto{\pgfqpoint{4.084242in}{0.672420in}}%
\pgfpathlineto{\pgfqpoint{4.159394in}{0.672507in}}%
\pgfpathlineto{\pgfqpoint{4.234545in}{0.672589in}}%
\pgfpathlineto{\pgfqpoint{4.309697in}{0.672665in}}%
\pgfpathlineto{\pgfqpoint{4.384848in}{0.672735in}}%
\pgfpathlineto{\pgfqpoint{4.460000in}{0.672798in}}%
\pgfpathlineto{\pgfqpoint{4.535151in}{0.672855in}}%
\pgfpathlineto{\pgfqpoint{4.610303in}{0.672906in}}%
\pgfpathlineto{\pgfqpoint{4.685454in}{0.672951in}}%
\pgfpathlineto{\pgfqpoint{4.760606in}{0.672990in}}%
\pgfpathlineto{\pgfqpoint{4.835757in}{0.673022in}}%
\pgfpathlineto{\pgfqpoint{4.910909in}{0.673048in}}%
\pgfpathlineto{\pgfqpoint{4.986060in}{0.673068in}}%
\pgfpathlineto{\pgfqpoint{5.061212in}{0.673082in}}%
\pgfpathlineto{\pgfqpoint{5.136363in}{0.673090in}}%
\pgfpathlineto{\pgfqpoint{5.211515in}{0.673092in}}%
\pgfpathlineto{\pgfqpoint{5.286666in}{0.673089in}}%
\pgfpathlineto{\pgfqpoint{5.286666in}{4.027071in}}%
\pgfpathlineto{\pgfqpoint{5.286666in}{4.027071in}}%
\pgfpathlineto{\pgfqpoint{5.211515in}{4.027074in}}%
\pgfpathlineto{\pgfqpoint{5.136363in}{4.027072in}}%
\pgfpathlineto{\pgfqpoint{5.061212in}{4.027064in}}%
\pgfpathlineto{\pgfqpoint{4.986060in}{4.027050in}}%
\pgfpathlineto{\pgfqpoint{4.910909in}{4.027030in}}%
\pgfpathlineto{\pgfqpoint{4.835757in}{4.027003in}}%
\pgfpathlineto{\pgfqpoint{4.760606in}{4.026971in}}%
\pgfpathlineto{\pgfqpoint{4.685454in}{4.026932in}}%
\pgfpathlineto{\pgfqpoint{4.610303in}{4.026887in}}%
\pgfpathlineto{\pgfqpoint{4.535151in}{4.026836in}}%
\pgfpathlineto{\pgfqpoint{4.460000in}{4.026779in}}%
\pgfpathlineto{\pgfqpoint{4.384848in}{4.026715in}}%
\pgfpathlineto{\pgfqpoint{4.309697in}{4.026645in}}%
\pgfpathlineto{\pgfqpoint{4.234545in}{4.026570in}}%
\pgfpathlineto{\pgfqpoint{4.159394in}{4.026488in}}%
\pgfpathlineto{\pgfqpoint{4.084242in}{4.026400in}}%
\pgfpathlineto{\pgfqpoint{4.009091in}{4.026306in}}%
\pgfpathlineto{\pgfqpoint{3.933939in}{4.026207in}}%
\pgfpathlineto{\pgfqpoint{3.858788in}{4.026102in}}%
\pgfpathlineto{\pgfqpoint{3.783636in}{4.025991in}}%
\pgfpathlineto{\pgfqpoint{3.708484in}{4.025875in}}%
\pgfpathlineto{\pgfqpoint{3.633333in}{4.025754in}}%
\pgfpathlineto{\pgfqpoint{3.558181in}{4.025628in}}%
\pgfpathlineto{\pgfqpoint{3.483030in}{4.025496in}}%
\pgfpathlineto{\pgfqpoint{3.407878in}{4.025360in}}%
\pgfpathlineto{\pgfqpoint{3.332727in}{4.025219in}}%
\pgfpathlineto{\pgfqpoint{3.257575in}{4.025074in}}%
\pgfpathlineto{\pgfqpoint{3.182424in}{4.024923in}}%
\pgfpathlineto{\pgfqpoint{3.107272in}{4.024769in}}%
\pgfpathlineto{\pgfqpoint{3.032121in}{4.024610in}}%
\pgfpathlineto{\pgfqpoint{2.956969in}{4.024447in}}%
\pgfpathlineto{\pgfqpoint{2.881818in}{4.024279in}}%
\pgfpathlineto{\pgfqpoint{2.806666in}{4.024108in}}%
\pgfpathlineto{\pgfqpoint{2.731515in}{4.023933in}}%
\pgfpathlineto{\pgfqpoint{2.656363in}{4.023753in}}%
\pgfpathlineto{\pgfqpoint{2.581212in}{4.023570in}}%
\pgfpathlineto{\pgfqpoint{2.506060in}{4.023383in}}%
\pgfpathlineto{\pgfqpoint{2.430909in}{4.023192in}}%
\pgfpathlineto{\pgfqpoint{2.355757in}{4.022997in}}%
\pgfpathlineto{\pgfqpoint{2.280606in}{4.022798in}}%
\pgfpathlineto{\pgfqpoint{2.205454in}{4.022596in}}%
\pgfpathlineto{\pgfqpoint{2.130303in}{4.022390in}}%
\pgfpathlineto{\pgfqpoint{2.055151in}{4.022180in}}%
\pgfpathlineto{\pgfqpoint{1.980000in}{4.021966in}}%
\pgfpathlineto{\pgfqpoint{1.904848in}{4.021775in}}%
\pgfpathlineto{\pgfqpoint{1.829697in}{4.021616in}}%
\pgfpathlineto{\pgfqpoint{1.754545in}{4.021485in}}%
\pgfpathlineto{\pgfqpoint{1.679394in}{4.021381in}}%
\pgfpathlineto{\pgfqpoint{1.604242in}{4.021299in}}%
\pgfpathlineto{\pgfqpoint{1.529091in}{4.021236in}}%
\pgfpathlineto{\pgfqpoint{1.453939in}{4.021190in}}%
\pgfpathlineto{\pgfqpoint{1.378788in}{4.021156in}}%
\pgfpathlineto{\pgfqpoint{1.303636in}{4.021132in}}%
\pgfpathlineto{\pgfqpoint{1.228484in}{4.021114in}}%
\pgfpathlineto{\pgfqpoint{1.153333in}{4.021101in}}%
\pgfpathlineto{\pgfqpoint{1.078181in}{4.021089in}}%
\pgfpathlineto{\pgfqpoint{1.003030in}{4.021079in}}%
\pgfpathlineto{\pgfqpoint{0.927878in}{4.021070in}}%
\pgfpathlineto{\pgfqpoint{0.852727in}{4.021063in}}%
\pgfpathlineto{\pgfqpoint{0.777575in}{4.021068in}}%
\pgfpathlineto{\pgfqpoint{0.777575in}{4.021068in}}%
\pgfpathclose%
\pgfusepath{stroke,fill}%
}%
\begin{pgfscope}%
\pgfsys@transformshift{0.000000in}{0.000000in}%
\pgfsys@useobject{currentmarker}{}%
\end{pgfscope}%
\end{pgfscope}%
\begin{pgfscope}%
\pgfpathrectangle{\pgfqpoint{0.552121in}{0.499074in}}{\pgfqpoint{4.960000in}{3.696000in}}%
\pgfusepath{clip}%
\pgfsetbuttcap%
\pgfsetroundjoin%
\pgfsetlinewidth{0.803000pt}%
\definecolor{currentstroke}{rgb}{0.501961,0.501961,0.501961}%
\pgfsetstrokecolor{currentstroke}%
\pgfsetstrokeopacity{0.700000}%
\pgfsetdash{{0.800000pt}{1.320000pt}}{0.000000pt}%
\pgfpathmoveto{\pgfqpoint{0.777575in}{0.499074in}}%
\pgfpathlineto{\pgfqpoint{0.777575in}{4.195074in}}%
\pgfusepath{stroke}%
\end{pgfscope}%
\begin{pgfscope}%
\pgfsetbuttcap%
\pgfsetroundjoin%
\definecolor{currentfill}{rgb}{0.000000,0.000000,0.000000}%
\pgfsetfillcolor{currentfill}%
\pgfsetlinewidth{0.803000pt}%
\definecolor{currentstroke}{rgb}{0.000000,0.000000,0.000000}%
\pgfsetstrokecolor{currentstroke}%
\pgfsetdash{}{0pt}%
\pgfsys@defobject{currentmarker}{\pgfqpoint{0.000000in}{-0.048611in}}{\pgfqpoint{0.000000in}{0.000000in}}{%
\pgfpathmoveto{\pgfqpoint{0.000000in}{0.000000in}}%
\pgfpathlineto{\pgfqpoint{0.000000in}{-0.048611in}}%
\pgfusepath{stroke,fill}%
}%
\begin{pgfscope}%
\pgfsys@transformshift{0.777575in}{0.499074in}%
\pgfsys@useobject{currentmarker}{}%
\end{pgfscope}%
\end{pgfscope}%
\begin{pgfscope}%
\definecolor{textcolor}{rgb}{0.000000,0.000000,0.000000}%
\pgfsetstrokecolor{textcolor}%
\pgfsetfillcolor{textcolor}%
\pgftext[x=0.777575in,y=0.401852in,,top]{\color{textcolor}{\rmfamily\fontsize{9.000000}{10.800000}\selectfont\catcode`\^=\active\def^{\ifmmode\sp\else\^{}\fi}\catcode`\%=\active\def%{\%}$\mathdefault{0.0}$}}%
\end{pgfscope}%
\begin{pgfscope}%
\pgfpathrectangle{\pgfqpoint{0.552121in}{0.499074in}}{\pgfqpoint{4.960000in}{3.696000in}}%
\pgfusepath{clip}%
\pgfsetbuttcap%
\pgfsetroundjoin%
\pgfsetlinewidth{0.803000pt}%
\definecolor{currentstroke}{rgb}{0.501961,0.501961,0.501961}%
\pgfsetstrokecolor{currentstroke}%
\pgfsetstrokeopacity{0.700000}%
\pgfsetdash{{0.800000pt}{1.320000pt}}{0.000000pt}%
\pgfpathmoveto{\pgfqpoint{1.679394in}{0.499074in}}%
\pgfpathlineto{\pgfqpoint{1.679394in}{4.195074in}}%
\pgfusepath{stroke}%
\end{pgfscope}%
\begin{pgfscope}%
\pgfsetbuttcap%
\pgfsetroundjoin%
\definecolor{currentfill}{rgb}{0.000000,0.000000,0.000000}%
\pgfsetfillcolor{currentfill}%
\pgfsetlinewidth{0.803000pt}%
\definecolor{currentstroke}{rgb}{0.000000,0.000000,0.000000}%
\pgfsetstrokecolor{currentstroke}%
\pgfsetdash{}{0pt}%
\pgfsys@defobject{currentmarker}{\pgfqpoint{0.000000in}{-0.048611in}}{\pgfqpoint{0.000000in}{0.000000in}}{%
\pgfpathmoveto{\pgfqpoint{0.000000in}{0.000000in}}%
\pgfpathlineto{\pgfqpoint{0.000000in}{-0.048611in}}%
\pgfusepath{stroke,fill}%
}%
\begin{pgfscope}%
\pgfsys@transformshift{1.679394in}{0.499074in}%
\pgfsys@useobject{currentmarker}{}%
\end{pgfscope}%
\end{pgfscope}%
\begin{pgfscope}%
\definecolor{textcolor}{rgb}{0.000000,0.000000,0.000000}%
\pgfsetstrokecolor{textcolor}%
\pgfsetfillcolor{textcolor}%
\pgftext[x=1.679394in,y=0.401852in,,top]{\color{textcolor}{\rmfamily\fontsize{9.000000}{10.800000}\selectfont\catcode`\^=\active\def^{\ifmmode\sp\else\^{}\fi}\catcode`\%=\active\def%{\%}$\mathdefault{0.2}$}}%
\end{pgfscope}%
\begin{pgfscope}%
\pgfpathrectangle{\pgfqpoint{0.552121in}{0.499074in}}{\pgfqpoint{4.960000in}{3.696000in}}%
\pgfusepath{clip}%
\pgfsetbuttcap%
\pgfsetroundjoin%
\pgfsetlinewidth{0.803000pt}%
\definecolor{currentstroke}{rgb}{0.501961,0.501961,0.501961}%
\pgfsetstrokecolor{currentstroke}%
\pgfsetstrokeopacity{0.700000}%
\pgfsetdash{{0.800000pt}{1.320000pt}}{0.000000pt}%
\pgfpathmoveto{\pgfqpoint{2.581212in}{0.499074in}}%
\pgfpathlineto{\pgfqpoint{2.581212in}{4.195074in}}%
\pgfusepath{stroke}%
\end{pgfscope}%
\begin{pgfscope}%
\pgfsetbuttcap%
\pgfsetroundjoin%
\definecolor{currentfill}{rgb}{0.000000,0.000000,0.000000}%
\pgfsetfillcolor{currentfill}%
\pgfsetlinewidth{0.803000pt}%
\definecolor{currentstroke}{rgb}{0.000000,0.000000,0.000000}%
\pgfsetstrokecolor{currentstroke}%
\pgfsetdash{}{0pt}%
\pgfsys@defobject{currentmarker}{\pgfqpoint{0.000000in}{-0.048611in}}{\pgfqpoint{0.000000in}{0.000000in}}{%
\pgfpathmoveto{\pgfqpoint{0.000000in}{0.000000in}}%
\pgfpathlineto{\pgfqpoint{0.000000in}{-0.048611in}}%
\pgfusepath{stroke,fill}%
}%
\begin{pgfscope}%
\pgfsys@transformshift{2.581212in}{0.499074in}%
\pgfsys@useobject{currentmarker}{}%
\end{pgfscope}%
\end{pgfscope}%
\begin{pgfscope}%
\definecolor{textcolor}{rgb}{0.000000,0.000000,0.000000}%
\pgfsetstrokecolor{textcolor}%
\pgfsetfillcolor{textcolor}%
\pgftext[x=2.581212in,y=0.401852in,,top]{\color{textcolor}{\rmfamily\fontsize{9.000000}{10.800000}\selectfont\catcode`\^=\active\def^{\ifmmode\sp\else\^{}\fi}\catcode`\%=\active\def%{\%}$\mathdefault{0.4}$}}%
\end{pgfscope}%
\begin{pgfscope}%
\pgfpathrectangle{\pgfqpoint{0.552121in}{0.499074in}}{\pgfqpoint{4.960000in}{3.696000in}}%
\pgfusepath{clip}%
\pgfsetbuttcap%
\pgfsetroundjoin%
\pgfsetlinewidth{0.803000pt}%
\definecolor{currentstroke}{rgb}{0.501961,0.501961,0.501961}%
\pgfsetstrokecolor{currentstroke}%
\pgfsetstrokeopacity{0.700000}%
\pgfsetdash{{0.800000pt}{1.320000pt}}{0.000000pt}%
\pgfpathmoveto{\pgfqpoint{3.483030in}{0.499074in}}%
\pgfpathlineto{\pgfqpoint{3.483030in}{4.195074in}}%
\pgfusepath{stroke}%
\end{pgfscope}%
\begin{pgfscope}%
\pgfsetbuttcap%
\pgfsetroundjoin%
\definecolor{currentfill}{rgb}{0.000000,0.000000,0.000000}%
\pgfsetfillcolor{currentfill}%
\pgfsetlinewidth{0.803000pt}%
\definecolor{currentstroke}{rgb}{0.000000,0.000000,0.000000}%
\pgfsetstrokecolor{currentstroke}%
\pgfsetdash{}{0pt}%
\pgfsys@defobject{currentmarker}{\pgfqpoint{0.000000in}{-0.048611in}}{\pgfqpoint{0.000000in}{0.000000in}}{%
\pgfpathmoveto{\pgfqpoint{0.000000in}{0.000000in}}%
\pgfpathlineto{\pgfqpoint{0.000000in}{-0.048611in}}%
\pgfusepath{stroke,fill}%
}%
\begin{pgfscope}%
\pgfsys@transformshift{3.483030in}{0.499074in}%
\pgfsys@useobject{currentmarker}{}%
\end{pgfscope}%
\end{pgfscope}%
\begin{pgfscope}%
\definecolor{textcolor}{rgb}{0.000000,0.000000,0.000000}%
\pgfsetstrokecolor{textcolor}%
\pgfsetfillcolor{textcolor}%
\pgftext[x=3.483030in,y=0.401852in,,top]{\color{textcolor}{\rmfamily\fontsize{9.000000}{10.800000}\selectfont\catcode`\^=\active\def^{\ifmmode\sp\else\^{}\fi}\catcode`\%=\active\def%{\%}$\mathdefault{0.6}$}}%
\end{pgfscope}%
\begin{pgfscope}%
\pgfpathrectangle{\pgfqpoint{0.552121in}{0.499074in}}{\pgfqpoint{4.960000in}{3.696000in}}%
\pgfusepath{clip}%
\pgfsetbuttcap%
\pgfsetroundjoin%
\pgfsetlinewidth{0.803000pt}%
\definecolor{currentstroke}{rgb}{0.501961,0.501961,0.501961}%
\pgfsetstrokecolor{currentstroke}%
\pgfsetstrokeopacity{0.700000}%
\pgfsetdash{{0.800000pt}{1.320000pt}}{0.000000pt}%
\pgfpathmoveto{\pgfqpoint{4.384848in}{0.499074in}}%
\pgfpathlineto{\pgfqpoint{4.384848in}{4.195074in}}%
\pgfusepath{stroke}%
\end{pgfscope}%
\begin{pgfscope}%
\pgfsetbuttcap%
\pgfsetroundjoin%
\definecolor{currentfill}{rgb}{0.000000,0.000000,0.000000}%
\pgfsetfillcolor{currentfill}%
\pgfsetlinewidth{0.803000pt}%
\definecolor{currentstroke}{rgb}{0.000000,0.000000,0.000000}%
\pgfsetstrokecolor{currentstroke}%
\pgfsetdash{}{0pt}%
\pgfsys@defobject{currentmarker}{\pgfqpoint{0.000000in}{-0.048611in}}{\pgfqpoint{0.000000in}{0.000000in}}{%
\pgfpathmoveto{\pgfqpoint{0.000000in}{0.000000in}}%
\pgfpathlineto{\pgfqpoint{0.000000in}{-0.048611in}}%
\pgfusepath{stroke,fill}%
}%
\begin{pgfscope}%
\pgfsys@transformshift{4.384848in}{0.499074in}%
\pgfsys@useobject{currentmarker}{}%
\end{pgfscope}%
\end{pgfscope}%
\begin{pgfscope}%
\definecolor{textcolor}{rgb}{0.000000,0.000000,0.000000}%
\pgfsetstrokecolor{textcolor}%
\pgfsetfillcolor{textcolor}%
\pgftext[x=4.384848in,y=0.401852in,,top]{\color{textcolor}{\rmfamily\fontsize{9.000000}{10.800000}\selectfont\catcode`\^=\active\def^{\ifmmode\sp\else\^{}\fi}\catcode`\%=\active\def%{\%}$\mathdefault{0.8}$}}%
\end{pgfscope}%
\begin{pgfscope}%
\pgfpathrectangle{\pgfqpoint{0.552121in}{0.499074in}}{\pgfqpoint{4.960000in}{3.696000in}}%
\pgfusepath{clip}%
\pgfsetbuttcap%
\pgfsetroundjoin%
\pgfsetlinewidth{0.803000pt}%
\definecolor{currentstroke}{rgb}{0.501961,0.501961,0.501961}%
\pgfsetstrokecolor{currentstroke}%
\pgfsetstrokeopacity{0.700000}%
\pgfsetdash{{0.800000pt}{1.320000pt}}{0.000000pt}%
\pgfpathmoveto{\pgfqpoint{5.286666in}{0.499074in}}%
\pgfpathlineto{\pgfqpoint{5.286666in}{4.195074in}}%
\pgfusepath{stroke}%
\end{pgfscope}%
\begin{pgfscope}%
\pgfsetbuttcap%
\pgfsetroundjoin%
\definecolor{currentfill}{rgb}{0.000000,0.000000,0.000000}%
\pgfsetfillcolor{currentfill}%
\pgfsetlinewidth{0.803000pt}%
\definecolor{currentstroke}{rgb}{0.000000,0.000000,0.000000}%
\pgfsetstrokecolor{currentstroke}%
\pgfsetdash{}{0pt}%
\pgfsys@defobject{currentmarker}{\pgfqpoint{0.000000in}{-0.048611in}}{\pgfqpoint{0.000000in}{0.000000in}}{%
\pgfpathmoveto{\pgfqpoint{0.000000in}{0.000000in}}%
\pgfpathlineto{\pgfqpoint{0.000000in}{-0.048611in}}%
\pgfusepath{stroke,fill}%
}%
\begin{pgfscope}%
\pgfsys@transformshift{5.286666in}{0.499074in}%
\pgfsys@useobject{currentmarker}{}%
\end{pgfscope}%
\end{pgfscope}%
\begin{pgfscope}%
\definecolor{textcolor}{rgb}{0.000000,0.000000,0.000000}%
\pgfsetstrokecolor{textcolor}%
\pgfsetfillcolor{textcolor}%
\pgftext[x=5.286666in,y=0.401852in,,top]{\color{textcolor}{\rmfamily\fontsize{9.000000}{10.800000}\selectfont\catcode`\^=\active\def^{\ifmmode\sp\else\^{}\fi}\catcode`\%=\active\def%{\%}$\mathdefault{1.0}$}}%
\end{pgfscope}%
\begin{pgfscope}%
\definecolor{textcolor}{rgb}{0.000000,0.000000,0.000000}%
\pgfsetstrokecolor{textcolor}%
\pgfsetfillcolor{textcolor}%
\pgftext[x=3.032121in,y=0.235185in,,top]{\color{textcolor}{\rmfamily\fontsize{11.000000}{13.200000}\selectfont\catcode`\^=\active\def^{\ifmmode\sp\else\^{}\fi}\catcode`\%=\active\def%{\%}Time [s]}}%
\end{pgfscope}%
\begin{pgfscope}%
\pgfpathrectangle{\pgfqpoint{0.552121in}{0.499074in}}{\pgfqpoint{4.960000in}{3.696000in}}%
\pgfusepath{clip}%
\pgfsetbuttcap%
\pgfsetroundjoin%
\pgfsetlinewidth{0.803000pt}%
\definecolor{currentstroke}{rgb}{0.501961,0.501961,0.501961}%
\pgfsetstrokecolor{currentstroke}%
\pgfsetstrokeopacity{0.700000}%
\pgfsetdash{{0.800000pt}{1.320000pt}}{0.000000pt}%
\pgfpathmoveto{\pgfqpoint{0.552121in}{0.601035in}}%
\pgfpathlineto{\pgfqpoint{5.512121in}{0.601035in}}%
\pgfusepath{stroke}%
\end{pgfscope}%
\begin{pgfscope}%
\pgfsetbuttcap%
\pgfsetroundjoin%
\definecolor{currentfill}{rgb}{0.000000,0.000000,0.000000}%
\pgfsetfillcolor{currentfill}%
\pgfsetlinewidth{0.803000pt}%
\definecolor{currentstroke}{rgb}{0.000000,0.000000,0.000000}%
\pgfsetstrokecolor{currentstroke}%
\pgfsetdash{}{0pt}%
\pgfsys@defobject{currentmarker}{\pgfqpoint{-0.048611in}{0.000000in}}{\pgfqpoint{-0.000000in}{0.000000in}}{%
\pgfpathmoveto{\pgfqpoint{-0.000000in}{0.000000in}}%
\pgfpathlineto{\pgfqpoint{-0.048611in}{0.000000in}}%
\pgfusepath{stroke,fill}%
}%
\begin{pgfscope}%
\pgfsys@transformshift{0.552121in}{0.601035in}%
\pgfsys@useobject{currentmarker}{}%
\end{pgfscope}%
\end{pgfscope}%
\begin{pgfscope}%
\definecolor{textcolor}{rgb}{0.000000,0.000000,0.000000}%
\pgfsetstrokecolor{textcolor}%
\pgfsetfillcolor{textcolor}%
\pgftext[x=0.290741in, y=0.557632in, left, base]{\color{textcolor}{\rmfamily\fontsize{9.000000}{10.800000}\selectfont\catcode`\^=\active\def^{\ifmmode\sp\else\^{}\fi}\catcode`\%=\active\def%{\%}$\mathdefault{\ensuremath{-}4}$}}%
\end{pgfscope}%
\begin{pgfscope}%
\pgfpathrectangle{\pgfqpoint{0.552121in}{0.499074in}}{\pgfqpoint{4.960000in}{3.696000in}}%
\pgfusepath{clip}%
\pgfsetbuttcap%
\pgfsetroundjoin%
\pgfsetlinewidth{0.803000pt}%
\definecolor{currentstroke}{rgb}{0.501961,0.501961,0.501961}%
\pgfsetstrokecolor{currentstroke}%
\pgfsetstrokeopacity{0.700000}%
\pgfsetdash{{0.800000pt}{1.320000pt}}{0.000000pt}%
\pgfpathmoveto{\pgfqpoint{0.552121in}{1.036794in}}%
\pgfpathlineto{\pgfqpoint{5.512121in}{1.036794in}}%
\pgfusepath{stroke}%
\end{pgfscope}%
\begin{pgfscope}%
\pgfsetbuttcap%
\pgfsetroundjoin%
\definecolor{currentfill}{rgb}{0.000000,0.000000,0.000000}%
\pgfsetfillcolor{currentfill}%
\pgfsetlinewidth{0.803000pt}%
\definecolor{currentstroke}{rgb}{0.000000,0.000000,0.000000}%
\pgfsetstrokecolor{currentstroke}%
\pgfsetdash{}{0pt}%
\pgfsys@defobject{currentmarker}{\pgfqpoint{-0.048611in}{0.000000in}}{\pgfqpoint{-0.000000in}{0.000000in}}{%
\pgfpathmoveto{\pgfqpoint{-0.000000in}{0.000000in}}%
\pgfpathlineto{\pgfqpoint{-0.048611in}{0.000000in}}%
\pgfusepath{stroke,fill}%
}%
\begin{pgfscope}%
\pgfsys@transformshift{0.552121in}{1.036794in}%
\pgfsys@useobject{currentmarker}{}%
\end{pgfscope}%
\end{pgfscope}%
\begin{pgfscope}%
\definecolor{textcolor}{rgb}{0.000000,0.000000,0.000000}%
\pgfsetstrokecolor{textcolor}%
\pgfsetfillcolor{textcolor}%
\pgftext[x=0.290741in, y=0.993391in, left, base]{\color{textcolor}{\rmfamily\fontsize{9.000000}{10.800000}\selectfont\catcode`\^=\active\def^{\ifmmode\sp\else\^{}\fi}\catcode`\%=\active\def%{\%}$\mathdefault{\ensuremath{-}3}$}}%
\end{pgfscope}%
\begin{pgfscope}%
\pgfpathrectangle{\pgfqpoint{0.552121in}{0.499074in}}{\pgfqpoint{4.960000in}{3.696000in}}%
\pgfusepath{clip}%
\pgfsetbuttcap%
\pgfsetroundjoin%
\pgfsetlinewidth{0.803000pt}%
\definecolor{currentstroke}{rgb}{0.501961,0.501961,0.501961}%
\pgfsetstrokecolor{currentstroke}%
\pgfsetstrokeopacity{0.700000}%
\pgfsetdash{{0.800000pt}{1.320000pt}}{0.000000pt}%
\pgfpathmoveto{\pgfqpoint{0.552121in}{1.472553in}}%
\pgfpathlineto{\pgfqpoint{5.512121in}{1.472553in}}%
\pgfusepath{stroke}%
\end{pgfscope}%
\begin{pgfscope}%
\pgfsetbuttcap%
\pgfsetroundjoin%
\definecolor{currentfill}{rgb}{0.000000,0.000000,0.000000}%
\pgfsetfillcolor{currentfill}%
\pgfsetlinewidth{0.803000pt}%
\definecolor{currentstroke}{rgb}{0.000000,0.000000,0.000000}%
\pgfsetstrokecolor{currentstroke}%
\pgfsetdash{}{0pt}%
\pgfsys@defobject{currentmarker}{\pgfqpoint{-0.048611in}{0.000000in}}{\pgfqpoint{-0.000000in}{0.000000in}}{%
\pgfpathmoveto{\pgfqpoint{-0.000000in}{0.000000in}}%
\pgfpathlineto{\pgfqpoint{-0.048611in}{0.000000in}}%
\pgfusepath{stroke,fill}%
}%
\begin{pgfscope}%
\pgfsys@transformshift{0.552121in}{1.472553in}%
\pgfsys@useobject{currentmarker}{}%
\end{pgfscope}%
\end{pgfscope}%
\begin{pgfscope}%
\definecolor{textcolor}{rgb}{0.000000,0.000000,0.000000}%
\pgfsetstrokecolor{textcolor}%
\pgfsetfillcolor{textcolor}%
\pgftext[x=0.290741in, y=1.429150in, left, base]{\color{textcolor}{\rmfamily\fontsize{9.000000}{10.800000}\selectfont\catcode`\^=\active\def^{\ifmmode\sp\else\^{}\fi}\catcode`\%=\active\def%{\%}$\mathdefault{\ensuremath{-}2}$}}%
\end{pgfscope}%
\begin{pgfscope}%
\pgfpathrectangle{\pgfqpoint{0.552121in}{0.499074in}}{\pgfqpoint{4.960000in}{3.696000in}}%
\pgfusepath{clip}%
\pgfsetbuttcap%
\pgfsetroundjoin%
\pgfsetlinewidth{0.803000pt}%
\definecolor{currentstroke}{rgb}{0.501961,0.501961,0.501961}%
\pgfsetstrokecolor{currentstroke}%
\pgfsetstrokeopacity{0.700000}%
\pgfsetdash{{0.800000pt}{1.320000pt}}{0.000000pt}%
\pgfpathmoveto{\pgfqpoint{0.552121in}{1.908312in}}%
\pgfpathlineto{\pgfqpoint{5.512121in}{1.908312in}}%
\pgfusepath{stroke}%
\end{pgfscope}%
\begin{pgfscope}%
\pgfsetbuttcap%
\pgfsetroundjoin%
\definecolor{currentfill}{rgb}{0.000000,0.000000,0.000000}%
\pgfsetfillcolor{currentfill}%
\pgfsetlinewidth{0.803000pt}%
\definecolor{currentstroke}{rgb}{0.000000,0.000000,0.000000}%
\pgfsetstrokecolor{currentstroke}%
\pgfsetdash{}{0pt}%
\pgfsys@defobject{currentmarker}{\pgfqpoint{-0.048611in}{0.000000in}}{\pgfqpoint{-0.000000in}{0.000000in}}{%
\pgfpathmoveto{\pgfqpoint{-0.000000in}{0.000000in}}%
\pgfpathlineto{\pgfqpoint{-0.048611in}{0.000000in}}%
\pgfusepath{stroke,fill}%
}%
\begin{pgfscope}%
\pgfsys@transformshift{0.552121in}{1.908312in}%
\pgfsys@useobject{currentmarker}{}%
\end{pgfscope}%
\end{pgfscope}%
\begin{pgfscope}%
\definecolor{textcolor}{rgb}{0.000000,0.000000,0.000000}%
\pgfsetstrokecolor{textcolor}%
\pgfsetfillcolor{textcolor}%
\pgftext[x=0.290741in, y=1.864909in, left, base]{\color{textcolor}{\rmfamily\fontsize{9.000000}{10.800000}\selectfont\catcode`\^=\active\def^{\ifmmode\sp\else\^{}\fi}\catcode`\%=\active\def%{\%}$\mathdefault{\ensuremath{-}1}$}}%
\end{pgfscope}%
\begin{pgfscope}%
\pgfpathrectangle{\pgfqpoint{0.552121in}{0.499074in}}{\pgfqpoint{4.960000in}{3.696000in}}%
\pgfusepath{clip}%
\pgfsetbuttcap%
\pgfsetroundjoin%
\pgfsetlinewidth{0.803000pt}%
\definecolor{currentstroke}{rgb}{0.501961,0.501961,0.501961}%
\pgfsetstrokecolor{currentstroke}%
\pgfsetstrokeopacity{0.700000}%
\pgfsetdash{{0.800000pt}{1.320000pt}}{0.000000pt}%
\pgfpathmoveto{\pgfqpoint{0.552121in}{2.344071in}}%
\pgfpathlineto{\pgfqpoint{5.512121in}{2.344071in}}%
\pgfusepath{stroke}%
\end{pgfscope}%
\begin{pgfscope}%
\pgfsetbuttcap%
\pgfsetroundjoin%
\definecolor{currentfill}{rgb}{0.000000,0.000000,0.000000}%
\pgfsetfillcolor{currentfill}%
\pgfsetlinewidth{0.803000pt}%
\definecolor{currentstroke}{rgb}{0.000000,0.000000,0.000000}%
\pgfsetstrokecolor{currentstroke}%
\pgfsetdash{}{0pt}%
\pgfsys@defobject{currentmarker}{\pgfqpoint{-0.048611in}{0.000000in}}{\pgfqpoint{-0.000000in}{0.000000in}}{%
\pgfpathmoveto{\pgfqpoint{-0.000000in}{0.000000in}}%
\pgfpathlineto{\pgfqpoint{-0.048611in}{0.000000in}}%
\pgfusepath{stroke,fill}%
}%
\begin{pgfscope}%
\pgfsys@transformshift{0.552121in}{2.344071in}%
\pgfsys@useobject{currentmarker}{}%
\end{pgfscope}%
\end{pgfscope}%
\begin{pgfscope}%
\definecolor{textcolor}{rgb}{0.000000,0.000000,0.000000}%
\pgfsetstrokecolor{textcolor}%
\pgfsetfillcolor{textcolor}%
\pgftext[x=0.390663in, y=2.300668in, left, base]{\color{textcolor}{\rmfamily\fontsize{9.000000}{10.800000}\selectfont\catcode`\^=\active\def^{\ifmmode\sp\else\^{}\fi}\catcode`\%=\active\def%{\%}$\mathdefault{0}$}}%
\end{pgfscope}%
\begin{pgfscope}%
\pgfpathrectangle{\pgfqpoint{0.552121in}{0.499074in}}{\pgfqpoint{4.960000in}{3.696000in}}%
\pgfusepath{clip}%
\pgfsetbuttcap%
\pgfsetroundjoin%
\pgfsetlinewidth{0.803000pt}%
\definecolor{currentstroke}{rgb}{0.501961,0.501961,0.501961}%
\pgfsetstrokecolor{currentstroke}%
\pgfsetstrokeopacity{0.700000}%
\pgfsetdash{{0.800000pt}{1.320000pt}}{0.000000pt}%
\pgfpathmoveto{\pgfqpoint{0.552121in}{2.779830in}}%
\pgfpathlineto{\pgfqpoint{5.512121in}{2.779830in}}%
\pgfusepath{stroke}%
\end{pgfscope}%
\begin{pgfscope}%
\pgfsetbuttcap%
\pgfsetroundjoin%
\definecolor{currentfill}{rgb}{0.000000,0.000000,0.000000}%
\pgfsetfillcolor{currentfill}%
\pgfsetlinewidth{0.803000pt}%
\definecolor{currentstroke}{rgb}{0.000000,0.000000,0.000000}%
\pgfsetstrokecolor{currentstroke}%
\pgfsetdash{}{0pt}%
\pgfsys@defobject{currentmarker}{\pgfqpoint{-0.048611in}{0.000000in}}{\pgfqpoint{-0.000000in}{0.000000in}}{%
\pgfpathmoveto{\pgfqpoint{-0.000000in}{0.000000in}}%
\pgfpathlineto{\pgfqpoint{-0.048611in}{0.000000in}}%
\pgfusepath{stroke,fill}%
}%
\begin{pgfscope}%
\pgfsys@transformshift{0.552121in}{2.779830in}%
\pgfsys@useobject{currentmarker}{}%
\end{pgfscope}%
\end{pgfscope}%
\begin{pgfscope}%
\definecolor{textcolor}{rgb}{0.000000,0.000000,0.000000}%
\pgfsetstrokecolor{textcolor}%
\pgfsetfillcolor{textcolor}%
\pgftext[x=0.390663in, y=2.736427in, left, base]{\color{textcolor}{\rmfamily\fontsize{9.000000}{10.800000}\selectfont\catcode`\^=\active\def^{\ifmmode\sp\else\^{}\fi}\catcode`\%=\active\def%{\%}$\mathdefault{1}$}}%
\end{pgfscope}%
\begin{pgfscope}%
\pgfpathrectangle{\pgfqpoint{0.552121in}{0.499074in}}{\pgfqpoint{4.960000in}{3.696000in}}%
\pgfusepath{clip}%
\pgfsetbuttcap%
\pgfsetroundjoin%
\pgfsetlinewidth{0.803000pt}%
\definecolor{currentstroke}{rgb}{0.501961,0.501961,0.501961}%
\pgfsetstrokecolor{currentstroke}%
\pgfsetstrokeopacity{0.700000}%
\pgfsetdash{{0.800000pt}{1.320000pt}}{0.000000pt}%
\pgfpathmoveto{\pgfqpoint{0.552121in}{3.215589in}}%
\pgfpathlineto{\pgfqpoint{5.512121in}{3.215589in}}%
\pgfusepath{stroke}%
\end{pgfscope}%
\begin{pgfscope}%
\pgfsetbuttcap%
\pgfsetroundjoin%
\definecolor{currentfill}{rgb}{0.000000,0.000000,0.000000}%
\pgfsetfillcolor{currentfill}%
\pgfsetlinewidth{0.803000pt}%
\definecolor{currentstroke}{rgb}{0.000000,0.000000,0.000000}%
\pgfsetstrokecolor{currentstroke}%
\pgfsetdash{}{0pt}%
\pgfsys@defobject{currentmarker}{\pgfqpoint{-0.048611in}{0.000000in}}{\pgfqpoint{-0.000000in}{0.000000in}}{%
\pgfpathmoveto{\pgfqpoint{-0.000000in}{0.000000in}}%
\pgfpathlineto{\pgfqpoint{-0.048611in}{0.000000in}}%
\pgfusepath{stroke,fill}%
}%
\begin{pgfscope}%
\pgfsys@transformshift{0.552121in}{3.215589in}%
\pgfsys@useobject{currentmarker}{}%
\end{pgfscope}%
\end{pgfscope}%
\begin{pgfscope}%
\definecolor{textcolor}{rgb}{0.000000,0.000000,0.000000}%
\pgfsetstrokecolor{textcolor}%
\pgfsetfillcolor{textcolor}%
\pgftext[x=0.390663in, y=3.172186in, left, base]{\color{textcolor}{\rmfamily\fontsize{9.000000}{10.800000}\selectfont\catcode`\^=\active\def^{\ifmmode\sp\else\^{}\fi}\catcode`\%=\active\def%{\%}$\mathdefault{2}$}}%
\end{pgfscope}%
\begin{pgfscope}%
\pgfpathrectangle{\pgfqpoint{0.552121in}{0.499074in}}{\pgfqpoint{4.960000in}{3.696000in}}%
\pgfusepath{clip}%
\pgfsetbuttcap%
\pgfsetroundjoin%
\pgfsetlinewidth{0.803000pt}%
\definecolor{currentstroke}{rgb}{0.501961,0.501961,0.501961}%
\pgfsetstrokecolor{currentstroke}%
\pgfsetstrokeopacity{0.700000}%
\pgfsetdash{{0.800000pt}{1.320000pt}}{0.000000pt}%
\pgfpathmoveto{\pgfqpoint{0.552121in}{3.651348in}}%
\pgfpathlineto{\pgfqpoint{5.512121in}{3.651348in}}%
\pgfusepath{stroke}%
\end{pgfscope}%
\begin{pgfscope}%
\pgfsetbuttcap%
\pgfsetroundjoin%
\definecolor{currentfill}{rgb}{0.000000,0.000000,0.000000}%
\pgfsetfillcolor{currentfill}%
\pgfsetlinewidth{0.803000pt}%
\definecolor{currentstroke}{rgb}{0.000000,0.000000,0.000000}%
\pgfsetstrokecolor{currentstroke}%
\pgfsetdash{}{0pt}%
\pgfsys@defobject{currentmarker}{\pgfqpoint{-0.048611in}{0.000000in}}{\pgfqpoint{-0.000000in}{0.000000in}}{%
\pgfpathmoveto{\pgfqpoint{-0.000000in}{0.000000in}}%
\pgfpathlineto{\pgfqpoint{-0.048611in}{0.000000in}}%
\pgfusepath{stroke,fill}%
}%
\begin{pgfscope}%
\pgfsys@transformshift{0.552121in}{3.651348in}%
\pgfsys@useobject{currentmarker}{}%
\end{pgfscope}%
\end{pgfscope}%
\begin{pgfscope}%
\definecolor{textcolor}{rgb}{0.000000,0.000000,0.000000}%
\pgfsetstrokecolor{textcolor}%
\pgfsetfillcolor{textcolor}%
\pgftext[x=0.390663in, y=3.607945in, left, base]{\color{textcolor}{\rmfamily\fontsize{9.000000}{10.800000}\selectfont\catcode`\^=\active\def^{\ifmmode\sp\else\^{}\fi}\catcode`\%=\active\def%{\%}$\mathdefault{3}$}}%
\end{pgfscope}%
\begin{pgfscope}%
\pgfpathrectangle{\pgfqpoint{0.552121in}{0.499074in}}{\pgfqpoint{4.960000in}{3.696000in}}%
\pgfusepath{clip}%
\pgfsetbuttcap%
\pgfsetroundjoin%
\pgfsetlinewidth{0.803000pt}%
\definecolor{currentstroke}{rgb}{0.501961,0.501961,0.501961}%
\pgfsetstrokecolor{currentstroke}%
\pgfsetstrokeopacity{0.700000}%
\pgfsetdash{{0.800000pt}{1.320000pt}}{0.000000pt}%
\pgfpathmoveto{\pgfqpoint{0.552121in}{4.087107in}}%
\pgfpathlineto{\pgfqpoint{5.512121in}{4.087107in}}%
\pgfusepath{stroke}%
\end{pgfscope}%
\begin{pgfscope}%
\pgfsetbuttcap%
\pgfsetroundjoin%
\definecolor{currentfill}{rgb}{0.000000,0.000000,0.000000}%
\pgfsetfillcolor{currentfill}%
\pgfsetlinewidth{0.803000pt}%
\definecolor{currentstroke}{rgb}{0.000000,0.000000,0.000000}%
\pgfsetstrokecolor{currentstroke}%
\pgfsetdash{}{0pt}%
\pgfsys@defobject{currentmarker}{\pgfqpoint{-0.048611in}{0.000000in}}{\pgfqpoint{-0.000000in}{0.000000in}}{%
\pgfpathmoveto{\pgfqpoint{-0.000000in}{0.000000in}}%
\pgfpathlineto{\pgfqpoint{-0.048611in}{0.000000in}}%
\pgfusepath{stroke,fill}%
}%
\begin{pgfscope}%
\pgfsys@transformshift{0.552121in}{4.087107in}%
\pgfsys@useobject{currentmarker}{}%
\end{pgfscope}%
\end{pgfscope}%
\begin{pgfscope}%
\definecolor{textcolor}{rgb}{0.000000,0.000000,0.000000}%
\pgfsetstrokecolor{textcolor}%
\pgfsetfillcolor{textcolor}%
\pgftext[x=0.390663in, y=4.043704in, left, base]{\color{textcolor}{\rmfamily\fontsize{9.000000}{10.800000}\selectfont\catcode`\^=\active\def^{\ifmmode\sp\else\^{}\fi}\catcode`\%=\active\def%{\%}$\mathdefault{4}$}}%
\end{pgfscope}%
\begin{pgfscope}%
\definecolor{textcolor}{rgb}{0.000000,0.000000,0.000000}%
\pgfsetstrokecolor{textcolor}%
\pgfsetfillcolor{textcolor}%
\pgftext[x=0.235185in,y=2.347074in,,bottom,rotate=90.000000]{\color{textcolor}{\rmfamily\fontsize{11.000000}{13.200000}\selectfont\catcode`\^=\active\def^{\ifmmode\sp\else\^{}\fi}\catcode`\%=\active\def%{\%}$v\xi$}}%
\end{pgfscope}%
\begin{pgfscope}%
\pgfpathrectangle{\pgfqpoint{0.552121in}{0.499074in}}{\pgfqpoint{4.960000in}{3.696000in}}%
\pgfusepath{clip}%
\pgfsetbuttcap%
\pgfsetroundjoin%
\pgfsetlinewidth{1.505625pt}%
\definecolor{currentstroke}{rgb}{1.000000,0.000000,0.000000}%
\pgfsetstrokecolor{currentstroke}%
\pgfsetdash{{1.500000pt}{2.475000pt}}{0.000000pt}%
\pgfpathmoveto{\pgfqpoint{0.777575in}{0.667074in}}%
\pgfpathlineto{\pgfqpoint{0.852727in}{0.667078in}}%
\pgfpathlineto{\pgfqpoint{0.927878in}{0.667085in}}%
\pgfpathlineto{\pgfqpoint{1.003030in}{0.667094in}}%
\pgfpathlineto{\pgfqpoint{1.078181in}{0.667105in}}%
\pgfpathlineto{\pgfqpoint{1.153333in}{0.667117in}}%
\pgfpathlineto{\pgfqpoint{1.228484in}{0.667131in}}%
\pgfpathlineto{\pgfqpoint{1.303636in}{0.667149in}}%
\pgfpathlineto{\pgfqpoint{1.378788in}{0.667173in}}%
\pgfpathlineto{\pgfqpoint{1.453939in}{0.667207in}}%
\pgfpathlineto{\pgfqpoint{1.529091in}{0.667253in}}%
\pgfpathlineto{\pgfqpoint{1.604242in}{0.667316in}}%
\pgfpathlineto{\pgfqpoint{1.679394in}{0.667398in}}%
\pgfpathlineto{\pgfqpoint{1.754545in}{0.667502in}}%
\pgfpathlineto{\pgfqpoint{1.829697in}{0.667633in}}%
\pgfpathlineto{\pgfqpoint{1.904848in}{0.667792in}}%
\pgfpathlineto{\pgfqpoint{1.980000in}{0.667983in}}%
\pgfpathlineto{\pgfqpoint{2.055151in}{0.668197in}}%
\pgfpathlineto{\pgfqpoint{2.130303in}{0.668407in}}%
\pgfpathlineto{\pgfqpoint{2.205454in}{0.668614in}}%
\pgfpathlineto{\pgfqpoint{2.280606in}{0.668816in}}%
\pgfpathlineto{\pgfqpoint{2.355757in}{0.669015in}}%
\pgfpathlineto{\pgfqpoint{2.430909in}{0.669210in}}%
\pgfpathlineto{\pgfqpoint{2.506060in}{0.669402in}}%
\pgfpathlineto{\pgfqpoint{2.581212in}{0.669589in}}%
\pgfpathlineto{\pgfqpoint{2.656363in}{0.669773in}}%
\pgfpathlineto{\pgfqpoint{2.731515in}{0.669952in}}%
\pgfpathlineto{\pgfqpoint{2.806666in}{0.670128in}}%
\pgfpathlineto{\pgfqpoint{2.881818in}{0.670299in}}%
\pgfpathlineto{\pgfqpoint{2.956969in}{0.670467in}}%
\pgfpathlineto{\pgfqpoint{3.032121in}{0.670630in}}%
\pgfpathlineto{\pgfqpoint{3.107272in}{0.670789in}}%
\pgfpathlineto{\pgfqpoint{3.182424in}{0.670944in}}%
\pgfpathlineto{\pgfqpoint{3.257575in}{0.671094in}}%
\pgfpathlineto{\pgfqpoint{3.332727in}{0.671239in}}%
\pgfpathlineto{\pgfqpoint{3.407878in}{0.671380in}}%
\pgfpathlineto{\pgfqpoint{3.483030in}{0.671517in}}%
\pgfpathlineto{\pgfqpoint{3.558181in}{0.671648in}}%
\pgfpathlineto{\pgfqpoint{3.633333in}{0.671774in}}%
\pgfpathlineto{\pgfqpoint{3.708484in}{0.671895in}}%
\pgfpathlineto{\pgfqpoint{3.783636in}{0.672011in}}%
\pgfpathlineto{\pgfqpoint{3.858788in}{0.672122in}}%
\pgfpathlineto{\pgfqpoint{3.933939in}{0.672227in}}%
\pgfpathlineto{\pgfqpoint{4.009091in}{0.672326in}}%
\pgfpathlineto{\pgfqpoint{4.084242in}{0.672420in}}%
\pgfpathlineto{\pgfqpoint{4.159394in}{0.672507in}}%
\pgfpathlineto{\pgfqpoint{4.234545in}{0.672589in}}%
\pgfpathlineto{\pgfqpoint{4.309697in}{0.672665in}}%
\pgfpathlineto{\pgfqpoint{4.384848in}{0.672735in}}%
\pgfpathlineto{\pgfqpoint{4.460000in}{0.672798in}}%
\pgfpathlineto{\pgfqpoint{4.535151in}{0.672855in}}%
\pgfpathlineto{\pgfqpoint{4.610303in}{0.672906in}}%
\pgfpathlineto{\pgfqpoint{4.685454in}{0.672951in}}%
\pgfpathlineto{\pgfqpoint{4.760606in}{0.672990in}}%
\pgfpathlineto{\pgfqpoint{4.835757in}{0.673022in}}%
\pgfpathlineto{\pgfqpoint{4.910909in}{0.673048in}}%
\pgfpathlineto{\pgfqpoint{4.986060in}{0.673068in}}%
\pgfpathlineto{\pgfqpoint{5.061212in}{0.673082in}}%
\pgfpathlineto{\pgfqpoint{5.136363in}{0.673090in}}%
\pgfpathlineto{\pgfqpoint{5.211515in}{0.673092in}}%
\pgfpathlineto{\pgfqpoint{5.286666in}{0.673089in}}%
\pgfusepath{stroke}%
\end{pgfscope}%
\begin{pgfscope}%
\pgfpathrectangle{\pgfqpoint{0.552121in}{0.499074in}}{\pgfqpoint{4.960000in}{3.696000in}}%
\pgfusepath{clip}%
\pgfsetbuttcap%
\pgfsetroundjoin%
\pgfsetlinewidth{1.505625pt}%
\definecolor{currentstroke}{rgb}{1.000000,0.000000,0.000000}%
\pgfsetstrokecolor{currentstroke}%
\pgfsetdash{{5.550000pt}{2.400000pt}}{0.000000pt}%
\pgfpathmoveto{\pgfqpoint{0.777575in}{4.021068in}}%
\pgfpathlineto{\pgfqpoint{0.852727in}{4.021063in}}%
\pgfpathlineto{\pgfqpoint{0.927878in}{4.021070in}}%
\pgfpathlineto{\pgfqpoint{1.003030in}{4.021079in}}%
\pgfpathlineto{\pgfqpoint{1.078181in}{4.021089in}}%
\pgfpathlineto{\pgfqpoint{1.153333in}{4.021101in}}%
\pgfpathlineto{\pgfqpoint{1.228484in}{4.021114in}}%
\pgfpathlineto{\pgfqpoint{1.303636in}{4.021132in}}%
\pgfpathlineto{\pgfqpoint{1.378788in}{4.021156in}}%
\pgfpathlineto{\pgfqpoint{1.453939in}{4.021190in}}%
\pgfpathlineto{\pgfqpoint{1.529091in}{4.021236in}}%
\pgfpathlineto{\pgfqpoint{1.604242in}{4.021299in}}%
\pgfpathlineto{\pgfqpoint{1.679394in}{4.021381in}}%
\pgfpathlineto{\pgfqpoint{1.754545in}{4.021485in}}%
\pgfpathlineto{\pgfqpoint{1.829697in}{4.021616in}}%
\pgfpathlineto{\pgfqpoint{1.904848in}{4.021775in}}%
\pgfpathlineto{\pgfqpoint{1.980000in}{4.021966in}}%
\pgfpathlineto{\pgfqpoint{2.055151in}{4.022180in}}%
\pgfpathlineto{\pgfqpoint{2.130303in}{4.022390in}}%
\pgfpathlineto{\pgfqpoint{2.205454in}{4.022596in}}%
\pgfpathlineto{\pgfqpoint{2.280606in}{4.022798in}}%
\pgfpathlineto{\pgfqpoint{2.355757in}{4.022997in}}%
\pgfpathlineto{\pgfqpoint{2.430909in}{4.023192in}}%
\pgfpathlineto{\pgfqpoint{2.506060in}{4.023383in}}%
\pgfpathlineto{\pgfqpoint{2.581212in}{4.023570in}}%
\pgfpathlineto{\pgfqpoint{2.656363in}{4.023753in}}%
\pgfpathlineto{\pgfqpoint{2.731515in}{4.023933in}}%
\pgfpathlineto{\pgfqpoint{2.806666in}{4.024108in}}%
\pgfpathlineto{\pgfqpoint{2.881818in}{4.024279in}}%
\pgfpathlineto{\pgfqpoint{2.956969in}{4.024447in}}%
\pgfpathlineto{\pgfqpoint{3.032121in}{4.024610in}}%
\pgfpathlineto{\pgfqpoint{3.107272in}{4.024769in}}%
\pgfpathlineto{\pgfqpoint{3.182424in}{4.024923in}}%
\pgfpathlineto{\pgfqpoint{3.257575in}{4.025074in}}%
\pgfpathlineto{\pgfqpoint{3.332727in}{4.025219in}}%
\pgfpathlineto{\pgfqpoint{3.407878in}{4.025360in}}%
\pgfpathlineto{\pgfqpoint{3.483030in}{4.025496in}}%
\pgfpathlineto{\pgfqpoint{3.558181in}{4.025628in}}%
\pgfpathlineto{\pgfqpoint{3.633333in}{4.025754in}}%
\pgfpathlineto{\pgfqpoint{3.708484in}{4.025875in}}%
\pgfpathlineto{\pgfqpoint{3.783636in}{4.025991in}}%
\pgfpathlineto{\pgfqpoint{3.858788in}{4.026102in}}%
\pgfpathlineto{\pgfqpoint{3.933939in}{4.026207in}}%
\pgfpathlineto{\pgfqpoint{4.009091in}{4.026306in}}%
\pgfpathlineto{\pgfqpoint{4.084242in}{4.026400in}}%
\pgfpathlineto{\pgfqpoint{4.159394in}{4.026488in}}%
\pgfpathlineto{\pgfqpoint{4.234545in}{4.026570in}}%
\pgfpathlineto{\pgfqpoint{4.309697in}{4.026645in}}%
\pgfpathlineto{\pgfqpoint{4.384848in}{4.026715in}}%
\pgfpathlineto{\pgfqpoint{4.460000in}{4.026779in}}%
\pgfpathlineto{\pgfqpoint{4.535151in}{4.026836in}}%
\pgfpathlineto{\pgfqpoint{4.610303in}{4.026887in}}%
\pgfpathlineto{\pgfqpoint{4.685454in}{4.026932in}}%
\pgfpathlineto{\pgfqpoint{4.760606in}{4.026971in}}%
\pgfpathlineto{\pgfqpoint{4.835757in}{4.027003in}}%
\pgfpathlineto{\pgfqpoint{4.910909in}{4.027030in}}%
\pgfpathlineto{\pgfqpoint{4.986060in}{4.027050in}}%
\pgfpathlineto{\pgfqpoint{5.061212in}{4.027064in}}%
\pgfpathlineto{\pgfqpoint{5.136363in}{4.027072in}}%
\pgfpathlineto{\pgfqpoint{5.211515in}{4.027074in}}%
\pgfpathlineto{\pgfqpoint{5.286666in}{4.027071in}}%
\pgfusepath{stroke}%
\end{pgfscope}%
\begin{pgfscope}%
\pgfpathrectangle{\pgfqpoint{0.552121in}{0.499074in}}{\pgfqpoint{4.960000in}{3.696000in}}%
\pgfusepath{clip}%
\pgfsetrectcap%
\pgfsetroundjoin%
\pgfsetlinewidth{1.505625pt}%
\definecolor{currentstroke}{rgb}{0.121569,0.466667,0.705882}%
\pgfsetstrokecolor{currentstroke}%
\pgfsetdash{}{0pt}%
\pgfpathmoveto{\pgfqpoint{0.777575in}{2.528962in}}%
\pgfpathlineto{\pgfqpoint{0.852727in}{2.441192in}}%
\pgfpathlineto{\pgfqpoint{0.927878in}{2.465751in}}%
\pgfpathlineto{\pgfqpoint{1.003030in}{2.464182in}}%
\pgfpathlineto{\pgfqpoint{1.078181in}{2.463741in}}%
\pgfpathlineto{\pgfqpoint{1.153333in}{2.463988in}}%
\pgfpathlineto{\pgfqpoint{1.228484in}{2.464695in}}%
\pgfpathlineto{\pgfqpoint{1.303636in}{2.465736in}}%
\pgfpathlineto{\pgfqpoint{1.378788in}{2.467042in}}%
\pgfpathlineto{\pgfqpoint{1.453939in}{2.468574in}}%
\pgfpathlineto{\pgfqpoint{1.529091in}{2.470311in}}%
\pgfpathlineto{\pgfqpoint{1.604242in}{2.472242in}}%
\pgfpathlineto{\pgfqpoint{1.679394in}{2.474364in}}%
\pgfpathlineto{\pgfqpoint{1.754545in}{2.476678in}}%
\pgfpathlineto{\pgfqpoint{1.829697in}{2.479187in}}%
\pgfpathlineto{\pgfqpoint{1.904848in}{2.481896in}}%
\pgfpathlineto{\pgfqpoint{1.980000in}{2.484812in}}%
\pgfpathlineto{\pgfqpoint{2.055151in}{2.487936in}}%
\pgfpathlineto{\pgfqpoint{2.130303in}{2.491262in}}%
\pgfpathlineto{\pgfqpoint{2.205454in}{2.494793in}}%
\pgfpathlineto{\pgfqpoint{2.280606in}{2.498536in}}%
\pgfpathlineto{\pgfqpoint{2.355757in}{2.502495in}}%
\pgfpathlineto{\pgfqpoint{2.430909in}{2.506676in}}%
\pgfpathlineto{\pgfqpoint{2.506060in}{2.511083in}}%
\pgfpathlineto{\pgfqpoint{2.581212in}{2.515723in}}%
\pgfpathlineto{\pgfqpoint{2.656363in}{2.520600in}}%
\pgfpathlineto{\pgfqpoint{2.731515in}{2.525719in}}%
\pgfpathlineto{\pgfqpoint{2.806666in}{2.531085in}}%
\pgfpathlineto{\pgfqpoint{2.881818in}{2.536702in}}%
\pgfpathlineto{\pgfqpoint{2.956969in}{2.542574in}}%
\pgfpathlineto{\pgfqpoint{3.032121in}{2.548704in}}%
\pgfpathlineto{\pgfqpoint{3.107272in}{2.555094in}}%
\pgfpathlineto{\pgfqpoint{3.182424in}{2.561746in}}%
\pgfpathlineto{\pgfqpoint{3.257575in}{2.568661in}}%
\pgfpathlineto{\pgfqpoint{3.332727in}{2.575840in}}%
\pgfpathlineto{\pgfqpoint{3.407878in}{2.583282in}}%
\pgfpathlineto{\pgfqpoint{3.483030in}{2.590986in}}%
\pgfpathlineto{\pgfqpoint{3.558181in}{2.598950in}}%
\pgfpathlineto{\pgfqpoint{3.633333in}{2.607173in}}%
\pgfpathlineto{\pgfqpoint{3.708484in}{2.615650in}}%
\pgfpathlineto{\pgfqpoint{3.783636in}{2.624377in}}%
\pgfpathlineto{\pgfqpoint{3.858788in}{2.633350in}}%
\pgfpathlineto{\pgfqpoint{3.933939in}{2.642563in}}%
\pgfpathlineto{\pgfqpoint{4.009091in}{2.652009in}}%
\pgfpathlineto{\pgfqpoint{4.084242in}{2.661680in}}%
\pgfpathlineto{\pgfqpoint{4.159394in}{2.671570in}}%
\pgfpathlineto{\pgfqpoint{4.234545in}{2.681670in}}%
\pgfpathlineto{\pgfqpoint{4.309697in}{2.691970in}}%
\pgfpathlineto{\pgfqpoint{4.384848in}{2.702462in}}%
\pgfpathlineto{\pgfqpoint{4.460000in}{2.713141in}}%
\pgfpathlineto{\pgfqpoint{4.535151in}{2.724004in}}%
\pgfpathlineto{\pgfqpoint{4.610303in}{2.735055in}}%
\pgfpathlineto{\pgfqpoint{4.685454in}{2.746311in}}%
\pgfpathlineto{\pgfqpoint{4.760606in}{2.757811in}}%
\pgfpathlineto{\pgfqpoint{4.835757in}{2.769624in}}%
\pgfpathlineto{\pgfqpoint{4.910909in}{2.781869in}}%
\pgfpathlineto{\pgfqpoint{4.986060in}{2.794737in}}%
\pgfpathlineto{\pgfqpoint{5.061212in}{2.808522in}}%
\pgfpathlineto{\pgfqpoint{5.136363in}{2.823664in}}%
\pgfpathlineto{\pgfqpoint{5.211515in}{2.840800in}}%
\pgfpathlineto{\pgfqpoint{5.286666in}{2.347559in}}%
\pgfusepath{stroke}%
\end{pgfscope}%
\begin{pgfscope}%
\pgfpathrectangle{\pgfqpoint{0.552121in}{0.499074in}}{\pgfqpoint{4.960000in}{3.696000in}}%
\pgfusepath{clip}%
\pgfsetrectcap%
\pgfsetroundjoin%
\pgfsetlinewidth{1.505625pt}%
\definecolor{currentstroke}{rgb}{1.000000,0.498039,0.054902}%
\pgfsetstrokecolor{currentstroke}%
\pgfsetdash{}{0pt}%
\pgfpathmoveto{\pgfqpoint{0.777575in}{2.344071in}}%
\pgfpathlineto{\pgfqpoint{0.852727in}{2.344071in}}%
\pgfpathlineto{\pgfqpoint{0.927878in}{2.344071in}}%
\pgfpathlineto{\pgfqpoint{1.003030in}{2.344071in}}%
\pgfpathlineto{\pgfqpoint{1.078181in}{2.344071in}}%
\pgfpathlineto{\pgfqpoint{1.153333in}{2.344072in}}%
\pgfpathlineto{\pgfqpoint{1.228484in}{2.344072in}}%
\pgfpathlineto{\pgfqpoint{1.303636in}{2.344072in}}%
\pgfpathlineto{\pgfqpoint{1.378788in}{2.344073in}}%
\pgfpathlineto{\pgfqpoint{1.453939in}{2.344073in}}%
\pgfpathlineto{\pgfqpoint{1.529091in}{2.344074in}}%
\pgfpathlineto{\pgfqpoint{1.604242in}{2.344076in}}%
\pgfpathlineto{\pgfqpoint{1.679394in}{2.344077in}}%
\pgfpathlineto{\pgfqpoint{1.754545in}{2.344079in}}%
\pgfpathlineto{\pgfqpoint{1.829697in}{2.344082in}}%
\pgfpathlineto{\pgfqpoint{1.904848in}{2.344085in}}%
\pgfpathlineto{\pgfqpoint{1.980000in}{2.344089in}}%
\pgfpathlineto{\pgfqpoint{2.055151in}{2.344093in}}%
\pgfpathlineto{\pgfqpoint{2.130303in}{2.344097in}}%
\pgfpathlineto{\pgfqpoint{2.205454in}{2.344102in}}%
\pgfpathlineto{\pgfqpoint{2.280606in}{2.344106in}}%
\pgfpathlineto{\pgfqpoint{2.355757in}{2.344110in}}%
\pgfpathlineto{\pgfqpoint{2.430909in}{2.344113in}}%
\pgfpathlineto{\pgfqpoint{2.506060in}{2.344117in}}%
\pgfpathlineto{\pgfqpoint{2.581212in}{2.344121in}}%
\pgfpathlineto{\pgfqpoint{2.656363in}{2.344125in}}%
\pgfpathlineto{\pgfqpoint{2.731515in}{2.344128in}}%
\pgfpathlineto{\pgfqpoint{2.806666in}{2.344132in}}%
\pgfpathlineto{\pgfqpoint{2.881818in}{2.344135in}}%
\pgfpathlineto{\pgfqpoint{2.956969in}{2.344139in}}%
\pgfpathlineto{\pgfqpoint{3.032121in}{2.344142in}}%
\pgfpathlineto{\pgfqpoint{3.107272in}{2.344145in}}%
\pgfpathlineto{\pgfqpoint{3.182424in}{2.344148in}}%
\pgfpathlineto{\pgfqpoint{3.257575in}{2.344151in}}%
\pgfpathlineto{\pgfqpoint{3.332727in}{2.344154in}}%
\pgfpathlineto{\pgfqpoint{3.407878in}{2.344157in}}%
\pgfpathlineto{\pgfqpoint{3.483030in}{2.344160in}}%
\pgfpathlineto{\pgfqpoint{3.558181in}{2.344162in}}%
\pgfpathlineto{\pgfqpoint{3.633333in}{2.344165in}}%
\pgfpathlineto{\pgfqpoint{3.708484in}{2.344167in}}%
\pgfpathlineto{\pgfqpoint{3.783636in}{2.344170in}}%
\pgfpathlineto{\pgfqpoint{3.858788in}{2.344172in}}%
\pgfpathlineto{\pgfqpoint{3.933939in}{2.344174in}}%
\pgfpathlineto{\pgfqpoint{4.009091in}{2.344176in}}%
\pgfpathlineto{\pgfqpoint{4.084242in}{2.344178in}}%
\pgfpathlineto{\pgfqpoint{4.159394in}{2.344179in}}%
\pgfpathlineto{\pgfqpoint{4.234545in}{2.344181in}}%
\pgfpathlineto{\pgfqpoint{4.309697in}{2.344183in}}%
\pgfpathlineto{\pgfqpoint{4.384848in}{2.344184in}}%
\pgfpathlineto{\pgfqpoint{4.460000in}{2.344185in}}%
\pgfpathlineto{\pgfqpoint{4.535151in}{2.344186in}}%
\pgfpathlineto{\pgfqpoint{4.610303in}{2.344187in}}%
\pgfpathlineto{\pgfqpoint{4.685454in}{2.344188in}}%
\pgfpathlineto{\pgfqpoint{4.760606in}{2.344189in}}%
\pgfpathlineto{\pgfqpoint{4.835757in}{2.344190in}}%
\pgfpathlineto{\pgfqpoint{4.910909in}{2.344190in}}%
\pgfpathlineto{\pgfqpoint{4.986060in}{2.344191in}}%
\pgfpathlineto{\pgfqpoint{5.061212in}{2.344191in}}%
\pgfpathlineto{\pgfqpoint{5.136363in}{2.344191in}}%
\pgfpathlineto{\pgfqpoint{5.211515in}{2.344191in}}%
\pgfpathlineto{\pgfqpoint{5.286666in}{2.344191in}}%
\pgfusepath{stroke}%
\end{pgfscope}%
\begin{pgfscope}%
\pgfsetrectcap%
\pgfsetmiterjoin%
\pgfsetlinewidth{0.803000pt}%
\definecolor{currentstroke}{rgb}{0.000000,0.000000,0.000000}%
\pgfsetstrokecolor{currentstroke}%
\pgfsetdash{}{0pt}%
\pgfpathmoveto{\pgfqpoint{0.552121in}{0.499074in}}%
\pgfpathlineto{\pgfqpoint{0.552121in}{4.195074in}}%
\pgfusepath{stroke}%
\end{pgfscope}%
\begin{pgfscope}%
\pgfsetrectcap%
\pgfsetmiterjoin%
\pgfsetlinewidth{0.803000pt}%
\definecolor{currentstroke}{rgb}{0.000000,0.000000,0.000000}%
\pgfsetstrokecolor{currentstroke}%
\pgfsetdash{}{0pt}%
\pgfpathmoveto{\pgfqpoint{5.512121in}{0.499074in}}%
\pgfpathlineto{\pgfqpoint{5.512121in}{4.195074in}}%
\pgfusepath{stroke}%
\end{pgfscope}%
\begin{pgfscope}%
\pgfsetrectcap%
\pgfsetmiterjoin%
\pgfsetlinewidth{0.803000pt}%
\definecolor{currentstroke}{rgb}{0.000000,0.000000,0.000000}%
\pgfsetstrokecolor{currentstroke}%
\pgfsetdash{}{0pt}%
\pgfpathmoveto{\pgfqpoint{0.552121in}{0.499074in}}%
\pgfpathlineto{\pgfqpoint{5.512121in}{0.499074in}}%
\pgfusepath{stroke}%
\end{pgfscope}%
\begin{pgfscope}%
\pgfsetrectcap%
\pgfsetmiterjoin%
\pgfsetlinewidth{0.803000pt}%
\definecolor{currentstroke}{rgb}{0.000000,0.000000,0.000000}%
\pgfsetstrokecolor{currentstroke}%
\pgfsetdash{}{0pt}%
\pgfpathmoveto{\pgfqpoint{0.552121in}{4.195074in}}%
\pgfpathlineto{\pgfqpoint{5.512121in}{4.195074in}}%
\pgfusepath{stroke}%
\end{pgfscope}%
\begin{pgfscope}%
\pgfsetbuttcap%
\pgfsetmiterjoin%
\definecolor{currentfill}{rgb}{1.000000,1.000000,1.000000}%
\pgfsetfillcolor{currentfill}%
\pgfsetfillopacity{0.800000}%
\pgfsetlinewidth{1.003750pt}%
\definecolor{currentstroke}{rgb}{0.800000,0.800000,0.800000}%
\pgfsetstrokecolor{currentstroke}%
\pgfsetstrokeopacity{0.800000}%
\pgfsetdash{}{0pt}%
\pgfpathmoveto{\pgfqpoint{0.649343in}{0.568518in}}%
\pgfpathlineto{\pgfqpoint{2.681754in}{0.568518in}}%
\pgfpathquadraticcurveto{\pgfqpoint{2.709531in}{0.568518in}}{\pgfqpoint{2.709531in}{0.596296in}}%
\pgfpathlineto{\pgfqpoint{2.709531in}{1.550771in}}%
\pgfpathquadraticcurveto{\pgfqpoint{2.709531in}{1.578549in}}{\pgfqpoint{2.681754in}{1.578549in}}%
\pgfpathlineto{\pgfqpoint{0.649343in}{1.578549in}}%
\pgfpathquadraticcurveto{\pgfqpoint{0.621565in}{1.578549in}}{\pgfqpoint{0.621565in}{1.550771in}}%
\pgfpathlineto{\pgfqpoint{0.621565in}{0.596296in}}%
\pgfpathquadraticcurveto{\pgfqpoint{0.621565in}{0.568518in}}{\pgfqpoint{0.649343in}{0.568518in}}%
\pgfpathlineto{\pgfqpoint{0.649343in}{0.568518in}}%
\pgfpathclose%
\pgfusepath{stroke,fill}%
\end{pgfscope}%
\begin{pgfscope}%
\pgfsetbuttcap%
\pgfsetmiterjoin%
\definecolor{currentfill}{rgb}{0.000000,0.501961,0.000000}%
\pgfsetfillcolor{currentfill}%
\pgfsetfillopacity{0.200000}%
\pgfsetlinewidth{1.003750pt}%
\definecolor{currentstroke}{rgb}{0.000000,0.501961,0.000000}%
\pgfsetstrokecolor{currentstroke}%
\pgfsetstrokeopacity{0.200000}%
\pgfsetdash{}{0pt}%
\pgfpathmoveto{\pgfqpoint{0.677121in}{1.425771in}}%
\pgfpathlineto{\pgfqpoint{0.954899in}{1.425771in}}%
\pgfpathlineto{\pgfqpoint{0.954899in}{1.522993in}}%
\pgfpathlineto{\pgfqpoint{0.677121in}{1.522993in}}%
\pgfpathlineto{\pgfqpoint{0.677121in}{1.425771in}}%
\pgfpathclose%
\pgfusepath{stroke,fill}%
\end{pgfscope}%
\begin{pgfscope}%
\definecolor{textcolor}{rgb}{0.000000,0.000000,0.000000}%
\pgfsetstrokecolor{textcolor}%
\pgfsetfillcolor{textcolor}%
\pgftext[x=1.066010in,y=1.425771in,left,base]{\color{textcolor}{\rmfamily\fontsize{10.000000}{12.000000}\selectfont\catcode`\^=\active\def^{\ifmmode\sp\else\^{}\fi}\catcode`\%=\active\def%{\%}Bounds}}%
\end{pgfscope}%
\begin{pgfscope}%
\pgfsetbuttcap%
\pgfsetroundjoin%
\pgfsetlinewidth{1.505625pt}%
\definecolor{currentstroke}{rgb}{1.000000,0.000000,0.000000}%
\pgfsetstrokecolor{currentstroke}%
\pgfsetdash{{1.500000pt}{2.475000pt}}{0.000000pt}%
\pgfpathmoveto{\pgfqpoint{0.677121in}{1.280710in}}%
\pgfpathlineto{\pgfqpoint{0.816010in}{1.280710in}}%
\pgfpathlineto{\pgfqpoint{0.954899in}{1.280710in}}%
\pgfusepath{stroke}%
\end{pgfscope}%
\begin{pgfscope}%
\definecolor{textcolor}{rgb}{0.000000,0.000000,0.000000}%
\pgfsetstrokecolor{textcolor}%
\pgfsetfillcolor{textcolor}%
\pgftext[x=1.066010in,y=1.232098in,left,base]{\color{textcolor}{\rmfamily\fontsize{10.000000}{12.000000}\selectfont\catcode`\^=\active\def^{\ifmmode\sp\else\^{}\fi}\catcode`\%=\active\def%{\%}Lower Bound}}%
\end{pgfscope}%
\begin{pgfscope}%
\pgfsetbuttcap%
\pgfsetroundjoin%
\pgfsetlinewidth{1.505625pt}%
\definecolor{currentstroke}{rgb}{1.000000,0.000000,0.000000}%
\pgfsetstrokecolor{currentstroke}%
\pgfsetdash{{5.550000pt}{2.400000pt}}{0.000000pt}%
\pgfpathmoveto{\pgfqpoint{0.677121in}{1.087037in}}%
\pgfpathlineto{\pgfqpoint{0.816010in}{1.087037in}}%
\pgfpathlineto{\pgfqpoint{0.954899in}{1.087037in}}%
\pgfusepath{stroke}%
\end{pgfscope}%
\begin{pgfscope}%
\definecolor{textcolor}{rgb}{0.000000,0.000000,0.000000}%
\pgfsetstrokecolor{textcolor}%
\pgfsetfillcolor{textcolor}%
\pgftext[x=1.066010in,y=1.038426in,left,base]{\color{textcolor}{\rmfamily\fontsize{10.000000}{12.000000}\selectfont\catcode`\^=\active\def^{\ifmmode\sp\else\^{}\fi}\catcode`\%=\active\def%{\%}Upper Bound}}%
\end{pgfscope}%
\begin{pgfscope}%
\pgfsetrectcap%
\pgfsetroundjoin%
\pgfsetlinewidth{1.505625pt}%
\definecolor{currentstroke}{rgb}{0.121569,0.466667,0.705882}%
\pgfsetstrokecolor{currentstroke}%
\pgfsetdash{}{0pt}%
\pgfpathmoveto{\pgfqpoint{0.677121in}{0.893364in}}%
\pgfpathlineto{\pgfqpoint{0.816010in}{0.893364in}}%
\pgfpathlineto{\pgfqpoint{0.954899in}{0.893364in}}%
\pgfusepath{stroke}%
\end{pgfscope}%
\begin{pgfscope}%
\definecolor{textcolor}{rgb}{0.000000,0.000000,0.000000}%
\pgfsetstrokecolor{textcolor}%
\pgfsetfillcolor{textcolor}%
\pgftext[x=1.066010in,y=0.844753in,left,base]{\color{textcolor}{\rmfamily\fontsize{10.000000}{12.000000}\selectfont\catcode`\^=\active\def^{\ifmmode\sp\else\^{}\fi}\catcode`\%=\active\def%{\%}Relaxation Variable Value}}%
\end{pgfscope}%
\begin{pgfscope}%
\pgfsetrectcap%
\pgfsetroundjoin%
\pgfsetlinewidth{1.505625pt}%
\definecolor{currentstroke}{rgb}{1.000000,0.498039,0.054902}%
\pgfsetstrokecolor{currentstroke}%
\pgfsetdash{}{0pt}%
\pgfpathmoveto{\pgfqpoint{0.677121in}{0.699691in}}%
\pgfpathlineto{\pgfqpoint{0.816010in}{0.699691in}}%
\pgfpathlineto{\pgfqpoint{0.954899in}{0.699691in}}%
\pgfusepath{stroke}%
\end{pgfscope}%
\begin{pgfscope}%
\definecolor{textcolor}{rgb}{0.000000,0.000000,0.000000}%
\pgfsetstrokecolor{textcolor}%
\pgfsetfillcolor{textcolor}%
\pgftext[x=1.066010in,y=0.651080in,left,base]{\color{textcolor}{\rmfamily\fontsize{10.000000}{12.000000}\selectfont\catcode`\^=\active\def^{\ifmmode\sp\else\^{}\fi}\catcode`\%=\active\def%{\%}Actual Bilinear Value}}%
\end{pgfscope}%
\end{pgfpicture}%
\makeatother%
\endgroup%
}
		\caption{State transition approximation for \( n \) using bilinear term relaxation.}
		\label{fig:mccormick_problem}
	\end{subfigure}
	\hfill
	\begin{subfigure}[b]{0.45\textwidth}
		\centering
		\resizebox{\textwidth}{!}{%% Creator: Matplotlib, PGF backend
%%
%% To include the figure in your LaTeX document, write
%%   \input{<filename>.pgf}
%%
%% Make sure the required packages are loaded in your preamble
%%   \usepackage{pgf}
%%
%% Also ensure that all the required font packages are loaded; for instance,
%% the lmodern package is sometimes necessary when using math font.
%%   \usepackage{lmodern}
%%
%% Figures using additional raster images can only be included by \input if
%% they are in the same directory as the main LaTeX file. For loading figures
%% from other directories you can use the `import` package
%%   \usepackage{import}
%%
%% and then include the figures with
%%   \import{<path to file>}{<filename>.pgf}
%%
%% Matplotlib used the following preamble
%%   \def\mathdefault#1{#1}
%%   \everymath=\expandafter{\the\everymath\displaystyle}
%%   
%%   \ifdefined\pdftexversion\else  % non-pdftex case.
%%     \usepackage{fontspec}
%%   \fi
%%   \makeatletter\@ifpackageloaded{underscore}{}{\usepackage[strings]{underscore}}\makeatother
%%
\begingroup%
\makeatletter%
\begin{pgfpicture}%
\pgfpathrectangle{\pgfpointorigin}{\pgfqpoint{5.612121in}{4.295074in}}%
\pgfusepath{use as bounding box, clip}%
\begin{pgfscope}%
\pgfsetbuttcap%
\pgfsetmiterjoin%
\definecolor{currentfill}{rgb}{1.000000,1.000000,1.000000}%
\pgfsetfillcolor{currentfill}%
\pgfsetlinewidth{0.000000pt}%
\definecolor{currentstroke}{rgb}{1.000000,1.000000,1.000000}%
\pgfsetstrokecolor{currentstroke}%
\pgfsetdash{}{0pt}%
\pgfpathmoveto{\pgfqpoint{0.000000in}{0.000000in}}%
\pgfpathlineto{\pgfqpoint{5.612121in}{0.000000in}}%
\pgfpathlineto{\pgfqpoint{5.612121in}{4.295074in}}%
\pgfpathlineto{\pgfqpoint{0.000000in}{4.295074in}}%
\pgfpathlineto{\pgfqpoint{0.000000in}{0.000000in}}%
\pgfpathclose%
\pgfusepath{fill}%
\end{pgfscope}%
\begin{pgfscope}%
\pgfsetbuttcap%
\pgfsetmiterjoin%
\definecolor{currentfill}{rgb}{1.000000,1.000000,1.000000}%
\pgfsetfillcolor{currentfill}%
\pgfsetlinewidth{0.000000pt}%
\definecolor{currentstroke}{rgb}{0.000000,0.000000,0.000000}%
\pgfsetstrokecolor{currentstroke}%
\pgfsetstrokeopacity{0.000000}%
\pgfsetdash{}{0pt}%
\pgfpathmoveto{\pgfqpoint{0.552121in}{0.499074in}}%
\pgfpathlineto{\pgfqpoint{5.512121in}{0.499074in}}%
\pgfpathlineto{\pgfqpoint{5.512121in}{4.195074in}}%
\pgfpathlineto{\pgfqpoint{0.552121in}{4.195074in}}%
\pgfpathlineto{\pgfqpoint{0.552121in}{0.499074in}}%
\pgfpathclose%
\pgfusepath{fill}%
\end{pgfscope}%
\begin{pgfscope}%
\pgfpathrectangle{\pgfqpoint{0.552121in}{0.499074in}}{\pgfqpoint{4.960000in}{3.696000in}}%
\pgfusepath{clip}%
\pgfsetbuttcap%
\pgfsetroundjoin%
\definecolor{currentfill}{rgb}{0.000000,0.501961,0.000000}%
\pgfsetfillcolor{currentfill}%
\pgfsetfillopacity{0.200000}%
\pgfsetlinewidth{1.003750pt}%
\definecolor{currentstroke}{rgb}{0.000000,0.501961,0.000000}%
\pgfsetstrokecolor{currentstroke}%
\pgfsetstrokeopacity{0.200000}%
\pgfsetdash{}{0pt}%
\pgfsys@defobject{currentmarker}{\pgfqpoint{0.777575in}{0.667074in}}{\pgfqpoint{5.286666in}{4.027074in}}{%
\pgfpathmoveto{\pgfqpoint{0.777575in}{2.312603in}}%
\pgfpathlineto{\pgfqpoint{0.777575in}{2.312603in}}%
\pgfpathlineto{\pgfqpoint{0.852727in}{2.284603in}}%
\pgfpathlineto{\pgfqpoint{0.927878in}{2.256599in}}%
\pgfpathlineto{\pgfqpoint{1.003030in}{2.228599in}}%
\pgfpathlineto{\pgfqpoint{1.078181in}{2.200598in}}%
\pgfpathlineto{\pgfqpoint{1.153333in}{2.172599in}}%
\pgfpathlineto{\pgfqpoint{1.228484in}{2.144601in}}%
\pgfpathlineto{\pgfqpoint{1.303636in}{2.116603in}}%
\pgfpathlineto{\pgfqpoint{1.378788in}{2.088608in}}%
\pgfpathlineto{\pgfqpoint{1.453939in}{2.060614in}}%
\pgfpathlineto{\pgfqpoint{1.529091in}{2.032624in}}%
\pgfpathlineto{\pgfqpoint{1.604242in}{2.004637in}}%
\pgfpathlineto{\pgfqpoint{1.679394in}{1.976656in}}%
\pgfpathlineto{\pgfqpoint{1.754545in}{1.948683in}}%
\pgfpathlineto{\pgfqpoint{1.829697in}{1.920720in}}%
\pgfpathlineto{\pgfqpoint{1.904848in}{1.892770in}}%
\pgfpathlineto{\pgfqpoint{1.980000in}{1.864839in}}%
\pgfpathlineto{\pgfqpoint{2.055151in}{1.836929in}}%
\pgfpathlineto{\pgfqpoint{2.130303in}{1.809037in}}%
\pgfpathlineto{\pgfqpoint{2.205454in}{1.781166in}}%
\pgfpathlineto{\pgfqpoint{2.280606in}{1.753318in}}%
\pgfpathlineto{\pgfqpoint{2.355757in}{1.725496in}}%
\pgfpathlineto{\pgfqpoint{2.430909in}{1.697701in}}%
\pgfpathlineto{\pgfqpoint{2.506060in}{1.669938in}}%
\pgfpathlineto{\pgfqpoint{2.581212in}{1.642207in}}%
\pgfpathlineto{\pgfqpoint{2.656363in}{1.614513in}}%
\pgfpathlineto{\pgfqpoint{2.731515in}{1.586857in}}%
\pgfpathlineto{\pgfqpoint{2.806666in}{1.559242in}}%
\pgfpathlineto{\pgfqpoint{2.881818in}{1.531670in}}%
\pgfpathlineto{\pgfqpoint{2.956969in}{1.504144in}}%
\pgfpathlineto{\pgfqpoint{3.032121in}{1.476665in}}%
\pgfpathlineto{\pgfqpoint{3.107272in}{1.449237in}}%
\pgfpathlineto{\pgfqpoint{3.182424in}{1.421859in}}%
\pgfpathlineto{\pgfqpoint{3.257575in}{1.394534in}}%
\pgfpathlineto{\pgfqpoint{3.332727in}{1.367263in}}%
\pgfpathlineto{\pgfqpoint{3.407878in}{1.340046in}}%
\pgfpathlineto{\pgfqpoint{3.483030in}{1.312884in}}%
\pgfpathlineto{\pgfqpoint{3.558181in}{1.285776in}}%
\pgfpathlineto{\pgfqpoint{3.633333in}{1.258722in}}%
\pgfpathlineto{\pgfqpoint{3.708484in}{1.231720in}}%
\pgfpathlineto{\pgfqpoint{3.783636in}{1.204769in}}%
\pgfpathlineto{\pgfqpoint{3.858788in}{1.177866in}}%
\pgfpathlineto{\pgfqpoint{3.933939in}{1.151009in}}%
\pgfpathlineto{\pgfqpoint{4.009091in}{1.124194in}}%
\pgfpathlineto{\pgfqpoint{4.084242in}{1.097416in}}%
\pgfpathlineto{\pgfqpoint{4.159394in}{1.070670in}}%
\pgfpathlineto{\pgfqpoint{4.234545in}{1.043951in}}%
\pgfpathlineto{\pgfqpoint{4.309697in}{1.017251in}}%
\pgfpathlineto{\pgfqpoint{4.384848in}{0.990562in}}%
\pgfpathlineto{\pgfqpoint{4.460000in}{0.963877in}}%
\pgfpathlineto{\pgfqpoint{4.535151in}{0.937185in}}%
\pgfpathlineto{\pgfqpoint{4.610303in}{0.910476in}}%
\pgfpathlineto{\pgfqpoint{4.685454in}{0.883736in}}%
\pgfpathlineto{\pgfqpoint{4.760606in}{0.856947in}}%
\pgfpathlineto{\pgfqpoint{4.835757in}{0.830098in}}%
\pgfpathlineto{\pgfqpoint{4.910909in}{0.803176in}}%
\pgfpathlineto{\pgfqpoint{4.986060in}{0.776172in}}%
\pgfpathlineto{\pgfqpoint{5.061212in}{0.749071in}}%
\pgfpathlineto{\pgfqpoint{5.136363in}{0.721863in}}%
\pgfpathlineto{\pgfqpoint{5.211515in}{0.694535in}}%
\pgfpathlineto{\pgfqpoint{5.286666in}{0.667074in}}%
\pgfpathlineto{\pgfqpoint{5.286666in}{4.027074in}}%
\pgfpathlineto{\pgfqpoint{5.286666in}{4.027074in}}%
\pgfpathlineto{\pgfqpoint{5.211515in}{3.998536in}}%
\pgfpathlineto{\pgfqpoint{5.136363in}{3.969866in}}%
\pgfpathlineto{\pgfqpoint{5.061212in}{3.941075in}}%
\pgfpathlineto{\pgfqpoint{4.986060in}{3.912176in}}%
\pgfpathlineto{\pgfqpoint{4.910909in}{3.883182in}}%
\pgfpathlineto{\pgfqpoint{4.835757in}{3.854104in}}%
\pgfpathlineto{\pgfqpoint{4.760606in}{3.824954in}}%
\pgfpathlineto{\pgfqpoint{4.685454in}{3.795744in}}%
\pgfpathlineto{\pgfqpoint{4.610303in}{3.766484in}}%
\pgfpathlineto{\pgfqpoint{4.535151in}{3.737194in}}%
\pgfpathlineto{\pgfqpoint{4.460000in}{3.707887in}}%
\pgfpathlineto{\pgfqpoint{4.384848in}{3.678572in}}%
\pgfpathlineto{\pgfqpoint{4.309697in}{3.649261in}}%
\pgfpathlineto{\pgfqpoint{4.234545in}{3.619961in}}%
\pgfpathlineto{\pgfqpoint{4.159394in}{3.590681in}}%
\pgfpathlineto{\pgfqpoint{4.084242in}{3.561427in}}%
\pgfpathlineto{\pgfqpoint{4.009091in}{3.532205in}}%
\pgfpathlineto{\pgfqpoint{3.933939in}{3.503021in}}%
\pgfpathlineto{\pgfqpoint{3.858788in}{3.473878in}}%
\pgfpathlineto{\pgfqpoint{3.783636in}{3.444782in}}%
\pgfpathlineto{\pgfqpoint{3.708484in}{3.415733in}}%
\pgfpathlineto{\pgfqpoint{3.633333in}{3.386736in}}%
\pgfpathlineto{\pgfqpoint{3.558181in}{3.357791in}}%
\pgfpathlineto{\pgfqpoint{3.483030in}{3.328899in}}%
\pgfpathlineto{\pgfqpoint{3.407878in}{3.300063in}}%
\pgfpathlineto{\pgfqpoint{3.332727in}{3.271281in}}%
\pgfpathlineto{\pgfqpoint{3.257575in}{3.242553in}}%
\pgfpathlineto{\pgfqpoint{3.182424in}{3.213879in}}%
\pgfpathlineto{\pgfqpoint{3.107272in}{3.185257in}}%
\pgfpathlineto{\pgfqpoint{3.032121in}{3.156687in}}%
\pgfpathlineto{\pgfqpoint{2.956969in}{3.128166in}}%
\pgfpathlineto{\pgfqpoint{2.881818in}{3.099692in}}%
\pgfpathlineto{\pgfqpoint{2.806666in}{3.071265in}}%
\pgfpathlineto{\pgfqpoint{2.731515in}{3.042880in}}%
\pgfpathlineto{\pgfqpoint{2.656363in}{3.014536in}}%
\pgfpathlineto{\pgfqpoint{2.581212in}{2.986231in}}%
\pgfpathlineto{\pgfqpoint{2.506060in}{2.957961in}}%
\pgfpathlineto{\pgfqpoint{2.430909in}{2.929724in}}%
\pgfpathlineto{\pgfqpoint{2.355757in}{2.901518in}}%
\pgfpathlineto{\pgfqpoint{2.280606in}{2.873340in}}%
\pgfpathlineto{\pgfqpoint{2.205454in}{2.845187in}}%
\pgfpathlineto{\pgfqpoint{2.130303in}{2.817057in}}%
\pgfpathlineto{\pgfqpoint{2.055151in}{2.788947in}}%
\pgfpathlineto{\pgfqpoint{1.980000in}{2.760856in}}%
\pgfpathlineto{\pgfqpoint{1.904848in}{2.732786in}}%
\pgfpathlineto{\pgfqpoint{1.829697in}{2.704734in}}%
\pgfpathlineto{\pgfqpoint{1.754545in}{2.676696in}}%
\pgfpathlineto{\pgfqpoint{1.679394in}{2.648668in}}%
\pgfpathlineto{\pgfqpoint{1.604242in}{2.620648in}}%
\pgfpathlineto{\pgfqpoint{1.529091in}{2.592634in}}%
\pgfpathlineto{\pgfqpoint{1.453939in}{2.564624in}}%
\pgfpathlineto{\pgfqpoint{1.378788in}{2.536617in}}%
\pgfpathlineto{\pgfqpoint{1.303636in}{2.508613in}}%
\pgfpathlineto{\pgfqpoint{1.228484in}{2.480610in}}%
\pgfpathlineto{\pgfqpoint{1.153333in}{2.452609in}}%
\pgfpathlineto{\pgfqpoint{1.078181in}{2.424608in}}%
\pgfpathlineto{\pgfqpoint{1.003030in}{2.396607in}}%
\pgfpathlineto{\pgfqpoint{0.927878in}{2.368607in}}%
\pgfpathlineto{\pgfqpoint{0.852727in}{2.340603in}}%
\pgfpathlineto{\pgfqpoint{0.777575in}{2.312603in}}%
\pgfpathlineto{\pgfqpoint{0.777575in}{2.312603in}}%
\pgfpathclose%
\pgfusepath{stroke,fill}%
}%
\begin{pgfscope}%
\pgfsys@transformshift{0.000000in}{0.000000in}%
\pgfsys@useobject{currentmarker}{}%
\end{pgfscope}%
\end{pgfscope}%
\begin{pgfscope}%
\pgfpathrectangle{\pgfqpoint{0.552121in}{0.499074in}}{\pgfqpoint{4.960000in}{3.696000in}}%
\pgfusepath{clip}%
\pgfsetbuttcap%
\pgfsetroundjoin%
\pgfsetlinewidth{0.803000pt}%
\definecolor{currentstroke}{rgb}{0.501961,0.501961,0.501961}%
\pgfsetstrokecolor{currentstroke}%
\pgfsetstrokeopacity{0.700000}%
\pgfsetdash{{0.800000pt}{1.320000pt}}{0.000000pt}%
\pgfpathmoveto{\pgfqpoint{0.777575in}{0.499074in}}%
\pgfpathlineto{\pgfqpoint{0.777575in}{4.195074in}}%
\pgfusepath{stroke}%
\end{pgfscope}%
\begin{pgfscope}%
\pgfsetbuttcap%
\pgfsetroundjoin%
\definecolor{currentfill}{rgb}{0.000000,0.000000,0.000000}%
\pgfsetfillcolor{currentfill}%
\pgfsetlinewidth{0.803000pt}%
\definecolor{currentstroke}{rgb}{0.000000,0.000000,0.000000}%
\pgfsetstrokecolor{currentstroke}%
\pgfsetdash{}{0pt}%
\pgfsys@defobject{currentmarker}{\pgfqpoint{0.000000in}{-0.048611in}}{\pgfqpoint{0.000000in}{0.000000in}}{%
\pgfpathmoveto{\pgfqpoint{0.000000in}{0.000000in}}%
\pgfpathlineto{\pgfqpoint{0.000000in}{-0.048611in}}%
\pgfusepath{stroke,fill}%
}%
\begin{pgfscope}%
\pgfsys@transformshift{0.777575in}{0.499074in}%
\pgfsys@useobject{currentmarker}{}%
\end{pgfscope}%
\end{pgfscope}%
\begin{pgfscope}%
\definecolor{textcolor}{rgb}{0.000000,0.000000,0.000000}%
\pgfsetstrokecolor{textcolor}%
\pgfsetfillcolor{textcolor}%
\pgftext[x=0.777575in,y=0.401852in,,top]{\color{textcolor}{\rmfamily\fontsize{9.000000}{10.800000}\selectfont\catcode`\^=\active\def^{\ifmmode\sp\else\^{}\fi}\catcode`\%=\active\def%{\%}$\mathdefault{0.0}$}}%
\end{pgfscope}%
\begin{pgfscope}%
\pgfpathrectangle{\pgfqpoint{0.552121in}{0.499074in}}{\pgfqpoint{4.960000in}{3.696000in}}%
\pgfusepath{clip}%
\pgfsetbuttcap%
\pgfsetroundjoin%
\pgfsetlinewidth{0.803000pt}%
\definecolor{currentstroke}{rgb}{0.501961,0.501961,0.501961}%
\pgfsetstrokecolor{currentstroke}%
\pgfsetstrokeopacity{0.700000}%
\pgfsetdash{{0.800000pt}{1.320000pt}}{0.000000pt}%
\pgfpathmoveto{\pgfqpoint{1.679394in}{0.499074in}}%
\pgfpathlineto{\pgfqpoint{1.679394in}{4.195074in}}%
\pgfusepath{stroke}%
\end{pgfscope}%
\begin{pgfscope}%
\pgfsetbuttcap%
\pgfsetroundjoin%
\definecolor{currentfill}{rgb}{0.000000,0.000000,0.000000}%
\pgfsetfillcolor{currentfill}%
\pgfsetlinewidth{0.803000pt}%
\definecolor{currentstroke}{rgb}{0.000000,0.000000,0.000000}%
\pgfsetstrokecolor{currentstroke}%
\pgfsetdash{}{0pt}%
\pgfsys@defobject{currentmarker}{\pgfqpoint{0.000000in}{-0.048611in}}{\pgfqpoint{0.000000in}{0.000000in}}{%
\pgfpathmoveto{\pgfqpoint{0.000000in}{0.000000in}}%
\pgfpathlineto{\pgfqpoint{0.000000in}{-0.048611in}}%
\pgfusepath{stroke,fill}%
}%
\begin{pgfscope}%
\pgfsys@transformshift{1.679394in}{0.499074in}%
\pgfsys@useobject{currentmarker}{}%
\end{pgfscope}%
\end{pgfscope}%
\begin{pgfscope}%
\definecolor{textcolor}{rgb}{0.000000,0.000000,0.000000}%
\pgfsetstrokecolor{textcolor}%
\pgfsetfillcolor{textcolor}%
\pgftext[x=1.679394in,y=0.401852in,,top]{\color{textcolor}{\rmfamily\fontsize{9.000000}{10.800000}\selectfont\catcode`\^=\active\def^{\ifmmode\sp\else\^{}\fi}\catcode`\%=\active\def%{\%}$\mathdefault{0.2}$}}%
\end{pgfscope}%
\begin{pgfscope}%
\pgfpathrectangle{\pgfqpoint{0.552121in}{0.499074in}}{\pgfqpoint{4.960000in}{3.696000in}}%
\pgfusepath{clip}%
\pgfsetbuttcap%
\pgfsetroundjoin%
\pgfsetlinewidth{0.803000pt}%
\definecolor{currentstroke}{rgb}{0.501961,0.501961,0.501961}%
\pgfsetstrokecolor{currentstroke}%
\pgfsetstrokeopacity{0.700000}%
\pgfsetdash{{0.800000pt}{1.320000pt}}{0.000000pt}%
\pgfpathmoveto{\pgfqpoint{2.581212in}{0.499074in}}%
\pgfpathlineto{\pgfqpoint{2.581212in}{4.195074in}}%
\pgfusepath{stroke}%
\end{pgfscope}%
\begin{pgfscope}%
\pgfsetbuttcap%
\pgfsetroundjoin%
\definecolor{currentfill}{rgb}{0.000000,0.000000,0.000000}%
\pgfsetfillcolor{currentfill}%
\pgfsetlinewidth{0.803000pt}%
\definecolor{currentstroke}{rgb}{0.000000,0.000000,0.000000}%
\pgfsetstrokecolor{currentstroke}%
\pgfsetdash{}{0pt}%
\pgfsys@defobject{currentmarker}{\pgfqpoint{0.000000in}{-0.048611in}}{\pgfqpoint{0.000000in}{0.000000in}}{%
\pgfpathmoveto{\pgfqpoint{0.000000in}{0.000000in}}%
\pgfpathlineto{\pgfqpoint{0.000000in}{-0.048611in}}%
\pgfusepath{stroke,fill}%
}%
\begin{pgfscope}%
\pgfsys@transformshift{2.581212in}{0.499074in}%
\pgfsys@useobject{currentmarker}{}%
\end{pgfscope}%
\end{pgfscope}%
\begin{pgfscope}%
\definecolor{textcolor}{rgb}{0.000000,0.000000,0.000000}%
\pgfsetstrokecolor{textcolor}%
\pgfsetfillcolor{textcolor}%
\pgftext[x=2.581212in,y=0.401852in,,top]{\color{textcolor}{\rmfamily\fontsize{9.000000}{10.800000}\selectfont\catcode`\^=\active\def^{\ifmmode\sp\else\^{}\fi}\catcode`\%=\active\def%{\%}$\mathdefault{0.4}$}}%
\end{pgfscope}%
\begin{pgfscope}%
\pgfpathrectangle{\pgfqpoint{0.552121in}{0.499074in}}{\pgfqpoint{4.960000in}{3.696000in}}%
\pgfusepath{clip}%
\pgfsetbuttcap%
\pgfsetroundjoin%
\pgfsetlinewidth{0.803000pt}%
\definecolor{currentstroke}{rgb}{0.501961,0.501961,0.501961}%
\pgfsetstrokecolor{currentstroke}%
\pgfsetstrokeopacity{0.700000}%
\pgfsetdash{{0.800000pt}{1.320000pt}}{0.000000pt}%
\pgfpathmoveto{\pgfqpoint{3.483030in}{0.499074in}}%
\pgfpathlineto{\pgfqpoint{3.483030in}{4.195074in}}%
\pgfusepath{stroke}%
\end{pgfscope}%
\begin{pgfscope}%
\pgfsetbuttcap%
\pgfsetroundjoin%
\definecolor{currentfill}{rgb}{0.000000,0.000000,0.000000}%
\pgfsetfillcolor{currentfill}%
\pgfsetlinewidth{0.803000pt}%
\definecolor{currentstroke}{rgb}{0.000000,0.000000,0.000000}%
\pgfsetstrokecolor{currentstroke}%
\pgfsetdash{}{0pt}%
\pgfsys@defobject{currentmarker}{\pgfqpoint{0.000000in}{-0.048611in}}{\pgfqpoint{0.000000in}{0.000000in}}{%
\pgfpathmoveto{\pgfqpoint{0.000000in}{0.000000in}}%
\pgfpathlineto{\pgfqpoint{0.000000in}{-0.048611in}}%
\pgfusepath{stroke,fill}%
}%
\begin{pgfscope}%
\pgfsys@transformshift{3.483030in}{0.499074in}%
\pgfsys@useobject{currentmarker}{}%
\end{pgfscope}%
\end{pgfscope}%
\begin{pgfscope}%
\definecolor{textcolor}{rgb}{0.000000,0.000000,0.000000}%
\pgfsetstrokecolor{textcolor}%
\pgfsetfillcolor{textcolor}%
\pgftext[x=3.483030in,y=0.401852in,,top]{\color{textcolor}{\rmfamily\fontsize{9.000000}{10.800000}\selectfont\catcode`\^=\active\def^{\ifmmode\sp\else\^{}\fi}\catcode`\%=\active\def%{\%}$\mathdefault{0.6}$}}%
\end{pgfscope}%
\begin{pgfscope}%
\pgfpathrectangle{\pgfqpoint{0.552121in}{0.499074in}}{\pgfqpoint{4.960000in}{3.696000in}}%
\pgfusepath{clip}%
\pgfsetbuttcap%
\pgfsetroundjoin%
\pgfsetlinewidth{0.803000pt}%
\definecolor{currentstroke}{rgb}{0.501961,0.501961,0.501961}%
\pgfsetstrokecolor{currentstroke}%
\pgfsetstrokeopacity{0.700000}%
\pgfsetdash{{0.800000pt}{1.320000pt}}{0.000000pt}%
\pgfpathmoveto{\pgfqpoint{4.384848in}{0.499074in}}%
\pgfpathlineto{\pgfqpoint{4.384848in}{4.195074in}}%
\pgfusepath{stroke}%
\end{pgfscope}%
\begin{pgfscope}%
\pgfsetbuttcap%
\pgfsetroundjoin%
\definecolor{currentfill}{rgb}{0.000000,0.000000,0.000000}%
\pgfsetfillcolor{currentfill}%
\pgfsetlinewidth{0.803000pt}%
\definecolor{currentstroke}{rgb}{0.000000,0.000000,0.000000}%
\pgfsetstrokecolor{currentstroke}%
\pgfsetdash{}{0pt}%
\pgfsys@defobject{currentmarker}{\pgfqpoint{0.000000in}{-0.048611in}}{\pgfqpoint{0.000000in}{0.000000in}}{%
\pgfpathmoveto{\pgfqpoint{0.000000in}{0.000000in}}%
\pgfpathlineto{\pgfqpoint{0.000000in}{-0.048611in}}%
\pgfusepath{stroke,fill}%
}%
\begin{pgfscope}%
\pgfsys@transformshift{4.384848in}{0.499074in}%
\pgfsys@useobject{currentmarker}{}%
\end{pgfscope}%
\end{pgfscope}%
\begin{pgfscope}%
\definecolor{textcolor}{rgb}{0.000000,0.000000,0.000000}%
\pgfsetstrokecolor{textcolor}%
\pgfsetfillcolor{textcolor}%
\pgftext[x=4.384848in,y=0.401852in,,top]{\color{textcolor}{\rmfamily\fontsize{9.000000}{10.800000}\selectfont\catcode`\^=\active\def^{\ifmmode\sp\else\^{}\fi}\catcode`\%=\active\def%{\%}$\mathdefault{0.8}$}}%
\end{pgfscope}%
\begin{pgfscope}%
\pgfpathrectangle{\pgfqpoint{0.552121in}{0.499074in}}{\pgfqpoint{4.960000in}{3.696000in}}%
\pgfusepath{clip}%
\pgfsetbuttcap%
\pgfsetroundjoin%
\pgfsetlinewidth{0.803000pt}%
\definecolor{currentstroke}{rgb}{0.501961,0.501961,0.501961}%
\pgfsetstrokecolor{currentstroke}%
\pgfsetstrokeopacity{0.700000}%
\pgfsetdash{{0.800000pt}{1.320000pt}}{0.000000pt}%
\pgfpathmoveto{\pgfqpoint{5.286666in}{0.499074in}}%
\pgfpathlineto{\pgfqpoint{5.286666in}{4.195074in}}%
\pgfusepath{stroke}%
\end{pgfscope}%
\begin{pgfscope}%
\pgfsetbuttcap%
\pgfsetroundjoin%
\definecolor{currentfill}{rgb}{0.000000,0.000000,0.000000}%
\pgfsetfillcolor{currentfill}%
\pgfsetlinewidth{0.803000pt}%
\definecolor{currentstroke}{rgb}{0.000000,0.000000,0.000000}%
\pgfsetstrokecolor{currentstroke}%
\pgfsetdash{}{0pt}%
\pgfsys@defobject{currentmarker}{\pgfqpoint{0.000000in}{-0.048611in}}{\pgfqpoint{0.000000in}{0.000000in}}{%
\pgfpathmoveto{\pgfqpoint{0.000000in}{0.000000in}}%
\pgfpathlineto{\pgfqpoint{0.000000in}{-0.048611in}}%
\pgfusepath{stroke,fill}%
}%
\begin{pgfscope}%
\pgfsys@transformshift{5.286666in}{0.499074in}%
\pgfsys@useobject{currentmarker}{}%
\end{pgfscope}%
\end{pgfscope}%
\begin{pgfscope}%
\definecolor{textcolor}{rgb}{0.000000,0.000000,0.000000}%
\pgfsetstrokecolor{textcolor}%
\pgfsetfillcolor{textcolor}%
\pgftext[x=5.286666in,y=0.401852in,,top]{\color{textcolor}{\rmfamily\fontsize{9.000000}{10.800000}\selectfont\catcode`\^=\active\def^{\ifmmode\sp\else\^{}\fi}\catcode`\%=\active\def%{\%}$\mathdefault{1.0}$}}%
\end{pgfscope}%
\begin{pgfscope}%
\definecolor{textcolor}{rgb}{0.000000,0.000000,0.000000}%
\pgfsetstrokecolor{textcolor}%
\pgfsetfillcolor{textcolor}%
\pgftext[x=3.032121in,y=0.235185in,,top]{\color{textcolor}{\rmfamily\fontsize{11.000000}{13.200000}\selectfont\catcode`\^=\active\def^{\ifmmode\sp\else\^{}\fi}\catcode`\%=\active\def%{\%}Time [s]}}%
\end{pgfscope}%
\begin{pgfscope}%
\pgfpathrectangle{\pgfqpoint{0.552121in}{0.499074in}}{\pgfqpoint{4.960000in}{3.696000in}}%
\pgfusepath{clip}%
\pgfsetbuttcap%
\pgfsetroundjoin%
\pgfsetlinewidth{0.803000pt}%
\definecolor{currentstroke}{rgb}{0.501961,0.501961,0.501961}%
\pgfsetstrokecolor{currentstroke}%
\pgfsetstrokeopacity{0.700000}%
\pgfsetdash{{0.800000pt}{1.320000pt}}{0.000000pt}%
\pgfpathmoveto{\pgfqpoint{0.552121in}{0.886570in}}%
\pgfpathlineto{\pgfqpoint{5.512121in}{0.886570in}}%
\pgfusepath{stroke}%
\end{pgfscope}%
\begin{pgfscope}%
\pgfsetbuttcap%
\pgfsetroundjoin%
\definecolor{currentfill}{rgb}{0.000000,0.000000,0.000000}%
\pgfsetfillcolor{currentfill}%
\pgfsetlinewidth{0.803000pt}%
\definecolor{currentstroke}{rgb}{0.000000,0.000000,0.000000}%
\pgfsetstrokecolor{currentstroke}%
\pgfsetdash{}{0pt}%
\pgfsys@defobject{currentmarker}{\pgfqpoint{-0.048611in}{0.000000in}}{\pgfqpoint{-0.000000in}{0.000000in}}{%
\pgfpathmoveto{\pgfqpoint{-0.000000in}{0.000000in}}%
\pgfpathlineto{\pgfqpoint{-0.048611in}{0.000000in}}%
\pgfusepath{stroke,fill}%
}%
\begin{pgfscope}%
\pgfsys@transformshift{0.552121in}{0.886570in}%
\pgfsys@useobject{currentmarker}{}%
\end{pgfscope}%
\end{pgfscope}%
\begin{pgfscope}%
\definecolor{textcolor}{rgb}{0.000000,0.000000,0.000000}%
\pgfsetstrokecolor{textcolor}%
\pgfsetfillcolor{textcolor}%
\pgftext[x=0.290741in, y=0.843167in, left, base]{\color{textcolor}{\rmfamily\fontsize{9.000000}{10.800000}\selectfont\catcode`\^=\active\def^{\ifmmode\sp\else\^{}\fi}\catcode`\%=\active\def%{\%}$\mathdefault{\ensuremath{-}2}$}}%
\end{pgfscope}%
\begin{pgfscope}%
\pgfpathrectangle{\pgfqpoint{0.552121in}{0.499074in}}{\pgfqpoint{4.960000in}{3.696000in}}%
\pgfusepath{clip}%
\pgfsetbuttcap%
\pgfsetroundjoin%
\pgfsetlinewidth{0.803000pt}%
\definecolor{currentstroke}{rgb}{0.501961,0.501961,0.501961}%
\pgfsetstrokecolor{currentstroke}%
\pgfsetstrokeopacity{0.700000}%
\pgfsetdash{{0.800000pt}{1.320000pt}}{0.000000pt}%
\pgfpathmoveto{\pgfqpoint{0.552121in}{1.599586in}}%
\pgfpathlineto{\pgfqpoint{5.512121in}{1.599586in}}%
\pgfusepath{stroke}%
\end{pgfscope}%
\begin{pgfscope}%
\pgfsetbuttcap%
\pgfsetroundjoin%
\definecolor{currentfill}{rgb}{0.000000,0.000000,0.000000}%
\pgfsetfillcolor{currentfill}%
\pgfsetlinewidth{0.803000pt}%
\definecolor{currentstroke}{rgb}{0.000000,0.000000,0.000000}%
\pgfsetstrokecolor{currentstroke}%
\pgfsetdash{}{0pt}%
\pgfsys@defobject{currentmarker}{\pgfqpoint{-0.048611in}{0.000000in}}{\pgfqpoint{-0.000000in}{0.000000in}}{%
\pgfpathmoveto{\pgfqpoint{-0.000000in}{0.000000in}}%
\pgfpathlineto{\pgfqpoint{-0.048611in}{0.000000in}}%
\pgfusepath{stroke,fill}%
}%
\begin{pgfscope}%
\pgfsys@transformshift{0.552121in}{1.599586in}%
\pgfsys@useobject{currentmarker}{}%
\end{pgfscope}%
\end{pgfscope}%
\begin{pgfscope}%
\definecolor{textcolor}{rgb}{0.000000,0.000000,0.000000}%
\pgfsetstrokecolor{textcolor}%
\pgfsetfillcolor{textcolor}%
\pgftext[x=0.290741in, y=1.556183in, left, base]{\color{textcolor}{\rmfamily\fontsize{9.000000}{10.800000}\selectfont\catcode`\^=\active\def^{\ifmmode\sp\else\^{}\fi}\catcode`\%=\active\def%{\%}$\mathdefault{\ensuremath{-}1}$}}%
\end{pgfscope}%
\begin{pgfscope}%
\pgfpathrectangle{\pgfqpoint{0.552121in}{0.499074in}}{\pgfqpoint{4.960000in}{3.696000in}}%
\pgfusepath{clip}%
\pgfsetbuttcap%
\pgfsetroundjoin%
\pgfsetlinewidth{0.803000pt}%
\definecolor{currentstroke}{rgb}{0.501961,0.501961,0.501961}%
\pgfsetstrokecolor{currentstroke}%
\pgfsetstrokeopacity{0.700000}%
\pgfsetdash{{0.800000pt}{1.320000pt}}{0.000000pt}%
\pgfpathmoveto{\pgfqpoint{0.552121in}{2.312603in}}%
\pgfpathlineto{\pgfqpoint{5.512121in}{2.312603in}}%
\pgfusepath{stroke}%
\end{pgfscope}%
\begin{pgfscope}%
\pgfsetbuttcap%
\pgfsetroundjoin%
\definecolor{currentfill}{rgb}{0.000000,0.000000,0.000000}%
\pgfsetfillcolor{currentfill}%
\pgfsetlinewidth{0.803000pt}%
\definecolor{currentstroke}{rgb}{0.000000,0.000000,0.000000}%
\pgfsetstrokecolor{currentstroke}%
\pgfsetdash{}{0pt}%
\pgfsys@defobject{currentmarker}{\pgfqpoint{-0.048611in}{0.000000in}}{\pgfqpoint{-0.000000in}{0.000000in}}{%
\pgfpathmoveto{\pgfqpoint{-0.000000in}{0.000000in}}%
\pgfpathlineto{\pgfqpoint{-0.048611in}{0.000000in}}%
\pgfusepath{stroke,fill}%
}%
\begin{pgfscope}%
\pgfsys@transformshift{0.552121in}{2.312603in}%
\pgfsys@useobject{currentmarker}{}%
\end{pgfscope}%
\end{pgfscope}%
\begin{pgfscope}%
\definecolor{textcolor}{rgb}{0.000000,0.000000,0.000000}%
\pgfsetstrokecolor{textcolor}%
\pgfsetfillcolor{textcolor}%
\pgftext[x=0.390663in, y=2.269200in, left, base]{\color{textcolor}{\rmfamily\fontsize{9.000000}{10.800000}\selectfont\catcode`\^=\active\def^{\ifmmode\sp\else\^{}\fi}\catcode`\%=\active\def%{\%}$\mathdefault{0}$}}%
\end{pgfscope}%
\begin{pgfscope}%
\pgfpathrectangle{\pgfqpoint{0.552121in}{0.499074in}}{\pgfqpoint{4.960000in}{3.696000in}}%
\pgfusepath{clip}%
\pgfsetbuttcap%
\pgfsetroundjoin%
\pgfsetlinewidth{0.803000pt}%
\definecolor{currentstroke}{rgb}{0.501961,0.501961,0.501961}%
\pgfsetstrokecolor{currentstroke}%
\pgfsetstrokeopacity{0.700000}%
\pgfsetdash{{0.800000pt}{1.320000pt}}{0.000000pt}%
\pgfpathmoveto{\pgfqpoint{0.552121in}{3.025619in}}%
\pgfpathlineto{\pgfqpoint{5.512121in}{3.025619in}}%
\pgfusepath{stroke}%
\end{pgfscope}%
\begin{pgfscope}%
\pgfsetbuttcap%
\pgfsetroundjoin%
\definecolor{currentfill}{rgb}{0.000000,0.000000,0.000000}%
\pgfsetfillcolor{currentfill}%
\pgfsetlinewidth{0.803000pt}%
\definecolor{currentstroke}{rgb}{0.000000,0.000000,0.000000}%
\pgfsetstrokecolor{currentstroke}%
\pgfsetdash{}{0pt}%
\pgfsys@defobject{currentmarker}{\pgfqpoint{-0.048611in}{0.000000in}}{\pgfqpoint{-0.000000in}{0.000000in}}{%
\pgfpathmoveto{\pgfqpoint{-0.000000in}{0.000000in}}%
\pgfpathlineto{\pgfqpoint{-0.048611in}{0.000000in}}%
\pgfusepath{stroke,fill}%
}%
\begin{pgfscope}%
\pgfsys@transformshift{0.552121in}{3.025619in}%
\pgfsys@useobject{currentmarker}{}%
\end{pgfscope}%
\end{pgfscope}%
\begin{pgfscope}%
\definecolor{textcolor}{rgb}{0.000000,0.000000,0.000000}%
\pgfsetstrokecolor{textcolor}%
\pgfsetfillcolor{textcolor}%
\pgftext[x=0.390663in, y=2.982216in, left, base]{\color{textcolor}{\rmfamily\fontsize{9.000000}{10.800000}\selectfont\catcode`\^=\active\def^{\ifmmode\sp\else\^{}\fi}\catcode`\%=\active\def%{\%}$\mathdefault{1}$}}%
\end{pgfscope}%
\begin{pgfscope}%
\pgfpathrectangle{\pgfqpoint{0.552121in}{0.499074in}}{\pgfqpoint{4.960000in}{3.696000in}}%
\pgfusepath{clip}%
\pgfsetbuttcap%
\pgfsetroundjoin%
\pgfsetlinewidth{0.803000pt}%
\definecolor{currentstroke}{rgb}{0.501961,0.501961,0.501961}%
\pgfsetstrokecolor{currentstroke}%
\pgfsetstrokeopacity{0.700000}%
\pgfsetdash{{0.800000pt}{1.320000pt}}{0.000000pt}%
\pgfpathmoveto{\pgfqpoint{0.552121in}{3.738636in}}%
\pgfpathlineto{\pgfqpoint{5.512121in}{3.738636in}}%
\pgfusepath{stroke}%
\end{pgfscope}%
\begin{pgfscope}%
\pgfsetbuttcap%
\pgfsetroundjoin%
\definecolor{currentfill}{rgb}{0.000000,0.000000,0.000000}%
\pgfsetfillcolor{currentfill}%
\pgfsetlinewidth{0.803000pt}%
\definecolor{currentstroke}{rgb}{0.000000,0.000000,0.000000}%
\pgfsetstrokecolor{currentstroke}%
\pgfsetdash{}{0pt}%
\pgfsys@defobject{currentmarker}{\pgfqpoint{-0.048611in}{0.000000in}}{\pgfqpoint{-0.000000in}{0.000000in}}{%
\pgfpathmoveto{\pgfqpoint{-0.000000in}{0.000000in}}%
\pgfpathlineto{\pgfqpoint{-0.048611in}{0.000000in}}%
\pgfusepath{stroke,fill}%
}%
\begin{pgfscope}%
\pgfsys@transformshift{0.552121in}{3.738636in}%
\pgfsys@useobject{currentmarker}{}%
\end{pgfscope}%
\end{pgfscope}%
\begin{pgfscope}%
\definecolor{textcolor}{rgb}{0.000000,0.000000,0.000000}%
\pgfsetstrokecolor{textcolor}%
\pgfsetfillcolor{textcolor}%
\pgftext[x=0.390663in, y=3.695233in, left, base]{\color{textcolor}{\rmfamily\fontsize{9.000000}{10.800000}\selectfont\catcode`\^=\active\def^{\ifmmode\sp\else\^{}\fi}\catcode`\%=\active\def%{\%}$\mathdefault{2}$}}%
\end{pgfscope}%
\begin{pgfscope}%
\definecolor{textcolor}{rgb}{0.000000,0.000000,0.000000}%
\pgfsetstrokecolor{textcolor}%
\pgfsetfillcolor{textcolor}%
\pgftext[x=0.235185in,y=2.347074in,,bottom,rotate=90.000000]{\color{textcolor}{\rmfamily\fontsize{11.000000}{13.200000}\selectfont\catcode`\^=\active\def^{\ifmmode\sp\else\^{}\fi}\catcode`\%=\active\def%{\%}$v\xi$}}%
\end{pgfscope}%
\begin{pgfscope}%
\pgfpathrectangle{\pgfqpoint{0.552121in}{0.499074in}}{\pgfqpoint{4.960000in}{3.696000in}}%
\pgfusepath{clip}%
\pgfsetbuttcap%
\pgfsetroundjoin%
\pgfsetlinewidth{1.505625pt}%
\definecolor{currentstroke}{rgb}{1.000000,0.000000,0.000000}%
\pgfsetstrokecolor{currentstroke}%
\pgfsetdash{{1.500000pt}{2.475000pt}}{0.000000pt}%
\pgfpathmoveto{\pgfqpoint{0.777575in}{2.312603in}}%
\pgfpathlineto{\pgfqpoint{0.852727in}{2.284603in}}%
\pgfpathlineto{\pgfqpoint{0.927878in}{2.256599in}}%
\pgfpathlineto{\pgfqpoint{1.003030in}{2.228599in}}%
\pgfpathlineto{\pgfqpoint{1.078181in}{2.200598in}}%
\pgfpathlineto{\pgfqpoint{1.153333in}{2.172599in}}%
\pgfpathlineto{\pgfqpoint{1.228484in}{2.144601in}}%
\pgfpathlineto{\pgfqpoint{1.303636in}{2.116603in}}%
\pgfpathlineto{\pgfqpoint{1.378788in}{2.088608in}}%
\pgfpathlineto{\pgfqpoint{1.453939in}{2.060614in}}%
\pgfpathlineto{\pgfqpoint{1.529091in}{2.032624in}}%
\pgfpathlineto{\pgfqpoint{1.604242in}{2.004637in}}%
\pgfpathlineto{\pgfqpoint{1.679394in}{1.976656in}}%
\pgfpathlineto{\pgfqpoint{1.754545in}{1.948683in}}%
\pgfpathlineto{\pgfqpoint{1.829697in}{1.920720in}}%
\pgfpathlineto{\pgfqpoint{1.904848in}{1.892770in}}%
\pgfpathlineto{\pgfqpoint{1.980000in}{1.864839in}}%
\pgfpathlineto{\pgfqpoint{2.055151in}{1.836929in}}%
\pgfpathlineto{\pgfqpoint{2.130303in}{1.809037in}}%
\pgfpathlineto{\pgfqpoint{2.205454in}{1.781166in}}%
\pgfpathlineto{\pgfqpoint{2.280606in}{1.753318in}}%
\pgfpathlineto{\pgfqpoint{2.355757in}{1.725496in}}%
\pgfpathlineto{\pgfqpoint{2.430909in}{1.697701in}}%
\pgfpathlineto{\pgfqpoint{2.506060in}{1.669938in}}%
\pgfpathlineto{\pgfqpoint{2.581212in}{1.642207in}}%
\pgfpathlineto{\pgfqpoint{2.656363in}{1.614513in}}%
\pgfpathlineto{\pgfqpoint{2.731515in}{1.586857in}}%
\pgfpathlineto{\pgfqpoint{2.806666in}{1.559242in}}%
\pgfpathlineto{\pgfqpoint{2.881818in}{1.531670in}}%
\pgfpathlineto{\pgfqpoint{2.956969in}{1.504144in}}%
\pgfpathlineto{\pgfqpoint{3.032121in}{1.476665in}}%
\pgfpathlineto{\pgfqpoint{3.107272in}{1.449237in}}%
\pgfpathlineto{\pgfqpoint{3.182424in}{1.421859in}}%
\pgfpathlineto{\pgfqpoint{3.257575in}{1.394534in}}%
\pgfpathlineto{\pgfqpoint{3.332727in}{1.367263in}}%
\pgfpathlineto{\pgfqpoint{3.407878in}{1.340046in}}%
\pgfpathlineto{\pgfqpoint{3.483030in}{1.312884in}}%
\pgfpathlineto{\pgfqpoint{3.558181in}{1.285776in}}%
\pgfpathlineto{\pgfqpoint{3.633333in}{1.258722in}}%
\pgfpathlineto{\pgfqpoint{3.708484in}{1.231720in}}%
\pgfpathlineto{\pgfqpoint{3.783636in}{1.204769in}}%
\pgfpathlineto{\pgfqpoint{3.858788in}{1.177866in}}%
\pgfpathlineto{\pgfqpoint{3.933939in}{1.151009in}}%
\pgfpathlineto{\pgfqpoint{4.009091in}{1.124194in}}%
\pgfpathlineto{\pgfqpoint{4.084242in}{1.097416in}}%
\pgfpathlineto{\pgfqpoint{4.159394in}{1.070670in}}%
\pgfpathlineto{\pgfqpoint{4.234545in}{1.043951in}}%
\pgfpathlineto{\pgfqpoint{4.309697in}{1.017251in}}%
\pgfpathlineto{\pgfqpoint{4.384848in}{0.990562in}}%
\pgfpathlineto{\pgfqpoint{4.460000in}{0.963877in}}%
\pgfpathlineto{\pgfqpoint{4.535151in}{0.937185in}}%
\pgfpathlineto{\pgfqpoint{4.610303in}{0.910476in}}%
\pgfpathlineto{\pgfqpoint{4.685454in}{0.883736in}}%
\pgfpathlineto{\pgfqpoint{4.760606in}{0.856947in}}%
\pgfpathlineto{\pgfqpoint{4.835757in}{0.830098in}}%
\pgfpathlineto{\pgfqpoint{4.910909in}{0.803176in}}%
\pgfpathlineto{\pgfqpoint{4.986060in}{0.776172in}}%
\pgfpathlineto{\pgfqpoint{5.061212in}{0.749071in}}%
\pgfpathlineto{\pgfqpoint{5.136363in}{0.721863in}}%
\pgfpathlineto{\pgfqpoint{5.211515in}{0.694535in}}%
\pgfpathlineto{\pgfqpoint{5.286666in}{0.667074in}}%
\pgfusepath{stroke}%
\end{pgfscope}%
\begin{pgfscope}%
\pgfpathrectangle{\pgfqpoint{0.552121in}{0.499074in}}{\pgfqpoint{4.960000in}{3.696000in}}%
\pgfusepath{clip}%
\pgfsetbuttcap%
\pgfsetroundjoin%
\pgfsetlinewidth{1.505625pt}%
\definecolor{currentstroke}{rgb}{1.000000,0.000000,0.000000}%
\pgfsetstrokecolor{currentstroke}%
\pgfsetdash{{5.550000pt}{2.400000pt}}{0.000000pt}%
\pgfpathmoveto{\pgfqpoint{0.777575in}{2.312603in}}%
\pgfpathlineto{\pgfqpoint{0.852727in}{2.340603in}}%
\pgfpathlineto{\pgfqpoint{0.927878in}{2.368607in}}%
\pgfpathlineto{\pgfqpoint{1.003030in}{2.396607in}}%
\pgfpathlineto{\pgfqpoint{1.078181in}{2.424608in}}%
\pgfpathlineto{\pgfqpoint{1.153333in}{2.452609in}}%
\pgfpathlineto{\pgfqpoint{1.228484in}{2.480610in}}%
\pgfpathlineto{\pgfqpoint{1.303636in}{2.508613in}}%
\pgfpathlineto{\pgfqpoint{1.378788in}{2.536617in}}%
\pgfpathlineto{\pgfqpoint{1.453939in}{2.564624in}}%
\pgfpathlineto{\pgfqpoint{1.529091in}{2.592634in}}%
\pgfpathlineto{\pgfqpoint{1.604242in}{2.620648in}}%
\pgfpathlineto{\pgfqpoint{1.679394in}{2.648668in}}%
\pgfpathlineto{\pgfqpoint{1.754545in}{2.676696in}}%
\pgfpathlineto{\pgfqpoint{1.829697in}{2.704734in}}%
\pgfpathlineto{\pgfqpoint{1.904848in}{2.732786in}}%
\pgfpathlineto{\pgfqpoint{1.980000in}{2.760856in}}%
\pgfpathlineto{\pgfqpoint{2.055151in}{2.788947in}}%
\pgfpathlineto{\pgfqpoint{2.130303in}{2.817057in}}%
\pgfpathlineto{\pgfqpoint{2.205454in}{2.845187in}}%
\pgfpathlineto{\pgfqpoint{2.280606in}{2.873340in}}%
\pgfpathlineto{\pgfqpoint{2.355757in}{2.901518in}}%
\pgfpathlineto{\pgfqpoint{2.430909in}{2.929724in}}%
\pgfpathlineto{\pgfqpoint{2.506060in}{2.957961in}}%
\pgfpathlineto{\pgfqpoint{2.581212in}{2.986231in}}%
\pgfpathlineto{\pgfqpoint{2.656363in}{3.014536in}}%
\pgfpathlineto{\pgfqpoint{2.731515in}{3.042880in}}%
\pgfpathlineto{\pgfqpoint{2.806666in}{3.071265in}}%
\pgfpathlineto{\pgfqpoint{2.881818in}{3.099692in}}%
\pgfpathlineto{\pgfqpoint{2.956969in}{3.128166in}}%
\pgfpathlineto{\pgfqpoint{3.032121in}{3.156687in}}%
\pgfpathlineto{\pgfqpoint{3.107272in}{3.185257in}}%
\pgfpathlineto{\pgfqpoint{3.182424in}{3.213879in}}%
\pgfpathlineto{\pgfqpoint{3.257575in}{3.242553in}}%
\pgfpathlineto{\pgfqpoint{3.332727in}{3.271281in}}%
\pgfpathlineto{\pgfqpoint{3.407878in}{3.300063in}}%
\pgfpathlineto{\pgfqpoint{3.483030in}{3.328899in}}%
\pgfpathlineto{\pgfqpoint{3.558181in}{3.357791in}}%
\pgfpathlineto{\pgfqpoint{3.633333in}{3.386736in}}%
\pgfpathlineto{\pgfqpoint{3.708484in}{3.415733in}}%
\pgfpathlineto{\pgfqpoint{3.783636in}{3.444782in}}%
\pgfpathlineto{\pgfqpoint{3.858788in}{3.473878in}}%
\pgfpathlineto{\pgfqpoint{3.933939in}{3.503021in}}%
\pgfpathlineto{\pgfqpoint{4.009091in}{3.532205in}}%
\pgfpathlineto{\pgfqpoint{4.084242in}{3.561427in}}%
\pgfpathlineto{\pgfqpoint{4.159394in}{3.590681in}}%
\pgfpathlineto{\pgfqpoint{4.234545in}{3.619961in}}%
\pgfpathlineto{\pgfqpoint{4.309697in}{3.649261in}}%
\pgfpathlineto{\pgfqpoint{4.384848in}{3.678572in}}%
\pgfpathlineto{\pgfqpoint{4.460000in}{3.707887in}}%
\pgfpathlineto{\pgfqpoint{4.535151in}{3.737194in}}%
\pgfpathlineto{\pgfqpoint{4.610303in}{3.766484in}}%
\pgfpathlineto{\pgfqpoint{4.685454in}{3.795744in}}%
\pgfpathlineto{\pgfqpoint{4.760606in}{3.824954in}}%
\pgfpathlineto{\pgfqpoint{4.835757in}{3.854104in}}%
\pgfpathlineto{\pgfqpoint{4.910909in}{3.883182in}}%
\pgfpathlineto{\pgfqpoint{4.986060in}{3.912176in}}%
\pgfpathlineto{\pgfqpoint{5.061212in}{3.941075in}}%
\pgfpathlineto{\pgfqpoint{5.136363in}{3.969866in}}%
\pgfpathlineto{\pgfqpoint{5.211515in}{3.998536in}}%
\pgfpathlineto{\pgfqpoint{5.286666in}{4.027074in}}%
\pgfusepath{stroke}%
\end{pgfscope}%
\begin{pgfscope}%
\pgfpathrectangle{\pgfqpoint{0.552121in}{0.499074in}}{\pgfqpoint{4.960000in}{3.696000in}}%
\pgfusepath{clip}%
\pgfsetrectcap%
\pgfsetroundjoin%
\pgfsetlinewidth{1.505625pt}%
\definecolor{currentstroke}{rgb}{0.121569,0.466667,0.705882}%
\pgfsetstrokecolor{currentstroke}%
\pgfsetdash{}{0pt}%
\pgfpathmoveto{\pgfqpoint{0.777575in}{2.312603in}}%
\pgfpathlineto{\pgfqpoint{0.852727in}{2.312184in}}%
\pgfpathlineto{\pgfqpoint{0.927878in}{2.313464in}}%
\pgfpathlineto{\pgfqpoint{1.003030in}{2.315661in}}%
\pgfpathlineto{\pgfqpoint{1.078181in}{2.318644in}}%
\pgfpathlineto{\pgfqpoint{1.153333in}{2.322292in}}%
\pgfpathlineto{\pgfqpoint{1.228484in}{2.326540in}}%
\pgfpathlineto{\pgfqpoint{1.303636in}{2.331358in}}%
\pgfpathlineto{\pgfqpoint{1.378788in}{2.336742in}}%
\pgfpathlineto{\pgfqpoint{1.453939in}{2.342701in}}%
\pgfpathlineto{\pgfqpoint{1.529091in}{2.349251in}}%
\pgfpathlineto{\pgfqpoint{1.604242in}{2.356414in}}%
\pgfpathlineto{\pgfqpoint{1.679394in}{2.364206in}}%
\pgfpathlineto{\pgfqpoint{1.754545in}{2.372643in}}%
\pgfpathlineto{\pgfqpoint{1.829697in}{2.381736in}}%
\pgfpathlineto{\pgfqpoint{1.904848in}{2.391489in}}%
\pgfpathlineto{\pgfqpoint{1.980000in}{2.401906in}}%
\pgfpathlineto{\pgfqpoint{2.055151in}{2.412984in}}%
\pgfpathlineto{\pgfqpoint{2.130303in}{2.424718in}}%
\pgfpathlineto{\pgfqpoint{2.205454in}{2.437106in}}%
\pgfpathlineto{\pgfqpoint{2.280606in}{2.450143in}}%
\pgfpathlineto{\pgfqpoint{2.355757in}{2.463824in}}%
\pgfpathlineto{\pgfqpoint{2.430909in}{2.478143in}}%
\pgfpathlineto{\pgfqpoint{2.506060in}{2.493095in}}%
\pgfpathlineto{\pgfqpoint{2.581212in}{2.508672in}}%
\pgfpathlineto{\pgfqpoint{2.656363in}{2.524866in}}%
\pgfpathlineto{\pgfqpoint{2.731515in}{2.541668in}}%
\pgfpathlineto{\pgfqpoint{2.806666in}{2.559067in}}%
\pgfpathlineto{\pgfqpoint{2.881818in}{2.577052in}}%
\pgfpathlineto{\pgfqpoint{2.956969in}{2.595608in}}%
\pgfpathlineto{\pgfqpoint{3.032121in}{2.614723in}}%
\pgfpathlineto{\pgfqpoint{3.107272in}{2.634379in}}%
\pgfpathlineto{\pgfqpoint{3.182424in}{2.654560in}}%
\pgfpathlineto{\pgfqpoint{3.257575in}{2.675249in}}%
\pgfpathlineto{\pgfqpoint{3.332727in}{2.696426in}}%
\pgfpathlineto{\pgfqpoint{3.407878in}{2.718071in}}%
\pgfpathlineto{\pgfqpoint{3.483030in}{2.740165in}}%
\pgfpathlineto{\pgfqpoint{3.558181in}{2.762686in}}%
\pgfpathlineto{\pgfqpoint{3.633333in}{2.785614in}}%
\pgfpathlineto{\pgfqpoint{3.708484in}{2.808928in}}%
\pgfpathlineto{\pgfqpoint{3.783636in}{2.832606in}}%
\pgfpathlineto{\pgfqpoint{3.858788in}{2.856629in}}%
\pgfpathlineto{\pgfqpoint{3.933939in}{2.880975in}}%
\pgfpathlineto{\pgfqpoint{4.009091in}{2.905626in}}%
\pgfpathlineto{\pgfqpoint{4.084242in}{2.930561in}}%
\pgfpathlineto{\pgfqpoint{4.159394in}{2.955763in}}%
\pgfpathlineto{\pgfqpoint{4.234545in}{2.981214in}}%
\pgfpathlineto{\pgfqpoint{4.309697in}{3.006898in}}%
\pgfpathlineto{\pgfqpoint{4.384848in}{3.032801in}}%
\pgfpathlineto{\pgfqpoint{4.460000in}{3.058909in}}%
\pgfpathlineto{\pgfqpoint{4.535151in}{3.085217in}}%
\pgfpathlineto{\pgfqpoint{4.610303in}{3.111725in}}%
\pgfpathlineto{\pgfqpoint{4.685454in}{3.137283in}}%
\pgfpathlineto{\pgfqpoint{4.760606in}{3.162928in}}%
\pgfpathlineto{\pgfqpoint{4.835757in}{3.188720in}}%
\pgfpathlineto{\pgfqpoint{4.910909in}{3.214781in}}%
\pgfpathlineto{\pgfqpoint{4.986060in}{3.241317in}}%
\pgfpathlineto{\pgfqpoint{5.061212in}{3.268663in}}%
\pgfpathlineto{\pgfqpoint{5.136363in}{3.297325in}}%
\pgfpathlineto{\pgfqpoint{5.211515in}{3.328036in}}%
\pgfpathlineto{\pgfqpoint{5.286666in}{2.325566in}}%
\pgfusepath{stroke}%
\end{pgfscope}%
\begin{pgfscope}%
\pgfpathrectangle{\pgfqpoint{0.552121in}{0.499074in}}{\pgfqpoint{4.960000in}{3.696000in}}%
\pgfusepath{clip}%
\pgfsetrectcap%
\pgfsetroundjoin%
\pgfsetlinewidth{1.505625pt}%
\definecolor{currentstroke}{rgb}{1.000000,0.498039,0.054902}%
\pgfsetstrokecolor{currentstroke}%
\pgfsetdash{}{0pt}%
\pgfpathmoveto{\pgfqpoint{0.777575in}{2.312603in}}%
\pgfpathlineto{\pgfqpoint{0.852727in}{2.312603in}}%
\pgfpathlineto{\pgfqpoint{0.927878in}{2.312603in}}%
\pgfpathlineto{\pgfqpoint{1.003030in}{2.312603in}}%
\pgfpathlineto{\pgfqpoint{1.078181in}{2.312603in}}%
\pgfpathlineto{\pgfqpoint{1.153333in}{2.312603in}}%
\pgfpathlineto{\pgfqpoint{1.228484in}{2.312603in}}%
\pgfpathlineto{\pgfqpoint{1.303636in}{2.312604in}}%
\pgfpathlineto{\pgfqpoint{1.378788in}{2.312605in}}%
\pgfpathlineto{\pgfqpoint{1.453939in}{2.312606in}}%
\pgfpathlineto{\pgfqpoint{1.529091in}{2.312607in}}%
\pgfpathlineto{\pgfqpoint{1.604242in}{2.312609in}}%
\pgfpathlineto{\pgfqpoint{1.679394in}{2.312611in}}%
\pgfpathlineto{\pgfqpoint{1.754545in}{2.312614in}}%
\pgfpathlineto{\pgfqpoint{1.829697in}{2.312618in}}%
\pgfpathlineto{\pgfqpoint{1.904848in}{2.312623in}}%
\pgfpathlineto{\pgfqpoint{1.980000in}{2.312630in}}%
\pgfpathlineto{\pgfqpoint{2.055151in}{2.312638in}}%
\pgfpathlineto{\pgfqpoint{2.130303in}{2.312647in}}%
\pgfpathlineto{\pgfqpoint{2.205454in}{2.312657in}}%
\pgfpathlineto{\pgfqpoint{2.280606in}{2.312669in}}%
\pgfpathlineto{\pgfqpoint{2.355757in}{2.312681in}}%
\pgfpathlineto{\pgfqpoint{2.430909in}{2.312695in}}%
\pgfpathlineto{\pgfqpoint{2.506060in}{2.312710in}}%
\pgfpathlineto{\pgfqpoint{2.581212in}{2.312727in}}%
\pgfpathlineto{\pgfqpoint{2.656363in}{2.312745in}}%
\pgfpathlineto{\pgfqpoint{2.731515in}{2.312764in}}%
\pgfpathlineto{\pgfqpoint{2.806666in}{2.312785in}}%
\pgfpathlineto{\pgfqpoint{2.881818in}{2.312808in}}%
\pgfpathlineto{\pgfqpoint{2.956969in}{2.312832in}}%
\pgfpathlineto{\pgfqpoint{3.032121in}{2.312857in}}%
\pgfpathlineto{\pgfqpoint{3.107272in}{2.312884in}}%
\pgfpathlineto{\pgfqpoint{3.182424in}{2.312912in}}%
\pgfpathlineto{\pgfqpoint{3.257575in}{2.312942in}}%
\pgfpathlineto{\pgfqpoint{3.332727in}{2.312973in}}%
\pgfpathlineto{\pgfqpoint{3.407878in}{2.313005in}}%
\pgfpathlineto{\pgfqpoint{3.483030in}{2.313039in}}%
\pgfpathlineto{\pgfqpoint{3.558181in}{2.313073in}}%
\pgfpathlineto{\pgfqpoint{3.633333in}{2.313109in}}%
\pgfpathlineto{\pgfqpoint{3.708484in}{2.313145in}}%
\pgfpathlineto{\pgfqpoint{3.783636in}{2.313182in}}%
\pgfpathlineto{\pgfqpoint{3.858788in}{2.313220in}}%
\pgfpathlineto{\pgfqpoint{3.933939in}{2.313258in}}%
\pgfpathlineto{\pgfqpoint{4.009091in}{2.313296in}}%
\pgfpathlineto{\pgfqpoint{4.084242in}{2.313334in}}%
\pgfpathlineto{\pgfqpoint{4.159394in}{2.313372in}}%
\pgfpathlineto{\pgfqpoint{4.234545in}{2.313409in}}%
\pgfpathlineto{\pgfqpoint{4.309697in}{2.313446in}}%
\pgfpathlineto{\pgfqpoint{4.384848in}{2.313481in}}%
\pgfpathlineto{\pgfqpoint{4.460000in}{2.313516in}}%
\pgfpathlineto{\pgfqpoint{4.535151in}{2.313548in}}%
\pgfpathlineto{\pgfqpoint{4.610303in}{2.313579in}}%
\pgfpathlineto{\pgfqpoint{4.685454in}{2.313608in}}%
\pgfpathlineto{\pgfqpoint{4.760606in}{2.313634in}}%
\pgfpathlineto{\pgfqpoint{4.835757in}{2.313656in}}%
\pgfpathlineto{\pgfqpoint{4.910909in}{2.313676in}}%
\pgfpathlineto{\pgfqpoint{4.986060in}{2.313691in}}%
\pgfpathlineto{\pgfqpoint{5.061212in}{2.313703in}}%
\pgfpathlineto{\pgfqpoint{5.136363in}{2.313711in}}%
\pgfpathlineto{\pgfqpoint{5.211515in}{2.313715in}}%
\pgfpathlineto{\pgfqpoint{5.286666in}{2.313715in}}%
\pgfusepath{stroke}%
\end{pgfscope}%
\begin{pgfscope}%
\pgfsetrectcap%
\pgfsetmiterjoin%
\pgfsetlinewidth{0.803000pt}%
\definecolor{currentstroke}{rgb}{0.000000,0.000000,0.000000}%
\pgfsetstrokecolor{currentstroke}%
\pgfsetdash{}{0pt}%
\pgfpathmoveto{\pgfqpoint{0.552121in}{0.499074in}}%
\pgfpathlineto{\pgfqpoint{0.552121in}{4.195074in}}%
\pgfusepath{stroke}%
\end{pgfscope}%
\begin{pgfscope}%
\pgfsetrectcap%
\pgfsetmiterjoin%
\pgfsetlinewidth{0.803000pt}%
\definecolor{currentstroke}{rgb}{0.000000,0.000000,0.000000}%
\pgfsetstrokecolor{currentstroke}%
\pgfsetdash{}{0pt}%
\pgfpathmoveto{\pgfqpoint{5.512121in}{0.499074in}}%
\pgfpathlineto{\pgfqpoint{5.512121in}{4.195074in}}%
\pgfusepath{stroke}%
\end{pgfscope}%
\begin{pgfscope}%
\pgfsetrectcap%
\pgfsetmiterjoin%
\pgfsetlinewidth{0.803000pt}%
\definecolor{currentstroke}{rgb}{0.000000,0.000000,0.000000}%
\pgfsetstrokecolor{currentstroke}%
\pgfsetdash{}{0pt}%
\pgfpathmoveto{\pgfqpoint{0.552121in}{0.499074in}}%
\pgfpathlineto{\pgfqpoint{5.512121in}{0.499074in}}%
\pgfusepath{stroke}%
\end{pgfscope}%
\begin{pgfscope}%
\pgfsetrectcap%
\pgfsetmiterjoin%
\pgfsetlinewidth{0.803000pt}%
\definecolor{currentstroke}{rgb}{0.000000,0.000000,0.000000}%
\pgfsetstrokecolor{currentstroke}%
\pgfsetdash{}{0pt}%
\pgfpathmoveto{\pgfqpoint{0.552121in}{4.195074in}}%
\pgfpathlineto{\pgfqpoint{5.512121in}{4.195074in}}%
\pgfusepath{stroke}%
\end{pgfscope}%
\begin{pgfscope}%
\pgfsetbuttcap%
\pgfsetmiterjoin%
\definecolor{currentfill}{rgb}{1.000000,1.000000,1.000000}%
\pgfsetfillcolor{currentfill}%
\pgfsetfillopacity{0.800000}%
\pgfsetlinewidth{1.003750pt}%
\definecolor{currentstroke}{rgb}{0.800000,0.800000,0.800000}%
\pgfsetstrokecolor{currentstroke}%
\pgfsetstrokeopacity{0.800000}%
\pgfsetdash{}{0pt}%
\pgfpathmoveto{\pgfqpoint{0.649343in}{3.115599in}}%
\pgfpathlineto{\pgfqpoint{2.681754in}{3.115599in}}%
\pgfpathquadraticcurveto{\pgfqpoint{2.709531in}{3.115599in}}{\pgfqpoint{2.709531in}{3.143377in}}%
\pgfpathlineto{\pgfqpoint{2.709531in}{4.097852in}}%
\pgfpathquadraticcurveto{\pgfqpoint{2.709531in}{4.125630in}}{\pgfqpoint{2.681754in}{4.125630in}}%
\pgfpathlineto{\pgfqpoint{0.649343in}{4.125630in}}%
\pgfpathquadraticcurveto{\pgfqpoint{0.621565in}{4.125630in}}{\pgfqpoint{0.621565in}{4.097852in}}%
\pgfpathlineto{\pgfqpoint{0.621565in}{3.143377in}}%
\pgfpathquadraticcurveto{\pgfqpoint{0.621565in}{3.115599in}}{\pgfqpoint{0.649343in}{3.115599in}}%
\pgfpathlineto{\pgfqpoint{0.649343in}{3.115599in}}%
\pgfpathclose%
\pgfusepath{stroke,fill}%
\end{pgfscope}%
\begin{pgfscope}%
\pgfsetbuttcap%
\pgfsetmiterjoin%
\definecolor{currentfill}{rgb}{0.000000,0.501961,0.000000}%
\pgfsetfillcolor{currentfill}%
\pgfsetfillopacity{0.200000}%
\pgfsetlinewidth{1.003750pt}%
\definecolor{currentstroke}{rgb}{0.000000,0.501961,0.000000}%
\pgfsetstrokecolor{currentstroke}%
\pgfsetstrokeopacity{0.200000}%
\pgfsetdash{}{0pt}%
\pgfpathmoveto{\pgfqpoint{0.677121in}{3.972852in}}%
\pgfpathlineto{\pgfqpoint{0.954899in}{3.972852in}}%
\pgfpathlineto{\pgfqpoint{0.954899in}{4.070074in}}%
\pgfpathlineto{\pgfqpoint{0.677121in}{4.070074in}}%
\pgfpathlineto{\pgfqpoint{0.677121in}{3.972852in}}%
\pgfpathclose%
\pgfusepath{stroke,fill}%
\end{pgfscope}%
\begin{pgfscope}%
\definecolor{textcolor}{rgb}{0.000000,0.000000,0.000000}%
\pgfsetstrokecolor{textcolor}%
\pgfsetfillcolor{textcolor}%
\pgftext[x=1.066010in,y=3.972852in,left,base]{\color{textcolor}{\rmfamily\fontsize{10.000000}{12.000000}\selectfont\catcode`\^=\active\def^{\ifmmode\sp\else\^{}\fi}\catcode`\%=\active\def%{\%}Bounds}}%
\end{pgfscope}%
\begin{pgfscope}%
\pgfsetbuttcap%
\pgfsetroundjoin%
\pgfsetlinewidth{1.505625pt}%
\definecolor{currentstroke}{rgb}{1.000000,0.000000,0.000000}%
\pgfsetstrokecolor{currentstroke}%
\pgfsetdash{{1.500000pt}{2.475000pt}}{0.000000pt}%
\pgfpathmoveto{\pgfqpoint{0.677121in}{3.827790in}}%
\pgfpathlineto{\pgfqpoint{0.816010in}{3.827790in}}%
\pgfpathlineto{\pgfqpoint{0.954899in}{3.827790in}}%
\pgfusepath{stroke}%
\end{pgfscope}%
\begin{pgfscope}%
\definecolor{textcolor}{rgb}{0.000000,0.000000,0.000000}%
\pgfsetstrokecolor{textcolor}%
\pgfsetfillcolor{textcolor}%
\pgftext[x=1.066010in,y=3.779179in,left,base]{\color{textcolor}{\rmfamily\fontsize{10.000000}{12.000000}\selectfont\catcode`\^=\active\def^{\ifmmode\sp\else\^{}\fi}\catcode`\%=\active\def%{\%}Lower Bound}}%
\end{pgfscope}%
\begin{pgfscope}%
\pgfsetbuttcap%
\pgfsetroundjoin%
\pgfsetlinewidth{1.505625pt}%
\definecolor{currentstroke}{rgb}{1.000000,0.000000,0.000000}%
\pgfsetstrokecolor{currentstroke}%
\pgfsetdash{{5.550000pt}{2.400000pt}}{0.000000pt}%
\pgfpathmoveto{\pgfqpoint{0.677121in}{3.634117in}}%
\pgfpathlineto{\pgfqpoint{0.816010in}{3.634117in}}%
\pgfpathlineto{\pgfqpoint{0.954899in}{3.634117in}}%
\pgfusepath{stroke}%
\end{pgfscope}%
\begin{pgfscope}%
\definecolor{textcolor}{rgb}{0.000000,0.000000,0.000000}%
\pgfsetstrokecolor{textcolor}%
\pgfsetfillcolor{textcolor}%
\pgftext[x=1.066010in,y=3.585506in,left,base]{\color{textcolor}{\rmfamily\fontsize{10.000000}{12.000000}\selectfont\catcode`\^=\active\def^{\ifmmode\sp\else\^{}\fi}\catcode`\%=\active\def%{\%}Upper Bound}}%
\end{pgfscope}%
\begin{pgfscope}%
\pgfsetrectcap%
\pgfsetroundjoin%
\pgfsetlinewidth{1.505625pt}%
\definecolor{currentstroke}{rgb}{0.121569,0.466667,0.705882}%
\pgfsetstrokecolor{currentstroke}%
\pgfsetdash{}{0pt}%
\pgfpathmoveto{\pgfqpoint{0.677121in}{3.440445in}}%
\pgfpathlineto{\pgfqpoint{0.816010in}{3.440445in}}%
\pgfpathlineto{\pgfqpoint{0.954899in}{3.440445in}}%
\pgfusepath{stroke}%
\end{pgfscope}%
\begin{pgfscope}%
\definecolor{textcolor}{rgb}{0.000000,0.000000,0.000000}%
\pgfsetstrokecolor{textcolor}%
\pgfsetfillcolor{textcolor}%
\pgftext[x=1.066010in,y=3.391833in,left,base]{\color{textcolor}{\rmfamily\fontsize{10.000000}{12.000000}\selectfont\catcode`\^=\active\def^{\ifmmode\sp\else\^{}\fi}\catcode`\%=\active\def%{\%}Relaxation Variable Value}}%
\end{pgfscope}%
\begin{pgfscope}%
\pgfsetrectcap%
\pgfsetroundjoin%
\pgfsetlinewidth{1.505625pt}%
\definecolor{currentstroke}{rgb}{1.000000,0.498039,0.054902}%
\pgfsetstrokecolor{currentstroke}%
\pgfsetdash{}{0pt}%
\pgfpathmoveto{\pgfqpoint{0.677121in}{3.246772in}}%
\pgfpathlineto{\pgfqpoint{0.816010in}{3.246772in}}%
\pgfpathlineto{\pgfqpoint{0.954899in}{3.246772in}}%
\pgfusepath{stroke}%
\end{pgfscope}%
\begin{pgfscope}%
\definecolor{textcolor}{rgb}{0.000000,0.000000,0.000000}%
\pgfsetstrokecolor{textcolor}%
\pgfsetfillcolor{textcolor}%
\pgftext[x=1.066010in,y=3.198161in,left,base]{\color{textcolor}{\rmfamily\fontsize{10.000000}{12.000000}\selectfont\catcode`\^=\active\def^{\ifmmode\sp\else\^{}\fi}\catcode`\%=\active\def%{\%}Actual Bilinear Value}}%
\end{pgfscope}%
\end{pgfpicture}%
\makeatother%
\endgroup%
}
		\caption{State transition approximation for \( n \) using tighter bounds.}
		\label{fig:mccormick_problem_better}
	\end{subfigure}
	\caption{Comparison of McCormick relaxation with and without tighter bounds.}
\end{figure}
Due to the rapid inaccuracy of the bounds, more frequent replanning is required.
This presents a significant trade-off compared to the point mass model, which necessitates less frequent replanning.
