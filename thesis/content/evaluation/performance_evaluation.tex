\section{Performance Evaluation} \label{sec:performance-evaluation}

\subsection{Objectives} \label{subsec:objectives}

Our trajectory planner aims to minimize a cost function that represents the driving behavior we desire.
The cost function is composed of several objectives, each weighted by a corresponding factor.
The main objectives we considered are:

The main objectives we considered are control effort, deviation from the reference
trajectory, and terminal state.
The control effort objective aims to minimize the numerical derivatives of control inputs to ensure smooth driving, represented by the cost function:
\begin{equation}
	J_{control} = \sum_{i=0}^{n-1} \left\| d_1(t_i) \right\|^2
\end{equation}
where $d_1(t_i)\in \mathbb{R}^{dim(u)}$ is a new introduced auxiliary variable, which is constrained by: \[
	d_1(t_i) = \frac{u(t_i) - u(t_{i-1})}{t_i - t_{i-1}}
\]

The tracking objective aims to minimize the deviation from the center of the road, represented by the cost function:
\begin{equation}
	J_{tracking} = \sum_{i=0}^{n} d_2(t_i)^2
\end{equation}
where $d_2(t_i)\in \mathbb{R}$ is a new introduced auxiliary variable, which is constrained by: \[
	0 \leq d_2(t_i) \leq \min \left\{ \overline{n}(s(t_i)) - n(t_i), n(t_i) - \underline{n}(s(t_i)) \right\}
\]
The terminal state objective aims to minimize the deviation from a desired terminal state $x_{final}$ at the final time step $t_n$, represented by the cost function:
\begin{equation}
	J_{terminal} = \|x(t_n) - x_{final}\|^2
\end{equation}

The total cost function $J$ is a weighted sum of these objectives:
\begin{equation}
	J = \alpha J_{control} + \beta J_{tracking} + \gamma J_{terminal} \end{equation} where $\alpha$, $\beta$, and $\gamma$ are the weights that determine
the relative importance of each objective.

By minimizing this cost function, our trajectory planner generates a trajectory that balances control effort, adherence to the reference trajectory,
and reaching the desired terminal state.

\subsection{Scenarios} \label{subsec:scenarios}

In order to evaluate the performance of our trajectory planner, we implemented several driving scenarios.
These scenarios are designed to test different aspects of the planner's capabilities.
The scenarios we implemented are:

The Straight Road scenario tests the planner's ability to maintain a straight path with minimal
control effort, ensuring smooth and efficient driving.

In the Left Turn scenario, the planner's performance is evaluated based on its ability to execute a smooth left turn while adhering closely to the
reference trajectory.

The Lane Change scenario assesses the planner's capability to perform a lane change maneuver safely and efficiently, highlighting its responsiveness
and precision.

The Slalom scenario challenges the planner to navigate through a series of closely spaced obstacles, requiring precise control and smooth transitions
between maneuvers.

The Elchtest, also known as the moose test, evaluates the planner's ability to perform a sudden evasive maneuver to avoid an obstacle, testing its
quick decision-making and control under pressure.
The Elchtest can be visualized as follows:
\begin{figure}[H]
	\centering
	\begin{tikzpicture}
		\draw[thick, dashed] (0,-0.5) -- (5,-0.5); % Road boundary
		\draw[thick, dashed] (0,0.5) -- (3,0.5); % Road boundary
		\draw[thick, dashed] (3,0.5) -- (3,1.5); % Road boundary
		\draw[thick, dashed] (5,0.5) -- (5,-0.5); % Road boundary
		\draw[thick, dashed] (5,0.5) -- (7,0.5); % Road boundary
		\draw[thick, dashed] (3,1.5) -- (9,1.5); % Road boundary
		\draw[thick, dashed] (9,0.5) -- (9,1.5); % Road boundary
		\draw[thick, dashed] (7,0.5) -- (7,-0.5); % Road boundary
		\draw[thick, dashed] (7,-0.5) -- (12,-0.5); % Road boundary
		\draw[thick, dashed] (9,0.5) -- (12,0.5); % Road boundary
		\node at (0,0) {Start};
		\node at (12,0) {End};
	\end{tikzpicture}
	\caption{Elchtest scenario visualization}
	\label{fig:elchtest}
\end{figure}

Finally, the Sharp U Turn scenario tests the planner's ability to execute a sharp U-turn, challenging its control effort and adherence to the desired
terminal state.

By evaluating the planner in these diverse scenarios, we can gain a comprehensive understanding of its strengths and areas for improvement.

\subsection{Simulation} \label{subsec:simulation}

For the actual simulation of the vehicle, we use a complex model and discretized its dynamics using Runge-Kutta which provides a more accurate than
the forward we used during planning.

For reproducibility, we provide the model which is given by the state variables and control inputs: \[ x = [p_x, p_y, \delta, v, \psi, \dot{\psi},
	\beta]^T, u = [a_x, v_{\delta}]^T \] where $p_x, p_y$ are the position coordinates, $\delta$ is the steering angle, $v$ is the velocity, $\psi$ is
the yaw angle, $\dot{\psi}$ is the yaw rate, $\beta$ is the slip angle, $a_x$ is the longitudinal acceleration, and $v_{\delta}$ is the steering
rate.

The dynamics of the model are given by, for $|v|\geq0.1$:
\[
	f(x, u) = \begin{bmatrix}
		v\cos(\psi + \beta)                                  \\
		v\sin(\psi + \beta)                                  \\
		v_\delta                                             \\
		a_x                                                  \\
		\dot{\psi}                                           \\
		\frac{\mu\,m}{I_{z}(l_{r}+l_{f})}\Bigl(
		l_{f}\,C_{S,f}\bigl(g\,l_{r}-a_xh_{cg}\bigr)\,\delta \\
		\;+                                                 \;\bigl[l_{r}\,C_{S,r}\bigl(g\,l_{f}+a_xh_{cg}\bigr)
			\;-\;l_{f}\,C_{S,f}\bigl(g\,l_{r}-a_xh_{cg}\bigr)\bigr]\,\beta
		\Bigr)                                               \\
		\quad -\;\Bigl[
		l_{f}^{2}\,C_{S,f}\bigl(g\,l_{r}-a_xh_{cg}\bigr)
		\;+\;
		l_{r}^{2}\,C_{S,r}\bigl(g\,l_{f}+a_xh_{cg}\bigr)
		\Bigr]
		\frac{\dot{\psi}}{v}                                 \\
		\frac{\mu}{v\,\bigl(l_{r}+l_{f}\bigr)}\Bigl(
		C_{S,f}\bigl(g\,l_{r}-a_xh_{cg}\bigr)\,\delta
		\;-\;
		\bigl[C_{S,r}\bigl(g\,l_{f}+a_xh_{cg}\bigr)          \\
			\;+\;
		C_{S,f}\bigl(g\,l_{r}-a_xh_{cg}\bigr)\bigr]\,\beta   \\
		\quad +\;\bigl[
			C_{S,r}\bigl(g\,l_{f}+a_xh_{cg}\bigr)\,l_{r}
			\;-\;
			C_{S,f}\bigl(g\,l_{r}-a_xh_{cg}\bigr)\,l_{f}
			\bigr]
		\frac{\dot{\psi}}{v}
		\Bigr)
		\;-\;
		\dot{\psi}
	\end{bmatrix}
\]
and for $|v|<0.1$:
\[
	f(x, u) = \begin{bmatrix}
		v\cos(\psi + \beta) \\
		v\sin(\psi + \beta) \\
		v_\delta            \\
		a_x                 \\
		\dot{\psi}          \\
		\frac{1}{l_{wb}}
		\biggl(
		a_x\,\cos( \beta)\,\tan(\delta)
		\;-\;
		v\,\sin( \beta)\,\tan(\delta)\,\dot{x}_{7}
		\;+\;
		\frac{v\,\cos( \beta)}{\cos^2(\delta)}\,
		v_{\delta}
		\biggr)
		\\
		\frac{1}{1 +
			\bigl(\tan(\delta)\tfrac{l_{r}}{l_{wb}}\bigr)^2}
		\;\cdot\;
		\frac{l_{r}}{l_{wb}\,\cos^2(\delta)}\,
		v_{\delta}
	\end{bmatrix}
\]

We consider a vehicle of length \(l = 4.298\,\mathrm{m}\) and width \(w = 1.674\,\mathrm{m}\), with total mass \(m = 1.225\times10^3\,\mathrm{kg}\)
and moment of inertia \(I_z = 1.538\times10^3\,\mathrm{kg\,m}^2\).
The center of gravity is located \(l_f = 0.883\,\mathrm{m}\) from the front axle and \(l_r = 1.508\,\mathrm{m}\) from the rear axle, at a height
\(h_{cg} = 0.557\,\mathrm{m}\).
The front and rear cornering stiffness coefficients are both \(C_{S,f} = C_{S,r} = 20.89\,\text{[1/rad]}\), and the friction coefficient is \(\mu =
1.048\).

In our approach, we employed a replanning strategy to improve the adaptability of the trajectory planner.
In addition to the time horizon used during the initial planning phase, we implemented a fixed time interval, shorter than the time horizon, after
which the planner recalculates the trajectory from the current position of the vehicle.
This replanning mechanism allows the planner to adjust to changes in the environment or deviations from the planned path.

\subsection{Results} \label{subsec:results}

In this work, we compare two modeling approaches for vehicle dynamics in scenarios where road alignment plays a key role.
Specifically, we evaluate a point mass model that is aligned to the road and a kinematic bicycle model that captures the vehicle's geometry and
steering kinematics more explicitly.

\subsection*{Point Mass (Road-Aligned) Model}

\paragraph{Assumptions and Simplifications.}
The point mass model simplifies the vehicle to a single mass concentrated at a point.
It assumes that the path is primarily driven by the longitudinal and lateral accelerations as constrained by the road alignment.
Key kinematic effects such as tire slip angles, lateral load transfer, or steering geometry are either omitted or treated in a highly simplified
manner.
Furthermore, it often assumes that the orientation of the mass aligns with the road direction, so the model is well suited for higher-level path
planning over a known trajectory, where the exact wheel and steering configuration are less critical.

\paragraph{Performance Characteristics.}
Because of its simpler nature, the point mass model generally offers:
\begin{itemize}
	\item \textbf{Lower computational cost}, enabling faster simulations and
	      easier real-time implementation for basic path tracking.
	\item \textbf{Reduced parameter dependency}, as it needs fewer vehicle
	      parameters (e.g., just mass and approximate friction limits).
	\item \textbf{Reduced accuracy in nonlinear conditions}, because it neglects
	      steering geometry, tire slip angles, and more nuanced lateral
	      dynamics---making it less accurate at high lateral accelerations
	      or large steering angles.
\end{itemize}

\subsection*{Kinematic Bicycle Model}

\paragraph{Assumptions and Structure.}
The kinematic bicycle model represents the vehicle by a single front tire and rear tire, capturing the geometric relationship among steering angle,
vehicle length, and lateral motion.
It models the steering input at the front axle while assuming no explicit slip at the tire contact patch (although slip angles may be implicitly
captured via kinematic relations).
Unlike a full dynamic model, it generally omits complex tire forces and weight transfer, but retains essential geometry required to describe the
vehicle's heading and turning radius accurately.

\paragraph{Performance Characteristics.}
Compared to the point mass model, the kinematic bicycle model typically has:
\begin{itemize}
	\item \textbf{Improved representation of vehicle posture}, because it
	      tracks yaw angle and the relation between front and rear axle motions.
	\item \textbf{Better handling of moderate steering maneuvers}, as it
	      can capture how steering inputs alter the effective turning radius
	      and orientation.
	\item \textbf{Moderate computational complexity}, still significantly
	      lighter than full dynamic models but more detailed than
	      a point mass approach.
	\item \textbf{Limitations at high speeds or extreme maneuvers}, as it
	      ignores tire slip forces and load transfers, thus limiting
	      predictive capability in aggressive cornering or low-adhesion
	      conditions.
\end{itemize}

\subsection*{Comparative Summary}

When evaluating performance for standard driving maneuvers and moderate turning radii, the kinematic bicycle model often yields more faithful
estimates of the vehicle trajectory than the point mass model, owing to its explicit handling of vehicle geometry.
The point mass model, however, can be beneficial in path planning or control algorithms that rely on computationally inexpensive dynamics and do not
require precise yaw-rate or steering-angle predictions.

In short, the road-aligned point mass model is a higher-level abstraction that excels in simplicity and speed but sacrifices accuracy in situations
with larger lateral dynamics.
The kinematic bicycle model achieves better fidelity by incorporating the essential geometry of steering, but at a slightly higher computational and
implementation cost.
Both models can be useful depending on the complexity and fidelity demands of the application.
