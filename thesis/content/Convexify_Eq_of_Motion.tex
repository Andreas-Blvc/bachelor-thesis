\chapter{Convexify Equations of Motion}

\[
	x_{tn} = \begin{bmatrix}
		s          \\
		n          \\
		\xi        \\
		\dot{s}    \\
		\dot{n}    \\
		\dot{\psi} \\
	\end{bmatrix}
\]

\[
	\dot{x}_{tn} = \begin{bmatrix}
		\dot{s}                                    \\
		\dot{n}                                    \\
		\dot{\psi} - C(s)\dot{s}                   \\
		\frac{
			a_{x, tn} + 2\dot{n}
		C(s)\dot{s} + nC'(s)\dot{s}^2 }{ 1-nC(s) } \\ a_{y,tn}-C(s)\dot{s}^2(1-nC(s)) \\ a_{\psi}\end{bmatrix} \]

1.
Equation

\[ \xi_{n+1} = \xi_n + (\dot{\psi} - C(s)\dot{s}) dt \]

Since $dt$ is a constant, one only has to
approximate $C(s)\dot{s}$ with a affine linear term.
This can be done by guessing $s$ using a reference velocity and evalutaing the curvature at this position, this would make $C(s)$ a constant in
planning.

2.
Equation

\[ \dot{s}_{n+1} = \dot{s}_n + \frac{ a_{x, tn} + 2\dot{n} C(s)\dot{s} + nC'(s)\dot{s}^2 }{ 1-nC(s) } dt \]

\begin{itemize}
	\item One can assume that $nC(s)$ is close to Zero: $\frac{1}{1-nC(s)}\approx1$.
	\item We can again make $C(s)$ a constant, that will be close to the true value.
	\item Use McCormick for $\dot{s}\dot{n}$
	\item Hardest Part: $n\dot{s}^2$: Use McCormick twice
\end{itemize}
