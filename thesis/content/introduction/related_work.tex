\section{Related Work} \label{sec:related_work}

Since the early demonstrations in the DARPA Grand and Urban Challenges [1][2], the motion planning literature for self-driving vehicles has evolved
to address a broad range of environments—from structured highways and city roads to unstructured parking lots and off-road terrains.
A unifying theme across these scenarios is the need to produce collision- free, dynamically-feasible trajectories that also account for comfort and
operational constraints.

\subsection{Classical Path Planning Approaches}
Initial research in autonomous driving often used graph-based or search-based algorithms, which discretize the state space into grids or motion
“primitives” and then apply search algorithms such as A* or D* [3].
While these methods can systematically find a solution if one exists, they can become computationally expensive in higher-dimensional state spaces,
especially when including vehicle dynamics.
Another popular family of methods is sampling-based planners, including Rapidly-exploring Random Trees (RRT) and Probabilistic Roadmaps (PRM) [4].
For on-road driving, variants like RRT* incorporate cost optimization and can quickly explore feasible regions.
However, sampling alone does not guarantee smoothness or dynamic feasibility, often requiring a secondary smoothing or optimization step.

\subsection{Optimization-Based Trajectory Planning}
In contrast to sampling-based methods, optimization-based trajectory planning explicitly encodes the cost criteria and constraints—such as collision
avoidance, road boundary constraints, and vehicle dynamics—into an objective function.
A trajectory is then optimized continuously rather than constructed through random sampling.
One prominent approach employs Frenet frames to transform the planning problem from a Cartesian to a curvilinear coordinate system, simplifying the
computation of collision and road boundary constraints [5].
This representation has proven effective for lane-keeping, lane changes, and freeway driving, where the environment can be reasonably approximated by
a reference path.

A closely related technique is Model Predictive Control (MPC) [6][7], which casts trajectory planning as a finite-horizon optimal control problem,
repeatedly solved online.
By incorporating vehicle dynamics, actuator limits, and predicted movements of surrounding vehicles, MPC can adapt in real time to changing traffic
conditions.
Variants such as Nonlinear MPC (NMPC) can handle more complex or higher-fidelity vehicle models (e.g., for aggressive maneuvers), but typically
require significant computational resources.
Recent work has explored stochastic MPC formulations to account for measurement and prediction uncertainties [8], aiming to provide robust
performance under noisy sensor data and uncertain driver intentions.

\subsection{Hierarchical and Multi-Layered Architectures}

Many autonomous driving systems employ a hierarchical architecture that separates global route planning from local trajectory optimization.
At the global level, simpler search-based or graph-based methods can select a route through the road network (e.g., from a start location to a
destination).
At the local level, an optimization-based planner refines this route into a dynamically-feasible, collision-free trajectory in real time [9].
This layered approach leverages the strengths of each method: global planners handle large- scale navigation, while local planners ensure feasibility
and handle dynamic obstacles, road curvature, and driving rules.

\subsection{Interaction-Aware Planning and Social Compliance}
A significant challenge in urban environments is predicting how other road users—vehicles, cyclists, and pedestrians—will move and interact.
Behavior prediction and interaction- aware planning thus play a critical role in modern autonomous driving pipelines [10].
One line of research integrates game-theoretic or multiagent models into the trajectory planning stage, enabling the autonomous vehicle to reason
about how human drivers might respond.
Alternatively, rules can be added to make the vehicle behave more cautiously around pedestrians or follow right-of-way rules at intersections.

\subsection{Learning-Based Approaches}
Recent advances in machine learning have led to the development of learning-based trajectory planning systems.
These can take the form of imitation learning—where the planner learns a cost function or policy from human driving data [11]—or deep reinforcement
learning, where the agent optimizes a reward function through simulation [12].
Nevertheless, pure data-driven methods can lack the stability and formal guarantees of classical optimization.
As a result, hybrid methods that use learning to shape the objective or constraints, while retaining an optimization-based trajectory generator for
low-level feasibility, have seen increased popularity [13].

\subsection{Summary}
In summary, the field of motion planning for autonomous driving has seen significant advancements over the years, evolving from classical path
planning approaches to more sophisticated optimization-based and learning-based methods.
Each approach has its strengths and weaknesses, and the choice of method often depends on the specific driving scenario and requirements.
Classical methods like graph-based and sampling-based planners provide systematic solutions but can struggle with high-dimensional state spaces and
dynamic feasibility.
Optimization-based methods offer a more continuous and constraint-aware approach, particularly effective in structured environments.
Hierarchical architectures combine global and local planning to handle large-scale navigation and real-time trajectory optimization.
Interaction-aware planning addresses the complexities of urban environments by predicting and responding to the behavior of other road users.
Finally, learning-based approaches leverage data-driven techniques to improve planning performance, though they often benefit from hybridization with
classical methods to ensure stability and feasibility.
As research continues, the integration of these diverse techniques promises to enhance the safety, efficiency, and robustness of autonomous driving
systems.
