\section{Introduction} \label{sec:introduction}

The problem of autonomous driving can be divided into four main components, each representing a crucial aspect of the system.

\begin{figure}[h!]
	\centering
	\begin{tikzpicture}[
			node distance=1cm,
			auto,
			thick,
			box/.style={rectangle, draw, text width=2cm, align=center, rounded corners, minimum height=1.5cm},
			arrow/.style={-Latex, thick},
			label/.style={font=\small, text width=2cm, align=center},
			highlighted/.style={fill=blue!30, font=\bfseries}
		]
		% Nodes
		\node[box] (route) {Route Planning};
		\node[box] (behavior) [right=of route] {Behavioral Layer};
		\node[box] (motion) [right=of behavior, highlighted] {Motion Planning Layer};
		\node[box] (local) [right=of motion] {Local Feedback};

		% Initial and final arrows
		\draw[arrow] (-2, 0) -- (route.west);
		\draw[arrow] (route.east) -- (behavior.west);
		\draw[arrow] (behavior.east) -- (motion.west);
		\draw[arrow] (motion.east) -- (local.west);
		\draw[arrow] (local.east) -- (11.8, 0);

		% Text labels below arrows
		% \node[label] at (-1.2, -1.25) {Destination};
		% \node[label] at (11.2, -1.25) {Control Commands};

	\end{tikzpicture}
	\caption{Overview of Autonomous Driving Problem Decomposition}
	\label{fig:autonomous_driving_overview}
\end{figure}

The process begins with the user providing a travel destination, which serves as the input to the first component, Route Planning.
In this phase, the system generates a sequence of waypoints through a predefined road network.

Next, the Behavioral Layer refines the waypoints by considering environmental factors such as other vehicles, obstacles, and road signs.
This layer defines the driving behavior at any given time, ensuring the vehicle adapts to dynamic traffic conditions.

With a behavior strategy in place, the Motion Planning Layer generates a trajectory that adheres to strict physical and safety constraints, ensuring
feasibility and compliance with rules of the road.

Finally, the Local Feedback component executes the plan by generating precise control commands—steering, throttle, and brake inputs—based on
real-time vehicle and environmental feedback.

Finding an exact solution to the motion planning problem is computationally intractable in most cases.
Therefore, numerical approaches are employed.
These methods fall into three main categories:

\begin{enumerate}
	\item Graph-Based Algorithms: These algorithms discretize the vehicle's possible states
	      and connect valid state transitions with edges.
	      A graph search algorithm can then be used to find an optimal trajectory.

	\item Incremental Tree Approaches: This category involves generating branches of feasible trajectories by randomly applying control commands and simulating
	      the resulting states.
	      The tree expands incrementally, exploring potential paths until a suitable trajectory is found.

	\item Optimization-Based Methods: These methods formulate the problem as an optimization task over a function space.
	      Optimization techniques aim to minimize an objective function (e.g., minimizing travel time or maximizing safety) while respecting constraints.
	      This is the approach we focus on.
\end{enumerate}

Our objective is to design a motion planner that consistently provides near real-time solutions.
We achieve this by leveraging modern, reliable solvers that can efficiently handle the optimization problem.
