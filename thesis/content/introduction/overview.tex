\section{Overview} \label{sec:overview}

\begin{figure}[h!]
	\centering
	\begin{tikzpicture}[
			node distance=1cm,
			auto,
			thick,
			box/.style={rectangle, draw, text width=2cm, align=center, rounded corners, minimum height=1.5cm},
			arrow/.style={-Latex, thick},
			label/.style={font=\small, text width=2cm, align=center},
			highlighted/.style={fill=blue!30, font=\bfseries}
		]
		% Nodes
		\node[box] (route) {Route Planning};
		\node[box] (behavior) [right=of route] {Behavioral Layer};
		\node[box] (motion) [right=of behavior, highlighted] {Motion Planning Layer};
		\node[box] (local) [right=of motion] {Local Feedback};

		% Initial and final arrows
		\draw[arrow] (-2, 0) -- (route.west);
		\draw[arrow] (route.east) -- (behavior.west);
		\draw[arrow] (behavior.east) -- (motion.west);
		\draw[arrow] (motion.east) -- (local.west);
		\draw[arrow] (local.east) -- (11.8, 0);

		% Text labels below arrows
		% \node[label] at (-1.2, -1.25) {Destination};
		% \node[label] at (11.2, -1.25) {Control Commands};

	\end{tikzpicture}
	\caption{Overview of Autonomous Driving Problem Decomposition}
	\label{fig:autonomous_driving_overview}
\end{figure}

The problem of autonomous driving can be divided into four main components, each representing an essential aspect of the system.
Figure 1.1 illustrates this decomposition, highlighting the motion planning layer as the focus of this work.

The process begins with the user providing a travel destination, which serves as the input to the Route Planning component.
This phase generates a sequence of waypoints through a predefined road network.
Next, the Behavioral Layer refines the waypoints by considering environmental factors such as other vehicles, obstacles, and road signs, ensuring the
vehicle adapts to dynamic traffic conditions.
Once a behavioral strategy is determined, the Motion Planning Layer generates a trajectory that satisfies physical and safety constraints, ensuring
feasibility and compliance with road rules.
Finally, the Local Feedback component executes the plan by generating precise control commands—steering, throttle, and brake inputs—based on
real-time vehicle and environmental feedback.

Finding an exact solution to the motion planning problem is computationally intractable in most cases.
As a result, numerical methods are commonly used to approximate solutions.
These approaches fall into three main categories:

\begin{enumerate}
	\item Graph-Based Algorithms discretize the vehicle's state space and connect valid
	      transitions with edges, allowing a graph search algorithm to determine an optimal
	      trajectory

	\item Incremental Tree Approaches expand a search tree by randomly applying control
	      commands and simulating state transitions until a feasible trajectory is found.

	\item Optimization-Based Methods formulate the problem as an optimization task over a function space, minimizing an objective function (e.g., travel time, energy efficiency)
	      while respecting constraints.
\end{enumerate}

We focus on optimization-based motion planning, which offers a structured and efficient approach to trajectory generation.
Our objective is to design a motion planner that consistently provides near real-time solutions.
To achieve this, we leverage modern optimization solvers.
