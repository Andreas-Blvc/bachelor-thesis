\section{Overview} \label{sec:overview}

\begin{figure}[b]
	\centering
	\begin{tikzpicture}[
			node distance=1cm,
			auto,
			thick,
			box/.style={rectangle, draw, text width=2cm, align=center, rounded corners, minimum height=1.5cm},
			arrow/.style={-Latex, thick},
			label/.style={font=\small, text width=2cm, align=center},
			highlighted/.style={fill=blue!30, font=\bfseries}
		]
		% Nodes
		\node[box] (route) {Route Planning};
		\node[box] (behavior) [right=of route] {Behavioral Layer};
		\node[box] (motion) [right=of behavior, highlighted] {Motion Planning Layer};
		\node[box] (local) [right=of motion] {Local Feedback};

		% Initial and final arrows
		\draw[arrow] (-2, 0) -- (route.west);
		\draw[arrow] (route.east) -- (behavior.west);
		\draw[arrow] (behavior.east) -- (motion.west);
		\draw[arrow] (motion.east) -- (local.west);
		\draw[arrow] (local.east) -- (11.8, 0);

		% Text labels below arrows
		% \node[label] at (-1.2, -1.25) {Destination};
		% \node[label] at (11.2, -1.25) {Control Commands};

	\end{tikzpicture}
	\caption{Overview of Motion Planning Problem Decomposition}
	\label{fig:motion_planning_overview}
\end{figure}

The motion planning system for autonomous driving is typically organized into a hierarchical, layered architecture that efficiently handles both
large-scale navigation and local trajectory optimization.
Figure \ref{fig:motion_planning_overview} illustrates this decomposition, adapted from ~\cite{paden_survey_2016}, and emphasizes the motion planning
layer, which is the focus of this work.

At the highest level, the Route Planning component takes a user-specified destination and uses search- or graph-based methods to generate a
high-level route through the road network.
This global planner addresses large-scale navigation by selecting a sequence of waypoints that outline the overall path from the start location to
the destination.

Following route planning, the Behavioral Layer refines the generated waypoints by incorporating environmental factors such as other vehicles,
obstacles, and road signs.
This step ensures that the vehicle adapts to dynamic traffic conditions and selects a behavioral strategy appropriate for the current driving
context.

Once the behavioral strategy is set, the Motion Planning Layer takes center stage.
Building on insights from early demonstrations in the DARPA Grand and Urban Challenges \cite{thrun_stanley_2006,montemerlo_junior_2008}, the motion
planning literature has evolved to tackle a wide variety of environments—from structured highways and urban roads to unstructured parking lots and
off-road terrains.
A unifying theme across these scenarios is the production of collision-free, dynamically feasible trajectories that account for comfort and
operational constraints.

In this context, an optimization-based local planner refines the global route into a smooth, continuous guiding trajectory in real
time~\cite{van_hierarchical_2020}.
Unlike the global route—represented as a series of discrete waypoints—this trajectory not only satisfies road rules and safety constraints but also
ensures fluid, comfortable motion that adapts to dynamic obstacles, road curvature, and other real-time factors.

Another critical element in the motion planning pipeline is the vehicle model.
This model defines the vehicle's position and orientation in the real world and predicts how these states evolve over time.
The choice of vehicle model significantly impacts both the accuracy and the computational complexity of the planning process: simpler models offer
efficiency at the cost of precision, whereas more complex models yield higher fidelity with increased computational demands.

Finally, the Local Feedback component executes the planned trajectory by generating precise control commands—steering, throttle, and brake
inputs—based on real-time vehicle and environmental feedback.
This closed-loop control ensures that the trajectory is followed accurately, even as conditions evolve.

% Next, the Behavioral Layer refines these waypoints by considering environmental factors such as other vehicles, obstacles, and road signs, ensuring
% the vehicle adapts to dynamic traffic conditions.
% Once a behavioral strategy is determined, the Motion Planning Layer takes center stage.
% Building on insights from early demonstrations in the DARPA Grand and Urban Challenges \cite{thrun_stanley_2006,montemerlo_junior_2008}, the motion
% planning literature has evolved to handle a wide variety of environments—from structured highways and urban roads to unstructured parking lots and
% off-road terrains.
% A unifying theme across these scenarios is the need to produce collision-free, dynamically feasible trajectories that also account for comfort and
% operational constraints.

% In this context, an optimization-based local planner refines the global route into a dynamically feasible, collision-free trajectory in real time
% \cite{van_hierarchical_2020}.
% Unlike the global route, which is represented as a series of waypoints, the resulting trajectory is a smooth, continuous guiding curve.
% It not only ensures compliance with road rules and safety constraints but also guarantees a fluid and comfortable motion that adapts to dynamic
% obstacles, road curvature, and other real-time factors.

% Another critical element in the motion planning pipeline is the vehicle model, which defines the position and orientation of the vehicle in the real
% world and predicts how these states evolve over time.
% The choice of model significantly impacts both the accuracy and computational complexity of the planning process.
% While simpler models are computationally efficient, they may lack precision; conversely, more complex models provide higher fidelity at the cost of
% increased computational resources.

% Finally, the Local Feedback component executes the plan by generating precise control commands—steering, throttle, and brake inputs—based on
% real-time vehicle and environmental feedback, ensuring that the trajectory is followed accurately.

% The motion planning problem can be divided into four main components, each representing an essential aspect of the system.
% Figure \ref{fig:motion_planning_overview} illustrates this decomposition, adapted from \cite{paden_survey_2016}, with an emphasis on the motion
% planning layer, which is the focus of this work.

% The process begins with the user providing a travel destination, which serves as the input to the Route Planning component.
% This phase generates a sequence of waypoints through a predefined road network.
% Next, the Behavioral Layer refines the waypoints by considering environmental factors such as other vehicles, obstacles, and road signs, ensuring the
% vehicle adapts to dynamic traffic conditions.
% Once a behavioral strategy is determined, the Motion Planning Layer generates a trajectory that satisfies physical and safety constraints, ensuring
% feasibility and compliance with road rules.
% Finally, the Local Feedback component executes the plan by generating precise control commands—steering, throttle, and brake inputs—based on
% real-time vehicle and environmental feedback.

Finding an exact solution to the motion planning problem is computationally intractable in most cases.
As a result, numerical methods are commonly used to approximate solutions.
These approaches fall into three main categories:

\begin{enumerate}
	\item Graph-Based Algorithms discretize the vehicle's state space and connect valid
	      transitions with edges, allowing a graph search algorithm to determine an optimal
	      trajectory

	\item Incremental Tree Approaches expand a search tree by randomly applying control
	      commands and simulating state transitions until a feasible trajectory is found.

	\item Optimization-Based Methods formulate the problem as an optimization task over a function space, minimizing an objective function (e.g., travel time, energy efficiency)
	      while respecting constraints.
\end{enumerate}

We focus on optimization-based motion planning, which offers a structured and efficient approach to trajectory generation while ensuring collision
avoidance and adherence to vehicle dynamics.
Compared to graph-based or Incremental Tree approaches, optimization-based methods provide smoother and more dynamically feasible trajectories by
directly incorporating vehicle dynamics constraints.
Additionally, they offer deterministic performance with guaranteed feasibility, avoiding the sampling inconsistencies and discontinuities often found
in tree-based methods.
By leveraging convex optimization and constraint reformulation, this approach efficiently computes safe and feasible trajectories while maintaining
real-time performance in dynamic environments.

Building on these advantages, we review existing optimization-based motion planning approaches, highlighting key methodologies, challenges, and
advancements in the field.
