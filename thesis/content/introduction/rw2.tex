
\subsection{Classical Motion Planning Approaches}

Early motion planning research focused on finding feasible paths in a geometric sense, often treating the vehicle as a point mass and ignoring
detailed dynamics.
Graph-based algorithms (e.g., A*, D*) discretize the state or configuration space into grids or lattices to search for collision-free paths, but they
can become computationally intractable in high-dimensional spaces.
Sampling-based planners, such as Probabilistic Roadmaps (PRM) and Rapidly-exploring Random Trees (RRT), offered a breakthrough in efficiency by
randomly sampling the continuous configuration space \cite{orthey_sampling-based_2024}.
PRM methods construct a graph of randomly-sampled collision-free configurations, while RRT incrementally grows a tree from the start state toward the
goal by sampling random inputs \cite{orthey_sampling-based_2024}.
These methods are probabilistically complete, meaning they will find a solution if one exists given enough time.
Variants like RRT* and PRM* were later developed to achieve asymptotic optimality (convergence towards the optimal path as samples increase) for
metrics like path length.

Despite their success in finding feasible paths even in high-dimensional or complex environments, classical planners have notable limitations.
They typically produce geometric paths without timing information, requiring a separate velocity profile or trajectory generation step to account for
kinematics and dynamics.
Moreover, they do not inherently optimize a specific cost function (beyond maybe path length for RRT*), so the resulting paths may be suboptimal in
terms of travel time, energy, or comfort.
In practice, an initial path from a sampling or graph search planner often needs refinement.
For example, trajectory optimization algorithms are commonly employed to smooth and shorten a raw path from an RRT or grid planner
\cite{schulman_finding_2013}, adjusting the path points or timing to satisfy dynamic constraints and improve optimality.
This two-stage approach (planning then post-optimization) can work, but it separates feasibility and optimality into distinct steps.
The transition to optimization-based methods aims to unify these steps by directly computing trajectories that are both feasible and optimal under
the given criteria.

\subsection{Optimization-Based Motion Planning}

Optimization-based motion planning formulates trajectory generation as a constrained optimization problem, ensuring that the computed trajectory is
both feasible and optimal within given physical and environmental constraints.
This approach leverages principles from optimal control theory, enabling planners to explicitly incorporate vehicle dynamics, collision avoidance,
and comfort-related cost functions.

Modern optimal control solvers often use direct transcription methods, which discretize the continuous-time trajectory planning problem into a
finite-dimensional constrained optimization problem.
In this formulation, the vehicle's state sequence $x_n$ and control sequence $u_n$ over a predefined horizon are treated as decision variables,
subject to vehicle dynamics constraints $x_{n+1} = f(x_n, u_n)$ \cite{tedrake2023trajopt}.
If the dynamics and constraints are linear (or properly approximated as such), the resulting optimization problem becomes a convex quadratic program
(QP) or linear program (LP), allowing for efficient real-time computation \cite{tedrake2023trajopt}.

While convex formulations provide computational efficiency and guaranteed global optimality, real-world motion planning problems are often non-convex
due to nonlinear vehicle dynamics and non-convex obstacle constraints.
To address this, researchers have explored strategies such as sequential convex programming, where a non-convex problem is iteratively approximated
by a sequence of convex subproblems \cite{schulman_finding_2013}.
However, these approaches rely on high-quality initializations and can still converge to local optima.

A widely used optimization-based motion planning framework is Model Predictive Control (MPC) \cite{falcone_predictive_2007, gray_robust_2013}.
MPC repeatedly solves a finite-horizon optimal control problem in a receding-horizon manner, applying only the first control input and re-optimizing
at each step.
This enables adaptation to dynamic environments and disturbances while enforcing constraints on vehicle dynamics and safety.
To ensure real-time feasibility, MPC implementations often rely on simplified vehicle models.
Depending on the application, planners may use kinematic or dynamic models, with linearized approximations used to maintain tractability
\cite{xia_survey_2024}.

Nonlinear MPC (NMPC), which accounts for full vehicle dynamics, offers greater accuracy but comes at a significant computational cost.
In contrast, linear time-varying (LTV) and linear time-invariant (LTI) MPC approximate the system dynamics using local linearizations, yielding
convex QP formulations that are computationally more efficient \cite{xia_survey_2024}.
Despite these advantages, even linearized MPC can be demanding for high-frequency trajectory planning.

To further enhance computational efficiency while maintaining feasibility, many trajectory planners adopt simplified motion models that ensure convex
formulations.
One such approach is the use of linear integrator models, which provide a structured yet computationally efficient representation of vehicle motion.
These models enable fast QP solvers while preserving key physical constraints, ensuring real-time feasibility in dynamic environments without
sacrificing trajectory smoothness or safety.
The next section explores how these models contribute to motion planning and how their limitations can be addressed through convex relaxation
techniques.

\subsection{Vehicle Dynamics Modeling in Optimization-Based Motion Planning}

\subsubsection{Linear Integrator Models for Convex Optimization}

Linear integrator models (e.g., double or triple integrators for position, velocity, and acceleration) are widely used in trajectory planning due to
their linear dynamics, which ensure convex optimization formulations.
These models allow motion planning to be framed as a quadratic program (QP) or mixed-integer QP (MIQP), enabling efficient real-time computation.

For instance, modeling motion with a triple integrator (incorporating acceleration and jerk) allows the direct imposition of acceleration limits as
linear constraints \cite{esterle_optimal_2020}.
Esterle et al.
leverage this property to encode non-holonomic behavior and collision avoidance constraints using a disjunctive MIQP formulation, ensuring that trajectory generation remains both feasible and computationally efficient \cite{esterle_optimal_2020}.
However, while integrator models offer significant computational advantages, they require additional modifications when applied to road-constrained
motion planning.

Planning on curved roads presents a fundamental challenge: simple integrator-based motion models assume independent lateral and longitudinal
dynamics, whereas real-world motion introduces coupling (e.g., speed-dependent lateral acceleration limits).
Eilbrecht and Stursberg demonstrate that when applied to curved roads, linear integrator models generate inherently non-convex coupled constraints
\cite{eilbrecht_challenges_2020}.
To address this, they propose inner convex approximations of the feasible set (using quantifier elimination), effectively relaxing the curvature
constraints while maintaining solution feasibility.
This strategy enables trajectory planners to respect road geometry constraints without sacrificing convexity.

\subsubsection{Kinematic and Dynamic Bicycle Models with Convex Constraints}

Vehicle dynamics are often modeled using either kinematic or dynamic bicycle models (single-track models).
Kinematic models approximate motion through geometric constraints (e.g., steering-to-curvature relations), while dynamic models incorporate tire
forces and slip dynamics.
The choice between these models introduces a trade-off between computational efficiency and dynamic accuracy.

Kong et al.
compare kinematic and dynamic bicycle models within an MPC framework, showing that with appropriate discretization, kinematic models can match the predictive accuracy of dynamic models while being significantly less computationally expensive \cite{kong_kinematic_2015}.
Their study demonstrates that at moderate speeds, kinematic models provide accurate trajectory predictions with lower computational requirements.
However, at high speeds (where lateral tire forces dominate), dynamic models become necessary to capture tire slip effects, at the cost of increased
computational complexity.

To ensure feasibility while maintaining a convex optimization formulation, researchers impose additional constraints on kinematic models.
Polack et al.
introduce a speed-dependent curvature constraint, effectively imposing a lateral acceleration limit that ensures the vehicle remains within its handling capabilities \cite{polack_guaranteeing_2018}.
By doing so, their kinematic MPC planner maintains convexity while ensuring trajectory feasibility across different speed regimes.
This type of constraint relaxation enables simpler models to approximate dynamic behavior without requiring full nonlinear dynamics.

\subsubsection{Real-Time Feasibility and Efficiency of Convex Approximations}

One of the main motivations for using convex optimization-based motion planning is its ability to ensure real-time feasibility.
Solving convex QPs is significantly faster than non-convex alternatives, making convex approximations crucial for high-frequency planning
applications.

Kong et al.
report that kinematic models allow larger discretization steps, reducing computational load while maintaining accurate trajectory tracking \cite{kong_kinematic_2015}.
Their experiments show that a kinematic model MPC with a longer update interval (200 ms) produced similar accuracy to a dynamic model with a 100 ms
update, highlighting the efficiency of convex formulations.

Esterle et al.
demonstrate that MIQP formulations using triple integrators can efficiently handle decision-level planning tasks, solving realistic intersection-turning scenarios in real-time \cite{esterle_optimal_2020}.
Meanwhile, pure QP-based convex trajectory formulations (e.g., using double integrators with convexified constraints) achieve even higher solve rates
(10-50 Hz) on automotive-grade hardware \cite{polack_guaranteeing_2018}.
These studies collectively emphasize that by restricting motion models to convex domains, planners can maintain computational efficiency without
sacrificing feasibility.

\subsubsection{Handling Road Constraints in a Convex Framework}

Beyond vehicle dynamics, road constraints play a critical role in motion planning.
Ensuring that planned trajectories remain within lane boundaries and follow road curvature constraints is essential but often introduces
non-convexity.

Eilbrecht and Stursberg discuss how using a Cartesian integrator model introduces road constraints (e.g., do not leave the road or follow the lane
curvature) that are inherently non-convex \cite{eilbrecht_challenges_2020}.
To maintain convexity, they propose polyhedral inner approximations of the feasible state-input space, effectively linearizing the speed-curvature
relationship into a set of linear inequalities.

Similarly, Polack et al.
demonstrate that embedding road topology constraints into kinematic MPC (via speed-conditioned steering limits) enables road curvature constraints to be implicitly satisfied \cite{polack_guaranteeing_2018}.
By combining linearized vehicle models with carefully designed convex constraints, these studies illustrate how road constraints can be efficiently
integrated into convex optimization frameworks without sacrificing real-time feasibility.

\subsection{Summary and Transition to Problem Formulation}

The reviewed literature highlights the critical trade-offs in motion planning: simplified models enable convex optimization and real-time
feasibility, but additional constraints must be imposed to ensure trajectory feasibility.
Linear integrator models provide a computationally efficient formulation but require convex approximations for road geometry and vehicle constraints.
Kinematic models, when constrained appropriately, offer a balance between efficiency and dynamic feasibility, while dynamic models provide higher
accuracy at a significant computational cost.

Building upon these insights, this thesis develops a motion planning framework that leverages convex formulations of vehicle dynamics and road
topology constraints.
By integrating quantifier elimination and convex relaxation techniques, our approach ensures feasibility while preserving computational efficiency,
bridging the gap between high-fidelity dynamic models and real-time motion planning.
The next section formalizes this problem statement and introduces our proposed methodology.

% The following chapter formally defines the problem statement and the proposed solution approach.
% We will detail how the vehicle model is chosen and formulated, what cost and constraints define the optimal trajectory, and how convexification
% techniques are applied.
% This formulation is directly informed by the gaps identified in existing work: it seeks to improve trajectory optimality and feasibility in scenarios
% where classical planners struggle, and to reduce computational burden compared to brute-force nonlinear optimization.
% In summary, by learning from the related work and leveraging convex optimization in a targeted way, the proposed method strives to advance motion
% planning performance for autonomous vehicles, especially in challenging yet tractable domains that demand both efficiency and realism.
