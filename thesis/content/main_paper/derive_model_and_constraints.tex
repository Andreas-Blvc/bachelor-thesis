\chapter{Motion Planning using the Point Mass Model} \label{ch:main-paper}

In Section \ref{section:basic_pm}, we introduced the point mass model with global coordinates and no orientation.
We now extend this model to the Frenet frame, which allows for a more structured approach to motion planning along a predefined road.

To achieve this, we first define the curvature of the reference path, which quantifies the rate of change of the tangent angle with respect to arc length:
\begin{equation}
	C := \frac{d\theta}{ds}.
\end{equation}
Additionally, let $\psi$ be the orientation of the vehicle, and define the alignment error $\xi$ as the difference between the vehicle's orientation and the reference path's tangent angle:
\begin{equation}
	\xi := \psi - \theta.
\end{equation}
This alignment error measures the deviation of the vehicle's heading from the reference path.

Using these definitions, along with established kinematic relationships and coordinate frame transformations, we can systematically derive the motion
dynamics in the Frenet frame.
These equations describe how the vehicle's velocity components in the body-fixed frame relate to changes in Frenet frame coordinates.

The first-order derivatives are given by:
\begin{align}
	\dot{s}(1 - nC(s)) & = v_x\cos{\xi} - v_y\sin{\xi}, \label{eq:first_derivative_long} \\
	\dot{n}            & = v_x\sin{\xi} + v_y\cos{\xi}. \label{eq:first_derivative_lat}
\end{align}

These equations describe the evolution of the vehicle's longitudinal and lateral positions in the Frenet frame.
The presence of the curvature term $C(s)$ ensures that the equations correctly account for the curvature of the reference path.

For the second derivatives, we obtain:
\begin{align}
	a_{x,tn} & = (a_x - v_y\dot{\psi})\cos{\xi} - (a_y + v_x\dot{\psi})\sin{\xi}, \label{eq:second_derivative_long} \\
	a_{y,tn} & = (a_x - v_y\dot{\psi})\sin{\xi} + (a_y + v_x\dot{\psi})\cos{\xi}. \label{eq:second_derivative_lat}
\end{align}

where the transformed acceleration terms in the Frenet frame are given by:
\begin{align}
	a_{x,tn} & := \ddot{s}(1 - nC(s)) - 2\dot{n}
	C(s)\dot{s} - nC'(s)\dot{s}^2, \label{def:axtn} \\ a_{y,tn} & := \ddot{n} + C(s)\dot{s}^2(1 - nC(s)).
	   \label{def:aytn}
\end{align}

These expressions incorporate the effects of curvature and alignment, ensuring that motion planning remains accurate and consistent with the road
geometry.
In subsequent sections, we will leverage these equations to develop motion planning strategies that optimize trajectory feasibility and control
performance.

\section{Model States in Frenet Frame}

We now define the state variables $x_{tn}$, which represent the vehicle's position, orientation, and velocities in the Frenet frame, and control inputs $u_{tn}$ for our point mass model in the Frenet frame:

\begin{equation}
	x_{tn} = \begin{bmatrix}
		s       \\
		n       \\
		\xi     \\
		\dot{s} \\
		\dot{n} \\
		\dot{\psi}
	\end{bmatrix}
\end{equation}

The state vector consists of the Frenet frame coordinates that define the vehicle's location, the alignment error representing the deviation of the
vehicle's orientation from the reference path, and the first-order time derivatives of these quantities, which capture the vehicle's velocity
components.

We define the following three control inputs:

\begin{equation}
	u_{tn} = \begin{bmatrix}
		a_{x} \\
		a_{y} \\
		a_\psi
	\end{bmatrix}
\end{equation}

Using the previously derived equations, we can formulate the model dynamics in the Frenet frame as:

\begin{equation}
	\label{eq:frenet_frame_pm_dynamics}
	f_{tn}(x_{tn}, u_{tn}) = \begin{bmatrix}
		\dot{s}                                   \\
		\dot{n}                                   \\
		\dot{\psi} - C(s)\dot{s}                  \\
		\frac{ a_{x,tn} + 2\dot{n}
		C(s)\dot{s}+nC'(s)\dot{s}^2 } { 1-nC(s) } \\ a_{y,tn} - C(s)\dot{s}^2(1-nC(s)) \\ a_\psi\end{bmatrix} \end{equation}

With this
formulation, we are now able to model vehicle motion in the Frenet frame along a predefined road.
This approach facilitates the definition of road topology constraints in a straightforward manner, ensuring that the resulting constraints align with
the principles of disciplined convex programming (DCP).
However, the inclusion of curvature terms introduces non-convexity into the system dynamics.
In the following section, we will explore methods to address this challenge and develop techniques to handle the resulting non-convex constraints
effectively.

\section{Derivation of the Integrator Model} \label{ch:der_int_mod}

First thing we are going to do is that we decouple the input.
If you take a look at the dynamics equations \ref{eq:frenet_frame_pm_dynamics}, the last two entries both contain the control input $a_x$, $a_y$ if
you plug in the equations \ref{def:axtn} and \ref{def:aytn}.
By making the following Assumption.
\subsubsection{Assumption: Alignment Error}

Orientation of the vehicle $\psi$ equals the angle of the road $\theta$:
\begin{equation}
	\xi = \psi - \theta = 0
\end{equation}
which directly implies the following three points, which will help us in further steps:
\begin{itemize}
	\item $[a_x, a_y] = [a_{x,tn}, a_{y,tn}]$
	\item $\dot{\psi} = \dot{\theta} = \frac{d\theta}{ds} \cdot \frac{ds}{dt} = C(s)\dot{s}$
	\item $a_\psi = \ddot{\psi} = \ddot{\theta} = C'(s) \dot{s}^2 + C(s)\ddot{s}$
\end{itemize}

At first glance seems like we are removing the orientation from our model, but we actually just force the vehicle to always be aligned with the road.
And since we allow lateral acceleration of the vehicle, it is not fixed to a constant offset to the reference and can still move left or right.
Once the trajectory has been planned we let further layers handle the problem of finding the correct control namely steering and longitudinal
acceleration for the vehicle, this will be based on more complex models.

With this assumption we end up with dynamics equation, which are affine linear in $a_{x,tn}$ and $a_{y,tn}$.
(TODO EXPLAIN WHY).
This allows us to introduce artificial control inputs to linearize the dynamics.
\begin{equation}
	\label{def:artificial_controls}
	\tilde{u} := \begin{bmatrix}
		u_t \\
		u_n
	\end{bmatrix} = \begin{bmatrix}
		\frac{
			a_{x,tn} + 2\dot{n}
		C(s)\dot{s} + nC'(s)\dot{s}^2 }{ 1 - nC(s) } \\ a_{y,tn} - C(s)\dot{s}^2(1 - nC(s))\end{bmatrix} \end{equation}

In summary, by
fixing one state variable, which leads to slightly less realistic vehicle model, but which led to affine linear dynamics in the controls inputs.
This enabled us to introduce artificial variables, which linearizes the dynamics entirely as follows.

\subsubsection{Resulting Simplified Model}

Since the orientation is fixed, we can remove it from the states variables, and we end up with the following states variables.
\[
	x_{tn} = \begin{bmatrix} s, & n, & \dot{s}, & \dot{n} \end{bmatrix}
\]
We now use the new defined artificial controls inputs, for controlling the system.
\[
	\tilde{u} = \begin{bmatrix} u_t, & u_n \end{bmatrix}
\]
The dynamics are given by:
\begin{equation}
	f(x_{tn}, \tilde{u}) = \begin{bmatrix}
		0 & 0 & 1 & 0 \\
		0 & 0 & 0 & 1 \\
		0 & 0 & 0 & 0 \\
		0 & 0 & 0 & 0 \\
	\end{bmatrix} x_{tn} + \begin{bmatrix}
		0 & 0 \\
		0 & 0 \\
		1 & 0 \\
		0 & 1 \\
	\end{bmatrix} \tilde{u}
\end{equation}

\subsection{Constraints}

We have tackled the dynamics constraints of our discrete-time optimal trajectory planning problem \ref{eq:discrete_dynamics}.
Next we are going to define our constraints coupling constraints on the states variables and the controls inputs and see which Problem arise and how
we can deal with them.

Let us first have a look on the vehicle constraints.
Let $\square$ be one of the variables used in trajectory planning, the upper bound is annotated with $\overline{\square}$ and the lower bound with
$\underline{\square}$ which are both constant during planning.

Constraints on the velocity is defined in a body fixed manner, with upper and lower bound, where we use the following annotations.

\begin{align}
	\underline{v_x} \leq v_x \leq \overline{v_x} \\
	\underline{v_y} \leq v_y \leq \overline{v_y}
\end{align}

We can easily apply \ref{eq:first_derivative_lat} and \ref{eq:first_derivative_long} to get resulting constraints on our state variables.
The equations reduce together with $\xi = 0$ to:

\begin{align}
	\underline{v_x}  \leq  \dot{s}(1-nC(s))  \leq  \overline{v_x} \\
	\underline{v_y}  \leq  \dot{n}           \leq  \overline{v_y}
\end{align}

For the acceleration two types of constraints are usually defined, the first one which constrains the relations of the longitudinal and lateral acceleration as:
\begin{equation}
	a_x^2 + a_y^2 \leq c \in \mathbb{R}^+
\end{equation}
for some constant radius $c$, the second similar to the velocity as follows:
\begin{align}
	\underline{a_x} \leq a_x \leq \overline{a_x} \\
	\underline{a_y} \leq a_y \leq \overline{a_y}
\end{align}

Using our derived equations we can define a mapping from the models state variables and artificial variables to our body fixed accelerations.
(TODO: the main paper mentions $a_b = a_{tn}$ which I believe should be \[
	a_b + \begin{bmatrix}
		-\dot{s}(1-nC(s))C(s)\dot{s} \\
		\dot{n}C(s)\dot{s}
	\end{bmatrix} = a_{tn}\])

\begin{align}
	g(x_{tn}, \tilde{u}) :=
	\begin{bmatrix}
		(1 - nC(s)) u_t - (2\dot{n}C(s)\dot{s} + nC' \dot{s}^2) -\dot{s}(1-nC(s))C(s)\dot{s} \\
		u_n + C(s) \dot{s}^2 (1 - nC(s)) +  \dot{n}C(s)\dot{s}
	\end{bmatrix}  = \begin{bmatrix}
		                 a_x \\
		                 a_y
	                 \end{bmatrix}
\end{align}
Combining the constraints on the vehicle derive from the body fixed notion and the straight forward constraint given by the road constraints we can define our coupling constraint set $\mathcal{C}$ as:
\begin{equation}
	\mathcal{C} := \left\{
	\begin{bmatrix} x_{tn} \\ \tilde{u} \end{bmatrix} \; \middle|\;
	\begin{aligned}
		 & \underline{s} \leq s \leq \overline{s},                                     \\
		 & \underline{n}(s) \leq n \leq \overline{n}(s),                               \\
		 & \underline{v_x}  \leq  \dot{s}(1-nC(s))  \leq  \overline{v_x}               \\
		 & \underline{v_y} \leq \dot{n} \leq \overline{v_y}                            \\
		 & \underline{\dot{\psi}} \leq C(s) \dot{s} \leq \overline{\dot{\psi}},        \\
		 & \underline{a_{\psi}} \leq C' \dot{s}^2 + C(s) u_t \leq \overline{a_{\psi}}, \\
		 & \begin{bmatrix}
			   \underline{a_x} \\ \underline{a_y}
		   \end{bmatrix} \leq g(x_{tn}, \tilde{u}) \leq \begin{bmatrix}
			                                                \overline{a_x} \\\overline{a_y}
		                                                \end{bmatrix} \\
		 & ||g(x_{tn}, \tilde{u})||^2 \leq c
	\end{aligned}
	\right\}
\end{equation}

This set is highly non-convex, and we now face the problem to 'convexify' those constraints.
We will do so by finding an inner polytope of the set $\mathcal{C}$ stated as follows.

\subsubsection{Problem Definition: Finding an Inner Polytope}
\label{problem:inner_polytope}
Given set $\mathcal{C}$ over the state variables $x_{tn}$ and control inputs $\tilde{u}$, find $\underline{\mathcal{C}}$, such that:
\begin{equation}
	\underline{\mathcal{C}} = \left\{ \begin{bmatrix}
		x_{tn} \\ \tilde{u} \end{bmatrix} \; \middle|\;
	N \begin{bmatrix}
		x_{tn} \\ \tilde{u} \end{bmatrix} \leq b
	\right\} \subseteq \mathcal{C}
\end{equation}

We will demonstrate two methods how one can archive, both them try to archive the following.

We want to find the set $\tilde{\underline{\mathcal{C}}}$ over the variables $\dot{s}$ and $\tilde{u}$, such that:

\begin{equation}
	\tilde{\underline{\mathcal{C}}} =
	\left\{ \;
	\begin{bmatrix}
		\dot{s} \\
		u_t     \\
		u_n
	\end{bmatrix}
	\middle|\;
	\begin{bmatrix}
		x_{tn} \\ \tilde{u}
	\end{bmatrix} \in \mathcal{C}, \text{for all } \begin{bmatrix}
		s \\
		n \\
		\dot{n}
	\end{bmatrix} \in \begin{bmatrix}
		\underline{s}, \overline{s} \\
		\underline{n}, \overline{n} \\
		\underline{\dot{n}},  \overline{\dot{n}}
	\end{bmatrix}
	\right\}
\end{equation}

To be more specific, we want to find a set of linear constraints, which each spans a half space and let $\tilde{\underline{\mathcal{C}}}$ be
intersection of all those half spaces, then the following should hold.

\begin{equation}
	\label{eq:forall_formula}
	\begin{bmatrix}
		\dot{s} \\
		u_t     \\
		u_n
	\end{bmatrix} \in \tilde{\underline{\mathcal{C}}}
	\implies
	\forall  \begin{bmatrix}
		s \\
		n \\
		\dot{n}
	\end{bmatrix} \in \begin{bmatrix}
		\underline{s}, \overline{s} \\
		\underline{n}, \overline{n} \\
		\underline{\dot{n}},  \overline{\dot{n}}
	\end{bmatrix}: \quad
	\begin{aligned}
		 & (\underline{v_x}  \leq  \dot{s}(1-nC(s))  \leq  \overline{v_x}             & \land \\
		 & \underline{\dot{\psi}} \leq C(s) \dot{s} \leq \overline{\dot{\psi}}        & \land \\
		 & \underline{a_{\psi}} \leq C' \dot{s}^2 + C(s) u_t \leq \overline{a_{\psi}} & \land \\
		 & \begin{bmatrix}
			   \underline{a_x} \\ \underline{a_y}
		   \end{bmatrix} \leq g(x_{tn}, \tilde{u}) \leq \begin{bmatrix}
			                                                \overline{a_x} \\\overline{a_y}
		                                                \end{bmatrix}               & \land   \\
		 & ||g(x_{tn}, \tilde{u})||^2 \leq c )
	\end{aligned}
\end{equation}

The problem therefore reduces to the Elimination of the $\forall$ quantifier.
