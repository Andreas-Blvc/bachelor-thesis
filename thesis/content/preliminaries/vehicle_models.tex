\section{Vehicle Models} \label{sec:vehicle_models}

A vehicle model describes the position and orientation of the vehicle in the real world and predicts how these states change over time.
Different models vary in complexity and accuracy.
Generally, higher complexity results in increased accuracy.

Here, we introduce two fundamental models.

The state variables $x_i$ represent the current position and pose of the vehicle, potentially including velocity, steering angle, or other relevant
parameters.
We group these variables into a state vector $x$.

The control inputs $u_i$ are external influences that modify the state.
These are grouped into a control vector $u$.
Both $x$ and $u$ are time-dependent.

\subsection{Point Mass Model} \label{subsec:point_mass_model}

The point mass model (PM) consists of four state variables:

\begin{equation}
	x = \begin{bmatrix} p_x \\ p_y \\ v_x \\ v_y \end{bmatrix}
	\label{eq:states_pm}
\end{equation}

Here, $p_x$ and $p_y$ represent the vehicle's position in a global fixed coordinate system.
Instead of an explicit orientation, velocity is divided into its components $v_x$ and $v_y$.

The model has two control inputs:

\begin{equation}
	u = \begin{bmatrix} a_x \\ a_y \end{bmatrix}
	\label{eq:controls_pm}
\end{equation}

These correspond to accelerations in the $x$ and $y$ directions.

The future position and velocity are determined by the following system of differential equations:

\begin{equation}
	\dot{x} = A x + B u
\end{equation}

where

\begin{equation}
	A = \begin{bmatrix} 0 & 0 & 1 & 0 \\ 0 & 0 & 0 & 1 \\ 0 & 0 & 0 & 0 \\ 0 & 0 & 0 & 0 \end{bmatrix}, \quad
	B = \begin{bmatrix} 0 & 0 \\ 0 & 0 \\ 1 & 0 \\ 0 & 1 \end{bmatrix}
\end{equation}

The control inputs are bounded by a circular constraint, where the radius $a_{\max}$ is a model parameter:

\begin{equation}
	\sqrt{u_1^2 + u_2^2} \leq a_{\max}
\end{equation}

This model is the simplest commonly used for motion planning.

\subsection{Bicycle Model} \label{subsec:bicycle_model}

The kinematic single track model (KST) consists of five state variables:

\begin{equation}
	x = \begin{bmatrix} p_x \\ p_y \\ \delta \\ v \\ \psi \end{bmatrix}
	\label{eq:states_kst}
\end{equation}

Similar to the point mass model, the first two state variables $p_x$ and $p_y$ define the vehicle's global position in a two-dimensional coordinate
system.
The vehicle is now modeled with an orientation $\psi$ relative to the global $x$-axis.
The velocity vector describes the velocity $v$ of the rear wheel, aligning with the orientation.
The front wheels can rotate around the yaw axis, and their angle relative to the orientation is represented by the steering angle $\delta$.

Two control inputs modify the velocity and steering, directly affecting the state variables:

\begin{equation}
	u = \begin{bmatrix} v_{\delta} \\ a_{\text{long}} \end{bmatrix}
	\label{eq:controls_kst}
\end{equation}
where $v_{\delta}$ is the steering velocity, and $a_{\text{long}}$ is the longitudinal acceleration.

The future state follows these differential equations:

\begin{align}
	 & \dot{p}_x = v\cos(\psi)                    \\
	 & \dot{p}_y = v\sin(\psi)                    \\
	 & \dot{\delta} = v_{\delta}                  \\
	 & \dot{v} = a_{\text{long}}                  \\
	 & \dot{\psi} = \frac{v}{l_{wb}} \tan(\delta) \\
\end{align}

The single-track name originates from simplifying front and rear wheels into single contact points, assuming no wheel slip, leading to a kinematic
model abstraction.
The following figure comprehends the whole model nicely.

\begin{figure}[h]
	\centering
	\begin{tikzpicture}
		% Axes
		\draw[->] (0,0) -- (2,0) node[right] {$x$};
		\draw[->] (0,0) -- (0,2) node[above] {$y$};

		% Rear Wheel
		\fill (2,2) circle (2pt); % Draws a small point at (2,2)

		% Vehicle body
		\draw[thick,rotate around={11.536959-90:(2,2)}] (1.8,1.3) rectangle (2.2,2.7);
		\draw[thick,rotate around={26.536959-90:(7,3)}] (6.8,2.3) rectangle (7.2,3.7);

		% Wheelbase
		\draw[-] (2,2) -- (7,3);
		\draw[dashed] (2,2) -- (1.7,3.5);
		\draw[dashed] (7,3) -- (6.7,4.5);
		\draw[dashed, <->] (1.8,3) -- (6.8,4) node[midway,above] {$l_{wb}$};

		% Velocity vector
		\draw[->] (2,2.1) -- (4,2.5) node[midway,above] {$v$};

		% Heading angle
		\draw[dashed] (3.25,2.25) -- (6,2.25);
		\draw[->] (6,2.25) arc (0:11.536959:2.75);
		\node at (5.7,2.5) {$\psi$};

		% Steering angle
		\draw[dashed] (7,3) -- (8.5,3.3);
		\draw[dashed] (7,3) -- ++(26.536959:1.5);
		\draw[->] (8.5,3.3) arc (11.536959:26.536959:1.5);
		\node at (8.2,3.43) {$\delta$};

		% Displacement vector
		\draw[dashed,thick,->] (0,0) -- (1.95,1.95)
		node[midway, left, shift={(-0,+0.4)}] {$\begin{bmatrix}s_x \\ s_y \end{bmatrix}$};
	\end{tikzpicture}
	\caption{Bicycle model representation of a vehicle.}
	\label{fig:bicycle_model}
\end{figure}

The following additional constraints are part of the model, a parameter $a_{max}$ is introduced:

\begin{equation}
	\sqrt{u_2^2 + (x_4\dot{x}_5)^2} \leq a_{\max}
\end{equation}

For both models, the vehicle is additionally constrained by its velocity range, steering angle range, and the rate of change of the steering angle.
These constraints are natural and should not be overlooked.

\subsection{Curve Following Coordinate System} \label{subsec:curve_following_coordinate_system}