\section{Problem Formulation} \label{sec:convex_discrete_time_optimization}

Motion planning for autonomous vehicles must simultaneously address two fundamental challenges: ensuring collision-free trajectories and accurately
capturing the vehicle's dynamic behavior.
In this section, we introduce a discrete-time formulation of the trajectory planning problem and discuss its inherent computational complexity.
This provides the groundwork for our convex optimization approach.

\subsection{Computational Complexity} \label{subsec:complexity}

Optimal trajectory planning is computationally intractable and classified as PSPACE-hard \cite{reif_complexity_1979}, with resource requirements
growing exponentially with problem size.
This complexity arises from the continuous state space, the non-convex nature of vehicle dynamics, and the need to enforce collision avoidance while
satisfying motion constraints.
These challenges, akin to the classic Movers' Problem, are further complicated by time dependencies and evolving vehicle behavior.
Exact solutions are impractical for real-time use, necessitating numerical optimization, heuristics, or approximate solvers for efficient
near-optimal solutions.

\subsection{Discrete-Time Problem Formulation}

To render the trajectory planning problem tractable, we discretize the continuous-time formulation into a finite set of time steps.
Let $\mathcal{X}$ denote the set of valid vehicle states and $\mathcal{U}$ the set of feasible control inputs.
We define the trajectory at discrete time points $\{t_i\}_{i=1,\dots ,m}$, with the state at time $t_i$ denoted by $\pi(t_i) = x_i$.
Our objective is to minimize a cumulative cost function $ J: \mathcal{X} \times \mathcal{U} \to \mathbb{R}, $ which penalizes deviations from desired
behaviors, including collisions and dynamic infeasibilities.

\subsubsection{Discrete-Time Optimal Trajectory Planning}\label{subsubsec:discrete_time_optimal_trajectory_planning}

Given a 7-tuple
$
	(\mathcal{X}, \mathcal{U}, x_{\text{initial}}, X_{\text{goal}}, f, J, \{t_i\}_{i=1,\dots ,m}),
$
the discrete-time optimal trajectory planning problem is defined as:
\begin{align}
	u^*     & = \underset{u \in \mathcal{U}^{m-1}}{\operatorname{arg\,min}} \sum_{i=1}^{m-1}
	J(x_{i+1}, u_{i}), \label{eq:obj}                                                                                                                                   \\ \text{s.t.
	} \quad & x_1  = x_{\text{initial}} \label{eq:init}                                                                                                                 \\
	        & x_m          \in X_{\text{goal}} \subseteq \mathcal{X} \label{eq:goal}                                                                                    \\
	        & (x_i, u_i)   \in \mathcal{C} \subseteq \mathcal{X} \times \mathcal{U}          & \text{for all}\, i \in \{1, \dots, m-1\} \label{eq:coupling_constraints} \\
	        & x_{i+1}      = f(x_i, u_i, \Delta t_i)                                         & \text{for all}\, i \in \{1, \dots, m-1\} \label{eq:discrete_dynamics}
\end{align}
where $\Delta t_i = t_{i+1} - t_i$, and $\mathcal{C}$ represents the set of coupling constraints that jointly enforce collision avoidance and the vehicle's dynamic limitations.
The system dynamics are approximated using an integration scheme, as specified in \eqref{eq:discrete_dynamics}, which estimates the state at time
$t_{i+1}$ based on the state at time $t_i$ and the control input $u_i$.

\subsubsection{Disciplined Convex Programming (DCP)}

The DCP framework imposes specific rules on how optimization problems must be formulated, which verifies that the resulting problem is convex.
This allows the use of standard convex solvers, which often solve such problems relatively efficiently, though actual performance depends on problem
size and numerical properties.
The key principles of DCP are as follows:
\begin{itemize}
	\item The objective function must be convex over the feasible set if it is to be minimized, or concave if it is to
	      be maximized.
	\item Constraints must be formulated in one of the following forms:
	      \begin{itemize}
		      \item An equality constraint between affine expressions: \(\text{affine} = \text{affine}\),
		      \item An inequality constraint where a convex function is bounded above by a concave function: \(\text{convex} \leq \text{concave}\).
	      \end{itemize}
\end{itemize}
The use of DCP in this work involves defining the cost function and constraints in a manner that satisfies the DCP rules.
We will use the term \textit{convex constraint} or \textit{convex form} to refer to constraints that adhere to these rules.
