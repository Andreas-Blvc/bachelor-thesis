\chapter{Format} \label{ch:format}
This chapter introduces the basic layout of your thesis. In general, you want to make sure that each chapter flows smoothly and is linked to the one before. One way to do this is to briefly (in half a sentence or less) summarize what was discussed in the previous chapter, though you do not want to become too repetitive. If this cannot be done well, then leave it out. More critical is a short overview of what your reader can expect in the current chapter. Consider the possibility that your reader may not only read your work from start to finish in one go, but may also re-read certain chapters individually or out-of-order. 

\begin{itemize}
	\item \textit{In the previous chapter, xyz was discussed. Now we turn to.../Now we will look at.../In this chapter...}
\end{itemize}

What follows is a general outline, although there is no one ``correct'' way to format your thesis. The sections given in this reference guide (Introduction, Approach, Evaluation, etc.) each refer to a different chapter of your thesis. The subsections presented here correlate therefore to chapter sections. Depending on what makes sense for your topic and how you formulate your argument, certain parts can be merged or split. As mentioned above, make sure to tell a complete story from beginning to end, i.e., structure each chapter as a mini-thesis with an introduction, main part, and conclusion (though not necessarily divided that way into sections). For more information, see e.g. this website\footnote{\url{https://student.unsw.edu.au/thesis-structure}}.


\section{Introduction} \label{sec:intro}
The first chapter of your text. Clearly state your problem, what your work attempts to do, as well as why it is important. This should be understandable for a general audience, as you do not know who will be reading your thesis. 

Present the related work. Explain what has already been accomplished/solved -- and then explain what your approach offers that previous ones do not. What are their disadvantages, and how does your approach improve upon their drawbacks? The literature review section should summarize and evaluate related works, thus proving to your reader that your presented problem exists. To find related work, see the advise presented in Ch.~\ref{ch:lit_review}.

Once you have presented the previous approaches, restate your approach, its novelty, and the methods you use. Present the challenges of your approach as well -- you are not going to save all the world's problems with your thesis, but you can try to make it easier to do so. Do not make sweeping claims and over-generalizations, e.g., \textit{This topic is the most important of this generation and many to come.}

End the introduction with an outline of your paper. Introduce each chapter/section and give a brief summary of what they contain. \textit{This paper/thesis/work is structured as follows: In Chapter 2...is presented in Chapter 3....Chapter 4 discusses...}

Often, it is useful to write the introduction after you have completed the rest of the paper: that way, you can correctly summarize what you set out to accomplish in the first place and how you did so. However, you should review related work as soon as you start your thesis to not reinvent the wheel.


\section{Approach}
This can be considered the main part of your work. Here, you want to go into detail about your approach. Explain the concept and methods used, as well as any shortcomings it may have. 

\subsection{Implementation/Method}
Depending on your approach, a description of your implementation may merit its own section or subsection. If the approach you are presenting is more general and/or can be applied to different areas, you may only have the time and resources for one example implementation.

This section should therefore describe your implementation process throughly and precisely, giving all necessary parameters, so that your reader can reproduce your results. If necessary, your source code can go in the appendix. 


\section{Numerical Experiments}
Here, you should give the results of your implementation and critically discuss them in detail. In order to prove the applicability/relevancy of your thesis, you typically also need to compare your approach to others. You should clearly define the setup and parameters of your experiment, then give your results and discuss them in detail. If your results are unexpected, figure out why and explain that to the reader. 


\section{Conclusion}
Here is the place to summarize what you have written: your goal, your approach, your results. Discuss what can be drawn from what you have achieved. End the thesis by describing your future work, i.e., what from your approach can be expanded and improved upon in further projects.