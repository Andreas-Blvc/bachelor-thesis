\chapter{Problem Framework}

The problem of motion planning using an optimization-based approach is defined as follows:

The objective is to find a function
$\pi(t): [0,T] \to \mathcal{X}$, where $T$ represents the planning horizon and $\mathcal{X}$ is the configuration space, which defines the set of
feasible configurations for the vehicle.

The vehicle starts in an initial configuration $x_{\text{initial}} \in \mathcal{X}$, and the trajectory should end in a set of goal configurations
$X_{\text{goal}} \subset \mathcal{X}$.

Additionally, a constraint is defined over the derivatives of $\pi$ with respect to $t \in [0, T]$, denoted as $D(\pi(t), \pi'(t), \pi''(t), \dots)$.

To formulate an optimization problem, we define an objective function and a feasible set.

Let $\Pi[\mathcal{X}, T]$ represent the set of all possible functions mapping $[0, T]$ to $\mathcal{X}$.
Further, let $J(\pi): \Pi[\mathcal{X}, T] \to \mathbb{R}$ be the objective function.

\subsubsection{Problem Definition: Optimal Trajectory Planning}

Given a 6-tuple $(\mathcal{X}, x_{\text{initial}}, X_{\text{goal}}, D, J, T)$, the objective is to find:

\begin{align}
	x^* & = \underset{\pi \in \Pi(\mathcal{X},T)}{\operatorname{arg\,min}}
	J(\pi)                                                                                                                        \\ \text{s.t.
	}   & \quad \pi(0) = x_{\text{initial}}                                                                                       \\
	    & \quad \pi(T) \in X_{\text{goal}}                                                                                        \\
	    & \quad \pi(t) \in \mathcal{X},                                    & \text{for all} \quad t \in [0,T]                     \\
	    & \quad D(\pi(t), \pi'(t), \pi''(t), \dots),                       & \text{for all} \quad t \in [0,T] \label{eq:dynamics}
\end{align}

\section{Complexity}

Finding an exact solution in a dynamic environment is highly challenging.
The problem is inherently non-convex, and solvers cannot directly operate over a function space.

\section{Numerical Approach}

To address this problem numerically, we first define the constraints by modeling the vehicle and its environment.
We then reformulate the problem, discretize it, and approximate it.
Our goal is to obtain a solution that is both computationally efficient and reliable.

To achieve this, we employ a convex solver, necessitating adherence to disciplined convex programming (DCP) rules.

\section{Vehicle Models}

A vehicle model describes the position and orientation of the vehicle in the real world and predicts how these states change over time.
Different models vary in complexity and accuracy.
Generally, higher complexity results in increased accuracy.

Here, we introduce two fundamental models.

\subsection{State and Control Variables}

The state variables $x_i$ represent the current position and pose of the vehicle, potentially including velocity, steering angle, or other relevant
parameters.
We group these variables into a state vector $x$.

The control inputs $u_i$ are external influences that modify the state.
These are grouped into a control vector $u$.
Both $x$ and $u$ are time-dependent.

\subsection{Point Mass Model}

The point mass model (PM) consists of four state variables:

\begin{equation}
	x = \begin{bmatrix} p_x \\ p_y \\ v_x \\ v_y \end{bmatrix}
	\label{eq:states_pm}
\end{equation}

Here, $p_x$ and $p_y$ represent the vehicle's position in a global fixed coordinate system.
Instead of an explicit orientation, velocity is divided into its components $v_x$ and $v_y$.

The model has two control inputs:

\begin{equation}
	u = \begin{bmatrix} a_x \\ a_y \end{bmatrix}
	\label{eq:controls_pm}
\end{equation}

These correspond to accelerations in the $x$ and $y$ directions.

The future position and velocity are determined by the following system of differential equations:

\begin{equation}
	\dot{x} = A x + B u
\end{equation}

where

\begin{equation}
	A = \begin{bmatrix} 0 & 0 & 1 & 0 \\ 0 & 0 & 0 & 1 \\ 0 & 0 & 0 & 0 \\ 0 & 0 & 0 & 0 \end{bmatrix}, \quad
	B = \begin{bmatrix} 0 & 0 \\ 0 & 0 \\ 1 & 0 \\ 0 & 1 \end{bmatrix}
\end{equation}

The control inputs are bounded by a circular constraint, where the radius $a_{\max}$ is a model parameter:

\begin{equation}
	\sqrt{u_1^2 + u_2^2} \leq a_{\max}
\end{equation}

This model is the simplest commonly used for motion planning.

\subsection{Kinematic Single Track Model}

The kinematic single track model (KST) consists of five state variables:

\begin{equation}
	x = \begin{bmatrix} p_x \\ p_y \\ \delta \\ v \\ \psi \end{bmatrix}
	\label{eq:states_kst}
\end{equation}

Similar to the point mass model, the first two state variables $p_x$ and $p_y$ define the vehicle's global position in a two-dimensional coordinate
system.
The vehicle is now modeled with an orientation $\psi$ relative to the global $x$-axis.
The velocity vector describes the velocity $v$ of the rear wheel, aligning with the orientation.
The front wheels can rotate around the yaw axis, and their angle relative to the orientation is represented by the steering angle $\delta$.

Two control inputs modify the velocity and steering, directly affecting the state variables:

\begin{equation}
	u = \begin{bmatrix} v_{\delta} \\ a_{\text{long}} \end{bmatrix}
	\label{eq:controls_kst}
\end{equation}
where $v_{\delta}$ is the steering velocity, and $a_{\text{long}}$ is the longitudinal acceleration.

The future state follows these differential equations:

\begin{align}
	 & \dot{p}_x = v\cos(\psi)                    \\
	 & \dot{p}_y = v\sin(\psi)                    \\
	 & \dot{\delta} = v_{\delta}                  \\
	 & \dot{v} = a_{\text{long}}                  \\
	 & \dot{\psi} = \frac{v}{l_{wb}} \tan(\delta) \\
\end{align}

The single-track name originates from simplifying front and rear wheels into single contact points, assuming no wheel slip, leading to a kinematic
model abstraction.
The following figure comprehends the whole model nicely.

\begin{figure}[h]
	\centering
	\begin{tikzpicture}
		% Axes
		\draw[->] (0,0) -- (2,0) node[right] {$x$};
		\draw[->] (0,0) -- (0,2) node[above] {$y$};

		% Rear Wheel
		\fill (2,2) circle (2pt); % Draws a small point at (2,2)

		% Vehicle body
		\draw[thick,rotate around={11.536959-90:(2,2)}] (1.8,1.3) rectangle (2.2,2.7);
		\draw[thick,rotate around={26.536959-90:(7,3)}] (6.8,2.3) rectangle (7.2,3.7);

		% Wheelbase
		\draw[-] (2,2) -- (7,3);
		\draw[dashed] (2,2) -- (1.7,3.5);
		\draw[dashed] (7,3) -- (6.7,4.5);
		\draw[dashed, <->] (1.8,3) -- (6.8,4) node[midway,above] {$l_{wb}$};

		% Velocity vector
		\draw[->] (2,2.1) -- (4,2.5) node[midway,above] {$v$};

		% Heading angle
		\draw[dashed] (3.25,2.25) -- (6,2.25);
		\draw[->] (6,2.25) arc (0:11.536959:2.75);
		\node at (5.7,2.5) {$\psi$};

		% Steering angle
		\draw[dashed] (7,3) -- (8.5,3.3);
		\draw[dashed] (7,3) -- ++(26.536959:1.5);
		\draw[->] (8.5,3.3) arc (11.536959:26.536959:1.5);
		\node at (8.2,3.43) {$\delta$};

		% Displacement vector
		\draw[dashed,thick,->] (0,0) -- (1.95,1.95)
		node[midway, left, shift={(-0,+0.4)}] {$\begin{bmatrix}s_x \\ s_y \end{bmatrix}$};
	\end{tikzpicture}
	\caption{Bicycle model representation of a vehicle.}
	\label{fig:bicycle_model}
\end{figure}

The following additional constraints are part of the model, a parameter $a_{max}$ is introduced:

\begin{equation}
	\sqrt{u_2^2 + (x_4\dot{x}_5)^2} \leq a_{\max}
\end{equation}

For both models the vehicle is additional constrained on its velocity range, steering angle range and how fast the steering angle can change.
Which are quite natural constraints, which should not be forgotten.

\section{Constraints}

Our constraints on the derivatives \ref{eq:dynamics} using one of the models can be then expressed as:
\begin{equation}
	\pi'(t) = f(\pi(t), u(t))
\end{equation}

\section{Reformulation of the Problem}

Since Solver cannot operate on a function space, the first task is to project our infinite-dimensional function space of trajectories to a
finite-dimensional vector space.
Additionally, we convert the constraints, which so far consist of set membership and predicates, into a set of equalities and inequalities, which is
the standard input format for the solver.

Outlook: Penalty or Barrier functions, which integrate the constraints as part of the objective.

Direct vs Indirect Methods

\subsection{Direct Methods}

To project the function space to a vector space one
defines a set of basis function which than span a subspace of the original function space.
A function of the subspace is then defined by a linear combination of the basis function.

\begin{equation}
	\tilde{\pi(t)} = \sum_{i=1}^{N}\pi_i\phi_i(t)
\end{equation}

Of course, we may lose our original optimal solution and restrict our search space by the basis function, but our problem now reduces to finding
finite-dimensional subspace.
One of the most common numerical method to approximate are Numerical Integrators with Collocation.
Collocation means that we require the trajectories to satisfies the constraints only in a set of discrete points $\{t_i\}_{i=1\dots m}$.
Then numerical integrations techniques are used to approximate the trajectory between those points.

\subsection{New Formulation}

Given the models and using direct method, we can formulate our optimization problem over a finite-dimensional vector space.

Let $\mathcal{X}$ be the set of valid state the vehicle model can be in and $\mathcal{U}$ bet set of all valid control inputs of the model.

We define our trajectory over discrete time points $\{t_i\}_{i=1\dots m}$ as $\pi(t_i)=x_i$ and our new objective over
$\mathcal{X}\times\mathcal{U}$, as $J: \mathcal{X}\times\mathcal{U}\to \mathbb{R}$.

\subsubsection{Problem Definition: Discrete Time Optimal Trajectory Planning}

Given a 7-tuple $(\mathcal{X}, \mathcal{U}, x_{\text{initial}}, X_{\text{goal}}, f, J, \{t_i\}_{i=1\dots m})$, the objective is to find:

\begin{align}
	u^* & = \underset{u\in \mathcal{U}^{T-1}}{\operatorname{arg\,min}} \sum_{i=1}^{T-1}
	J(x_{i+1}, u_{i})                                                                                                                                                \\ \text{s.t.
	}   & \quad x_1 = x_{\text{initial}}                                                                                                                             \\
	    & \quad x_T \in X_{\text{goal}} \subseteq \mathcal{X}                                                                                                        \\
	    & \quad x_{i},u_{i}, \in \mathcal{C} \subseteq \mathcal{X}\times\mathcal{U}     & \text{for all} \quad i \in \{1,\dots,m-1\} \label{eq:coupling_constraints} \\
	    & \quad x_{i+1} = x_i + (t_{i+1} - t_i)f(x_i,u_i)                               & \text{for all} \quad i \in\{1,\dots,m-1\}	\label{eq:discrete_dynamics}
\end{align}

Given an initial state $x_{\text{initial}}$ and a control sequence over the whole sequence of time points, we can use our model's dynamics $f$ and
the numerical integration approach to compute the state for each time point.
Our problem formulation introduces a new coupling constraint $C$ on a state and the control input applied to this state.

We will face the problem that the problem does so far not apply to the DCP Problems.
We will present how one can come around this problem with both of the models we introduced.
Doing so we will explain some modeling techniques, that come along with some approximations and investigate them.
We will first focus on the point mass model.

\subsection{Frenet Frame}

So far, our models used a global Cartesian coordinate system to describe the vehicle positions and pose.
This is fine for the dynamics of the vehicle, but once one starts to model the constraints which arise from the road topology it gets quite complex.
Given the fact that road topology is known in advance we will use it to model or vehicles states that will empower us to model constraints based on
the road in a more natural way, which also leads to convex road topology constraints.

