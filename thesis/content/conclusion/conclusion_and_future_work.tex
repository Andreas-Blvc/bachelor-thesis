\section{Conclusion} \label{subsec:conclusion}

In this thesis, we investigated predictive planning and control strategies using two distinct vehicle models (a double integrator model and a
kinematic bicycle model) under a variety of road scenarios and velocity conditions.
Our principal objectives were to analyze solver performance, quantify completion rates, and examine how friction-circle constraints affect high-speed
maneuvers.

\paragraph{Solver Times}
The double integrator model consistently outperformed the bicycle model in terms of solver time, requiring on average around 3.8--3.9,ms compared to
9.4--9.5,ms for the bicycle model under both time discretization configurations.
These results also demonstrated that the double integrator model exhibited more stable performance, indicated by smaller time deviations.

\paragraph{Completion Rates}
With respect to scenario feasibility, both models failed all runs of the intentionally infeasible curve scenario.
When the curve was enlarged to a marginally feasible radius at lower speeds, both models succeeded.
As speeds increased, failure rates rose across the board, underscoring the critical impact of higher velocities on path-following constraints.
Notably, the bicycle model faced more failures in sharp-turn scenarios (Elchtest, Lane Change), especially at high speeds, mainly due to restrictive
friction-circle approximations.
Conversely, the double integrator model, while simpler, proved more robust in these scenarios because it does not dynamically couple speed and
steering to friction limits.

\paragraph{Friction-Circle Constraints}
Our investigations into the friction circle revealed that high-speed maneuvers became infeasible for the bicycle model under strict friction
approximations: once the model selects a higher velocity, allowable steering angles shrink, making sharp maneuvers impossible.
The terminal cost function further exacerbated this by incentivizing high speed at the expense of feasible steering control.
By introducing soft constraints on road-boundary proximity, some numerical instabilities were alleviated, but the fundamental trade-off between
maintaining high speed and ensuring adequate steering authority remained a challenge.

Overall, the double integrator model achieved faster and more stable solver performance, but at the cost of less physical realism regarding friction
limits.
The kinematic bicycle model provided a more accurate vehicle representation, but suffered from higher solver times and increased failure rates in
tightly constrained, high-speed scenarios.

\section{Future Work} \label{subsec:future_work}

While the presented results underline the strengths and weaknesses of both models, several avenues for future work arise:

\begin{itemize} \item \textbf{Improved Friction Modeling:}
	      The friction-circle approximation can be refined by incorporating velocity-dependent tire models or more sophisticated dynamics (e.g., Pacejka
	      models).
	      These enhancements would reduce infeasibilities at high speeds, but may require advanced solvers or tailored approximations to maintain computational
	      efficiency.
	\item \textbf{Adaptive Time Discretization:}
	      Although two time-discretization configurations were explored, a more adaptive approach could dynamically adjust the horizon or time-step sizes
	      during runtime based on scenario complexity or solver convergence metrics.
	      This may further optimize computational resources, especially in high-speed maneuvers.

	\item \textbf{Soft Constraints and Penalty Tuning:}
	      Our preliminary introduction of soft constraints highlights the potential for balancing feasibility against other objectives.
	      Systematic tuning of penalty weights (e.g., for road-boundary offsets) and investigating robust methods to handle constraint violations could improve
	      success rates for the bicycle model.

	\item \textbf{Multi-Stage Approaches:}
	      A hierarchical or multi-stage planning approach (e.g., generating coarse global paths first, then refining with a local high-fidelity model) could
	      blend the computational simplicity of the double integrator model with the accuracy of the bicycle model.
	      This may reduce the need for global friction-circle approximations without sacrificing real-world realism.

	\item \textbf{Hardware Deployment and Real-Time Performance:}
	      Deploying the proposed methods on embedded hardware or real test vehicles remains an important step.
	      Evaluating trade-offs between model fidelity and real-time feasibility could guide the choice of model and solver in production systems.

	\item \textbf{Extended Cost Functions:}
	      Future extensions might include advanced cost functions considering comfort, energy efficiency, and safety margins.
	      Additionally, refining velocity maximization objectives to account for friction and steering constraints explicitly may prevent infeasibilities
	      observed in certain high-speed or narrow-lane scenarios.
\end{itemize}

By addressing these areas, subsequent research can build upon the findings of this thesis to develop more robust, efficient, and realistic motion
planners.
The interplay between solver performance, model fidelity, and scenario constraints remains an active and critical frontier in predictive control for
autonomous driving.
