\section{Conclusion} \label{sec:conclusion}

This thesis presented an optimization-based motion planning framework for autonomous vehicles, formulated within a convex framework using the Frenet
coordinate system.
By developing and comparing two distinct vehicle models—a double integrator and a kinematic bicycle—the work addresses the fundamental trade-off
between computational speed and model accuracy.
Our experimental evaluation demonstrated that the double integrator model offers superior computational efficiency and stability, making it
well-suited for real-time applications.
In contrast, the kinematic bicycle model, while imposing a higher computational load, provides a more realistic representation of vehicle dynamics,
particularly in scenarios that demand accurate handling of steering and acceleration constraints.
Overall, these results underscore the necessity of balancing model simplicity with dynamic fidelity to ensure both feasible trajectory generation and
adherence to physical constraints in autonomous driving.

\section{Future Work} \label{sec:future_work}

Building on the insights of this thesis, future research can further bridge the gap between computational efficiency and model realism.
One promising direction is the development of adaptive planning strategies that leverage the speed of simpler models while selectively invoking
high-fidelity dynamics when needed.
This could involve hierarchical or multi-stage planning approaches where a computationally efficient global planner generates initial trajectories
that are later refined by a local planner with detailed vehicle dynamics.
Additionally, exploring adaptive time discretization techniques and robust constraint handling methods may enhance the planner's performance in
dynamic and complex environments.
Finally, extending the framework to real-world implementations—by deploying the methods on embedded hardware or actual test vehicles—will be critical
to validate these approaches under practical conditions and to further advance motion planning for autonomous driving applications.
