\section{Motion Planning using Bicycle Model} \label{sec:motion_planning_using_bicylce}

We have already introduced the kinematic single track model in which a steering angle and an orientation, which will now combine with the idea of the
previous chapter.
We want to model our state variables in the road aligned frame, which is the Frenet frame.
The states and controls variables of the kinematic single track model are defined in the global coordinate system are defined as in \ref{eq:states_kst} and \ref{eq:controls_kst}:

\begin{figure}[h]
	\centering
	\begin{subfigure}[b]{0.45\textwidth}
		\centering
		$x = \begin{bmatrix} p_x \\ p_y \\ \delta \\ v \\ \psi \end{bmatrix}$
		\caption{State Variables}
	\end{subfigure}
	\hfill
	\begin{subfigure}[b]{0.45\textwidth}
		\centering
		$u = \begin{bmatrix} v_{\delta} \\ a_{\text{long}} \end{bmatrix}$
		\caption{Control Inputs}
	\end{subfigure}
\end{figure}

Our goal is not model dynamics and state variables in the Frenet frame.

\subsection{Transforming Global Cartesian Coordinates to Frenet Frame} \label{subsec:bicycle_conversion_of_cartesian_to_frenet}

In this section, we derive the state transition model for the Frenet frame, which is a commonly used coordinate system in vehicle dynamics and
control.
The Frenet frame is particularly useful for path tracking and motion planning as it simplifies the representation of the vehicle's motion relative to
a reference path.

We start by considering the dynamics of the kinematic single track model in the global coordinate system.
This model captures the essential behavior of a vehicle by representing it as a single point mass with a front and rear axle.
The state variables include the position coordinates $(p_x, p_y)$, the orientation angle $\psi$, the steering angle $\delta$, and the longitudinal
velocity $v$.
The control inputs are the longitudinal acceleration $a_{\text{long}}$ and the steering rate $v_{\delta}$.

The dynamics of the kinematic single track model in the global coordinate system are:

\begin{align*}
	 & \dot{p}_x = v\cos(\psi)                    \\
	 & \dot{p}_y = v\sin(\psi)                    \\
	 & \dot{\delta} = v_{\delta}                  \\
	 & \dot{v} = a_{\text{long}}                  \\
	 & \dot{\psi} = \frac{v}{l_{wb}} \tan(\delta) \\
\end{align*}

Building on the orientation model from the previous chapter, we can extend the point mass model to include body-fixed control inputs.
This approach helps us achieve our goal.

Using the previously derived equations \ref{eq:first_derivative_long} and \ref{eq:first_derivative_lat}, and incorporating the
definitions $\xi = \psi - \theta$ and $\dot{\theta}=\frac{d\theta}{ds}\frac{ds}{dt}=C(s)\dot{s}$, we obtain
the following equations:

\begin{align*}
	\dot{s} = \frac{v_x\cos{\xi}}{1-nC(s)} \\
	\dot{n} = v_x\sin{\xi}                 \\
	\dot{\xi} = \dot{\psi} - C(s)\dot{s}   \\
\end{align*}

Combining these equations with the kinematic single track model, we derive the state transition model for the Frenet frame.
This model provides a comprehensive representation of the vehicle's motion in the Frenet coordinate system, which is essential for accurate path
tracking and motion planning.

Next, we define the state variables and control inputs for the single-track model.

\subsection{Model Dynamics Approximation} \label{subsec:approximation_of_model_dynamics}

The state variables and control inputs for the single-track model in the Frenet frame are defined as follows:
\begin{figure}[h]
	\centering
	\begin{subfigure}[b]{0.45\textwidth}
		\centering
		$x = \begin{bmatrix}
				s \\ n \\ \xi \\ v \\ \delta
			\end{bmatrix}$
		\caption{State Variables}
	\end{subfigure}
	\hfill
	\begin{subfigure}[b]{0.45\textwidth}
		\centering
		$u = \begin{bmatrix}
				a_{x,b} \\ v_\delta
			\end{bmatrix}$
		\caption{Control Inputs}
	\end{subfigure}
\end{figure}

where the state consists of the Frenet frame coordinates $(s, n)$, the heading alignment error $\xi$, the longitudinal vehicle velocity $v$, and the steering angle $\delta$.
The state evolution is described by the following equations:

\begin{equation}
	f(x, u) =
	\begin{bmatrix}
		\frac{v \cos\xi}{1 - nC(s)}                \\[8pt]
		v \sin\xi                                  \\[8pt]
		\frac{1}{l_{wb}}v \tan\delta - C(s)\dot{s} \\[8pt]
		a_{x,b}                                    \\[8pt]
		v_\delta
	\end{bmatrix}.
\end{equation}

For this model, we will use body-fixed control inputs, which provide convex road topology constraints.
To linearize the model dynamics, we will use two new techniques.
These techniques allow us to maintain the constraints without shifting, as was necessary in the previous chapter.
This ensures a more accurate and computationally efficient representation of the vehicle's motion.

To simplify the model, the following assumptions are made: $nC(s)$ is close to zero.
This is a valid since $n$ models the vehicle's lateral position relative to the reference path, and $C(s)$ is the curvature of the reference path
which is typically small enough for this assumption to be valid.
We split up the problem by looking the terms which make the model non-linear.

\subsubsection{Non-Linear Terms}

We have four non-linear terms in the state transition model.
In the following, we linearize them using approximations.
\begin{align}
	 & \frac{v \cos\xi}{1 - nC(s)} \\
	 & v \sin\xi                   \\
	 & v \tan\delta                \\
	 & C(s)\dot{s}
\end{align}

With our assumptions, that $nC(s)\approx 0$, we can immediately simplify the first term.

\[
	\frac{v \cos\xi}{1 - nC(s)} \approx v \cos\xi
\]

\subsubsection{First Order Taylor Polynomial}

To linearize the non-linear terms, we use the first order Taylor polynomial approximation around a reference point.
The first order Taylor expansion of a function \(f(x)\) around a point \(x_0\) is given by:

\[ f(x) \approx f(x_0) +
	\frac{df}{dx} (x_0) (x - x_0) \]

Applying the first order Taylor polynomial to the trigonometric functions $\sin$, $\cos$, and
$\tan$ around the reference points $\xi_0$ and $\delta_0$, we obtain the following approximations:

\[ \cos(\xi) \approx
	\cos(\xi_0) - \sin(\xi_0) (\xi - \xi_0) \] \[ \sin(\xi) \approx \sin(\xi_0) + \cos(\xi_0) (\xi - \xi_0) \] \[ \tan(\delta) \approx \tan(\delta_0) +
	\frac{1}{\cos^2(\delta_0)} (\delta - \delta_0) \]

These approximations are known as small angle approximations, which are valid
when the angles $\xi$ and $\delta$ are small.
In vehicle dynamics, it is often reasonable to assume that the heading alignment error $\xi$ and the steering angle $\delta$ are small, especially
when the vehicle is closely following a reference path.
This allows us to simplify the trigonometric functions using their first order Taylor expansions.

Substituting these first order Taylor approximations into the state transition model, we obtain the following terms:

\begin{align*}
	 & v (\cos(\xi_0) - \sin(\xi_0) (\xi - \xi_0))                         \\
	 & v (\sin(\xi_0) + \cos(\xi_0) (\xi - \xi_0))                         \\
	 & v (\tan(\delta_0) + \frac{1}{\cos^2(\delta_0)} (\delta - \delta_0)) \\
	 & C(s)\dot{s}
\end{align*}

Since our reference values $\xi_0 = 0$ and $\delta_0 = 0$ are constant the only remaining non-linear terms are: $$v \xi, v \delta, C(s)\dot{s}$$

Bilinear terms arise when the product of two variables appears in the equations, making the system non-linear.
In our state transition model, the terms \(v \xi\), \(v \delta\), and \(C(s)\dot{s}\) are bilinear.
Since \(C(s)\) is a function of \(s\), we will introduce another assumption.
We have already seen that we can model our road in segments, which we used for the Point Mass Model to obtain less conservative coupling constraints.
Here, we will reuse this approach to linearize our dynamics.

\subsubsection{Assumption: Piece Wise Linear Curvature}

We assume that the curvature can be expressed as a piece-wise linear function.

\[
	C(s) = \begin{cases}
		a_1s+b_1 & \text{if } s \in [s_0, s_1]     \\
		a_2s+b_2 & \text{if } s \in [s_1, s_2]     \\
		\vdots                                     \\
		a_ns+b_n & \text{if } s \in [s_{n-1}, s_n]
	\end{cases}
\]

This reduces the non-linear term \(C(s)\dot{s}\) to a new bilinear term \(s\dot{s}\), which we need to linearize.

Next, we will focus on linearizing the bilinear terms.

\subsection{Convex Relaxation of Bilinear Terms} \label{subsec:convex_relaxation_for_bilinear_terms}

To handle bilinear terms of the form \(xy\), we introduce a new variable \(w\) and apply the McCormick relaxation.
The McCormick relaxation is a technique that linearizes bilinear terms by introducing auxiliary variables and constraints.
This technique allows us to represent the bilinear terms as a set of linear constraints, which can be solved efficiently using convex optimization
methods.
This relaxation only works if the variables \(x\) and \(y\) are bounded:

\[ \underline{x} \leq x \leq \overline{x}, \qquad
	\underline{y} \leq y \leq \overline{y} \] with constants \(\underline{x}, \overline{x}, \underline{y}, \overline{y} \in \mathbb{R}\).

We introduce a new auxiliary variable \(w\), which will be used as an approximation of the bilinear term \(xy\).
We use linear constraints to bound the auxiliary variable \(w\) and ensure that it approximates the bilinear term \(xy\).
The linear constraints are derived from the bounds of the variables \(x\), \(y\), and the bilinear term \(xy\), as shown below:

\[
	\begin{aligned}
		w & \geq \underline{x} y + x \underline{y} - \underline{x} \underline{y}, \\
		w & \geq \overline{x} y + x \overline{y} - \overline{x} \overline{y},     \\
		w & \leq \overline{x} y + x \underline{y} - \overline{x} \underline{y},   \\
		w & \leq \underline{x} y + x \overline{y} - \underline{x} \overline{y}.
	\end{aligned}
\]

The idea behind these constraints is to create a convex lower bound and a concave upper bound for the bilinear term \(xy\).
These bounds are constructed as follows: \[ a = (x - \underline{x}) \geq 0 \] \[ b = (y - \underline{y}) \geq 0 \] Since a and b are both positive,
we can multiply them to get: \[ ab = xy - \underline{x}y - \underline{y}x + \underline{x}\underline{y} \geq 0 \] Thus, we get our first under
estimator for $xy$: \[ xy \geq \underline{x}y + \underline{y}x - \underline{x}\underline{y} \] To get the all the possible lower and upper bounds by
what is given we can apply this pattern to $a\in\{x - \underline{x}, \overline{x} - x\}$ and $b\in\{y - \underline{y}, \overline{y} - y\}$.
For any of the four possible combination of $a$ and $b$ we get $ab\geq0$ and can therefor come up with either an upper or lower bound for $xy$.

\begin{figure}[h]
	\centering
	\begin{subfigure}[b]{0.45\textwidth}
		\centering
		\resizebox{\textwidth}{!}{\input{figures/mccormick/mccormick-bounds-0-upper.pgf}}
		\caption{Difference to the upper bound}
		\label{fig:mccormick_0_upper}
	\end{subfigure}
	\hfill
	\begin{subfigure}[b]{0.45\textwidth}
		\centering
		\resizebox{\textwidth}{!}{\input{figures/mccormick/mccormick-bounds-0-lower.pgf}}
		\caption{Difference to the lower bound}
		\label{fig:mccormick_0_lower}
	\end{subfigure}
	\caption{McCormick relaxation bounds for the bilinear term \( xy \).}
	\label{fig:mccormick_bounds_0}
\end{figure}

Figure \ref{fig:mccormick_0_upper} illustrates the deviation between the actual bilinear term \(xy \) and the smallest upper bound provided by the
relaxation constraints.
Similarly, Figure \ref{fig:mccormick_0_lower} shows the deviation to the greatest lower bound.
For the range \( -2 \leq x \leq 2 \) and \( 0 \leq y \leq 50 \).
It is evident that the bounds improve as \( x \) and \( y \) approach their respective limits.

\begin{figure}[h]
	\centering
	\begin{subfigure}[b]{0.45\textwidth}
		\centering
		\resizebox{\textwidth}{!}{%% Creator: Matplotlib, PGF backend
%%
%% To include the figure in your LaTeX document, write
%%   \input{<filename>.pgf}
%%
%% Make sure the required packages are loaded in your preamble
%%   \usepackage{pgf}
%%
%% Also ensure that all the required font packages are loaded; for instance,
%% the lmodern package is sometimes necessary when using math font.
%%   \usepackage{lmodern}
%%
%% Figures using additional raster images can only be included by \input if
%% they are in the same directory as the main LaTeX file. For loading figures
%% from other directories you can use the `import` package
%%   \usepackage{import}
%%
%% and then include the figures with
%%   \import{<path to file>}{<filename>.pgf}
%%
%% Matplotlib used the following preamble
%%   \def\mathdefault#1{#1}
%%   \everymath=\expandafter{\the\everymath\displaystyle}
%%   
%%   \ifdefined\pdftexversion\else  % non-pdftex case.
%%     \usepackage{fontspec}
%%   \fi
%%   \makeatletter\@ifpackageloaded{underscore}{}{\usepackage[strings]{underscore}}\makeatother
%%
\begingroup%
\makeatletter%
\begin{pgfpicture}%
\pgfpathrectangle{\pgfpointorigin}{\pgfqpoint{4.377484in}{3.839102in}}%
\pgfusepath{use as bounding box, clip}%
\begin{pgfscope}%
\pgfsetbuttcap%
\pgfsetmiterjoin%
\definecolor{currentfill}{rgb}{1.000000,1.000000,1.000000}%
\pgfsetfillcolor{currentfill}%
\pgfsetlinewidth{0.000000pt}%
\definecolor{currentstroke}{rgb}{1.000000,1.000000,1.000000}%
\pgfsetstrokecolor{currentstroke}%
\pgfsetdash{}{0pt}%
\pgfpathmoveto{\pgfqpoint{0.000000in}{0.000000in}}%
\pgfpathlineto{\pgfqpoint{4.377484in}{0.000000in}}%
\pgfpathlineto{\pgfqpoint{4.377484in}{3.839102in}}%
\pgfpathlineto{\pgfqpoint{0.000000in}{3.839102in}}%
\pgfpathlineto{\pgfqpoint{0.000000in}{0.000000in}}%
\pgfpathclose%
\pgfusepath{fill}%
\end{pgfscope}%
\begin{pgfscope}%
\pgfsetbuttcap%
\pgfsetmiterjoin%
\definecolor{currentfill}{rgb}{1.000000,1.000000,1.000000}%
\pgfsetfillcolor{currentfill}%
\pgfsetlinewidth{0.000000pt}%
\definecolor{currentstroke}{rgb}{0.000000,0.000000,0.000000}%
\pgfsetstrokecolor{currentstroke}%
\pgfsetstrokeopacity{0.000000}%
\pgfsetdash{}{0pt}%
\pgfpathmoveto{\pgfqpoint{0.516434in}{0.499074in}}%
\pgfpathlineto{\pgfqpoint{3.713060in}{0.499074in}}%
\pgfpathlineto{\pgfqpoint{3.713060in}{3.695699in}}%
\pgfpathlineto{\pgfqpoint{0.516434in}{3.695699in}}%
\pgfpathlineto{\pgfqpoint{0.516434in}{0.499074in}}%
\pgfpathclose%
\pgfusepath{fill}%
\end{pgfscope}%
\begin{pgfscope}%
\pgfpathrectangle{\pgfqpoint{0.516434in}{0.499074in}}{\pgfqpoint{3.196625in}{3.196625in}}%
\pgfusepath{clip}%
\pgfsys@transformshift{0.516434in}{0.499074in}%
\pgftext[left,bottom]{\includegraphics[interpolate=true,width=3.200000in,height=3.200000in]{figures/mccormick/mccormick-bounds-1-upper-img0.png}}%
\end{pgfscope}%
\begin{pgfscope}%
\pgfsetbuttcap%
\pgfsetroundjoin%
\definecolor{currentfill}{rgb}{0.000000,0.000000,0.000000}%
\pgfsetfillcolor{currentfill}%
\pgfsetlinewidth{0.803000pt}%
\definecolor{currentstroke}{rgb}{0.000000,0.000000,0.000000}%
\pgfsetstrokecolor{currentstroke}%
\pgfsetdash{}{0pt}%
\pgfsys@defobject{currentmarker}{\pgfqpoint{0.000000in}{-0.048611in}}{\pgfqpoint{0.000000in}{0.000000in}}{%
\pgfpathmoveto{\pgfqpoint{0.000000in}{0.000000in}}%
\pgfpathlineto{\pgfqpoint{0.000000in}{-0.048611in}}%
\pgfusepath{stroke,fill}%
}%
\begin{pgfscope}%
\pgfsys@transformshift{0.516434in}{0.499074in}%
\pgfsys@useobject{currentmarker}{}%
\end{pgfscope}%
\end{pgfscope}%
\begin{pgfscope}%
\definecolor{textcolor}{rgb}{0.000000,0.000000,0.000000}%
\pgfsetstrokecolor{textcolor}%
\pgfsetfillcolor{textcolor}%
\pgftext[x=0.516434in,y=0.401852in,,top]{\color{textcolor}{\rmfamily\fontsize{9.000000}{10.800000}\selectfont\catcode`\^=\active\def^{\ifmmode\sp\else\^{}\fi}\catcode`\%=\active\def%{\%}$\mathdefault{\ensuremath{-}2.00}$}}%
\end{pgfscope}%
\begin{pgfscope}%
\pgfsetbuttcap%
\pgfsetroundjoin%
\definecolor{currentfill}{rgb}{0.000000,0.000000,0.000000}%
\pgfsetfillcolor{currentfill}%
\pgfsetlinewidth{0.803000pt}%
\definecolor{currentstroke}{rgb}{0.000000,0.000000,0.000000}%
\pgfsetstrokecolor{currentstroke}%
\pgfsetdash{}{0pt}%
\pgfsys@defobject{currentmarker}{\pgfqpoint{0.000000in}{-0.048611in}}{\pgfqpoint{0.000000in}{0.000000in}}{%
\pgfpathmoveto{\pgfqpoint{0.000000in}{0.000000in}}%
\pgfpathlineto{\pgfqpoint{0.000000in}{-0.048611in}}%
\pgfusepath{stroke,fill}%
}%
\begin{pgfscope}%
\pgfsys@transformshift{0.916012in}{0.499074in}%
\pgfsys@useobject{currentmarker}{}%
\end{pgfscope}%
\end{pgfscope}%
\begin{pgfscope}%
\definecolor{textcolor}{rgb}{0.000000,0.000000,0.000000}%
\pgfsetstrokecolor{textcolor}%
\pgfsetfillcolor{textcolor}%
\pgftext[x=0.916012in,y=0.401852in,,top]{\color{textcolor}{\rmfamily\fontsize{9.000000}{10.800000}\selectfont\catcode`\^=\active\def^{\ifmmode\sp\else\^{}\fi}\catcode`\%=\active\def%{\%}$\mathdefault{\ensuremath{-}1.75}$}}%
\end{pgfscope}%
\begin{pgfscope}%
\pgfsetbuttcap%
\pgfsetroundjoin%
\definecolor{currentfill}{rgb}{0.000000,0.000000,0.000000}%
\pgfsetfillcolor{currentfill}%
\pgfsetlinewidth{0.803000pt}%
\definecolor{currentstroke}{rgb}{0.000000,0.000000,0.000000}%
\pgfsetstrokecolor{currentstroke}%
\pgfsetdash{}{0pt}%
\pgfsys@defobject{currentmarker}{\pgfqpoint{0.000000in}{-0.048611in}}{\pgfqpoint{0.000000in}{0.000000in}}{%
\pgfpathmoveto{\pgfqpoint{0.000000in}{0.000000in}}%
\pgfpathlineto{\pgfqpoint{0.000000in}{-0.048611in}}%
\pgfusepath{stroke,fill}%
}%
\begin{pgfscope}%
\pgfsys@transformshift{1.315591in}{0.499074in}%
\pgfsys@useobject{currentmarker}{}%
\end{pgfscope}%
\end{pgfscope}%
\begin{pgfscope}%
\definecolor{textcolor}{rgb}{0.000000,0.000000,0.000000}%
\pgfsetstrokecolor{textcolor}%
\pgfsetfillcolor{textcolor}%
\pgftext[x=1.315591in,y=0.401852in,,top]{\color{textcolor}{\rmfamily\fontsize{9.000000}{10.800000}\selectfont\catcode`\^=\active\def^{\ifmmode\sp\else\^{}\fi}\catcode`\%=\active\def%{\%}$\mathdefault{\ensuremath{-}1.50}$}}%
\end{pgfscope}%
\begin{pgfscope}%
\pgfsetbuttcap%
\pgfsetroundjoin%
\definecolor{currentfill}{rgb}{0.000000,0.000000,0.000000}%
\pgfsetfillcolor{currentfill}%
\pgfsetlinewidth{0.803000pt}%
\definecolor{currentstroke}{rgb}{0.000000,0.000000,0.000000}%
\pgfsetstrokecolor{currentstroke}%
\pgfsetdash{}{0pt}%
\pgfsys@defobject{currentmarker}{\pgfqpoint{0.000000in}{-0.048611in}}{\pgfqpoint{0.000000in}{0.000000in}}{%
\pgfpathmoveto{\pgfqpoint{0.000000in}{0.000000in}}%
\pgfpathlineto{\pgfqpoint{0.000000in}{-0.048611in}}%
\pgfusepath{stroke,fill}%
}%
\begin{pgfscope}%
\pgfsys@transformshift{1.715169in}{0.499074in}%
\pgfsys@useobject{currentmarker}{}%
\end{pgfscope}%
\end{pgfscope}%
\begin{pgfscope}%
\definecolor{textcolor}{rgb}{0.000000,0.000000,0.000000}%
\pgfsetstrokecolor{textcolor}%
\pgfsetfillcolor{textcolor}%
\pgftext[x=1.715169in,y=0.401852in,,top]{\color{textcolor}{\rmfamily\fontsize{9.000000}{10.800000}\selectfont\catcode`\^=\active\def^{\ifmmode\sp\else\^{}\fi}\catcode`\%=\active\def%{\%}$\mathdefault{\ensuremath{-}1.25}$}}%
\end{pgfscope}%
\begin{pgfscope}%
\pgfsetbuttcap%
\pgfsetroundjoin%
\definecolor{currentfill}{rgb}{0.000000,0.000000,0.000000}%
\pgfsetfillcolor{currentfill}%
\pgfsetlinewidth{0.803000pt}%
\definecolor{currentstroke}{rgb}{0.000000,0.000000,0.000000}%
\pgfsetstrokecolor{currentstroke}%
\pgfsetdash{}{0pt}%
\pgfsys@defobject{currentmarker}{\pgfqpoint{0.000000in}{-0.048611in}}{\pgfqpoint{0.000000in}{0.000000in}}{%
\pgfpathmoveto{\pgfqpoint{0.000000in}{0.000000in}}%
\pgfpathlineto{\pgfqpoint{0.000000in}{-0.048611in}}%
\pgfusepath{stroke,fill}%
}%
\begin{pgfscope}%
\pgfsys@transformshift{2.114747in}{0.499074in}%
\pgfsys@useobject{currentmarker}{}%
\end{pgfscope}%
\end{pgfscope}%
\begin{pgfscope}%
\definecolor{textcolor}{rgb}{0.000000,0.000000,0.000000}%
\pgfsetstrokecolor{textcolor}%
\pgfsetfillcolor{textcolor}%
\pgftext[x=2.114747in,y=0.401852in,,top]{\color{textcolor}{\rmfamily\fontsize{9.000000}{10.800000}\selectfont\catcode`\^=\active\def^{\ifmmode\sp\else\^{}\fi}\catcode`\%=\active\def%{\%}$\mathdefault{\ensuremath{-}1.00}$}}%
\end{pgfscope}%
\begin{pgfscope}%
\pgfsetbuttcap%
\pgfsetroundjoin%
\definecolor{currentfill}{rgb}{0.000000,0.000000,0.000000}%
\pgfsetfillcolor{currentfill}%
\pgfsetlinewidth{0.803000pt}%
\definecolor{currentstroke}{rgb}{0.000000,0.000000,0.000000}%
\pgfsetstrokecolor{currentstroke}%
\pgfsetdash{}{0pt}%
\pgfsys@defobject{currentmarker}{\pgfqpoint{0.000000in}{-0.048611in}}{\pgfqpoint{0.000000in}{0.000000in}}{%
\pgfpathmoveto{\pgfqpoint{0.000000in}{0.000000in}}%
\pgfpathlineto{\pgfqpoint{0.000000in}{-0.048611in}}%
\pgfusepath{stroke,fill}%
}%
\begin{pgfscope}%
\pgfsys@transformshift{2.514325in}{0.499074in}%
\pgfsys@useobject{currentmarker}{}%
\end{pgfscope}%
\end{pgfscope}%
\begin{pgfscope}%
\definecolor{textcolor}{rgb}{0.000000,0.000000,0.000000}%
\pgfsetstrokecolor{textcolor}%
\pgfsetfillcolor{textcolor}%
\pgftext[x=2.514325in,y=0.401852in,,top]{\color{textcolor}{\rmfamily\fontsize{9.000000}{10.800000}\selectfont\catcode`\^=\active\def^{\ifmmode\sp\else\^{}\fi}\catcode`\%=\active\def%{\%}$\mathdefault{\ensuremath{-}0.75}$}}%
\end{pgfscope}%
\begin{pgfscope}%
\pgfsetbuttcap%
\pgfsetroundjoin%
\definecolor{currentfill}{rgb}{0.000000,0.000000,0.000000}%
\pgfsetfillcolor{currentfill}%
\pgfsetlinewidth{0.803000pt}%
\definecolor{currentstroke}{rgb}{0.000000,0.000000,0.000000}%
\pgfsetstrokecolor{currentstroke}%
\pgfsetdash{}{0pt}%
\pgfsys@defobject{currentmarker}{\pgfqpoint{0.000000in}{-0.048611in}}{\pgfqpoint{0.000000in}{0.000000in}}{%
\pgfpathmoveto{\pgfqpoint{0.000000in}{0.000000in}}%
\pgfpathlineto{\pgfqpoint{0.000000in}{-0.048611in}}%
\pgfusepath{stroke,fill}%
}%
\begin{pgfscope}%
\pgfsys@transformshift{2.913903in}{0.499074in}%
\pgfsys@useobject{currentmarker}{}%
\end{pgfscope}%
\end{pgfscope}%
\begin{pgfscope}%
\definecolor{textcolor}{rgb}{0.000000,0.000000,0.000000}%
\pgfsetstrokecolor{textcolor}%
\pgfsetfillcolor{textcolor}%
\pgftext[x=2.913903in,y=0.401852in,,top]{\color{textcolor}{\rmfamily\fontsize{9.000000}{10.800000}\selectfont\catcode`\^=\active\def^{\ifmmode\sp\else\^{}\fi}\catcode`\%=\active\def%{\%}$\mathdefault{\ensuremath{-}0.50}$}}%
\end{pgfscope}%
\begin{pgfscope}%
\pgfsetbuttcap%
\pgfsetroundjoin%
\definecolor{currentfill}{rgb}{0.000000,0.000000,0.000000}%
\pgfsetfillcolor{currentfill}%
\pgfsetlinewidth{0.803000pt}%
\definecolor{currentstroke}{rgb}{0.000000,0.000000,0.000000}%
\pgfsetstrokecolor{currentstroke}%
\pgfsetdash{}{0pt}%
\pgfsys@defobject{currentmarker}{\pgfqpoint{0.000000in}{-0.048611in}}{\pgfqpoint{0.000000in}{0.000000in}}{%
\pgfpathmoveto{\pgfqpoint{0.000000in}{0.000000in}}%
\pgfpathlineto{\pgfqpoint{0.000000in}{-0.048611in}}%
\pgfusepath{stroke,fill}%
}%
\begin{pgfscope}%
\pgfsys@transformshift{3.313482in}{0.499074in}%
\pgfsys@useobject{currentmarker}{}%
\end{pgfscope}%
\end{pgfscope}%
\begin{pgfscope}%
\definecolor{textcolor}{rgb}{0.000000,0.000000,0.000000}%
\pgfsetstrokecolor{textcolor}%
\pgfsetfillcolor{textcolor}%
\pgftext[x=3.313482in,y=0.401852in,,top]{\color{textcolor}{\rmfamily\fontsize{9.000000}{10.800000}\selectfont\catcode`\^=\active\def^{\ifmmode\sp\else\^{}\fi}\catcode`\%=\active\def%{\%}$\mathdefault{\ensuremath{-}0.25}$}}%
\end{pgfscope}%
\begin{pgfscope}%
\pgfsetbuttcap%
\pgfsetroundjoin%
\definecolor{currentfill}{rgb}{0.000000,0.000000,0.000000}%
\pgfsetfillcolor{currentfill}%
\pgfsetlinewidth{0.803000pt}%
\definecolor{currentstroke}{rgb}{0.000000,0.000000,0.000000}%
\pgfsetstrokecolor{currentstroke}%
\pgfsetdash{}{0pt}%
\pgfsys@defobject{currentmarker}{\pgfqpoint{0.000000in}{-0.048611in}}{\pgfqpoint{0.000000in}{0.000000in}}{%
\pgfpathmoveto{\pgfqpoint{0.000000in}{0.000000in}}%
\pgfpathlineto{\pgfqpoint{0.000000in}{-0.048611in}}%
\pgfusepath{stroke,fill}%
}%
\begin{pgfscope}%
\pgfsys@transformshift{3.713060in}{0.499074in}%
\pgfsys@useobject{currentmarker}{}%
\end{pgfscope}%
\end{pgfscope}%
\begin{pgfscope}%
\definecolor{textcolor}{rgb}{0.000000,0.000000,0.000000}%
\pgfsetstrokecolor{textcolor}%
\pgfsetfillcolor{textcolor}%
\pgftext[x=3.713060in,y=0.401852in,,top]{\color{textcolor}{\rmfamily\fontsize{9.000000}{10.800000}\selectfont\catcode`\^=\active\def^{\ifmmode\sp\else\^{}\fi}\catcode`\%=\active\def%{\%}$\mathdefault{0.00}$}}%
\end{pgfscope}%
\begin{pgfscope}%
\definecolor{textcolor}{rgb}{0.000000,0.000000,0.000000}%
\pgfsetstrokecolor{textcolor}%
\pgfsetfillcolor{textcolor}%
\pgftext[x=2.114747in,y=0.235185in,,top]{\color{textcolor}{\rmfamily\fontsize{11.000000}{13.200000}\selectfont\catcode`\^=\active\def^{\ifmmode\sp\else\^{}\fi}\catcode`\%=\active\def%{\%}$v_1$}}%
\end{pgfscope}%
\begin{pgfscope}%
\pgfsetbuttcap%
\pgfsetroundjoin%
\definecolor{currentfill}{rgb}{0.000000,0.000000,0.000000}%
\pgfsetfillcolor{currentfill}%
\pgfsetlinewidth{0.803000pt}%
\definecolor{currentstroke}{rgb}{0.000000,0.000000,0.000000}%
\pgfsetstrokecolor{currentstroke}%
\pgfsetdash{}{0pt}%
\pgfsys@defobject{currentmarker}{\pgfqpoint{-0.048611in}{0.000000in}}{\pgfqpoint{-0.000000in}{0.000000in}}{%
\pgfpathmoveto{\pgfqpoint{-0.000000in}{0.000000in}}%
\pgfpathlineto{\pgfqpoint{-0.048611in}{0.000000in}}%
\pgfusepath{stroke,fill}%
}%
\begin{pgfscope}%
\pgfsys@transformshift{0.516434in}{0.499074in}%
\pgfsys@useobject{currentmarker}{}%
\end{pgfscope}%
\end{pgfscope}%
\begin{pgfscope}%
\definecolor{textcolor}{rgb}{0.000000,0.000000,0.000000}%
\pgfsetstrokecolor{textcolor}%
\pgfsetfillcolor{textcolor}%
\pgftext[x=0.354976in, y=0.455671in, left, base]{\color{textcolor}{\rmfamily\fontsize{9.000000}{10.800000}\selectfont\catcode`\^=\active\def^{\ifmmode\sp\else\^{}\fi}\catcode`\%=\active\def%{\%}$\mathdefault{0}$}}%
\end{pgfscope}%
\begin{pgfscope}%
\pgfsetbuttcap%
\pgfsetroundjoin%
\definecolor{currentfill}{rgb}{0.000000,0.000000,0.000000}%
\pgfsetfillcolor{currentfill}%
\pgfsetlinewidth{0.803000pt}%
\definecolor{currentstroke}{rgb}{0.000000,0.000000,0.000000}%
\pgfsetstrokecolor{currentstroke}%
\pgfsetdash{}{0pt}%
\pgfsys@defobject{currentmarker}{\pgfqpoint{-0.048611in}{0.000000in}}{\pgfqpoint{-0.000000in}{0.000000in}}{%
\pgfpathmoveto{\pgfqpoint{-0.000000in}{0.000000in}}%
\pgfpathlineto{\pgfqpoint{-0.048611in}{0.000000in}}%
\pgfusepath{stroke,fill}%
}%
\begin{pgfscope}%
\pgfsys@transformshift{0.516434in}{1.138399in}%
\pgfsys@useobject{currentmarker}{}%
\end{pgfscope}%
\end{pgfscope}%
\begin{pgfscope}%
\definecolor{textcolor}{rgb}{0.000000,0.000000,0.000000}%
\pgfsetstrokecolor{textcolor}%
\pgfsetfillcolor{textcolor}%
\pgftext[x=0.290741in, y=1.094996in, left, base]{\color{textcolor}{\rmfamily\fontsize{9.000000}{10.800000}\selectfont\catcode`\^=\active\def^{\ifmmode\sp\else\^{}\fi}\catcode`\%=\active\def%{\%}$\mathdefault{10}$}}%
\end{pgfscope}%
\begin{pgfscope}%
\pgfsetbuttcap%
\pgfsetroundjoin%
\definecolor{currentfill}{rgb}{0.000000,0.000000,0.000000}%
\pgfsetfillcolor{currentfill}%
\pgfsetlinewidth{0.803000pt}%
\definecolor{currentstroke}{rgb}{0.000000,0.000000,0.000000}%
\pgfsetstrokecolor{currentstroke}%
\pgfsetdash{}{0pt}%
\pgfsys@defobject{currentmarker}{\pgfqpoint{-0.048611in}{0.000000in}}{\pgfqpoint{-0.000000in}{0.000000in}}{%
\pgfpathmoveto{\pgfqpoint{-0.000000in}{0.000000in}}%
\pgfpathlineto{\pgfqpoint{-0.048611in}{0.000000in}}%
\pgfusepath{stroke,fill}%
}%
\begin{pgfscope}%
\pgfsys@transformshift{0.516434in}{1.777724in}%
\pgfsys@useobject{currentmarker}{}%
\end{pgfscope}%
\end{pgfscope}%
\begin{pgfscope}%
\definecolor{textcolor}{rgb}{0.000000,0.000000,0.000000}%
\pgfsetstrokecolor{textcolor}%
\pgfsetfillcolor{textcolor}%
\pgftext[x=0.290741in, y=1.734321in, left, base]{\color{textcolor}{\rmfamily\fontsize{9.000000}{10.800000}\selectfont\catcode`\^=\active\def^{\ifmmode\sp\else\^{}\fi}\catcode`\%=\active\def%{\%}$\mathdefault{20}$}}%
\end{pgfscope}%
\begin{pgfscope}%
\pgfsetbuttcap%
\pgfsetroundjoin%
\definecolor{currentfill}{rgb}{0.000000,0.000000,0.000000}%
\pgfsetfillcolor{currentfill}%
\pgfsetlinewidth{0.803000pt}%
\definecolor{currentstroke}{rgb}{0.000000,0.000000,0.000000}%
\pgfsetstrokecolor{currentstroke}%
\pgfsetdash{}{0pt}%
\pgfsys@defobject{currentmarker}{\pgfqpoint{-0.048611in}{0.000000in}}{\pgfqpoint{-0.000000in}{0.000000in}}{%
\pgfpathmoveto{\pgfqpoint{-0.000000in}{0.000000in}}%
\pgfpathlineto{\pgfqpoint{-0.048611in}{0.000000in}}%
\pgfusepath{stroke,fill}%
}%
\begin{pgfscope}%
\pgfsys@transformshift{0.516434in}{2.417049in}%
\pgfsys@useobject{currentmarker}{}%
\end{pgfscope}%
\end{pgfscope}%
\begin{pgfscope}%
\definecolor{textcolor}{rgb}{0.000000,0.000000,0.000000}%
\pgfsetstrokecolor{textcolor}%
\pgfsetfillcolor{textcolor}%
\pgftext[x=0.290741in, y=2.373646in, left, base]{\color{textcolor}{\rmfamily\fontsize{9.000000}{10.800000}\selectfont\catcode`\^=\active\def^{\ifmmode\sp\else\^{}\fi}\catcode`\%=\active\def%{\%}$\mathdefault{30}$}}%
\end{pgfscope}%
\begin{pgfscope}%
\pgfsetbuttcap%
\pgfsetroundjoin%
\definecolor{currentfill}{rgb}{0.000000,0.000000,0.000000}%
\pgfsetfillcolor{currentfill}%
\pgfsetlinewidth{0.803000pt}%
\definecolor{currentstroke}{rgb}{0.000000,0.000000,0.000000}%
\pgfsetstrokecolor{currentstroke}%
\pgfsetdash{}{0pt}%
\pgfsys@defobject{currentmarker}{\pgfqpoint{-0.048611in}{0.000000in}}{\pgfqpoint{-0.000000in}{0.000000in}}{%
\pgfpathmoveto{\pgfqpoint{-0.000000in}{0.000000in}}%
\pgfpathlineto{\pgfqpoint{-0.048611in}{0.000000in}}%
\pgfusepath{stroke,fill}%
}%
\begin{pgfscope}%
\pgfsys@transformshift{0.516434in}{3.056374in}%
\pgfsys@useobject{currentmarker}{}%
\end{pgfscope}%
\end{pgfscope}%
\begin{pgfscope}%
\definecolor{textcolor}{rgb}{0.000000,0.000000,0.000000}%
\pgfsetstrokecolor{textcolor}%
\pgfsetfillcolor{textcolor}%
\pgftext[x=0.290741in, y=3.012972in, left, base]{\color{textcolor}{\rmfamily\fontsize{9.000000}{10.800000}\selectfont\catcode`\^=\active\def^{\ifmmode\sp\else\^{}\fi}\catcode`\%=\active\def%{\%}$\mathdefault{40}$}}%
\end{pgfscope}%
\begin{pgfscope}%
\pgfsetbuttcap%
\pgfsetroundjoin%
\definecolor{currentfill}{rgb}{0.000000,0.000000,0.000000}%
\pgfsetfillcolor{currentfill}%
\pgfsetlinewidth{0.803000pt}%
\definecolor{currentstroke}{rgb}{0.000000,0.000000,0.000000}%
\pgfsetstrokecolor{currentstroke}%
\pgfsetdash{}{0pt}%
\pgfsys@defobject{currentmarker}{\pgfqpoint{-0.048611in}{0.000000in}}{\pgfqpoint{-0.000000in}{0.000000in}}{%
\pgfpathmoveto{\pgfqpoint{-0.000000in}{0.000000in}}%
\pgfpathlineto{\pgfqpoint{-0.048611in}{0.000000in}}%
\pgfusepath{stroke,fill}%
}%
\begin{pgfscope}%
\pgfsys@transformshift{0.516434in}{3.695699in}%
\pgfsys@useobject{currentmarker}{}%
\end{pgfscope}%
\end{pgfscope}%
\begin{pgfscope}%
\definecolor{textcolor}{rgb}{0.000000,0.000000,0.000000}%
\pgfsetstrokecolor{textcolor}%
\pgfsetfillcolor{textcolor}%
\pgftext[x=0.290741in, y=3.652297in, left, base]{\color{textcolor}{\rmfamily\fontsize{9.000000}{10.800000}\selectfont\catcode`\^=\active\def^{\ifmmode\sp\else\^{}\fi}\catcode`\%=\active\def%{\%}$\mathdefault{50}$}}%
\end{pgfscope}%
\begin{pgfscope}%
\definecolor{textcolor}{rgb}{0.000000,0.000000,0.000000}%
\pgfsetstrokecolor{textcolor}%
\pgfsetfillcolor{textcolor}%
\pgftext[x=0.235185in,y=2.097387in,,bottom,rotate=90.000000]{\color{textcolor}{\rmfamily\fontsize{11.000000}{13.200000}\selectfont\catcode`\^=\active\def^{\ifmmode\sp\else\^{}\fi}\catcode`\%=\active\def%{\%}$v_2$}}%
\end{pgfscope}%
\begin{pgfscope}%
\pgfsetrectcap%
\pgfsetmiterjoin%
\pgfsetlinewidth{0.803000pt}%
\definecolor{currentstroke}{rgb}{0.000000,0.000000,0.000000}%
\pgfsetstrokecolor{currentstroke}%
\pgfsetdash{}{0pt}%
\pgfpathmoveto{\pgfqpoint{0.516434in}{0.499074in}}%
\pgfpathlineto{\pgfqpoint{0.516434in}{3.695699in}}%
\pgfusepath{stroke}%
\end{pgfscope}%
\begin{pgfscope}%
\pgfsetrectcap%
\pgfsetmiterjoin%
\pgfsetlinewidth{0.803000pt}%
\definecolor{currentstroke}{rgb}{0.000000,0.000000,0.000000}%
\pgfsetstrokecolor{currentstroke}%
\pgfsetdash{}{0pt}%
\pgfpathmoveto{\pgfqpoint{3.713060in}{0.499074in}}%
\pgfpathlineto{\pgfqpoint{3.713060in}{3.695699in}}%
\pgfusepath{stroke}%
\end{pgfscope}%
\begin{pgfscope}%
\pgfsetrectcap%
\pgfsetmiterjoin%
\pgfsetlinewidth{0.803000pt}%
\definecolor{currentstroke}{rgb}{0.000000,0.000000,0.000000}%
\pgfsetstrokecolor{currentstroke}%
\pgfsetdash{}{0pt}%
\pgfpathmoveto{\pgfqpoint{0.516434in}{0.499074in}}%
\pgfpathlineto{\pgfqpoint{3.713060in}{0.499074in}}%
\pgfusepath{stroke}%
\end{pgfscope}%
\begin{pgfscope}%
\pgfsetrectcap%
\pgfsetmiterjoin%
\pgfsetlinewidth{0.803000pt}%
\definecolor{currentstroke}{rgb}{0.000000,0.000000,0.000000}%
\pgfsetstrokecolor{currentstroke}%
\pgfsetdash{}{0pt}%
\pgfpathmoveto{\pgfqpoint{0.516434in}{3.695699in}}%
\pgfpathlineto{\pgfqpoint{3.713060in}{3.695699in}}%
\pgfusepath{stroke}%
\end{pgfscope}%
\begin{pgfscope}%
\pgfsetbuttcap%
\pgfsetmiterjoin%
\definecolor{currentfill}{rgb}{1.000000,1.000000,1.000000}%
\pgfsetfillcolor{currentfill}%
\pgfsetlinewidth{0.000000pt}%
\definecolor{currentstroke}{rgb}{0.000000,0.000000,0.000000}%
\pgfsetstrokecolor{currentstroke}%
\pgfsetstrokeopacity{0.000000}%
\pgfsetdash{}{0pt}%
\pgfpathmoveto{\pgfqpoint{3.891959in}{0.499074in}}%
\pgfpathlineto{\pgfqpoint{4.051791in}{0.499074in}}%
\pgfpathlineto{\pgfqpoint{4.051791in}{3.695699in}}%
\pgfpathlineto{\pgfqpoint{3.891959in}{3.695699in}}%
\pgfpathlineto{\pgfqpoint{3.891959in}{0.499074in}}%
\pgfpathclose%
\pgfusepath{fill}%
\end{pgfscope}%
\begin{pgfscope}%
\pgfsys@transformshift{3.890000in}{0.509102in}%
\pgftext[left,bottom]{\includegraphics[interpolate=true,width=0.160000in,height=3.200000in]{figures/mccormick/mccormick-bounds-1-upper-img1.png}}%
\end{pgfscope}%
\begin{pgfscope}%
\pgfsetbuttcap%
\pgfsetroundjoin%
\definecolor{currentfill}{rgb}{0.000000,0.000000,0.000000}%
\pgfsetfillcolor{currentfill}%
\pgfsetlinewidth{0.803000pt}%
\definecolor{currentstroke}{rgb}{0.000000,0.000000,0.000000}%
\pgfsetstrokecolor{currentstroke}%
\pgfsetdash{}{0pt}%
\pgfsys@defobject{currentmarker}{\pgfqpoint{0.000000in}{0.000000in}}{\pgfqpoint{0.048611in}{0.000000in}}{%
\pgfpathmoveto{\pgfqpoint{0.000000in}{0.000000in}}%
\pgfpathlineto{\pgfqpoint{0.048611in}{0.000000in}}%
\pgfusepath{stroke,fill}%
}%
\begin{pgfscope}%
\pgfsys@transformshift{4.051791in}{0.499074in}%
\pgfsys@useobject{currentmarker}{}%
\end{pgfscope}%
\end{pgfscope}%
\begin{pgfscope}%
\definecolor{textcolor}{rgb}{0.000000,0.000000,0.000000}%
\pgfsetstrokecolor{textcolor}%
\pgfsetfillcolor{textcolor}%
\pgftext[x=4.149013in, y=0.455671in, left, base]{\color{textcolor}{\rmfamily\fontsize{9.000000}{10.800000}\selectfont\catcode`\^=\active\def^{\ifmmode\sp\else\^{}\fi}\catcode`\%=\active\def%{\%}$\mathdefault{0}$}}%
\end{pgfscope}%
\begin{pgfscope}%
\pgfsetbuttcap%
\pgfsetroundjoin%
\definecolor{currentfill}{rgb}{0.000000,0.000000,0.000000}%
\pgfsetfillcolor{currentfill}%
\pgfsetlinewidth{0.803000pt}%
\definecolor{currentstroke}{rgb}{0.000000,0.000000,0.000000}%
\pgfsetstrokecolor{currentstroke}%
\pgfsetdash{}{0pt}%
\pgfsys@defobject{currentmarker}{\pgfqpoint{0.000000in}{0.000000in}}{\pgfqpoint{0.048611in}{0.000000in}}{%
\pgfpathmoveto{\pgfqpoint{0.000000in}{0.000000in}}%
\pgfpathlineto{\pgfqpoint{0.048611in}{0.000000in}}%
\pgfusepath{stroke,fill}%
}%
\begin{pgfscope}%
\pgfsys@transformshift{4.051791in}{1.138399in}%
\pgfsys@useobject{currentmarker}{}%
\end{pgfscope}%
\end{pgfscope}%
\begin{pgfscope}%
\definecolor{textcolor}{rgb}{0.000000,0.000000,0.000000}%
\pgfsetstrokecolor{textcolor}%
\pgfsetfillcolor{textcolor}%
\pgftext[x=4.149013in, y=1.094996in, left, base]{\color{textcolor}{\rmfamily\fontsize{9.000000}{10.800000}\selectfont\catcode`\^=\active\def^{\ifmmode\sp\else\^{}\fi}\catcode`\%=\active\def%{\%}$\mathdefault{10}$}}%
\end{pgfscope}%
\begin{pgfscope}%
\pgfsetbuttcap%
\pgfsetroundjoin%
\definecolor{currentfill}{rgb}{0.000000,0.000000,0.000000}%
\pgfsetfillcolor{currentfill}%
\pgfsetlinewidth{0.803000pt}%
\definecolor{currentstroke}{rgb}{0.000000,0.000000,0.000000}%
\pgfsetstrokecolor{currentstroke}%
\pgfsetdash{}{0pt}%
\pgfsys@defobject{currentmarker}{\pgfqpoint{0.000000in}{0.000000in}}{\pgfqpoint{0.048611in}{0.000000in}}{%
\pgfpathmoveto{\pgfqpoint{0.000000in}{0.000000in}}%
\pgfpathlineto{\pgfqpoint{0.048611in}{0.000000in}}%
\pgfusepath{stroke,fill}%
}%
\begin{pgfscope}%
\pgfsys@transformshift{4.051791in}{1.777724in}%
\pgfsys@useobject{currentmarker}{}%
\end{pgfscope}%
\end{pgfscope}%
\begin{pgfscope}%
\definecolor{textcolor}{rgb}{0.000000,0.000000,0.000000}%
\pgfsetstrokecolor{textcolor}%
\pgfsetfillcolor{textcolor}%
\pgftext[x=4.149013in, y=1.734321in, left, base]{\color{textcolor}{\rmfamily\fontsize{9.000000}{10.800000}\selectfont\catcode`\^=\active\def^{\ifmmode\sp\else\^{}\fi}\catcode`\%=\active\def%{\%}$\mathdefault{20}$}}%
\end{pgfscope}%
\begin{pgfscope}%
\pgfsetbuttcap%
\pgfsetroundjoin%
\definecolor{currentfill}{rgb}{0.000000,0.000000,0.000000}%
\pgfsetfillcolor{currentfill}%
\pgfsetlinewidth{0.803000pt}%
\definecolor{currentstroke}{rgb}{0.000000,0.000000,0.000000}%
\pgfsetstrokecolor{currentstroke}%
\pgfsetdash{}{0pt}%
\pgfsys@defobject{currentmarker}{\pgfqpoint{0.000000in}{0.000000in}}{\pgfqpoint{0.048611in}{0.000000in}}{%
\pgfpathmoveto{\pgfqpoint{0.000000in}{0.000000in}}%
\pgfpathlineto{\pgfqpoint{0.048611in}{0.000000in}}%
\pgfusepath{stroke,fill}%
}%
\begin{pgfscope}%
\pgfsys@transformshift{4.051791in}{2.417049in}%
\pgfsys@useobject{currentmarker}{}%
\end{pgfscope}%
\end{pgfscope}%
\begin{pgfscope}%
\definecolor{textcolor}{rgb}{0.000000,0.000000,0.000000}%
\pgfsetstrokecolor{textcolor}%
\pgfsetfillcolor{textcolor}%
\pgftext[x=4.149013in, y=2.373646in, left, base]{\color{textcolor}{\rmfamily\fontsize{9.000000}{10.800000}\selectfont\catcode`\^=\active\def^{\ifmmode\sp\else\^{}\fi}\catcode`\%=\active\def%{\%}$\mathdefault{30}$}}%
\end{pgfscope}%
\begin{pgfscope}%
\pgfsetbuttcap%
\pgfsetroundjoin%
\definecolor{currentfill}{rgb}{0.000000,0.000000,0.000000}%
\pgfsetfillcolor{currentfill}%
\pgfsetlinewidth{0.803000pt}%
\definecolor{currentstroke}{rgb}{0.000000,0.000000,0.000000}%
\pgfsetstrokecolor{currentstroke}%
\pgfsetdash{}{0pt}%
\pgfsys@defobject{currentmarker}{\pgfqpoint{0.000000in}{0.000000in}}{\pgfqpoint{0.048611in}{0.000000in}}{%
\pgfpathmoveto{\pgfqpoint{0.000000in}{0.000000in}}%
\pgfpathlineto{\pgfqpoint{0.048611in}{0.000000in}}%
\pgfusepath{stroke,fill}%
}%
\begin{pgfscope}%
\pgfsys@transformshift{4.051791in}{3.056374in}%
\pgfsys@useobject{currentmarker}{}%
\end{pgfscope}%
\end{pgfscope}%
\begin{pgfscope}%
\definecolor{textcolor}{rgb}{0.000000,0.000000,0.000000}%
\pgfsetstrokecolor{textcolor}%
\pgfsetfillcolor{textcolor}%
\pgftext[x=4.149013in, y=3.012972in, left, base]{\color{textcolor}{\rmfamily\fontsize{9.000000}{10.800000}\selectfont\catcode`\^=\active\def^{\ifmmode\sp\else\^{}\fi}\catcode`\%=\active\def%{\%}$\mathdefault{40}$}}%
\end{pgfscope}%
\begin{pgfscope}%
\pgfsetbuttcap%
\pgfsetroundjoin%
\definecolor{currentfill}{rgb}{0.000000,0.000000,0.000000}%
\pgfsetfillcolor{currentfill}%
\pgfsetlinewidth{0.803000pt}%
\definecolor{currentstroke}{rgb}{0.000000,0.000000,0.000000}%
\pgfsetstrokecolor{currentstroke}%
\pgfsetdash{}{0pt}%
\pgfsys@defobject{currentmarker}{\pgfqpoint{0.000000in}{0.000000in}}{\pgfqpoint{0.048611in}{0.000000in}}{%
\pgfpathmoveto{\pgfqpoint{0.000000in}{0.000000in}}%
\pgfpathlineto{\pgfqpoint{0.048611in}{0.000000in}}%
\pgfusepath{stroke,fill}%
}%
\begin{pgfscope}%
\pgfsys@transformshift{4.051791in}{3.695699in}%
\pgfsys@useobject{currentmarker}{}%
\end{pgfscope}%
\end{pgfscope}%
\begin{pgfscope}%
\definecolor{textcolor}{rgb}{0.000000,0.000000,0.000000}%
\pgfsetstrokecolor{textcolor}%
\pgfsetfillcolor{textcolor}%
\pgftext[x=4.149013in, y=3.652297in, left, base]{\color{textcolor}{\rmfamily\fontsize{9.000000}{10.800000}\selectfont\catcode`\^=\active\def^{\ifmmode\sp\else\^{}\fi}\catcode`\%=\active\def%{\%}$\mathdefault{50}$}}%
\end{pgfscope}%
\begin{pgfscope}%
\pgfsetrectcap%
\pgfsetmiterjoin%
\pgfsetlinewidth{0.803000pt}%
\definecolor{currentstroke}{rgb}{0.000000,0.000000,0.000000}%
\pgfsetstrokecolor{currentstroke}%
\pgfsetdash{}{0pt}%
\pgfpathmoveto{\pgfqpoint{3.891959in}{0.499074in}}%
\pgfpathlineto{\pgfqpoint{3.971875in}{0.499074in}}%
\pgfpathlineto{\pgfqpoint{4.051791in}{0.499074in}}%
\pgfpathlineto{\pgfqpoint{4.051791in}{3.695699in}}%
\pgfpathlineto{\pgfqpoint{3.971875in}{3.695699in}}%
\pgfpathlineto{\pgfqpoint{3.891959in}{3.695699in}}%
\pgfpathlineto{\pgfqpoint{3.891959in}{0.499074in}}%
\pgfpathclose%
\pgfusepath{stroke}%
\end{pgfscope}%
\end{pgfpicture}%
\makeatother%
\endgroup%
}
		\caption{Difference to the upper bound}
		\label{fig:mccormick_1_upper}
	\end{subfigure}
	\hfill
	\begin{subfigure}[b]{0.45\textwidth}
		\centering
		\resizebox{\textwidth}{!}{\input{figures/mccormick/mccormick-bounds-1-lower.pgf}}
		\caption{Difference to the lower bound}
		\label{fig:mccormick_1_lower}
	\end{subfigure}
	\caption{McCormick relaxation bounds for the bilinear term \(  xy \) with stricter bounds on \( x \).}
	\label{fig:mccormick_bounds_1}
\end{figure}

Figures \ref{fig:mccormick_1_upper} and \ref{fig:mccormick_1_lower} present the results when \( x \) is more tightly bounded, specifically \( -2 \leq
x \leq 0 \).
One can observe that the maximum deviation is considerably reduced compared to the previous scenario, indicating that tighter bounds yield a more
accurate relaxation.

\subsubsection{Improve Bounds}

To improve the bounds of the McCormick relaxation, we can use tighter bounds for the variables involved in the bilinear terms.
By reducing the range of the variables, we can achieve a more accurate approximation of the bilinear terms.
This can be done by analyzing the specific problem and determining the realistic bounds for the variables.

For example, if we know that the variable \( x \) is always within a smaller range, such as \( -1 \leq x \leq 1 \), we can use these tighter bounds
to improve the McCormick relaxation.
Similarly, if the variable \( y \) is within a smaller range, such as \( 0 \leq y \leq 25 \), we can use these bounds to achieve a more accurate
approximation.

To apply tighter bounds to the velocity, we can use the known current velocity and the maximum and minimum acceleration values.
This allows us to provide a more accurate range for the velocity at each time point, especially for the first few time points.

Let \( v_0 \) be the current velocity, \( \underline{a} \) be the minimum acceleration, and \( \overline{a} \) be the maximum acceleration.
The velocity at the next time point \( v_1 \) can be bounded as follows:

\[ v_1 \in [v_0 + \underline{a} \Delta t, v_0 +
		\overline{a} \Delta t] \] where \( \Delta t \) is the time step.

For subsequent time points, we can iteratively apply these bounds to obtain tighter ranges for the velocity.
For example, the velocity at the second time point \( v_2 \) can be bounded as:

\[ v_2 \in [v_0 + 2\underline{a} \Delta t, v_0 +
		2\overline{a} \Delta t] \]

By applying these tighter bounds, we can improve the accuracy of the McCormick relaxation for the
bilinear terms involving the velocity.
This results in a more precise and efficient representation of the vehicle's motion, which is essential for accurate path tracking and motion
planning.

In summary, improving the bounds of the McCormick relaxation involves: 1.
Analyzing the specific problem to determine realistic bounds for the variables.
2.
Using tighter bounds for the variables involved in the bilinear terms.
3.
Applying these tighter bounds to achieve a more accurate approximation of the bilinear terms.

This approach ensures that the McCormick relaxation provides a more precise and efficient representation of the bilinear terms, which is crucial for
the accurate modeling and control of vehicle dynamics.

\subsection{Final Model} \label{subsec:bicycle_resulting_model}

The final linearized state transition model for the Frenet frame, incorporating the linearized non-linear terms and the McCormick relaxation for the bilinear terms, is given by:

\begin{equation}
	\label{eq:kst_final_dynamics}
	f(s, n, \xi, v, \delta, a_{x,b}, v_\delta )\approx
	\begin{bmatrix}
		v (\cos(\xi_0) + \sin(\xi_0)\xi_0) - \sin(\xi_0) w_{v,\xi}                                                                                                 \\[8pt]
		v (\sin(\xi_0) - \cos(\xi_0)\xi_0) + \cos(\xi_0) w_{v,\xi}                                                                                                 \\[8pt]
		\frac{v}{l_{wb}} (\tan(\delta_0) - \frac{\delta_0}{\cos^2(\delta_0)}) + \frac{w_{v,\delta}}{\cos^2(\delta_0)}  - a_{i(s)} w_{s,\dot{s}} - b_{i(s)} \dot{s} \\[8pt]
		a_{x,b}                                                                                                                                                    \\[8pt]
		v_\delta
	\end{bmatrix}
\end{equation}
Where $w_{v,\xi}$, $w_{v,\delta}$, and $w_{s,\dot{s}}$ are auxiliary variables introduced by the McCormick relaxation to approximate the bilinear
terms $v\xi$, $v\delta$, and $s\dot{s}$, respectively.
The Curvature is treated as a piece-wise linear function, where $a_{i(s)}$ and $b_{i(s)}$ are the slope and intercept of the linear function in the
interval $[s_{i-1}, s_i]$, which can be select by the position $s$.
The index will be treated as a constant during planing for each time point, which we can achieve by predicted the road segment at each time point.
The additional constraints for the McCormick relaxation are:
\begin{figure}[h]
	\centering
	\begin{subfigure}[b]{0.3\textwidth}
		\centering
		\[
			\begin{aligned}
				w_{v,\xi} & \geq \underline{v} \xi + v \underline{\xi} - \underline{v} \underline{\xi}, \\
				w_{v,\xi} & \geq \overline{v} \xi + v \overline{\xi} - \overline{v} \overline{\xi},     \\
				w_{v,\xi} & \leq \overline{v} \xi + v \underline{\xi} - \overline{v} \underline{\xi},   \\
				w_{v,\xi} & \leq \underline{v} \xi + v \overline{\xi} - \underline{v} \overline{\xi}.
			\end{aligned}
		\]
	\end{subfigure}
	\hfill
	\begin{subfigure}[b]{0.3\textwidth}
		\centering
		\[
			\begin{aligned}
				w_{v,\delta} & \geq \underline{v} \delta + v \underline{\delta} - \underline{v} \underline{\delta}, \\
				w_{v,\delta} & \geq \overline{v} \delta + v \overline{\delta} - \overline{v} \overline{\delta},     \\
				w_{v,\delta} & \leq \overline{v} \delta + v \underline{\delta} - \overline{v} \underline{\delta},   \\
				w_{v,\delta} & \leq \underline{v} \delta + v \overline{\delta} - \underline{v} \overline{\delta}.
			\end{aligned}
		\]
	\end{subfigure}
	\hfill
	\begin{subfigure}[b]{0.3\textwidth}
		\centering
		\[
			\begin{aligned}
				w_{s,\dot{s}} & \geq \underline{s} \dot{s} + s \underline{\dot{s}} - \underline{s} \underline{\dot{s}}, \\
				w_{s,\dot{s}} & \geq \overline{s} \dot{s} + s \overline{\dot{s}} - \overline{s} \overline{\dot{s}},     \\
				w_{s,\dot{s}} & \leq \overline{s} \dot{s} + s \underline{\dot{s}} - \overline{s} \underline{\dot{s}},   \\
				w_{s,\dot{s}} & \leq \underline{s} \dot{s} + s \overline{\dot{s}} - \underline{s} \overline{\dot{s}}.
			\end{aligned}
		\]
	\end{subfigure}
	\caption{McCormick relaxation constraints for the bilinear terms.}
	\label{fig:mccormick_constraints}
\end{figure}

\subsubsection{Coupling Constraints} \label{sec:kst_coupling_constraints}
The coupling constraints are simple as \[ s\in[\underline{s}, \overline{s}], n\in[\underline{n}(s), \overline{n}(s)], \xi \in [\underline{\xi},
		\overline{\xi}], v\in[\underline{v}, \overline{v}], \delta \in [\underline{\delta}, \overline{\delta}], \] \[ v_\delta \in [\underline{v_\delta},
		\overline{v_\delta}], a_{x,b} \in [\underline{a}_{x,b}, \overline{a}_{x,b}\min\{1, \frac{v_S}{v}\}] \] where $v_S$ is parameter which is used to
encounter limiting engine power and breaking power.

Last but not least we have to consider the friction circle.
\[
	\sqrt{a_{x,b}^2 + (\frac{v^2}{l_{wb}} \tan(\delta))^2} \leq a_{max}
\]
Utilizing that $\tan(\delta) \leq \frac{\tan(\overline{\delta})}{\overline{\delta}}\delta$ leads to a bit more conservative constraint.
\[
	a_{x,b}^2 + (\frac{1}{l_{wb}}\frac{\tan(\overline{\delta})}{\overline{\delta}})^2 v^4 \delta^2 \leq a_{max}^2
\]

Next, we derive an upper bound for the term \(v^4 \delta^2\).
This term appears in the friction circle constraint and can significantly impact the vehicle's dynamics.
By finding an upper bound, we can simplify the constraint and ensure that the model remains computationally efficient while maintaining accuracy.

\subsubsection{Upper Bound for \(v^4 \delta^2\)}

Our goal is to find the parameters for the following term: \[ a v^2 + b \delta^2 \] Such that this term equals the term \(v^4 \delta^2\) for the
absolute maximum values of \(v\) and \(\delta\) $v^* = max\{|\underline{v}|, |\overline{v}|\}$, $\delta^* = max\{|\underline{\delta}|,
	|\overline{\delta}|\}$.

We end up with the following values:

\[ a = max\left\{\frac{(v^*)^4 (\delta^*)^2 - \frac{(\delta^*)^3}{v^* + \delta^*}}{(v^*)^2},
	0\right\} \]

\[ b = \begin{cases}
		(v^*)^4                         & \text{if } a = 0 \\
		\frac{\delta^*}{v^* + \delta^*} & \text{otherwise}
	\end{cases} \]

We can use this upper bound to enforce the friction circle by the following constraint:

\[ a_{x,b}^2 +
	(\frac{1}{l_{wb}}\frac{\tan(\overline{\delta})}{\overline{\delta}})^2 (a v^2 + b \delta^2) \leq a_{max}^2 \]

Since $a_{x,b}$,
$v$, and $\delta$ are the only variables in this constraint and each of them only occur squared, without any interaction, we have a convex constraint
which we can use to enforce the friction circle.

% \subsection{Linearize Quadratic Term}

% In this section, we will linearize the quadratic term \(s^2\) in the state transition model.
% For quadratic terms, which occur in equality constraints, we can use the McCormick relaxation technique, which is typically used for bilinear terms,
% but can also be applied to quadratic terms.
% However, in the case of quadratic terms, we can achieve better bounds.
% For the lower bound, which must be convex, we can simply use the term itself, as a quadratic term is always convex.
% The best upper bound can be obtained by connecting the points defined by the bounds on \(s\).
% \[
% 	s^2 \leq (\underline{s} + \overline{s})s - \underline{s}\overline{s}
% \]
