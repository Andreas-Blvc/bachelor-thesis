% \section{Modeling Techniques} \label{ch:modeling_techniques}

% In the previous chapters, we have explored various modeling techniques that are essential for solving complex optimization problems.
% We began by discussing the decoupling of variables using quantifier elimination, a powerful method that simplifies the problem by reducing the number
% of variables involved.
% This technique is particularly useful in scenarios where the relationships between variables are intricate and non-linear.

% Next, we examined convex relaxations, specifically focusing on McCormick envelopes for bilinear terms.
% Convex relaxations are important for transforming non-convex problems into convex ones, which are easier to solve.
% McCormick envelopes provide a way to approximate bilinear terms with convex functions, thereby enabling the use of efficient convex optimization
% algorithms.

% In this section, we present a collection of commonly used modeling methods for convex programming.
% These methods serve as a practical guide for formulating and solving convex optimization problems.

% \subsection{Penalty Methods}

% Penalty methods are another approach to handle constraints in optimization problems.
% These methods incorporate constraints into the objective function by adding a penalty term that increases the objective value when constraints are
% violated.
% This approach transforms a constrained optimization problem into an unconstrained one, which can be easier to solve.

% Consider an optimization problem with a constraint \( g(x) \leq 0 \).
% In a penalty method, we modify the objective function to include a penalty term \( P(g(x)) \) that penalizes constraint violations.
% A common choice for the penalty term is a quadratic function, such as \( P(g(x)) = \mu \max(0, g(x))^2 \), where \( \mu \) is a positive penalty
% parameter.

% The modified optimization problem can be written as:
% \begin{align*}
% 	\min_{x} \quad & f(x) + \mu \max(0, g(x))^2
% \end{align*}

% By adjusting the value of \( \mu \), we can control the severity of the penalty for constraint violations.
% A larger \( \mu \) places more emphasis on satisfying the constraint, while a smaller \( \mu \) allows for greater flexibility in violating the
% constraint.

% Penalty methods are particularly useful when dealing with complex constraints that are difficult to handle directly.
% By incorporating these constraints into the objective function, we can leverage efficient unconstrained optimization algorithms to find solutions.

% \subsection{Conclusion}

% In this chapter, we have explored various modeling techniques that are essential for solving complex optimization problems.
% We discussed the use of soft constraints, auxiliary variables and penalty methods, each of which offers unique advantages for handling different
% types of constraints and objectives.
% By understanding and applying these techniques, we can formulate and solve optimization problems more effectively, leading to better solutions and
% improved performance in practical applications.
