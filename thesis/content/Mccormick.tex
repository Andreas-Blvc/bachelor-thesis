\chapter{McCormick Relaxation}

To address bilinear terms of the form \( w = xy \), we introduce the following constraints based on the bounds of \( x \) and \( y \):

\[ x^L \leq x \leq x^U, \qquad y^L \leq y \leq y^U.
\]

The resulting McCormick relaxation constraints for \( w \) are:

\[
	\begin{aligned}
		w & \geq x^L y + x y^L - x^L y^L, \\
		w & \geq x^U y + x y^U - x^U y^U, \\
		w & \leq x^U y + x y^L - x^U y^L, \\
		w & \leq x^L y + x y^U - x^L y^U.
	\end{aligned}
\]

These constraints establish an overestimation and underestimation of the bilinear term \( w \), which can be visualized to assess their accuracy
compared to the actual bilinear relationship.

\begin{figure}[h]
	\centering
	\begin{subfigure}[b]{0.45\textwidth}
		\centering
		\resizebox{\textwidth}{!}{\input{figures/mccormick/mccormick-bounds-0-upper.pgf}}
		\caption{Difference to the upper bound}
		\label{fig:mccormick_0_upper}
	\end{subfigure}
	\hfill
	\begin{subfigure}[b]{0.45\textwidth}
		\centering
		\resizebox{\textwidth}{!}{\input{figures/mccormick/mccormick-bounds-0-lower.pgf}}
		\caption{Difference to the lower bound}
		\label{fig:mccormick_0_lower}
	\end{subfigure}
	\caption{McCormick relaxation bounds for the bilinear term \( w = xy \).}
	\label{fig:mccormick_bounds_0}
\end{figure}

Figure \ref{fig:mccormick_0_upper} illustrates the deviation between the actual bilinear term \( w = xy \) and the smallest upper bound provided by
the relaxation constraints.
Similarly, Figure \ref{fig:mccormick_0_lower} shows the deviation to the greatest lower bound.
For the range \( -2 \leq x \leq 2 \) and \( 0 \leq y \leq 50 \).
It is evident that the bounds improve as \( x \) and \( y \) approach their respective limits.

\begin{figure}[h]
	\centering
	\begin{subfigure}[b]{0.45\textwidth}
		\centering
		\resizebox{\textwidth}{!}{%% Creator: Matplotlib, PGF backend
%%
%% To include the figure in your LaTeX document, write
%%   \input{<filename>.pgf}
%%
%% Make sure the required packages are loaded in your preamble
%%   \usepackage{pgf}
%%
%% Also ensure that all the required font packages are loaded; for instance,
%% the lmodern package is sometimes necessary when using math font.
%%   \usepackage{lmodern}
%%
%% Figures using additional raster images can only be included by \input if
%% they are in the same directory as the main LaTeX file. For loading figures
%% from other directories you can use the `import` package
%%   \usepackage{import}
%%
%% and then include the figures with
%%   \import{<path to file>}{<filename>.pgf}
%%
%% Matplotlib used the following preamble
%%   \def\mathdefault#1{#1}
%%   \everymath=\expandafter{\the\everymath\displaystyle}
%%   
%%   \ifdefined\pdftexversion\else  % non-pdftex case.
%%     \usepackage{fontspec}
%%   \fi
%%   \makeatletter\@ifpackageloaded{underscore}{}{\usepackage[strings]{underscore}}\makeatother
%%
\begingroup%
\makeatletter%
\begin{pgfpicture}%
\pgfpathrectangle{\pgfpointorigin}{\pgfqpoint{4.377484in}{3.839102in}}%
\pgfusepath{use as bounding box, clip}%
\begin{pgfscope}%
\pgfsetbuttcap%
\pgfsetmiterjoin%
\definecolor{currentfill}{rgb}{1.000000,1.000000,1.000000}%
\pgfsetfillcolor{currentfill}%
\pgfsetlinewidth{0.000000pt}%
\definecolor{currentstroke}{rgb}{1.000000,1.000000,1.000000}%
\pgfsetstrokecolor{currentstroke}%
\pgfsetdash{}{0pt}%
\pgfpathmoveto{\pgfqpoint{0.000000in}{0.000000in}}%
\pgfpathlineto{\pgfqpoint{4.377484in}{0.000000in}}%
\pgfpathlineto{\pgfqpoint{4.377484in}{3.839102in}}%
\pgfpathlineto{\pgfqpoint{0.000000in}{3.839102in}}%
\pgfpathlineto{\pgfqpoint{0.000000in}{0.000000in}}%
\pgfpathclose%
\pgfusepath{fill}%
\end{pgfscope}%
\begin{pgfscope}%
\pgfsetbuttcap%
\pgfsetmiterjoin%
\definecolor{currentfill}{rgb}{1.000000,1.000000,1.000000}%
\pgfsetfillcolor{currentfill}%
\pgfsetlinewidth{0.000000pt}%
\definecolor{currentstroke}{rgb}{0.000000,0.000000,0.000000}%
\pgfsetstrokecolor{currentstroke}%
\pgfsetstrokeopacity{0.000000}%
\pgfsetdash{}{0pt}%
\pgfpathmoveto{\pgfqpoint{0.516434in}{0.499074in}}%
\pgfpathlineto{\pgfqpoint{3.713060in}{0.499074in}}%
\pgfpathlineto{\pgfqpoint{3.713060in}{3.695699in}}%
\pgfpathlineto{\pgfqpoint{0.516434in}{3.695699in}}%
\pgfpathlineto{\pgfqpoint{0.516434in}{0.499074in}}%
\pgfpathclose%
\pgfusepath{fill}%
\end{pgfscope}%
\begin{pgfscope}%
\pgfpathrectangle{\pgfqpoint{0.516434in}{0.499074in}}{\pgfqpoint{3.196625in}{3.196625in}}%
\pgfusepath{clip}%
\pgfsys@transformshift{0.516434in}{0.499074in}%
\pgftext[left,bottom]{\includegraphics[interpolate=true,width=3.200000in,height=3.200000in]{figures/mccormick/mccormick-bounds-1-upper-img0.png}}%
\end{pgfscope}%
\begin{pgfscope}%
\pgfsetbuttcap%
\pgfsetroundjoin%
\definecolor{currentfill}{rgb}{0.000000,0.000000,0.000000}%
\pgfsetfillcolor{currentfill}%
\pgfsetlinewidth{0.803000pt}%
\definecolor{currentstroke}{rgb}{0.000000,0.000000,0.000000}%
\pgfsetstrokecolor{currentstroke}%
\pgfsetdash{}{0pt}%
\pgfsys@defobject{currentmarker}{\pgfqpoint{0.000000in}{-0.048611in}}{\pgfqpoint{0.000000in}{0.000000in}}{%
\pgfpathmoveto{\pgfqpoint{0.000000in}{0.000000in}}%
\pgfpathlineto{\pgfqpoint{0.000000in}{-0.048611in}}%
\pgfusepath{stroke,fill}%
}%
\begin{pgfscope}%
\pgfsys@transformshift{0.516434in}{0.499074in}%
\pgfsys@useobject{currentmarker}{}%
\end{pgfscope}%
\end{pgfscope}%
\begin{pgfscope}%
\definecolor{textcolor}{rgb}{0.000000,0.000000,0.000000}%
\pgfsetstrokecolor{textcolor}%
\pgfsetfillcolor{textcolor}%
\pgftext[x=0.516434in,y=0.401852in,,top]{\color{textcolor}{\rmfamily\fontsize{9.000000}{10.800000}\selectfont\catcode`\^=\active\def^{\ifmmode\sp\else\^{}\fi}\catcode`\%=\active\def%{\%}$\mathdefault{\ensuremath{-}2.00}$}}%
\end{pgfscope}%
\begin{pgfscope}%
\pgfsetbuttcap%
\pgfsetroundjoin%
\definecolor{currentfill}{rgb}{0.000000,0.000000,0.000000}%
\pgfsetfillcolor{currentfill}%
\pgfsetlinewidth{0.803000pt}%
\definecolor{currentstroke}{rgb}{0.000000,0.000000,0.000000}%
\pgfsetstrokecolor{currentstroke}%
\pgfsetdash{}{0pt}%
\pgfsys@defobject{currentmarker}{\pgfqpoint{0.000000in}{-0.048611in}}{\pgfqpoint{0.000000in}{0.000000in}}{%
\pgfpathmoveto{\pgfqpoint{0.000000in}{0.000000in}}%
\pgfpathlineto{\pgfqpoint{0.000000in}{-0.048611in}}%
\pgfusepath{stroke,fill}%
}%
\begin{pgfscope}%
\pgfsys@transformshift{0.916012in}{0.499074in}%
\pgfsys@useobject{currentmarker}{}%
\end{pgfscope}%
\end{pgfscope}%
\begin{pgfscope}%
\definecolor{textcolor}{rgb}{0.000000,0.000000,0.000000}%
\pgfsetstrokecolor{textcolor}%
\pgfsetfillcolor{textcolor}%
\pgftext[x=0.916012in,y=0.401852in,,top]{\color{textcolor}{\rmfamily\fontsize{9.000000}{10.800000}\selectfont\catcode`\^=\active\def^{\ifmmode\sp\else\^{}\fi}\catcode`\%=\active\def%{\%}$\mathdefault{\ensuremath{-}1.75}$}}%
\end{pgfscope}%
\begin{pgfscope}%
\pgfsetbuttcap%
\pgfsetroundjoin%
\definecolor{currentfill}{rgb}{0.000000,0.000000,0.000000}%
\pgfsetfillcolor{currentfill}%
\pgfsetlinewidth{0.803000pt}%
\definecolor{currentstroke}{rgb}{0.000000,0.000000,0.000000}%
\pgfsetstrokecolor{currentstroke}%
\pgfsetdash{}{0pt}%
\pgfsys@defobject{currentmarker}{\pgfqpoint{0.000000in}{-0.048611in}}{\pgfqpoint{0.000000in}{0.000000in}}{%
\pgfpathmoveto{\pgfqpoint{0.000000in}{0.000000in}}%
\pgfpathlineto{\pgfqpoint{0.000000in}{-0.048611in}}%
\pgfusepath{stroke,fill}%
}%
\begin{pgfscope}%
\pgfsys@transformshift{1.315591in}{0.499074in}%
\pgfsys@useobject{currentmarker}{}%
\end{pgfscope}%
\end{pgfscope}%
\begin{pgfscope}%
\definecolor{textcolor}{rgb}{0.000000,0.000000,0.000000}%
\pgfsetstrokecolor{textcolor}%
\pgfsetfillcolor{textcolor}%
\pgftext[x=1.315591in,y=0.401852in,,top]{\color{textcolor}{\rmfamily\fontsize{9.000000}{10.800000}\selectfont\catcode`\^=\active\def^{\ifmmode\sp\else\^{}\fi}\catcode`\%=\active\def%{\%}$\mathdefault{\ensuremath{-}1.50}$}}%
\end{pgfscope}%
\begin{pgfscope}%
\pgfsetbuttcap%
\pgfsetroundjoin%
\definecolor{currentfill}{rgb}{0.000000,0.000000,0.000000}%
\pgfsetfillcolor{currentfill}%
\pgfsetlinewidth{0.803000pt}%
\definecolor{currentstroke}{rgb}{0.000000,0.000000,0.000000}%
\pgfsetstrokecolor{currentstroke}%
\pgfsetdash{}{0pt}%
\pgfsys@defobject{currentmarker}{\pgfqpoint{0.000000in}{-0.048611in}}{\pgfqpoint{0.000000in}{0.000000in}}{%
\pgfpathmoveto{\pgfqpoint{0.000000in}{0.000000in}}%
\pgfpathlineto{\pgfqpoint{0.000000in}{-0.048611in}}%
\pgfusepath{stroke,fill}%
}%
\begin{pgfscope}%
\pgfsys@transformshift{1.715169in}{0.499074in}%
\pgfsys@useobject{currentmarker}{}%
\end{pgfscope}%
\end{pgfscope}%
\begin{pgfscope}%
\definecolor{textcolor}{rgb}{0.000000,0.000000,0.000000}%
\pgfsetstrokecolor{textcolor}%
\pgfsetfillcolor{textcolor}%
\pgftext[x=1.715169in,y=0.401852in,,top]{\color{textcolor}{\rmfamily\fontsize{9.000000}{10.800000}\selectfont\catcode`\^=\active\def^{\ifmmode\sp\else\^{}\fi}\catcode`\%=\active\def%{\%}$\mathdefault{\ensuremath{-}1.25}$}}%
\end{pgfscope}%
\begin{pgfscope}%
\pgfsetbuttcap%
\pgfsetroundjoin%
\definecolor{currentfill}{rgb}{0.000000,0.000000,0.000000}%
\pgfsetfillcolor{currentfill}%
\pgfsetlinewidth{0.803000pt}%
\definecolor{currentstroke}{rgb}{0.000000,0.000000,0.000000}%
\pgfsetstrokecolor{currentstroke}%
\pgfsetdash{}{0pt}%
\pgfsys@defobject{currentmarker}{\pgfqpoint{0.000000in}{-0.048611in}}{\pgfqpoint{0.000000in}{0.000000in}}{%
\pgfpathmoveto{\pgfqpoint{0.000000in}{0.000000in}}%
\pgfpathlineto{\pgfqpoint{0.000000in}{-0.048611in}}%
\pgfusepath{stroke,fill}%
}%
\begin{pgfscope}%
\pgfsys@transformshift{2.114747in}{0.499074in}%
\pgfsys@useobject{currentmarker}{}%
\end{pgfscope}%
\end{pgfscope}%
\begin{pgfscope}%
\definecolor{textcolor}{rgb}{0.000000,0.000000,0.000000}%
\pgfsetstrokecolor{textcolor}%
\pgfsetfillcolor{textcolor}%
\pgftext[x=2.114747in,y=0.401852in,,top]{\color{textcolor}{\rmfamily\fontsize{9.000000}{10.800000}\selectfont\catcode`\^=\active\def^{\ifmmode\sp\else\^{}\fi}\catcode`\%=\active\def%{\%}$\mathdefault{\ensuremath{-}1.00}$}}%
\end{pgfscope}%
\begin{pgfscope}%
\pgfsetbuttcap%
\pgfsetroundjoin%
\definecolor{currentfill}{rgb}{0.000000,0.000000,0.000000}%
\pgfsetfillcolor{currentfill}%
\pgfsetlinewidth{0.803000pt}%
\definecolor{currentstroke}{rgb}{0.000000,0.000000,0.000000}%
\pgfsetstrokecolor{currentstroke}%
\pgfsetdash{}{0pt}%
\pgfsys@defobject{currentmarker}{\pgfqpoint{0.000000in}{-0.048611in}}{\pgfqpoint{0.000000in}{0.000000in}}{%
\pgfpathmoveto{\pgfqpoint{0.000000in}{0.000000in}}%
\pgfpathlineto{\pgfqpoint{0.000000in}{-0.048611in}}%
\pgfusepath{stroke,fill}%
}%
\begin{pgfscope}%
\pgfsys@transformshift{2.514325in}{0.499074in}%
\pgfsys@useobject{currentmarker}{}%
\end{pgfscope}%
\end{pgfscope}%
\begin{pgfscope}%
\definecolor{textcolor}{rgb}{0.000000,0.000000,0.000000}%
\pgfsetstrokecolor{textcolor}%
\pgfsetfillcolor{textcolor}%
\pgftext[x=2.514325in,y=0.401852in,,top]{\color{textcolor}{\rmfamily\fontsize{9.000000}{10.800000}\selectfont\catcode`\^=\active\def^{\ifmmode\sp\else\^{}\fi}\catcode`\%=\active\def%{\%}$\mathdefault{\ensuremath{-}0.75}$}}%
\end{pgfscope}%
\begin{pgfscope}%
\pgfsetbuttcap%
\pgfsetroundjoin%
\definecolor{currentfill}{rgb}{0.000000,0.000000,0.000000}%
\pgfsetfillcolor{currentfill}%
\pgfsetlinewidth{0.803000pt}%
\definecolor{currentstroke}{rgb}{0.000000,0.000000,0.000000}%
\pgfsetstrokecolor{currentstroke}%
\pgfsetdash{}{0pt}%
\pgfsys@defobject{currentmarker}{\pgfqpoint{0.000000in}{-0.048611in}}{\pgfqpoint{0.000000in}{0.000000in}}{%
\pgfpathmoveto{\pgfqpoint{0.000000in}{0.000000in}}%
\pgfpathlineto{\pgfqpoint{0.000000in}{-0.048611in}}%
\pgfusepath{stroke,fill}%
}%
\begin{pgfscope}%
\pgfsys@transformshift{2.913903in}{0.499074in}%
\pgfsys@useobject{currentmarker}{}%
\end{pgfscope}%
\end{pgfscope}%
\begin{pgfscope}%
\definecolor{textcolor}{rgb}{0.000000,0.000000,0.000000}%
\pgfsetstrokecolor{textcolor}%
\pgfsetfillcolor{textcolor}%
\pgftext[x=2.913903in,y=0.401852in,,top]{\color{textcolor}{\rmfamily\fontsize{9.000000}{10.800000}\selectfont\catcode`\^=\active\def^{\ifmmode\sp\else\^{}\fi}\catcode`\%=\active\def%{\%}$\mathdefault{\ensuremath{-}0.50}$}}%
\end{pgfscope}%
\begin{pgfscope}%
\pgfsetbuttcap%
\pgfsetroundjoin%
\definecolor{currentfill}{rgb}{0.000000,0.000000,0.000000}%
\pgfsetfillcolor{currentfill}%
\pgfsetlinewidth{0.803000pt}%
\definecolor{currentstroke}{rgb}{0.000000,0.000000,0.000000}%
\pgfsetstrokecolor{currentstroke}%
\pgfsetdash{}{0pt}%
\pgfsys@defobject{currentmarker}{\pgfqpoint{0.000000in}{-0.048611in}}{\pgfqpoint{0.000000in}{0.000000in}}{%
\pgfpathmoveto{\pgfqpoint{0.000000in}{0.000000in}}%
\pgfpathlineto{\pgfqpoint{0.000000in}{-0.048611in}}%
\pgfusepath{stroke,fill}%
}%
\begin{pgfscope}%
\pgfsys@transformshift{3.313482in}{0.499074in}%
\pgfsys@useobject{currentmarker}{}%
\end{pgfscope}%
\end{pgfscope}%
\begin{pgfscope}%
\definecolor{textcolor}{rgb}{0.000000,0.000000,0.000000}%
\pgfsetstrokecolor{textcolor}%
\pgfsetfillcolor{textcolor}%
\pgftext[x=3.313482in,y=0.401852in,,top]{\color{textcolor}{\rmfamily\fontsize{9.000000}{10.800000}\selectfont\catcode`\^=\active\def^{\ifmmode\sp\else\^{}\fi}\catcode`\%=\active\def%{\%}$\mathdefault{\ensuremath{-}0.25}$}}%
\end{pgfscope}%
\begin{pgfscope}%
\pgfsetbuttcap%
\pgfsetroundjoin%
\definecolor{currentfill}{rgb}{0.000000,0.000000,0.000000}%
\pgfsetfillcolor{currentfill}%
\pgfsetlinewidth{0.803000pt}%
\definecolor{currentstroke}{rgb}{0.000000,0.000000,0.000000}%
\pgfsetstrokecolor{currentstroke}%
\pgfsetdash{}{0pt}%
\pgfsys@defobject{currentmarker}{\pgfqpoint{0.000000in}{-0.048611in}}{\pgfqpoint{0.000000in}{0.000000in}}{%
\pgfpathmoveto{\pgfqpoint{0.000000in}{0.000000in}}%
\pgfpathlineto{\pgfqpoint{0.000000in}{-0.048611in}}%
\pgfusepath{stroke,fill}%
}%
\begin{pgfscope}%
\pgfsys@transformshift{3.713060in}{0.499074in}%
\pgfsys@useobject{currentmarker}{}%
\end{pgfscope}%
\end{pgfscope}%
\begin{pgfscope}%
\definecolor{textcolor}{rgb}{0.000000,0.000000,0.000000}%
\pgfsetstrokecolor{textcolor}%
\pgfsetfillcolor{textcolor}%
\pgftext[x=3.713060in,y=0.401852in,,top]{\color{textcolor}{\rmfamily\fontsize{9.000000}{10.800000}\selectfont\catcode`\^=\active\def^{\ifmmode\sp\else\^{}\fi}\catcode`\%=\active\def%{\%}$\mathdefault{0.00}$}}%
\end{pgfscope}%
\begin{pgfscope}%
\definecolor{textcolor}{rgb}{0.000000,0.000000,0.000000}%
\pgfsetstrokecolor{textcolor}%
\pgfsetfillcolor{textcolor}%
\pgftext[x=2.114747in,y=0.235185in,,top]{\color{textcolor}{\rmfamily\fontsize{11.000000}{13.200000}\selectfont\catcode`\^=\active\def^{\ifmmode\sp\else\^{}\fi}\catcode`\%=\active\def%{\%}$v_1$}}%
\end{pgfscope}%
\begin{pgfscope}%
\pgfsetbuttcap%
\pgfsetroundjoin%
\definecolor{currentfill}{rgb}{0.000000,0.000000,0.000000}%
\pgfsetfillcolor{currentfill}%
\pgfsetlinewidth{0.803000pt}%
\definecolor{currentstroke}{rgb}{0.000000,0.000000,0.000000}%
\pgfsetstrokecolor{currentstroke}%
\pgfsetdash{}{0pt}%
\pgfsys@defobject{currentmarker}{\pgfqpoint{-0.048611in}{0.000000in}}{\pgfqpoint{-0.000000in}{0.000000in}}{%
\pgfpathmoveto{\pgfqpoint{-0.000000in}{0.000000in}}%
\pgfpathlineto{\pgfqpoint{-0.048611in}{0.000000in}}%
\pgfusepath{stroke,fill}%
}%
\begin{pgfscope}%
\pgfsys@transformshift{0.516434in}{0.499074in}%
\pgfsys@useobject{currentmarker}{}%
\end{pgfscope}%
\end{pgfscope}%
\begin{pgfscope}%
\definecolor{textcolor}{rgb}{0.000000,0.000000,0.000000}%
\pgfsetstrokecolor{textcolor}%
\pgfsetfillcolor{textcolor}%
\pgftext[x=0.354976in, y=0.455671in, left, base]{\color{textcolor}{\rmfamily\fontsize{9.000000}{10.800000}\selectfont\catcode`\^=\active\def^{\ifmmode\sp\else\^{}\fi}\catcode`\%=\active\def%{\%}$\mathdefault{0}$}}%
\end{pgfscope}%
\begin{pgfscope}%
\pgfsetbuttcap%
\pgfsetroundjoin%
\definecolor{currentfill}{rgb}{0.000000,0.000000,0.000000}%
\pgfsetfillcolor{currentfill}%
\pgfsetlinewidth{0.803000pt}%
\definecolor{currentstroke}{rgb}{0.000000,0.000000,0.000000}%
\pgfsetstrokecolor{currentstroke}%
\pgfsetdash{}{0pt}%
\pgfsys@defobject{currentmarker}{\pgfqpoint{-0.048611in}{0.000000in}}{\pgfqpoint{-0.000000in}{0.000000in}}{%
\pgfpathmoveto{\pgfqpoint{-0.000000in}{0.000000in}}%
\pgfpathlineto{\pgfqpoint{-0.048611in}{0.000000in}}%
\pgfusepath{stroke,fill}%
}%
\begin{pgfscope}%
\pgfsys@transformshift{0.516434in}{1.138399in}%
\pgfsys@useobject{currentmarker}{}%
\end{pgfscope}%
\end{pgfscope}%
\begin{pgfscope}%
\definecolor{textcolor}{rgb}{0.000000,0.000000,0.000000}%
\pgfsetstrokecolor{textcolor}%
\pgfsetfillcolor{textcolor}%
\pgftext[x=0.290741in, y=1.094996in, left, base]{\color{textcolor}{\rmfamily\fontsize{9.000000}{10.800000}\selectfont\catcode`\^=\active\def^{\ifmmode\sp\else\^{}\fi}\catcode`\%=\active\def%{\%}$\mathdefault{10}$}}%
\end{pgfscope}%
\begin{pgfscope}%
\pgfsetbuttcap%
\pgfsetroundjoin%
\definecolor{currentfill}{rgb}{0.000000,0.000000,0.000000}%
\pgfsetfillcolor{currentfill}%
\pgfsetlinewidth{0.803000pt}%
\definecolor{currentstroke}{rgb}{0.000000,0.000000,0.000000}%
\pgfsetstrokecolor{currentstroke}%
\pgfsetdash{}{0pt}%
\pgfsys@defobject{currentmarker}{\pgfqpoint{-0.048611in}{0.000000in}}{\pgfqpoint{-0.000000in}{0.000000in}}{%
\pgfpathmoveto{\pgfqpoint{-0.000000in}{0.000000in}}%
\pgfpathlineto{\pgfqpoint{-0.048611in}{0.000000in}}%
\pgfusepath{stroke,fill}%
}%
\begin{pgfscope}%
\pgfsys@transformshift{0.516434in}{1.777724in}%
\pgfsys@useobject{currentmarker}{}%
\end{pgfscope}%
\end{pgfscope}%
\begin{pgfscope}%
\definecolor{textcolor}{rgb}{0.000000,0.000000,0.000000}%
\pgfsetstrokecolor{textcolor}%
\pgfsetfillcolor{textcolor}%
\pgftext[x=0.290741in, y=1.734321in, left, base]{\color{textcolor}{\rmfamily\fontsize{9.000000}{10.800000}\selectfont\catcode`\^=\active\def^{\ifmmode\sp\else\^{}\fi}\catcode`\%=\active\def%{\%}$\mathdefault{20}$}}%
\end{pgfscope}%
\begin{pgfscope}%
\pgfsetbuttcap%
\pgfsetroundjoin%
\definecolor{currentfill}{rgb}{0.000000,0.000000,0.000000}%
\pgfsetfillcolor{currentfill}%
\pgfsetlinewidth{0.803000pt}%
\definecolor{currentstroke}{rgb}{0.000000,0.000000,0.000000}%
\pgfsetstrokecolor{currentstroke}%
\pgfsetdash{}{0pt}%
\pgfsys@defobject{currentmarker}{\pgfqpoint{-0.048611in}{0.000000in}}{\pgfqpoint{-0.000000in}{0.000000in}}{%
\pgfpathmoveto{\pgfqpoint{-0.000000in}{0.000000in}}%
\pgfpathlineto{\pgfqpoint{-0.048611in}{0.000000in}}%
\pgfusepath{stroke,fill}%
}%
\begin{pgfscope}%
\pgfsys@transformshift{0.516434in}{2.417049in}%
\pgfsys@useobject{currentmarker}{}%
\end{pgfscope}%
\end{pgfscope}%
\begin{pgfscope}%
\definecolor{textcolor}{rgb}{0.000000,0.000000,0.000000}%
\pgfsetstrokecolor{textcolor}%
\pgfsetfillcolor{textcolor}%
\pgftext[x=0.290741in, y=2.373646in, left, base]{\color{textcolor}{\rmfamily\fontsize{9.000000}{10.800000}\selectfont\catcode`\^=\active\def^{\ifmmode\sp\else\^{}\fi}\catcode`\%=\active\def%{\%}$\mathdefault{30}$}}%
\end{pgfscope}%
\begin{pgfscope}%
\pgfsetbuttcap%
\pgfsetroundjoin%
\definecolor{currentfill}{rgb}{0.000000,0.000000,0.000000}%
\pgfsetfillcolor{currentfill}%
\pgfsetlinewidth{0.803000pt}%
\definecolor{currentstroke}{rgb}{0.000000,0.000000,0.000000}%
\pgfsetstrokecolor{currentstroke}%
\pgfsetdash{}{0pt}%
\pgfsys@defobject{currentmarker}{\pgfqpoint{-0.048611in}{0.000000in}}{\pgfqpoint{-0.000000in}{0.000000in}}{%
\pgfpathmoveto{\pgfqpoint{-0.000000in}{0.000000in}}%
\pgfpathlineto{\pgfqpoint{-0.048611in}{0.000000in}}%
\pgfusepath{stroke,fill}%
}%
\begin{pgfscope}%
\pgfsys@transformshift{0.516434in}{3.056374in}%
\pgfsys@useobject{currentmarker}{}%
\end{pgfscope}%
\end{pgfscope}%
\begin{pgfscope}%
\definecolor{textcolor}{rgb}{0.000000,0.000000,0.000000}%
\pgfsetstrokecolor{textcolor}%
\pgfsetfillcolor{textcolor}%
\pgftext[x=0.290741in, y=3.012972in, left, base]{\color{textcolor}{\rmfamily\fontsize{9.000000}{10.800000}\selectfont\catcode`\^=\active\def^{\ifmmode\sp\else\^{}\fi}\catcode`\%=\active\def%{\%}$\mathdefault{40}$}}%
\end{pgfscope}%
\begin{pgfscope}%
\pgfsetbuttcap%
\pgfsetroundjoin%
\definecolor{currentfill}{rgb}{0.000000,0.000000,0.000000}%
\pgfsetfillcolor{currentfill}%
\pgfsetlinewidth{0.803000pt}%
\definecolor{currentstroke}{rgb}{0.000000,0.000000,0.000000}%
\pgfsetstrokecolor{currentstroke}%
\pgfsetdash{}{0pt}%
\pgfsys@defobject{currentmarker}{\pgfqpoint{-0.048611in}{0.000000in}}{\pgfqpoint{-0.000000in}{0.000000in}}{%
\pgfpathmoveto{\pgfqpoint{-0.000000in}{0.000000in}}%
\pgfpathlineto{\pgfqpoint{-0.048611in}{0.000000in}}%
\pgfusepath{stroke,fill}%
}%
\begin{pgfscope}%
\pgfsys@transformshift{0.516434in}{3.695699in}%
\pgfsys@useobject{currentmarker}{}%
\end{pgfscope}%
\end{pgfscope}%
\begin{pgfscope}%
\definecolor{textcolor}{rgb}{0.000000,0.000000,0.000000}%
\pgfsetstrokecolor{textcolor}%
\pgfsetfillcolor{textcolor}%
\pgftext[x=0.290741in, y=3.652297in, left, base]{\color{textcolor}{\rmfamily\fontsize{9.000000}{10.800000}\selectfont\catcode`\^=\active\def^{\ifmmode\sp\else\^{}\fi}\catcode`\%=\active\def%{\%}$\mathdefault{50}$}}%
\end{pgfscope}%
\begin{pgfscope}%
\definecolor{textcolor}{rgb}{0.000000,0.000000,0.000000}%
\pgfsetstrokecolor{textcolor}%
\pgfsetfillcolor{textcolor}%
\pgftext[x=0.235185in,y=2.097387in,,bottom,rotate=90.000000]{\color{textcolor}{\rmfamily\fontsize{11.000000}{13.200000}\selectfont\catcode`\^=\active\def^{\ifmmode\sp\else\^{}\fi}\catcode`\%=\active\def%{\%}$v_2$}}%
\end{pgfscope}%
\begin{pgfscope}%
\pgfsetrectcap%
\pgfsetmiterjoin%
\pgfsetlinewidth{0.803000pt}%
\definecolor{currentstroke}{rgb}{0.000000,0.000000,0.000000}%
\pgfsetstrokecolor{currentstroke}%
\pgfsetdash{}{0pt}%
\pgfpathmoveto{\pgfqpoint{0.516434in}{0.499074in}}%
\pgfpathlineto{\pgfqpoint{0.516434in}{3.695699in}}%
\pgfusepath{stroke}%
\end{pgfscope}%
\begin{pgfscope}%
\pgfsetrectcap%
\pgfsetmiterjoin%
\pgfsetlinewidth{0.803000pt}%
\definecolor{currentstroke}{rgb}{0.000000,0.000000,0.000000}%
\pgfsetstrokecolor{currentstroke}%
\pgfsetdash{}{0pt}%
\pgfpathmoveto{\pgfqpoint{3.713060in}{0.499074in}}%
\pgfpathlineto{\pgfqpoint{3.713060in}{3.695699in}}%
\pgfusepath{stroke}%
\end{pgfscope}%
\begin{pgfscope}%
\pgfsetrectcap%
\pgfsetmiterjoin%
\pgfsetlinewidth{0.803000pt}%
\definecolor{currentstroke}{rgb}{0.000000,0.000000,0.000000}%
\pgfsetstrokecolor{currentstroke}%
\pgfsetdash{}{0pt}%
\pgfpathmoveto{\pgfqpoint{0.516434in}{0.499074in}}%
\pgfpathlineto{\pgfqpoint{3.713060in}{0.499074in}}%
\pgfusepath{stroke}%
\end{pgfscope}%
\begin{pgfscope}%
\pgfsetrectcap%
\pgfsetmiterjoin%
\pgfsetlinewidth{0.803000pt}%
\definecolor{currentstroke}{rgb}{0.000000,0.000000,0.000000}%
\pgfsetstrokecolor{currentstroke}%
\pgfsetdash{}{0pt}%
\pgfpathmoveto{\pgfqpoint{0.516434in}{3.695699in}}%
\pgfpathlineto{\pgfqpoint{3.713060in}{3.695699in}}%
\pgfusepath{stroke}%
\end{pgfscope}%
\begin{pgfscope}%
\pgfsetbuttcap%
\pgfsetmiterjoin%
\definecolor{currentfill}{rgb}{1.000000,1.000000,1.000000}%
\pgfsetfillcolor{currentfill}%
\pgfsetlinewidth{0.000000pt}%
\definecolor{currentstroke}{rgb}{0.000000,0.000000,0.000000}%
\pgfsetstrokecolor{currentstroke}%
\pgfsetstrokeopacity{0.000000}%
\pgfsetdash{}{0pt}%
\pgfpathmoveto{\pgfqpoint{3.891959in}{0.499074in}}%
\pgfpathlineto{\pgfqpoint{4.051791in}{0.499074in}}%
\pgfpathlineto{\pgfqpoint{4.051791in}{3.695699in}}%
\pgfpathlineto{\pgfqpoint{3.891959in}{3.695699in}}%
\pgfpathlineto{\pgfqpoint{3.891959in}{0.499074in}}%
\pgfpathclose%
\pgfusepath{fill}%
\end{pgfscope}%
\begin{pgfscope}%
\pgfsys@transformshift{3.890000in}{0.509102in}%
\pgftext[left,bottom]{\includegraphics[interpolate=true,width=0.160000in,height=3.200000in]{figures/mccormick/mccormick-bounds-1-upper-img1.png}}%
\end{pgfscope}%
\begin{pgfscope}%
\pgfsetbuttcap%
\pgfsetroundjoin%
\definecolor{currentfill}{rgb}{0.000000,0.000000,0.000000}%
\pgfsetfillcolor{currentfill}%
\pgfsetlinewidth{0.803000pt}%
\definecolor{currentstroke}{rgb}{0.000000,0.000000,0.000000}%
\pgfsetstrokecolor{currentstroke}%
\pgfsetdash{}{0pt}%
\pgfsys@defobject{currentmarker}{\pgfqpoint{0.000000in}{0.000000in}}{\pgfqpoint{0.048611in}{0.000000in}}{%
\pgfpathmoveto{\pgfqpoint{0.000000in}{0.000000in}}%
\pgfpathlineto{\pgfqpoint{0.048611in}{0.000000in}}%
\pgfusepath{stroke,fill}%
}%
\begin{pgfscope}%
\pgfsys@transformshift{4.051791in}{0.499074in}%
\pgfsys@useobject{currentmarker}{}%
\end{pgfscope}%
\end{pgfscope}%
\begin{pgfscope}%
\definecolor{textcolor}{rgb}{0.000000,0.000000,0.000000}%
\pgfsetstrokecolor{textcolor}%
\pgfsetfillcolor{textcolor}%
\pgftext[x=4.149013in, y=0.455671in, left, base]{\color{textcolor}{\rmfamily\fontsize{9.000000}{10.800000}\selectfont\catcode`\^=\active\def^{\ifmmode\sp\else\^{}\fi}\catcode`\%=\active\def%{\%}$\mathdefault{0}$}}%
\end{pgfscope}%
\begin{pgfscope}%
\pgfsetbuttcap%
\pgfsetroundjoin%
\definecolor{currentfill}{rgb}{0.000000,0.000000,0.000000}%
\pgfsetfillcolor{currentfill}%
\pgfsetlinewidth{0.803000pt}%
\definecolor{currentstroke}{rgb}{0.000000,0.000000,0.000000}%
\pgfsetstrokecolor{currentstroke}%
\pgfsetdash{}{0pt}%
\pgfsys@defobject{currentmarker}{\pgfqpoint{0.000000in}{0.000000in}}{\pgfqpoint{0.048611in}{0.000000in}}{%
\pgfpathmoveto{\pgfqpoint{0.000000in}{0.000000in}}%
\pgfpathlineto{\pgfqpoint{0.048611in}{0.000000in}}%
\pgfusepath{stroke,fill}%
}%
\begin{pgfscope}%
\pgfsys@transformshift{4.051791in}{1.138399in}%
\pgfsys@useobject{currentmarker}{}%
\end{pgfscope}%
\end{pgfscope}%
\begin{pgfscope}%
\definecolor{textcolor}{rgb}{0.000000,0.000000,0.000000}%
\pgfsetstrokecolor{textcolor}%
\pgfsetfillcolor{textcolor}%
\pgftext[x=4.149013in, y=1.094996in, left, base]{\color{textcolor}{\rmfamily\fontsize{9.000000}{10.800000}\selectfont\catcode`\^=\active\def^{\ifmmode\sp\else\^{}\fi}\catcode`\%=\active\def%{\%}$\mathdefault{10}$}}%
\end{pgfscope}%
\begin{pgfscope}%
\pgfsetbuttcap%
\pgfsetroundjoin%
\definecolor{currentfill}{rgb}{0.000000,0.000000,0.000000}%
\pgfsetfillcolor{currentfill}%
\pgfsetlinewidth{0.803000pt}%
\definecolor{currentstroke}{rgb}{0.000000,0.000000,0.000000}%
\pgfsetstrokecolor{currentstroke}%
\pgfsetdash{}{0pt}%
\pgfsys@defobject{currentmarker}{\pgfqpoint{0.000000in}{0.000000in}}{\pgfqpoint{0.048611in}{0.000000in}}{%
\pgfpathmoveto{\pgfqpoint{0.000000in}{0.000000in}}%
\pgfpathlineto{\pgfqpoint{0.048611in}{0.000000in}}%
\pgfusepath{stroke,fill}%
}%
\begin{pgfscope}%
\pgfsys@transformshift{4.051791in}{1.777724in}%
\pgfsys@useobject{currentmarker}{}%
\end{pgfscope}%
\end{pgfscope}%
\begin{pgfscope}%
\definecolor{textcolor}{rgb}{0.000000,0.000000,0.000000}%
\pgfsetstrokecolor{textcolor}%
\pgfsetfillcolor{textcolor}%
\pgftext[x=4.149013in, y=1.734321in, left, base]{\color{textcolor}{\rmfamily\fontsize{9.000000}{10.800000}\selectfont\catcode`\^=\active\def^{\ifmmode\sp\else\^{}\fi}\catcode`\%=\active\def%{\%}$\mathdefault{20}$}}%
\end{pgfscope}%
\begin{pgfscope}%
\pgfsetbuttcap%
\pgfsetroundjoin%
\definecolor{currentfill}{rgb}{0.000000,0.000000,0.000000}%
\pgfsetfillcolor{currentfill}%
\pgfsetlinewidth{0.803000pt}%
\definecolor{currentstroke}{rgb}{0.000000,0.000000,0.000000}%
\pgfsetstrokecolor{currentstroke}%
\pgfsetdash{}{0pt}%
\pgfsys@defobject{currentmarker}{\pgfqpoint{0.000000in}{0.000000in}}{\pgfqpoint{0.048611in}{0.000000in}}{%
\pgfpathmoveto{\pgfqpoint{0.000000in}{0.000000in}}%
\pgfpathlineto{\pgfqpoint{0.048611in}{0.000000in}}%
\pgfusepath{stroke,fill}%
}%
\begin{pgfscope}%
\pgfsys@transformshift{4.051791in}{2.417049in}%
\pgfsys@useobject{currentmarker}{}%
\end{pgfscope}%
\end{pgfscope}%
\begin{pgfscope}%
\definecolor{textcolor}{rgb}{0.000000,0.000000,0.000000}%
\pgfsetstrokecolor{textcolor}%
\pgfsetfillcolor{textcolor}%
\pgftext[x=4.149013in, y=2.373646in, left, base]{\color{textcolor}{\rmfamily\fontsize{9.000000}{10.800000}\selectfont\catcode`\^=\active\def^{\ifmmode\sp\else\^{}\fi}\catcode`\%=\active\def%{\%}$\mathdefault{30}$}}%
\end{pgfscope}%
\begin{pgfscope}%
\pgfsetbuttcap%
\pgfsetroundjoin%
\definecolor{currentfill}{rgb}{0.000000,0.000000,0.000000}%
\pgfsetfillcolor{currentfill}%
\pgfsetlinewidth{0.803000pt}%
\definecolor{currentstroke}{rgb}{0.000000,0.000000,0.000000}%
\pgfsetstrokecolor{currentstroke}%
\pgfsetdash{}{0pt}%
\pgfsys@defobject{currentmarker}{\pgfqpoint{0.000000in}{0.000000in}}{\pgfqpoint{0.048611in}{0.000000in}}{%
\pgfpathmoveto{\pgfqpoint{0.000000in}{0.000000in}}%
\pgfpathlineto{\pgfqpoint{0.048611in}{0.000000in}}%
\pgfusepath{stroke,fill}%
}%
\begin{pgfscope}%
\pgfsys@transformshift{4.051791in}{3.056374in}%
\pgfsys@useobject{currentmarker}{}%
\end{pgfscope}%
\end{pgfscope}%
\begin{pgfscope}%
\definecolor{textcolor}{rgb}{0.000000,0.000000,0.000000}%
\pgfsetstrokecolor{textcolor}%
\pgfsetfillcolor{textcolor}%
\pgftext[x=4.149013in, y=3.012972in, left, base]{\color{textcolor}{\rmfamily\fontsize{9.000000}{10.800000}\selectfont\catcode`\^=\active\def^{\ifmmode\sp\else\^{}\fi}\catcode`\%=\active\def%{\%}$\mathdefault{40}$}}%
\end{pgfscope}%
\begin{pgfscope}%
\pgfsetbuttcap%
\pgfsetroundjoin%
\definecolor{currentfill}{rgb}{0.000000,0.000000,0.000000}%
\pgfsetfillcolor{currentfill}%
\pgfsetlinewidth{0.803000pt}%
\definecolor{currentstroke}{rgb}{0.000000,0.000000,0.000000}%
\pgfsetstrokecolor{currentstroke}%
\pgfsetdash{}{0pt}%
\pgfsys@defobject{currentmarker}{\pgfqpoint{0.000000in}{0.000000in}}{\pgfqpoint{0.048611in}{0.000000in}}{%
\pgfpathmoveto{\pgfqpoint{0.000000in}{0.000000in}}%
\pgfpathlineto{\pgfqpoint{0.048611in}{0.000000in}}%
\pgfusepath{stroke,fill}%
}%
\begin{pgfscope}%
\pgfsys@transformshift{4.051791in}{3.695699in}%
\pgfsys@useobject{currentmarker}{}%
\end{pgfscope}%
\end{pgfscope}%
\begin{pgfscope}%
\definecolor{textcolor}{rgb}{0.000000,0.000000,0.000000}%
\pgfsetstrokecolor{textcolor}%
\pgfsetfillcolor{textcolor}%
\pgftext[x=4.149013in, y=3.652297in, left, base]{\color{textcolor}{\rmfamily\fontsize{9.000000}{10.800000}\selectfont\catcode`\^=\active\def^{\ifmmode\sp\else\^{}\fi}\catcode`\%=\active\def%{\%}$\mathdefault{50}$}}%
\end{pgfscope}%
\begin{pgfscope}%
\pgfsetrectcap%
\pgfsetmiterjoin%
\pgfsetlinewidth{0.803000pt}%
\definecolor{currentstroke}{rgb}{0.000000,0.000000,0.000000}%
\pgfsetstrokecolor{currentstroke}%
\pgfsetdash{}{0pt}%
\pgfpathmoveto{\pgfqpoint{3.891959in}{0.499074in}}%
\pgfpathlineto{\pgfqpoint{3.971875in}{0.499074in}}%
\pgfpathlineto{\pgfqpoint{4.051791in}{0.499074in}}%
\pgfpathlineto{\pgfqpoint{4.051791in}{3.695699in}}%
\pgfpathlineto{\pgfqpoint{3.971875in}{3.695699in}}%
\pgfpathlineto{\pgfqpoint{3.891959in}{3.695699in}}%
\pgfpathlineto{\pgfqpoint{3.891959in}{0.499074in}}%
\pgfpathclose%
\pgfusepath{stroke}%
\end{pgfscope}%
\end{pgfpicture}%
\makeatother%
\endgroup%
}
		\caption{Difference to the upper bound}
		\label{fig:mccormick_1_upper}
	\end{subfigure}
	\hfill
	\begin{subfigure}[b]{0.45\textwidth}
		\centering
		\resizebox{\textwidth}{!}{\input{figures/mccormick/mccormick-bounds-1-lower.pgf}}
		\caption{Difference to the lower bound}
		\label{fig:mccormick_1_lower}
	\end{subfigure}
	\caption{McCormick relaxation bounds for the bilinear term \( w = xy \) with stricter bounds on \( x \).}
	\label{fig:mccormick_bounds_1}
\end{figure}

Figures \ref{fig:mccormick_1_upper} and \ref{fig:mccormick_1_lower} present the results when \( x \) is more tightly bounded, specifically \( -2 \leq
x \leq 0 \).
One can observe that the maximum deviation is considerably reduced compared to the previous scenario, indicating that tighter bounds yield a more
accurate relaxation.

To illustrate the application of these relaxations in practice, consider a path-planning scenario with \( v_{min} = 1 \), \( v_{max} = 4 \), and \(
v_{start} = 1 \).
In this scenario, the bilinear term \( v\xi \), which appears in the equation of motion for \(\dot{n} = v \sin{\xi} \approx v\xi\), is approximated
using McCormick relaxations.

\begin{figure}[h]
	\centering
	\resizebox{1\textwidth}{!}{%% Creator: Matplotlib, PGF backend
%%
%% To include the figure in your LaTeX document, write
%%   \input{<filename>.pgf}
%%
%% Make sure the required packages are loaded in your preamble
%%   \usepackage{pgf}
%%
%% Also ensure that all the required font packages are loaded; for instance,
%% the lmodern package is sometimes necessary when using math font.
%%   \usepackage{lmodern}
%%
%% Figures using additional raster images can only be included by \input if
%% they are in the same directory as the main LaTeX file. For loading figures
%% from other directories you can use the `import` package
%%   \usepackage{import}
%%
%% and then include the figures with
%%   \import{<path to file>}{<filename>.pgf}
%%
%% Matplotlib used the following preamble
%%   \def\mathdefault#1{#1}
%%   \everymath=\expandafter{\the\everymath\displaystyle}
%%   
%%   \ifdefined\pdftexversion\else  % non-pdftex case.
%%     \usepackage{fontspec}
%%   \fi
%%   \makeatletter\@ifpackageloaded{underscore}{}{\usepackage[strings]{underscore}}\makeatother
%%
\begingroup%
\makeatletter%
\begin{pgfpicture}%
\pgfpathrectangle{\pgfpointorigin}{\pgfqpoint{9.855329in}{1.757149in}}%
\pgfusepath{use as bounding box, clip}%
\begin{pgfscope}%
\pgfsetbuttcap%
\pgfsetmiterjoin%
\definecolor{currentfill}{rgb}{1.000000,1.000000,1.000000}%
\pgfsetfillcolor{currentfill}%
\pgfsetlinewidth{0.000000pt}%
\definecolor{currentstroke}{rgb}{1.000000,1.000000,1.000000}%
\pgfsetstrokecolor{currentstroke}%
\pgfsetdash{}{0pt}%
\pgfpathmoveto{\pgfqpoint{0.000000in}{0.000000in}}%
\pgfpathlineto{\pgfqpoint{9.855329in}{0.000000in}}%
\pgfpathlineto{\pgfqpoint{9.855329in}{1.757149in}}%
\pgfpathlineto{\pgfqpoint{0.000000in}{1.757149in}}%
\pgfpathlineto{\pgfqpoint{0.000000in}{0.000000in}}%
\pgfpathclose%
\pgfusepath{fill}%
\end{pgfscope}%
\begin{pgfscope}%
\pgfsetbuttcap%
\pgfsetmiterjoin%
\definecolor{currentfill}{rgb}{1.000000,1.000000,1.000000}%
\pgfsetfillcolor{currentfill}%
\pgfsetlinewidth{0.000000pt}%
\definecolor{currentstroke}{rgb}{0.000000,0.000000,0.000000}%
\pgfsetstrokecolor{currentstroke}%
\pgfsetstrokeopacity{0.000000}%
\pgfsetdash{}{0pt}%
\pgfpathmoveto{\pgfqpoint{0.452199in}{0.515972in}}%
\pgfpathlineto{\pgfqpoint{9.755329in}{0.515972in}}%
\pgfpathlineto{\pgfqpoint{9.755329in}{1.657149in}}%
\pgfpathlineto{\pgfqpoint{0.452199in}{1.657149in}}%
\pgfpathlineto{\pgfqpoint{0.452199in}{0.515972in}}%
\pgfpathclose%
\pgfusepath{fill}%
\end{pgfscope}%
\begin{pgfscope}%
\pgfsetbuttcap%
\pgfsetroundjoin%
\definecolor{currentfill}{rgb}{0.000000,0.000000,0.000000}%
\pgfsetfillcolor{currentfill}%
\pgfsetlinewidth{0.803000pt}%
\definecolor{currentstroke}{rgb}{0.000000,0.000000,0.000000}%
\pgfsetstrokecolor{currentstroke}%
\pgfsetdash{}{0pt}%
\pgfsys@defobject{currentmarker}{\pgfqpoint{0.000000in}{-0.048611in}}{\pgfqpoint{0.000000in}{0.000000in}}{%
\pgfpathmoveto{\pgfqpoint{0.000000in}{0.000000in}}%
\pgfpathlineto{\pgfqpoint{0.000000in}{-0.048611in}}%
\pgfusepath{stroke,fill}%
}%
\begin{pgfscope}%
\pgfsys@transformshift{0.875068in}{0.515972in}%
\pgfsys@useobject{currentmarker}{}%
\end{pgfscope}%
\end{pgfscope}%
\begin{pgfscope}%
\definecolor{textcolor}{rgb}{0.000000,0.000000,0.000000}%
\pgfsetstrokecolor{textcolor}%
\pgfsetfillcolor{textcolor}%
\pgftext[x=0.875068in,y=0.418750in,,top]{\color{textcolor}{\rmfamily\fontsize{9.000000}{10.800000}\selectfont\catcode`\^=\active\def^{\ifmmode\sp\else\^{}\fi}\catcode`\%=\active\def%{\%}$\mathdefault{0.0}$}}%
\end{pgfscope}%
\begin{pgfscope}%
\pgfsetbuttcap%
\pgfsetroundjoin%
\definecolor{currentfill}{rgb}{0.000000,0.000000,0.000000}%
\pgfsetfillcolor{currentfill}%
\pgfsetlinewidth{0.803000pt}%
\definecolor{currentstroke}{rgb}{0.000000,0.000000,0.000000}%
\pgfsetstrokecolor{currentstroke}%
\pgfsetdash{}{0pt}%
\pgfsys@defobject{currentmarker}{\pgfqpoint{0.000000in}{-0.048611in}}{\pgfqpoint{0.000000in}{0.000000in}}{%
\pgfpathmoveto{\pgfqpoint{0.000000in}{0.000000in}}%
\pgfpathlineto{\pgfqpoint{0.000000in}{-0.048611in}}%
\pgfusepath{stroke,fill}%
}%
\begin{pgfscope}%
\pgfsys@transformshift{1.932242in}{0.515972in}%
\pgfsys@useobject{currentmarker}{}%
\end{pgfscope}%
\end{pgfscope}%
\begin{pgfscope}%
\definecolor{textcolor}{rgb}{0.000000,0.000000,0.000000}%
\pgfsetstrokecolor{textcolor}%
\pgfsetfillcolor{textcolor}%
\pgftext[x=1.932242in,y=0.418750in,,top]{\color{textcolor}{\rmfamily\fontsize{9.000000}{10.800000}\selectfont\catcode`\^=\active\def^{\ifmmode\sp\else\^{}\fi}\catcode`\%=\active\def%{\%}$\mathdefault{0.5}$}}%
\end{pgfscope}%
\begin{pgfscope}%
\pgfsetbuttcap%
\pgfsetroundjoin%
\definecolor{currentfill}{rgb}{0.000000,0.000000,0.000000}%
\pgfsetfillcolor{currentfill}%
\pgfsetlinewidth{0.803000pt}%
\definecolor{currentstroke}{rgb}{0.000000,0.000000,0.000000}%
\pgfsetstrokecolor{currentstroke}%
\pgfsetdash{}{0pt}%
\pgfsys@defobject{currentmarker}{\pgfqpoint{0.000000in}{-0.048611in}}{\pgfqpoint{0.000000in}{0.000000in}}{%
\pgfpathmoveto{\pgfqpoint{0.000000in}{0.000000in}}%
\pgfpathlineto{\pgfqpoint{0.000000in}{-0.048611in}}%
\pgfusepath{stroke,fill}%
}%
\begin{pgfscope}%
\pgfsys@transformshift{2.989416in}{0.515972in}%
\pgfsys@useobject{currentmarker}{}%
\end{pgfscope}%
\end{pgfscope}%
\begin{pgfscope}%
\definecolor{textcolor}{rgb}{0.000000,0.000000,0.000000}%
\pgfsetstrokecolor{textcolor}%
\pgfsetfillcolor{textcolor}%
\pgftext[x=2.989416in,y=0.418750in,,top]{\color{textcolor}{\rmfamily\fontsize{9.000000}{10.800000}\selectfont\catcode`\^=\active\def^{\ifmmode\sp\else\^{}\fi}\catcode`\%=\active\def%{\%}$\mathdefault{1.0}$}}%
\end{pgfscope}%
\begin{pgfscope}%
\pgfsetbuttcap%
\pgfsetroundjoin%
\definecolor{currentfill}{rgb}{0.000000,0.000000,0.000000}%
\pgfsetfillcolor{currentfill}%
\pgfsetlinewidth{0.803000pt}%
\definecolor{currentstroke}{rgb}{0.000000,0.000000,0.000000}%
\pgfsetstrokecolor{currentstroke}%
\pgfsetdash{}{0pt}%
\pgfsys@defobject{currentmarker}{\pgfqpoint{0.000000in}{-0.048611in}}{\pgfqpoint{0.000000in}{0.000000in}}{%
\pgfpathmoveto{\pgfqpoint{0.000000in}{0.000000in}}%
\pgfpathlineto{\pgfqpoint{0.000000in}{-0.048611in}}%
\pgfusepath{stroke,fill}%
}%
\begin{pgfscope}%
\pgfsys@transformshift{4.046590in}{0.515972in}%
\pgfsys@useobject{currentmarker}{}%
\end{pgfscope}%
\end{pgfscope}%
\begin{pgfscope}%
\definecolor{textcolor}{rgb}{0.000000,0.000000,0.000000}%
\pgfsetstrokecolor{textcolor}%
\pgfsetfillcolor{textcolor}%
\pgftext[x=4.046590in,y=0.418750in,,top]{\color{textcolor}{\rmfamily\fontsize{9.000000}{10.800000}\selectfont\catcode`\^=\active\def^{\ifmmode\sp\else\^{}\fi}\catcode`\%=\active\def%{\%}$\mathdefault{1.5}$}}%
\end{pgfscope}%
\begin{pgfscope}%
\pgfsetbuttcap%
\pgfsetroundjoin%
\definecolor{currentfill}{rgb}{0.000000,0.000000,0.000000}%
\pgfsetfillcolor{currentfill}%
\pgfsetlinewidth{0.803000pt}%
\definecolor{currentstroke}{rgb}{0.000000,0.000000,0.000000}%
\pgfsetstrokecolor{currentstroke}%
\pgfsetdash{}{0pt}%
\pgfsys@defobject{currentmarker}{\pgfqpoint{0.000000in}{-0.048611in}}{\pgfqpoint{0.000000in}{0.000000in}}{%
\pgfpathmoveto{\pgfqpoint{0.000000in}{0.000000in}}%
\pgfpathlineto{\pgfqpoint{0.000000in}{-0.048611in}}%
\pgfusepath{stroke,fill}%
}%
\begin{pgfscope}%
\pgfsys@transformshift{5.103764in}{0.515972in}%
\pgfsys@useobject{currentmarker}{}%
\end{pgfscope}%
\end{pgfscope}%
\begin{pgfscope}%
\definecolor{textcolor}{rgb}{0.000000,0.000000,0.000000}%
\pgfsetstrokecolor{textcolor}%
\pgfsetfillcolor{textcolor}%
\pgftext[x=5.103764in,y=0.418750in,,top]{\color{textcolor}{\rmfamily\fontsize{9.000000}{10.800000}\selectfont\catcode`\^=\active\def^{\ifmmode\sp\else\^{}\fi}\catcode`\%=\active\def%{\%}$\mathdefault{2.0}$}}%
\end{pgfscope}%
\begin{pgfscope}%
\pgfsetbuttcap%
\pgfsetroundjoin%
\definecolor{currentfill}{rgb}{0.000000,0.000000,0.000000}%
\pgfsetfillcolor{currentfill}%
\pgfsetlinewidth{0.803000pt}%
\definecolor{currentstroke}{rgb}{0.000000,0.000000,0.000000}%
\pgfsetstrokecolor{currentstroke}%
\pgfsetdash{}{0pt}%
\pgfsys@defobject{currentmarker}{\pgfqpoint{0.000000in}{-0.048611in}}{\pgfqpoint{0.000000in}{0.000000in}}{%
\pgfpathmoveto{\pgfqpoint{0.000000in}{0.000000in}}%
\pgfpathlineto{\pgfqpoint{0.000000in}{-0.048611in}}%
\pgfusepath{stroke,fill}%
}%
\begin{pgfscope}%
\pgfsys@transformshift{6.160938in}{0.515972in}%
\pgfsys@useobject{currentmarker}{}%
\end{pgfscope}%
\end{pgfscope}%
\begin{pgfscope}%
\definecolor{textcolor}{rgb}{0.000000,0.000000,0.000000}%
\pgfsetstrokecolor{textcolor}%
\pgfsetfillcolor{textcolor}%
\pgftext[x=6.160938in,y=0.418750in,,top]{\color{textcolor}{\rmfamily\fontsize{9.000000}{10.800000}\selectfont\catcode`\^=\active\def^{\ifmmode\sp\else\^{}\fi}\catcode`\%=\active\def%{\%}$\mathdefault{2.5}$}}%
\end{pgfscope}%
\begin{pgfscope}%
\pgfsetbuttcap%
\pgfsetroundjoin%
\definecolor{currentfill}{rgb}{0.000000,0.000000,0.000000}%
\pgfsetfillcolor{currentfill}%
\pgfsetlinewidth{0.803000pt}%
\definecolor{currentstroke}{rgb}{0.000000,0.000000,0.000000}%
\pgfsetstrokecolor{currentstroke}%
\pgfsetdash{}{0pt}%
\pgfsys@defobject{currentmarker}{\pgfqpoint{0.000000in}{-0.048611in}}{\pgfqpoint{0.000000in}{0.000000in}}{%
\pgfpathmoveto{\pgfqpoint{0.000000in}{0.000000in}}%
\pgfpathlineto{\pgfqpoint{0.000000in}{-0.048611in}}%
\pgfusepath{stroke,fill}%
}%
\begin{pgfscope}%
\pgfsys@transformshift{7.218112in}{0.515972in}%
\pgfsys@useobject{currentmarker}{}%
\end{pgfscope}%
\end{pgfscope}%
\begin{pgfscope}%
\definecolor{textcolor}{rgb}{0.000000,0.000000,0.000000}%
\pgfsetstrokecolor{textcolor}%
\pgfsetfillcolor{textcolor}%
\pgftext[x=7.218112in,y=0.418750in,,top]{\color{textcolor}{\rmfamily\fontsize{9.000000}{10.800000}\selectfont\catcode`\^=\active\def^{\ifmmode\sp\else\^{}\fi}\catcode`\%=\active\def%{\%}$\mathdefault{3.0}$}}%
\end{pgfscope}%
\begin{pgfscope}%
\pgfsetbuttcap%
\pgfsetroundjoin%
\definecolor{currentfill}{rgb}{0.000000,0.000000,0.000000}%
\pgfsetfillcolor{currentfill}%
\pgfsetlinewidth{0.803000pt}%
\definecolor{currentstroke}{rgb}{0.000000,0.000000,0.000000}%
\pgfsetstrokecolor{currentstroke}%
\pgfsetdash{}{0pt}%
\pgfsys@defobject{currentmarker}{\pgfqpoint{0.000000in}{-0.048611in}}{\pgfqpoint{0.000000in}{0.000000in}}{%
\pgfpathmoveto{\pgfqpoint{0.000000in}{0.000000in}}%
\pgfpathlineto{\pgfqpoint{0.000000in}{-0.048611in}}%
\pgfusepath{stroke,fill}%
}%
\begin{pgfscope}%
\pgfsys@transformshift{8.275286in}{0.515972in}%
\pgfsys@useobject{currentmarker}{}%
\end{pgfscope}%
\end{pgfscope}%
\begin{pgfscope}%
\definecolor{textcolor}{rgb}{0.000000,0.000000,0.000000}%
\pgfsetstrokecolor{textcolor}%
\pgfsetfillcolor{textcolor}%
\pgftext[x=8.275286in,y=0.418750in,,top]{\color{textcolor}{\rmfamily\fontsize{9.000000}{10.800000}\selectfont\catcode`\^=\active\def^{\ifmmode\sp\else\^{}\fi}\catcode`\%=\active\def%{\%}$\mathdefault{3.5}$}}%
\end{pgfscope}%
\begin{pgfscope}%
\pgfsetbuttcap%
\pgfsetroundjoin%
\definecolor{currentfill}{rgb}{0.000000,0.000000,0.000000}%
\pgfsetfillcolor{currentfill}%
\pgfsetlinewidth{0.803000pt}%
\definecolor{currentstroke}{rgb}{0.000000,0.000000,0.000000}%
\pgfsetstrokecolor{currentstroke}%
\pgfsetdash{}{0pt}%
\pgfsys@defobject{currentmarker}{\pgfqpoint{0.000000in}{-0.048611in}}{\pgfqpoint{0.000000in}{0.000000in}}{%
\pgfpathmoveto{\pgfqpoint{0.000000in}{0.000000in}}%
\pgfpathlineto{\pgfqpoint{0.000000in}{-0.048611in}}%
\pgfusepath{stroke,fill}%
}%
\begin{pgfscope}%
\pgfsys@transformshift{9.332460in}{0.515972in}%
\pgfsys@useobject{currentmarker}{}%
\end{pgfscope}%
\end{pgfscope}%
\begin{pgfscope}%
\definecolor{textcolor}{rgb}{0.000000,0.000000,0.000000}%
\pgfsetstrokecolor{textcolor}%
\pgfsetfillcolor{textcolor}%
\pgftext[x=9.332460in,y=0.418750in,,top]{\color{textcolor}{\rmfamily\fontsize{9.000000}{10.800000}\selectfont\catcode`\^=\active\def^{\ifmmode\sp\else\^{}\fi}\catcode`\%=\active\def%{\%}$\mathdefault{4.0}$}}%
\end{pgfscope}%
\begin{pgfscope}%
\definecolor{textcolor}{rgb}{0.000000,0.000000,0.000000}%
\pgfsetstrokecolor{textcolor}%
\pgfsetfillcolor{textcolor}%
\pgftext[x=5.103764in,y=0.252083in,,top]{\color{textcolor}{\rmfamily\fontsize{11.000000}{13.200000}\selectfont\catcode`\^=\active\def^{\ifmmode\sp\else\^{}\fi}\catcode`\%=\active\def%{\%}Time [s]}}%
\end{pgfscope}%
\begin{pgfscope}%
\pgfsetbuttcap%
\pgfsetroundjoin%
\definecolor{currentfill}{rgb}{0.000000,0.000000,0.000000}%
\pgfsetfillcolor{currentfill}%
\pgfsetlinewidth{0.803000pt}%
\definecolor{currentstroke}{rgb}{0.000000,0.000000,0.000000}%
\pgfsetstrokecolor{currentstroke}%
\pgfsetdash{}{0pt}%
\pgfsys@defobject{currentmarker}{\pgfqpoint{-0.048611in}{0.000000in}}{\pgfqpoint{-0.000000in}{0.000000in}}{%
\pgfpathmoveto{\pgfqpoint{-0.000000in}{0.000000in}}%
\pgfpathlineto{\pgfqpoint{-0.048611in}{0.000000in}}%
\pgfusepath{stroke,fill}%
}%
\begin{pgfscope}%
\pgfsys@transformshift{0.452199in}{0.567844in}%
\pgfsys@useobject{currentmarker}{}%
\end{pgfscope}%
\end{pgfscope}%
\begin{pgfscope}%
\definecolor{textcolor}{rgb}{0.000000,0.000000,0.000000}%
\pgfsetstrokecolor{textcolor}%
\pgfsetfillcolor{textcolor}%
\pgftext[x=0.290741in, y=0.524441in, left, base]{\color{textcolor}{\rmfamily\fontsize{9.000000}{10.800000}\selectfont\catcode`\^=\active\def^{\ifmmode\sp\else\^{}\fi}\catcode`\%=\active\def%{\%}$\mathdefault{1}$}}%
\end{pgfscope}%
\begin{pgfscope}%
\pgfsetbuttcap%
\pgfsetroundjoin%
\definecolor{currentfill}{rgb}{0.000000,0.000000,0.000000}%
\pgfsetfillcolor{currentfill}%
\pgfsetlinewidth{0.803000pt}%
\definecolor{currentstroke}{rgb}{0.000000,0.000000,0.000000}%
\pgfsetstrokecolor{currentstroke}%
\pgfsetdash{}{0pt}%
\pgfsys@defobject{currentmarker}{\pgfqpoint{-0.048611in}{0.000000in}}{\pgfqpoint{-0.000000in}{0.000000in}}{%
\pgfpathmoveto{\pgfqpoint{-0.000000in}{0.000000in}}%
\pgfpathlineto{\pgfqpoint{-0.048611in}{0.000000in}}%
\pgfusepath{stroke,fill}%
}%
\begin{pgfscope}%
\pgfsys@transformshift{0.452199in}{0.913655in}%
\pgfsys@useobject{currentmarker}{}%
\end{pgfscope}%
\end{pgfscope}%
\begin{pgfscope}%
\definecolor{textcolor}{rgb}{0.000000,0.000000,0.000000}%
\pgfsetstrokecolor{textcolor}%
\pgfsetfillcolor{textcolor}%
\pgftext[x=0.290741in, y=0.870252in, left, base]{\color{textcolor}{\rmfamily\fontsize{9.000000}{10.800000}\selectfont\catcode`\^=\active\def^{\ifmmode\sp\else\^{}\fi}\catcode`\%=\active\def%{\%}$\mathdefault{2}$}}%
\end{pgfscope}%
\begin{pgfscope}%
\pgfsetbuttcap%
\pgfsetroundjoin%
\definecolor{currentfill}{rgb}{0.000000,0.000000,0.000000}%
\pgfsetfillcolor{currentfill}%
\pgfsetlinewidth{0.803000pt}%
\definecolor{currentstroke}{rgb}{0.000000,0.000000,0.000000}%
\pgfsetstrokecolor{currentstroke}%
\pgfsetdash{}{0pt}%
\pgfsys@defobject{currentmarker}{\pgfqpoint{-0.048611in}{0.000000in}}{\pgfqpoint{-0.000000in}{0.000000in}}{%
\pgfpathmoveto{\pgfqpoint{-0.000000in}{0.000000in}}%
\pgfpathlineto{\pgfqpoint{-0.048611in}{0.000000in}}%
\pgfusepath{stroke,fill}%
}%
\begin{pgfscope}%
\pgfsys@transformshift{0.452199in}{1.259466in}%
\pgfsys@useobject{currentmarker}{}%
\end{pgfscope}%
\end{pgfscope}%
\begin{pgfscope}%
\definecolor{textcolor}{rgb}{0.000000,0.000000,0.000000}%
\pgfsetstrokecolor{textcolor}%
\pgfsetfillcolor{textcolor}%
\pgftext[x=0.290741in, y=1.216063in, left, base]{\color{textcolor}{\rmfamily\fontsize{9.000000}{10.800000}\selectfont\catcode`\^=\active\def^{\ifmmode\sp\else\^{}\fi}\catcode`\%=\active\def%{\%}$\mathdefault{3}$}}%
\end{pgfscope}%
\begin{pgfscope}%
\pgfsetbuttcap%
\pgfsetroundjoin%
\definecolor{currentfill}{rgb}{0.000000,0.000000,0.000000}%
\pgfsetfillcolor{currentfill}%
\pgfsetlinewidth{0.803000pt}%
\definecolor{currentstroke}{rgb}{0.000000,0.000000,0.000000}%
\pgfsetstrokecolor{currentstroke}%
\pgfsetdash{}{0pt}%
\pgfsys@defobject{currentmarker}{\pgfqpoint{-0.048611in}{0.000000in}}{\pgfqpoint{-0.000000in}{0.000000in}}{%
\pgfpathmoveto{\pgfqpoint{-0.000000in}{0.000000in}}%
\pgfpathlineto{\pgfqpoint{-0.048611in}{0.000000in}}%
\pgfusepath{stroke,fill}%
}%
\begin{pgfscope}%
\pgfsys@transformshift{0.452199in}{1.605277in}%
\pgfsys@useobject{currentmarker}{}%
\end{pgfscope}%
\end{pgfscope}%
\begin{pgfscope}%
\definecolor{textcolor}{rgb}{0.000000,0.000000,0.000000}%
\pgfsetstrokecolor{textcolor}%
\pgfsetfillcolor{textcolor}%
\pgftext[x=0.290741in, y=1.561874in, left, base]{\color{textcolor}{\rmfamily\fontsize{9.000000}{10.800000}\selectfont\catcode`\^=\active\def^{\ifmmode\sp\else\^{}\fi}\catcode`\%=\active\def%{\%}$\mathdefault{4}$}}%
\end{pgfscope}%
\begin{pgfscope}%
\definecolor{textcolor}{rgb}{0.000000,0.000000,0.000000}%
\pgfsetstrokecolor{textcolor}%
\pgfsetfillcolor{textcolor}%
\pgftext[x=0.235185in,y=1.086561in,,bottom,rotate=90.000000]{\color{textcolor}{\rmfamily\fontsize{11.000000}{13.200000}\selectfont\catcode`\^=\active\def^{\ifmmode\sp\else\^{}\fi}\catcode`\%=\active\def%{\%}Velocity}}%
\end{pgfscope}%
\begin{pgfscope}%
\pgfpathrectangle{\pgfqpoint{0.452199in}{0.515972in}}{\pgfqpoint{9.303131in}{1.141177in}}%
\pgfusepath{clip}%
\pgfsetrectcap%
\pgfsetroundjoin%
\pgfsetlinewidth{1.505625pt}%
\definecolor{currentstroke}{rgb}{0.000000,0.000000,1.000000}%
\pgfsetstrokecolor{currentstroke}%
\pgfsetdash{}{0pt}%
\pgfpathmoveto{\pgfqpoint{0.875068in}{0.567844in}}%
\pgfpathlineto{\pgfqpoint{0.945546in}{0.590898in}}%
\pgfpathlineto{\pgfqpoint{1.016025in}{0.613952in}}%
\pgfpathlineto{\pgfqpoint{1.086503in}{0.637006in}}%
\pgfpathlineto{\pgfqpoint{1.156981in}{0.660060in}}%
\pgfpathlineto{\pgfqpoint{1.227460in}{0.683114in}}%
\pgfpathlineto{\pgfqpoint{1.297938in}{0.706168in}}%
\pgfpathlineto{\pgfqpoint{1.368416in}{0.729222in}}%
\pgfpathlineto{\pgfqpoint{1.438894in}{0.752276in}}%
\pgfpathlineto{\pgfqpoint{1.509373in}{0.775330in}}%
\pgfpathlineto{\pgfqpoint{1.579851in}{0.798385in}}%
\pgfpathlineto{\pgfqpoint{1.650329in}{0.821439in}}%
\pgfpathlineto{\pgfqpoint{1.720807in}{0.844493in}}%
\pgfpathlineto{\pgfqpoint{1.791286in}{0.867547in}}%
\pgfpathlineto{\pgfqpoint{1.861764in}{0.890601in}}%
\pgfpathlineto{\pgfqpoint{1.932242in}{0.913655in}}%
\pgfpathlineto{\pgfqpoint{2.002720in}{0.936709in}}%
\pgfpathlineto{\pgfqpoint{2.073199in}{0.959763in}}%
\pgfpathlineto{\pgfqpoint{2.143677in}{0.982817in}}%
\pgfpathlineto{\pgfqpoint{2.214155in}{1.005871in}}%
\pgfpathlineto{\pgfqpoint{2.284633in}{1.028925in}}%
\pgfpathlineto{\pgfqpoint{2.355112in}{1.051979in}}%
\pgfpathlineto{\pgfqpoint{2.425590in}{1.075033in}}%
\pgfpathlineto{\pgfqpoint{2.496068in}{1.098088in}}%
\pgfpathlineto{\pgfqpoint{2.566547in}{1.121142in}}%
\pgfpathlineto{\pgfqpoint{2.637025in}{1.144196in}}%
\pgfpathlineto{\pgfqpoint{2.707503in}{1.167250in}}%
\pgfpathlineto{\pgfqpoint{2.777981in}{1.190304in}}%
\pgfpathlineto{\pgfqpoint{2.848460in}{1.213358in}}%
\pgfpathlineto{\pgfqpoint{2.918938in}{1.236412in}}%
\pgfpathlineto{\pgfqpoint{2.989416in}{1.259466in}}%
\pgfpathlineto{\pgfqpoint{3.059894in}{1.282520in}}%
\pgfpathlineto{\pgfqpoint{3.130373in}{1.305574in}}%
\pgfpathlineto{\pgfqpoint{3.200851in}{1.328628in}}%
\pgfpathlineto{\pgfqpoint{3.271329in}{1.351682in}}%
\pgfpathlineto{\pgfqpoint{3.341807in}{1.374736in}}%
\pgfpathlineto{\pgfqpoint{3.412286in}{1.397791in}}%
\pgfpathlineto{\pgfqpoint{3.482764in}{1.420845in}}%
\pgfpathlineto{\pgfqpoint{3.553242in}{1.443899in}}%
\pgfpathlineto{\pgfqpoint{3.623721in}{1.466953in}}%
\pgfpathlineto{\pgfqpoint{3.694199in}{1.490007in}}%
\pgfpathlineto{\pgfqpoint{3.764677in}{1.513061in}}%
\pgfpathlineto{\pgfqpoint{3.835155in}{1.536115in}}%
\pgfpathlineto{\pgfqpoint{3.905634in}{1.559169in}}%
\pgfpathlineto{\pgfqpoint{3.976112in}{1.582223in}}%
\pgfpathlineto{\pgfqpoint{4.046590in}{1.605277in}}%
\pgfpathlineto{\pgfqpoint{4.117068in}{1.605277in}}%
\pgfpathlineto{\pgfqpoint{4.187547in}{1.605277in}}%
\pgfpathlineto{\pgfqpoint{4.258025in}{1.605277in}}%
\pgfpathlineto{\pgfqpoint{4.328503in}{1.605277in}}%
\pgfpathlineto{\pgfqpoint{4.398981in}{1.605277in}}%
\pgfpathlineto{\pgfqpoint{4.469460in}{1.605277in}}%
\pgfpathlineto{\pgfqpoint{4.539938in}{1.605277in}}%
\pgfpathlineto{\pgfqpoint{4.610416in}{1.605277in}}%
\pgfpathlineto{\pgfqpoint{4.680894in}{1.605277in}}%
\pgfpathlineto{\pgfqpoint{4.751373in}{1.605277in}}%
\pgfpathlineto{\pgfqpoint{4.821851in}{1.605277in}}%
\pgfpathlineto{\pgfqpoint{4.892329in}{1.605277in}}%
\pgfpathlineto{\pgfqpoint{4.962808in}{1.605277in}}%
\pgfpathlineto{\pgfqpoint{5.033286in}{1.605277in}}%
\pgfpathlineto{\pgfqpoint{5.103764in}{1.605277in}}%
\pgfpathlineto{\pgfqpoint{5.174242in}{1.605277in}}%
\pgfpathlineto{\pgfqpoint{5.244721in}{1.605277in}}%
\pgfpathlineto{\pgfqpoint{5.315199in}{1.605277in}}%
\pgfpathlineto{\pgfqpoint{5.385677in}{1.605277in}}%
\pgfpathlineto{\pgfqpoint{5.456155in}{1.605277in}}%
\pgfpathlineto{\pgfqpoint{5.526634in}{1.605277in}}%
\pgfpathlineto{\pgfqpoint{5.597112in}{1.605277in}}%
\pgfpathlineto{\pgfqpoint{5.667590in}{1.605277in}}%
\pgfpathlineto{\pgfqpoint{5.738068in}{1.605277in}}%
\pgfpathlineto{\pgfqpoint{5.808547in}{1.605277in}}%
\pgfpathlineto{\pgfqpoint{5.879025in}{1.605277in}}%
\pgfpathlineto{\pgfqpoint{5.949503in}{1.605277in}}%
\pgfpathlineto{\pgfqpoint{6.019981in}{1.605277in}}%
\pgfpathlineto{\pgfqpoint{6.090460in}{1.605277in}}%
\pgfpathlineto{\pgfqpoint{6.160938in}{1.605277in}}%
\pgfpathlineto{\pgfqpoint{6.231416in}{1.605277in}}%
\pgfpathlineto{\pgfqpoint{6.301895in}{1.605277in}}%
\pgfpathlineto{\pgfqpoint{6.372373in}{1.605277in}}%
\pgfpathlineto{\pgfqpoint{6.442851in}{1.605277in}}%
\pgfpathlineto{\pgfqpoint{6.513329in}{1.605277in}}%
\pgfpathlineto{\pgfqpoint{6.583808in}{1.605277in}}%
\pgfpathlineto{\pgfqpoint{6.654286in}{1.605277in}}%
\pgfpathlineto{\pgfqpoint{6.724764in}{1.605277in}}%
\pgfpathlineto{\pgfqpoint{6.795242in}{1.605277in}}%
\pgfpathlineto{\pgfqpoint{6.865721in}{1.605277in}}%
\pgfpathlineto{\pgfqpoint{6.936199in}{1.605277in}}%
\pgfpathlineto{\pgfqpoint{7.006677in}{1.605277in}}%
\pgfpathlineto{\pgfqpoint{7.077155in}{1.605277in}}%
\pgfpathlineto{\pgfqpoint{7.147634in}{1.605277in}}%
\pgfpathlineto{\pgfqpoint{7.218112in}{1.605277in}}%
\pgfpathlineto{\pgfqpoint{7.288590in}{1.605277in}}%
\pgfpathlineto{\pgfqpoint{7.359069in}{1.605277in}}%
\pgfpathlineto{\pgfqpoint{7.429547in}{1.605277in}}%
\pgfpathlineto{\pgfqpoint{7.500025in}{1.605277in}}%
\pgfpathlineto{\pgfqpoint{7.570503in}{1.605277in}}%
\pgfpathlineto{\pgfqpoint{7.640982in}{1.605277in}}%
\pgfpathlineto{\pgfqpoint{7.711460in}{1.605277in}}%
\pgfpathlineto{\pgfqpoint{7.781938in}{1.605277in}}%
\pgfpathlineto{\pgfqpoint{7.852416in}{1.605277in}}%
\pgfpathlineto{\pgfqpoint{7.922895in}{1.605277in}}%
\pgfpathlineto{\pgfqpoint{7.993373in}{1.605277in}}%
\pgfpathlineto{\pgfqpoint{8.063851in}{1.605277in}}%
\pgfpathlineto{\pgfqpoint{8.134329in}{1.605277in}}%
\pgfpathlineto{\pgfqpoint{8.204808in}{1.605277in}}%
\pgfpathlineto{\pgfqpoint{8.275286in}{1.605277in}}%
\pgfpathlineto{\pgfqpoint{8.345764in}{1.605277in}}%
\pgfpathlineto{\pgfqpoint{8.416242in}{1.605277in}}%
\pgfpathlineto{\pgfqpoint{8.486721in}{1.605277in}}%
\pgfpathlineto{\pgfqpoint{8.557199in}{1.605277in}}%
\pgfpathlineto{\pgfqpoint{8.627677in}{1.605277in}}%
\pgfpathlineto{\pgfqpoint{8.698156in}{1.605277in}}%
\pgfpathlineto{\pgfqpoint{8.768634in}{1.605277in}}%
\pgfpathlineto{\pgfqpoint{8.839112in}{1.605277in}}%
\pgfpathlineto{\pgfqpoint{8.909590in}{1.605277in}}%
\pgfpathlineto{\pgfqpoint{8.980069in}{1.605277in}}%
\pgfpathlineto{\pgfqpoint{9.050547in}{1.605277in}}%
\pgfpathlineto{\pgfqpoint{9.121025in}{1.605277in}}%
\pgfpathlineto{\pgfqpoint{9.191503in}{1.605277in}}%
\pgfpathlineto{\pgfqpoint{9.261982in}{1.605277in}}%
\pgfpathlineto{\pgfqpoint{9.332460in}{1.605277in}}%
\pgfusepath{stroke}%
\end{pgfscope}%
\begin{pgfscope}%
\pgfpathrectangle{\pgfqpoint{0.452199in}{0.515972in}}{\pgfqpoint{9.303131in}{1.141177in}}%
\pgfusepath{clip}%
\pgfsetrectcap%
\pgfsetroundjoin%
\pgfsetlinewidth{1.505625pt}%
\definecolor{currentstroke}{rgb}{1.000000,0.000000,0.000000}%
\pgfsetstrokecolor{currentstroke}%
\pgfsetdash{}{0pt}%
\pgfusepath{stroke}%
\end{pgfscope}%
\begin{pgfscope}%
\pgfpathrectangle{\pgfqpoint{0.452199in}{0.515972in}}{\pgfqpoint{9.303131in}{1.141177in}}%
\pgfusepath{clip}%
\pgfsetrectcap%
\pgfsetroundjoin%
\pgfsetlinewidth{1.505625pt}%
\definecolor{currentstroke}{rgb}{0.000000,0.501961,0.000000}%
\pgfsetstrokecolor{currentstroke}%
\pgfsetdash{}{0pt}%
\pgfusepath{stroke}%
\end{pgfscope}%
\begin{pgfscope}%
\pgfsetrectcap%
\pgfsetmiterjoin%
\pgfsetlinewidth{0.803000pt}%
\definecolor{currentstroke}{rgb}{0.000000,0.000000,0.000000}%
\pgfsetstrokecolor{currentstroke}%
\pgfsetdash{}{0pt}%
\pgfpathmoveto{\pgfqpoint{0.452199in}{0.515972in}}%
\pgfpathlineto{\pgfqpoint{0.452199in}{1.657149in}}%
\pgfusepath{stroke}%
\end{pgfscope}%
\begin{pgfscope}%
\pgfsetrectcap%
\pgfsetmiterjoin%
\pgfsetlinewidth{0.803000pt}%
\definecolor{currentstroke}{rgb}{0.000000,0.000000,0.000000}%
\pgfsetstrokecolor{currentstroke}%
\pgfsetdash{}{0pt}%
\pgfpathmoveto{\pgfqpoint{9.755329in}{0.515972in}}%
\pgfpathlineto{\pgfqpoint{9.755329in}{1.657149in}}%
\pgfusepath{stroke}%
\end{pgfscope}%
\begin{pgfscope}%
\pgfsetrectcap%
\pgfsetmiterjoin%
\pgfsetlinewidth{0.803000pt}%
\definecolor{currentstroke}{rgb}{0.000000,0.000000,0.000000}%
\pgfsetstrokecolor{currentstroke}%
\pgfsetdash{}{0pt}%
\pgfpathmoveto{\pgfqpoint{0.452199in}{0.515972in}}%
\pgfpathlineto{\pgfqpoint{9.755329in}{0.515972in}}%
\pgfusepath{stroke}%
\end{pgfscope}%
\begin{pgfscope}%
\pgfsetrectcap%
\pgfsetmiterjoin%
\pgfsetlinewidth{0.803000pt}%
\definecolor{currentstroke}{rgb}{0.000000,0.000000,0.000000}%
\pgfsetstrokecolor{currentstroke}%
\pgfsetdash{}{0pt}%
\pgfpathmoveto{\pgfqpoint{0.452199in}{1.657149in}}%
\pgfpathlineto{\pgfqpoint{9.755329in}{1.657149in}}%
\pgfusepath{stroke}%
\end{pgfscope}%
\begin{pgfscope}%
\pgfsetbuttcap%
\pgfsetmiterjoin%
\definecolor{currentfill}{rgb}{1.000000,1.000000,1.000000}%
\pgfsetfillcolor{currentfill}%
\pgfsetfillopacity{0.800000}%
\pgfsetlinewidth{1.003750pt}%
\definecolor{currentstroke}{rgb}{0.800000,0.800000,0.800000}%
\pgfsetstrokecolor{currentstroke}%
\pgfsetstrokeopacity{0.800000}%
\pgfsetdash{}{0pt}%
\pgfpathmoveto{\pgfqpoint{8.040821in}{0.921038in}}%
\pgfpathlineto{\pgfqpoint{9.658107in}{0.921038in}}%
\pgfpathquadraticcurveto{\pgfqpoint{9.685885in}{0.921038in}}{\pgfqpoint{9.685885in}{0.948816in}}%
\pgfpathlineto{\pgfqpoint{9.685885in}{1.559927in}}%
\pgfpathquadraticcurveto{\pgfqpoint{9.685885in}{1.587704in}}{\pgfqpoint{9.658107in}{1.587704in}}%
\pgfpathlineto{\pgfqpoint{8.040821in}{1.587704in}}%
\pgfpathquadraticcurveto{\pgfqpoint{8.013043in}{1.587704in}}{\pgfqpoint{8.013043in}{1.559927in}}%
\pgfpathlineto{\pgfqpoint{8.013043in}{0.948816in}}%
\pgfpathquadraticcurveto{\pgfqpoint{8.013043in}{0.921038in}}{\pgfqpoint{8.040821in}{0.921038in}}%
\pgfpathlineto{\pgfqpoint{8.040821in}{0.921038in}}%
\pgfpathclose%
\pgfusepath{stroke,fill}%
\end{pgfscope}%
\begin{pgfscope}%
\pgfsetrectcap%
\pgfsetroundjoin%
\pgfsetlinewidth{1.505625pt}%
\definecolor{currentstroke}{rgb}{0.000000,0.000000,1.000000}%
\pgfsetstrokecolor{currentstroke}%
\pgfsetdash{}{0pt}%
\pgfpathmoveto{\pgfqpoint{8.068599in}{1.476593in}}%
\pgfpathlineto{\pgfqpoint{8.207488in}{1.476593in}}%
\pgfpathlineto{\pgfqpoint{8.346377in}{1.476593in}}%
\pgfusepath{stroke}%
\end{pgfscope}%
\begin{pgfscope}%
\definecolor{textcolor}{rgb}{0.000000,0.000000,0.000000}%
\pgfsetstrokecolor{textcolor}%
\pgfsetfillcolor{textcolor}%
\pgftext[x=8.457488in,y=1.427982in,left,base]{\color{textcolor}{\rmfamily\fontsize{10.000000}{12.000000}\selectfont\catcode`\^=\active\def^{\ifmmode\sp\else\^{}\fi}\catcode`\%=\active\def%{\%}Velocity (Positive)}}%
\end{pgfscope}%
\begin{pgfscope}%
\pgfsetrectcap%
\pgfsetroundjoin%
\pgfsetlinewidth{1.505625pt}%
\definecolor{currentstroke}{rgb}{1.000000,0.000000,0.000000}%
\pgfsetstrokecolor{currentstroke}%
\pgfsetdash{}{0pt}%
\pgfpathmoveto{\pgfqpoint{8.068599in}{1.268260in}}%
\pgfpathlineto{\pgfqpoint{8.207488in}{1.268260in}}%
\pgfpathlineto{\pgfqpoint{8.346377in}{1.268260in}}%
\pgfusepath{stroke}%
\end{pgfscope}%
\begin{pgfscope}%
\definecolor{textcolor}{rgb}{0.000000,0.000000,0.000000}%
\pgfsetstrokecolor{textcolor}%
\pgfsetfillcolor{textcolor}%
\pgftext[x=8.457488in,y=1.219649in,left,base]{\color{textcolor}{\rmfamily\fontsize{10.000000}{12.000000}\selectfont\catcode`\^=\active\def^{\ifmmode\sp\else\^{}\fi}\catcode`\%=\active\def%{\%}Velocity (Negative)}}%
\end{pgfscope}%
\begin{pgfscope}%
\pgfsetrectcap%
\pgfsetroundjoin%
\pgfsetlinewidth{1.505625pt}%
\definecolor{currentstroke}{rgb}{0.000000,0.501961,0.000000}%
\pgfsetstrokecolor{currentstroke}%
\pgfsetdash{}{0pt}%
\pgfpathmoveto{\pgfqpoint{8.068599in}{1.059927in}}%
\pgfpathlineto{\pgfqpoint{8.207488in}{1.059927in}}%
\pgfpathlineto{\pgfqpoint{8.346377in}{1.059927in}}%
\pgfusepath{stroke}%
\end{pgfscope}%
\begin{pgfscope}%
\definecolor{textcolor}{rgb}{0.000000,0.000000,0.000000}%
\pgfsetstrokecolor{textcolor}%
\pgfsetfillcolor{textcolor}%
\pgftext[x=8.457488in,y=1.011316in,left,base]{\color{textcolor}{\rmfamily\fontsize{10.000000}{12.000000}\selectfont\catcode`\^=\active\def^{\ifmmode\sp\else\^{}\fi}\catcode`\%=\active\def%{\%}Velocity (Zero)}}%
\end{pgfscope}%
\end{pgfpicture}%
\makeatother%
\endgroup%
}
	\caption{Planned velocity profile.}
	\label{fig:velocity}
\end{figure}

Figure \ref{fig:velocity} shows the planned velocity profile, which quickly reaches its upper limit.

\begin{figure}[h]
	\centering
	\resizebox{1\textwidth}{!}{%% Creator: Matplotlib, PGF backend
%%
%% To include the figure in your LaTeX document, write
%%   \input{<filename>.pgf}
%%
%% Make sure the required packages are loaded in your preamble
%%   \usepackage{pgf}
%%
%% Also ensure that all the required font packages are loaded; for instance,
%% the lmodern package is sometimes necessary when using math font.
%%   \usepackage{lmodern}
%%
%% Figures using additional raster images can only be included by \input if
%% they are in the same directory as the main LaTeX file. For loading figures
%% from other directories you can use the `import` package
%%   \usepackage{import}
%%
%% and then include the figures with
%%   \import{<path to file>}{<filename>.pgf}
%%
%% Matplotlib used the following preamble
%%   \def\mathdefault#1{#1}
%%   \everymath=\expandafter{\the\everymath\displaystyle}
%%   
%%   \ifdefined\pdftexversion\else  % non-pdftex case.
%%     \usepackage{fontspec}
%%   \fi
%%   \makeatletter\@ifpackageloaded{underscore}{}{\usepackage[strings]{underscore}}\makeatother
%%
\begingroup%
\makeatletter%
\begin{pgfpicture}%
\pgfpathrectangle{\pgfpointorigin}{\pgfqpoint{9.863821in}{1.757149in}}%
\pgfusepath{use as bounding box, clip}%
\begin{pgfscope}%
\pgfsetbuttcap%
\pgfsetmiterjoin%
\definecolor{currentfill}{rgb}{1.000000,1.000000,1.000000}%
\pgfsetfillcolor{currentfill}%
\pgfsetlinewidth{0.000000pt}%
\definecolor{currentstroke}{rgb}{1.000000,1.000000,1.000000}%
\pgfsetstrokecolor{currentstroke}%
\pgfsetdash{}{0pt}%
\pgfpathmoveto{\pgfqpoint{0.000000in}{0.000000in}}%
\pgfpathlineto{\pgfqpoint{9.863821in}{0.000000in}}%
\pgfpathlineto{\pgfqpoint{9.863821in}{1.757149in}}%
\pgfpathlineto{\pgfqpoint{0.000000in}{1.757149in}}%
\pgfpathlineto{\pgfqpoint{0.000000in}{0.000000in}}%
\pgfpathclose%
\pgfusepath{fill}%
\end{pgfscope}%
\begin{pgfscope}%
\pgfsetbuttcap%
\pgfsetmiterjoin%
\definecolor{currentfill}{rgb}{1.000000,1.000000,1.000000}%
\pgfsetfillcolor{currentfill}%
\pgfsetlinewidth{0.000000pt}%
\definecolor{currentstroke}{rgb}{0.000000,0.000000,0.000000}%
\pgfsetstrokecolor{currentstroke}%
\pgfsetstrokeopacity{0.000000}%
\pgfsetdash{}{0pt}%
\pgfpathmoveto{\pgfqpoint{0.616357in}{0.515972in}}%
\pgfpathlineto{\pgfqpoint{9.763821in}{0.515972in}}%
\pgfpathlineto{\pgfqpoint{9.763821in}{1.657149in}}%
\pgfpathlineto{\pgfqpoint{0.616357in}{1.657149in}}%
\pgfpathlineto{\pgfqpoint{0.616357in}{0.515972in}}%
\pgfpathclose%
\pgfusepath{fill}%
\end{pgfscope}%
\begin{pgfscope}%
\pgfsetbuttcap%
\pgfsetroundjoin%
\definecolor{currentfill}{rgb}{0.000000,0.000000,0.000000}%
\pgfsetfillcolor{currentfill}%
\pgfsetlinewidth{0.803000pt}%
\definecolor{currentstroke}{rgb}{0.000000,0.000000,0.000000}%
\pgfsetstrokecolor{currentstroke}%
\pgfsetdash{}{0pt}%
\pgfsys@defobject{currentmarker}{\pgfqpoint{0.000000in}{-0.048611in}}{\pgfqpoint{0.000000in}{0.000000in}}{%
\pgfpathmoveto{\pgfqpoint{0.000000in}{0.000000in}}%
\pgfpathlineto{\pgfqpoint{0.000000in}{-0.048611in}}%
\pgfusepath{stroke,fill}%
}%
\begin{pgfscope}%
\pgfsys@transformshift{1.032150in}{0.515972in}%
\pgfsys@useobject{currentmarker}{}%
\end{pgfscope}%
\end{pgfscope}%
\begin{pgfscope}%
\definecolor{textcolor}{rgb}{0.000000,0.000000,0.000000}%
\pgfsetstrokecolor{textcolor}%
\pgfsetfillcolor{textcolor}%
\pgftext[x=1.032150in,y=0.418750in,,top]{\color{textcolor}{\rmfamily\fontsize{9.000000}{10.800000}\selectfont\catcode`\^=\active\def^{\ifmmode\sp\else\^{}\fi}\catcode`\%=\active\def%{\%}$\mathdefault{0.0}$}}%
\end{pgfscope}%
\begin{pgfscope}%
\pgfsetbuttcap%
\pgfsetroundjoin%
\definecolor{currentfill}{rgb}{0.000000,0.000000,0.000000}%
\pgfsetfillcolor{currentfill}%
\pgfsetlinewidth{0.803000pt}%
\definecolor{currentstroke}{rgb}{0.000000,0.000000,0.000000}%
\pgfsetstrokecolor{currentstroke}%
\pgfsetdash{}{0pt}%
\pgfsys@defobject{currentmarker}{\pgfqpoint{0.000000in}{-0.048611in}}{\pgfqpoint{0.000000in}{0.000000in}}{%
\pgfpathmoveto{\pgfqpoint{0.000000in}{0.000000in}}%
\pgfpathlineto{\pgfqpoint{0.000000in}{-0.048611in}}%
\pgfusepath{stroke,fill}%
}%
\begin{pgfscope}%
\pgfsys@transformshift{2.071635in}{0.515972in}%
\pgfsys@useobject{currentmarker}{}%
\end{pgfscope}%
\end{pgfscope}%
\begin{pgfscope}%
\definecolor{textcolor}{rgb}{0.000000,0.000000,0.000000}%
\pgfsetstrokecolor{textcolor}%
\pgfsetfillcolor{textcolor}%
\pgftext[x=2.071635in,y=0.418750in,,top]{\color{textcolor}{\rmfamily\fontsize{9.000000}{10.800000}\selectfont\catcode`\^=\active\def^{\ifmmode\sp\else\^{}\fi}\catcode`\%=\active\def%{\%}$\mathdefault{0.5}$}}%
\end{pgfscope}%
\begin{pgfscope}%
\pgfsetbuttcap%
\pgfsetroundjoin%
\definecolor{currentfill}{rgb}{0.000000,0.000000,0.000000}%
\pgfsetfillcolor{currentfill}%
\pgfsetlinewidth{0.803000pt}%
\definecolor{currentstroke}{rgb}{0.000000,0.000000,0.000000}%
\pgfsetstrokecolor{currentstroke}%
\pgfsetdash{}{0pt}%
\pgfsys@defobject{currentmarker}{\pgfqpoint{0.000000in}{-0.048611in}}{\pgfqpoint{0.000000in}{0.000000in}}{%
\pgfpathmoveto{\pgfqpoint{0.000000in}{0.000000in}}%
\pgfpathlineto{\pgfqpoint{0.000000in}{-0.048611in}}%
\pgfusepath{stroke,fill}%
}%
\begin{pgfscope}%
\pgfsys@transformshift{3.111120in}{0.515972in}%
\pgfsys@useobject{currentmarker}{}%
\end{pgfscope}%
\end{pgfscope}%
\begin{pgfscope}%
\definecolor{textcolor}{rgb}{0.000000,0.000000,0.000000}%
\pgfsetstrokecolor{textcolor}%
\pgfsetfillcolor{textcolor}%
\pgftext[x=3.111120in,y=0.418750in,,top]{\color{textcolor}{\rmfamily\fontsize{9.000000}{10.800000}\selectfont\catcode`\^=\active\def^{\ifmmode\sp\else\^{}\fi}\catcode`\%=\active\def%{\%}$\mathdefault{1.0}$}}%
\end{pgfscope}%
\begin{pgfscope}%
\pgfsetbuttcap%
\pgfsetroundjoin%
\definecolor{currentfill}{rgb}{0.000000,0.000000,0.000000}%
\pgfsetfillcolor{currentfill}%
\pgfsetlinewidth{0.803000pt}%
\definecolor{currentstroke}{rgb}{0.000000,0.000000,0.000000}%
\pgfsetstrokecolor{currentstroke}%
\pgfsetdash{}{0pt}%
\pgfsys@defobject{currentmarker}{\pgfqpoint{0.000000in}{-0.048611in}}{\pgfqpoint{0.000000in}{0.000000in}}{%
\pgfpathmoveto{\pgfqpoint{0.000000in}{0.000000in}}%
\pgfpathlineto{\pgfqpoint{0.000000in}{-0.048611in}}%
\pgfusepath{stroke,fill}%
}%
\begin{pgfscope}%
\pgfsys@transformshift{4.150604in}{0.515972in}%
\pgfsys@useobject{currentmarker}{}%
\end{pgfscope}%
\end{pgfscope}%
\begin{pgfscope}%
\definecolor{textcolor}{rgb}{0.000000,0.000000,0.000000}%
\pgfsetstrokecolor{textcolor}%
\pgfsetfillcolor{textcolor}%
\pgftext[x=4.150604in,y=0.418750in,,top]{\color{textcolor}{\rmfamily\fontsize{9.000000}{10.800000}\selectfont\catcode`\^=\active\def^{\ifmmode\sp\else\^{}\fi}\catcode`\%=\active\def%{\%}$\mathdefault{1.5}$}}%
\end{pgfscope}%
\begin{pgfscope}%
\pgfsetbuttcap%
\pgfsetroundjoin%
\definecolor{currentfill}{rgb}{0.000000,0.000000,0.000000}%
\pgfsetfillcolor{currentfill}%
\pgfsetlinewidth{0.803000pt}%
\definecolor{currentstroke}{rgb}{0.000000,0.000000,0.000000}%
\pgfsetstrokecolor{currentstroke}%
\pgfsetdash{}{0pt}%
\pgfsys@defobject{currentmarker}{\pgfqpoint{0.000000in}{-0.048611in}}{\pgfqpoint{0.000000in}{0.000000in}}{%
\pgfpathmoveto{\pgfqpoint{0.000000in}{0.000000in}}%
\pgfpathlineto{\pgfqpoint{0.000000in}{-0.048611in}}%
\pgfusepath{stroke,fill}%
}%
\begin{pgfscope}%
\pgfsys@transformshift{5.190089in}{0.515972in}%
\pgfsys@useobject{currentmarker}{}%
\end{pgfscope}%
\end{pgfscope}%
\begin{pgfscope}%
\definecolor{textcolor}{rgb}{0.000000,0.000000,0.000000}%
\pgfsetstrokecolor{textcolor}%
\pgfsetfillcolor{textcolor}%
\pgftext[x=5.190089in,y=0.418750in,,top]{\color{textcolor}{\rmfamily\fontsize{9.000000}{10.800000}\selectfont\catcode`\^=\active\def^{\ifmmode\sp\else\^{}\fi}\catcode`\%=\active\def%{\%}$\mathdefault{2.0}$}}%
\end{pgfscope}%
\begin{pgfscope}%
\pgfsetbuttcap%
\pgfsetroundjoin%
\definecolor{currentfill}{rgb}{0.000000,0.000000,0.000000}%
\pgfsetfillcolor{currentfill}%
\pgfsetlinewidth{0.803000pt}%
\definecolor{currentstroke}{rgb}{0.000000,0.000000,0.000000}%
\pgfsetstrokecolor{currentstroke}%
\pgfsetdash{}{0pt}%
\pgfsys@defobject{currentmarker}{\pgfqpoint{0.000000in}{-0.048611in}}{\pgfqpoint{0.000000in}{0.000000in}}{%
\pgfpathmoveto{\pgfqpoint{0.000000in}{0.000000in}}%
\pgfpathlineto{\pgfqpoint{0.000000in}{-0.048611in}}%
\pgfusepath{stroke,fill}%
}%
\begin{pgfscope}%
\pgfsys@transformshift{6.229573in}{0.515972in}%
\pgfsys@useobject{currentmarker}{}%
\end{pgfscope}%
\end{pgfscope}%
\begin{pgfscope}%
\definecolor{textcolor}{rgb}{0.000000,0.000000,0.000000}%
\pgfsetstrokecolor{textcolor}%
\pgfsetfillcolor{textcolor}%
\pgftext[x=6.229573in,y=0.418750in,,top]{\color{textcolor}{\rmfamily\fontsize{9.000000}{10.800000}\selectfont\catcode`\^=\active\def^{\ifmmode\sp\else\^{}\fi}\catcode`\%=\active\def%{\%}$\mathdefault{2.5}$}}%
\end{pgfscope}%
\begin{pgfscope}%
\pgfsetbuttcap%
\pgfsetroundjoin%
\definecolor{currentfill}{rgb}{0.000000,0.000000,0.000000}%
\pgfsetfillcolor{currentfill}%
\pgfsetlinewidth{0.803000pt}%
\definecolor{currentstroke}{rgb}{0.000000,0.000000,0.000000}%
\pgfsetstrokecolor{currentstroke}%
\pgfsetdash{}{0pt}%
\pgfsys@defobject{currentmarker}{\pgfqpoint{0.000000in}{-0.048611in}}{\pgfqpoint{0.000000in}{0.000000in}}{%
\pgfpathmoveto{\pgfqpoint{0.000000in}{0.000000in}}%
\pgfpathlineto{\pgfqpoint{0.000000in}{-0.048611in}}%
\pgfusepath{stroke,fill}%
}%
\begin{pgfscope}%
\pgfsys@transformshift{7.269058in}{0.515972in}%
\pgfsys@useobject{currentmarker}{}%
\end{pgfscope}%
\end{pgfscope}%
\begin{pgfscope}%
\definecolor{textcolor}{rgb}{0.000000,0.000000,0.000000}%
\pgfsetstrokecolor{textcolor}%
\pgfsetfillcolor{textcolor}%
\pgftext[x=7.269058in,y=0.418750in,,top]{\color{textcolor}{\rmfamily\fontsize{9.000000}{10.800000}\selectfont\catcode`\^=\active\def^{\ifmmode\sp\else\^{}\fi}\catcode`\%=\active\def%{\%}$\mathdefault{3.0}$}}%
\end{pgfscope}%
\begin{pgfscope}%
\pgfsetbuttcap%
\pgfsetroundjoin%
\definecolor{currentfill}{rgb}{0.000000,0.000000,0.000000}%
\pgfsetfillcolor{currentfill}%
\pgfsetlinewidth{0.803000pt}%
\definecolor{currentstroke}{rgb}{0.000000,0.000000,0.000000}%
\pgfsetstrokecolor{currentstroke}%
\pgfsetdash{}{0pt}%
\pgfsys@defobject{currentmarker}{\pgfqpoint{0.000000in}{-0.048611in}}{\pgfqpoint{0.000000in}{0.000000in}}{%
\pgfpathmoveto{\pgfqpoint{0.000000in}{0.000000in}}%
\pgfpathlineto{\pgfqpoint{0.000000in}{-0.048611in}}%
\pgfusepath{stroke,fill}%
}%
\begin{pgfscope}%
\pgfsys@transformshift{8.308543in}{0.515972in}%
\pgfsys@useobject{currentmarker}{}%
\end{pgfscope}%
\end{pgfscope}%
\begin{pgfscope}%
\definecolor{textcolor}{rgb}{0.000000,0.000000,0.000000}%
\pgfsetstrokecolor{textcolor}%
\pgfsetfillcolor{textcolor}%
\pgftext[x=8.308543in,y=0.418750in,,top]{\color{textcolor}{\rmfamily\fontsize{9.000000}{10.800000}\selectfont\catcode`\^=\active\def^{\ifmmode\sp\else\^{}\fi}\catcode`\%=\active\def%{\%}$\mathdefault{3.5}$}}%
\end{pgfscope}%
\begin{pgfscope}%
\pgfsetbuttcap%
\pgfsetroundjoin%
\definecolor{currentfill}{rgb}{0.000000,0.000000,0.000000}%
\pgfsetfillcolor{currentfill}%
\pgfsetlinewidth{0.803000pt}%
\definecolor{currentstroke}{rgb}{0.000000,0.000000,0.000000}%
\pgfsetstrokecolor{currentstroke}%
\pgfsetdash{}{0pt}%
\pgfsys@defobject{currentmarker}{\pgfqpoint{0.000000in}{-0.048611in}}{\pgfqpoint{0.000000in}{0.000000in}}{%
\pgfpathmoveto{\pgfqpoint{0.000000in}{0.000000in}}%
\pgfpathlineto{\pgfqpoint{0.000000in}{-0.048611in}}%
\pgfusepath{stroke,fill}%
}%
\begin{pgfscope}%
\pgfsys@transformshift{9.348027in}{0.515972in}%
\pgfsys@useobject{currentmarker}{}%
\end{pgfscope}%
\end{pgfscope}%
\begin{pgfscope}%
\definecolor{textcolor}{rgb}{0.000000,0.000000,0.000000}%
\pgfsetstrokecolor{textcolor}%
\pgfsetfillcolor{textcolor}%
\pgftext[x=9.348027in,y=0.418750in,,top]{\color{textcolor}{\rmfamily\fontsize{9.000000}{10.800000}\selectfont\catcode`\^=\active\def^{\ifmmode\sp\else\^{}\fi}\catcode`\%=\active\def%{\%}$\mathdefault{4.0}$}}%
\end{pgfscope}%
\begin{pgfscope}%
\definecolor{textcolor}{rgb}{0.000000,0.000000,0.000000}%
\pgfsetstrokecolor{textcolor}%
\pgfsetfillcolor{textcolor}%
\pgftext[x=5.190089in,y=0.252083in,,top]{\color{textcolor}{\rmfamily\fontsize{11.000000}{13.200000}\selectfont\catcode`\^=\active\def^{\ifmmode\sp\else\^{}\fi}\catcode`\%=\active\def%{\%}Time [s]}}%
\end{pgfscope}%
\begin{pgfscope}%
\pgfsetbuttcap%
\pgfsetroundjoin%
\definecolor{currentfill}{rgb}{0.000000,0.000000,0.000000}%
\pgfsetfillcolor{currentfill}%
\pgfsetlinewidth{0.803000pt}%
\definecolor{currentstroke}{rgb}{0.000000,0.000000,0.000000}%
\pgfsetstrokecolor{currentstroke}%
\pgfsetdash{}{0pt}%
\pgfsys@defobject{currentmarker}{\pgfqpoint{-0.048611in}{0.000000in}}{\pgfqpoint{-0.000000in}{0.000000in}}{%
\pgfpathmoveto{\pgfqpoint{-0.000000in}{0.000000in}}%
\pgfpathlineto{\pgfqpoint{-0.048611in}{0.000000in}}%
\pgfusepath{stroke,fill}%
}%
\begin{pgfscope}%
\pgfsys@transformshift{0.616357in}{0.809942in}%
\pgfsys@useobject{currentmarker}{}%
\end{pgfscope}%
\end{pgfscope}%
\begin{pgfscope}%
\definecolor{textcolor}{rgb}{0.000000,0.000000,0.000000}%
\pgfsetstrokecolor{textcolor}%
\pgfsetfillcolor{textcolor}%
\pgftext[x=0.290741in, y=0.766539in, left, base]{\color{textcolor}{\rmfamily\fontsize{9.000000}{10.800000}\selectfont\catcode`\^=\active\def^{\ifmmode\sp\else\^{}\fi}\catcode`\%=\active\def%{\%}$\mathdefault{0.00}$}}%
\end{pgfscope}%
\begin{pgfscope}%
\pgfsetbuttcap%
\pgfsetroundjoin%
\definecolor{currentfill}{rgb}{0.000000,0.000000,0.000000}%
\pgfsetfillcolor{currentfill}%
\pgfsetlinewidth{0.803000pt}%
\definecolor{currentstroke}{rgb}{0.000000,0.000000,0.000000}%
\pgfsetstrokecolor{currentstroke}%
\pgfsetdash{}{0pt}%
\pgfsys@defobject{currentmarker}{\pgfqpoint{-0.048611in}{0.000000in}}{\pgfqpoint{-0.000000in}{0.000000in}}{%
\pgfpathmoveto{\pgfqpoint{-0.000000in}{0.000000in}}%
\pgfpathlineto{\pgfqpoint{-0.048611in}{0.000000in}}%
\pgfusepath{stroke,fill}%
}%
\begin{pgfscope}%
\pgfsys@transformshift{0.616357in}{1.145179in}%
\pgfsys@useobject{currentmarker}{}%
\end{pgfscope}%
\end{pgfscope}%
\begin{pgfscope}%
\definecolor{textcolor}{rgb}{0.000000,0.000000,0.000000}%
\pgfsetstrokecolor{textcolor}%
\pgfsetfillcolor{textcolor}%
\pgftext[x=0.290741in, y=1.101776in, left, base]{\color{textcolor}{\rmfamily\fontsize{9.000000}{10.800000}\selectfont\catcode`\^=\active\def^{\ifmmode\sp\else\^{}\fi}\catcode`\%=\active\def%{\%}$\mathdefault{0.05}$}}%
\end{pgfscope}%
\begin{pgfscope}%
\pgfsetbuttcap%
\pgfsetroundjoin%
\definecolor{currentfill}{rgb}{0.000000,0.000000,0.000000}%
\pgfsetfillcolor{currentfill}%
\pgfsetlinewidth{0.803000pt}%
\definecolor{currentstroke}{rgb}{0.000000,0.000000,0.000000}%
\pgfsetstrokecolor{currentstroke}%
\pgfsetdash{}{0pt}%
\pgfsys@defobject{currentmarker}{\pgfqpoint{-0.048611in}{0.000000in}}{\pgfqpoint{-0.000000in}{0.000000in}}{%
\pgfpathmoveto{\pgfqpoint{-0.000000in}{0.000000in}}%
\pgfpathlineto{\pgfqpoint{-0.048611in}{0.000000in}}%
\pgfusepath{stroke,fill}%
}%
\begin{pgfscope}%
\pgfsys@transformshift{0.616357in}{1.480416in}%
\pgfsys@useobject{currentmarker}{}%
\end{pgfscope}%
\end{pgfscope}%
\begin{pgfscope}%
\definecolor{textcolor}{rgb}{0.000000,0.000000,0.000000}%
\pgfsetstrokecolor{textcolor}%
\pgfsetfillcolor{textcolor}%
\pgftext[x=0.290741in, y=1.437013in, left, base]{\color{textcolor}{\rmfamily\fontsize{9.000000}{10.800000}\selectfont\catcode`\^=\active\def^{\ifmmode\sp\else\^{}\fi}\catcode`\%=\active\def%{\%}$\mathdefault{0.10}$}}%
\end{pgfscope}%
\begin{pgfscope}%
\definecolor{textcolor}{rgb}{0.000000,0.000000,0.000000}%
\pgfsetstrokecolor{textcolor}%
\pgfsetfillcolor{textcolor}%
\pgftext[x=0.235185in,y=1.086561in,,bottom,rotate=90.000000]{\color{textcolor}{\rmfamily\fontsize{11.000000}{13.200000}\selectfont\catcode`\^=\active\def^{\ifmmode\sp\else\^{}\fi}\catcode`\%=\active\def%{\%}Alignment Error}}%
\end{pgfscope}%
\begin{pgfscope}%
\pgfpathrectangle{\pgfqpoint{0.616357in}{0.515972in}}{\pgfqpoint{9.147465in}{1.141177in}}%
\pgfusepath{clip}%
\pgfsetrectcap%
\pgfsetroundjoin%
\pgfsetlinewidth{1.505625pt}%
\definecolor{currentstroke}{rgb}{0.000000,0.000000,1.000000}%
\pgfsetstrokecolor{currentstroke}%
\pgfsetdash{}{0pt}%
\pgfpathmoveto{\pgfqpoint{1.101449in}{0.823910in}}%
\pgfpathlineto{\pgfqpoint{1.170748in}{0.837508in}}%
\pgfpathlineto{\pgfqpoint{1.240047in}{0.850690in}}%
\pgfpathlineto{\pgfqpoint{1.309346in}{0.863463in}}%
\pgfpathlineto{\pgfqpoint{1.378645in}{0.875824in}}%
\pgfpathlineto{\pgfqpoint{1.447944in}{0.887760in}}%
\pgfpathlineto{\pgfqpoint{1.517243in}{0.899245in}}%
\pgfpathlineto{\pgfqpoint{1.586542in}{0.910240in}}%
\pgfpathlineto{\pgfqpoint{1.655841in}{0.920705in}}%
\pgfpathlineto{\pgfqpoint{1.725140in}{0.930589in}}%
\pgfpathlineto{\pgfqpoint{1.794439in}{0.939844in}}%
\pgfpathlineto{\pgfqpoint{1.863738in}{0.948416in}}%
\pgfpathlineto{\pgfqpoint{1.933037in}{0.956251in}}%
\pgfpathlineto{\pgfqpoint{2.002336in}{0.963296in}}%
\pgfpathlineto{\pgfqpoint{2.071635in}{0.969498in}}%
\pgfpathlineto{\pgfqpoint{2.140934in}{0.974806in}}%
\pgfpathlineto{\pgfqpoint{2.210233in}{0.979172in}}%
\pgfpathlineto{\pgfqpoint{2.279532in}{0.982554in}}%
\pgfpathlineto{\pgfqpoint{2.348831in}{0.984917in}}%
\pgfpathlineto{\pgfqpoint{2.418130in}{0.986232in}}%
\pgfpathlineto{\pgfqpoint{2.487429in}{0.986481in}}%
\pgfpathlineto{\pgfqpoint{2.556728in}{0.985656in}}%
\pgfpathlineto{\pgfqpoint{2.626027in}{0.983756in}}%
\pgfpathlineto{\pgfqpoint{2.695326in}{0.980794in}}%
\pgfpathlineto{\pgfqpoint{2.764625in}{0.976790in}}%
\pgfpathlineto{\pgfqpoint{2.833924in}{0.971775in}}%
\pgfpathlineto{\pgfqpoint{2.903223in}{0.965790in}}%
\pgfpathlineto{\pgfqpoint{2.972522in}{0.958885in}}%
\pgfpathlineto{\pgfqpoint{3.041821in}{0.951119in}}%
\pgfpathlineto{\pgfqpoint{3.111120in}{0.942557in}}%
\pgfpathlineto{\pgfqpoint{3.180419in}{0.933264in}}%
\pgfpathlineto{\pgfqpoint{3.249718in}{0.923298in}}%
\pgfpathlineto{\pgfqpoint{3.319017in}{0.912713in}}%
\pgfpathlineto{\pgfqpoint{3.388316in}{0.901554in}}%
\pgfpathlineto{\pgfqpoint{3.457614in}{0.889872in}}%
\pgfpathlineto{\pgfqpoint{3.526913in}{0.877719in}}%
\pgfpathlineto{\pgfqpoint{3.596212in}{0.865158in}}%
\pgfpathlineto{\pgfqpoint{3.665511in}{0.852260in}}%
\pgfpathlineto{\pgfqpoint{3.734810in}{0.839096in}}%
\pgfpathlineto{\pgfqpoint{3.804109in}{0.825737in}}%
\pgfpathlineto{\pgfqpoint{3.873408in}{0.812255in}}%
\pgfpathmoveto{\pgfqpoint{7.823450in}{0.820764in}}%
\pgfpathlineto{\pgfqpoint{7.892749in}{0.842825in}}%
\pgfpathlineto{\pgfqpoint{7.962048in}{0.866079in}}%
\pgfpathlineto{\pgfqpoint{8.031347in}{0.890558in}}%
\pgfpathlineto{\pgfqpoint{8.100646in}{0.916297in}}%
\pgfpathlineto{\pgfqpoint{8.169945in}{0.943326in}}%
\pgfpathlineto{\pgfqpoint{8.239244in}{0.971677in}}%
\pgfpathlineto{\pgfqpoint{8.308543in}{1.001375in}}%
\pgfpathlineto{\pgfqpoint{8.377842in}{1.032447in}}%
\pgfpathlineto{\pgfqpoint{8.447141in}{1.064911in}}%
\pgfpathlineto{\pgfqpoint{8.516440in}{1.098784in}}%
\pgfpathlineto{\pgfqpoint{8.585739in}{1.134074in}}%
\pgfpathlineto{\pgfqpoint{8.655038in}{1.170781in}}%
\pgfpathlineto{\pgfqpoint{8.724337in}{1.208898in}}%
\pgfpathlineto{\pgfqpoint{8.793636in}{1.248403in}}%
\pgfpathlineto{\pgfqpoint{8.862934in}{1.289260in}}%
\pgfpathlineto{\pgfqpoint{8.932233in}{1.331416in}}%
\pgfpathlineto{\pgfqpoint{9.001532in}{1.374796in}}%
\pgfpathlineto{\pgfqpoint{9.070831in}{1.419299in}}%
\pgfpathlineto{\pgfqpoint{9.140130in}{1.464793in}}%
\pgfpathlineto{\pgfqpoint{9.209429in}{1.511106in}}%
\pgfpathlineto{\pgfqpoint{9.278728in}{1.558024in}}%
\pgfpathlineto{\pgfqpoint{9.348027in}{1.605277in}}%
\pgfusepath{stroke}%
\end{pgfscope}%
\begin{pgfscope}%
\pgfpathrectangle{\pgfqpoint{0.616357in}{0.515972in}}{\pgfqpoint{9.147465in}{1.141177in}}%
\pgfusepath{clip}%
\pgfsetrectcap%
\pgfsetroundjoin%
\pgfsetlinewidth{1.505625pt}%
\definecolor{currentstroke}{rgb}{1.000000,0.000000,0.000000}%
\pgfsetstrokecolor{currentstroke}%
\pgfsetdash{}{0pt}%
\pgfpathmoveto{\pgfqpoint{3.942707in}{0.798735in}}%
\pgfpathlineto{\pgfqpoint{4.012006in}{0.785278in}}%
\pgfpathlineto{\pgfqpoint{4.081305in}{0.772001in}}%
\pgfpathlineto{\pgfqpoint{4.150604in}{0.759053in}}%
\pgfpathlineto{\pgfqpoint{4.219903in}{0.746619in}}%
\pgfpathlineto{\pgfqpoint{4.289202in}{0.734059in}}%
\pgfpathlineto{\pgfqpoint{4.358501in}{0.721559in}}%
\pgfpathlineto{\pgfqpoint{4.427800in}{0.709258in}}%
\pgfpathlineto{\pgfqpoint{4.497099in}{0.697263in}}%
\pgfpathlineto{\pgfqpoint{4.566398in}{0.685652in}}%
\pgfpathlineto{\pgfqpoint{4.635697in}{0.674483in}}%
\pgfpathlineto{\pgfqpoint{4.704996in}{0.663801in}}%
\pgfpathlineto{\pgfqpoint{4.774295in}{0.653637in}}%
\pgfpathlineto{\pgfqpoint{4.843594in}{0.644012in}}%
\pgfpathlineto{\pgfqpoint{4.912893in}{0.634942in}}%
\pgfpathlineto{\pgfqpoint{4.982192in}{0.626438in}}%
\pgfpathlineto{\pgfqpoint{5.051491in}{0.618507in}}%
\pgfpathlineto{\pgfqpoint{5.120790in}{0.611150in}}%
\pgfpathlineto{\pgfqpoint{5.190089in}{0.604371in}}%
\pgfpathlineto{\pgfqpoint{5.259388in}{0.598170in}}%
\pgfpathlineto{\pgfqpoint{5.328687in}{0.592546in}}%
\pgfpathlineto{\pgfqpoint{5.397986in}{0.587499in}}%
\pgfpathlineto{\pgfqpoint{5.467285in}{0.583027in}}%
\pgfpathlineto{\pgfqpoint{5.536584in}{0.579131in}}%
\pgfpathlineto{\pgfqpoint{5.605883in}{0.575808in}}%
\pgfpathlineto{\pgfqpoint{5.675182in}{0.573060in}}%
\pgfpathlineto{\pgfqpoint{5.744481in}{0.570888in}}%
\pgfpathlineto{\pgfqpoint{5.813780in}{0.569293in}}%
\pgfpathlineto{\pgfqpoint{5.883079in}{0.568277in}}%
\pgfpathlineto{\pgfqpoint{5.952378in}{0.567844in}}%
\pgfpathlineto{\pgfqpoint{6.021677in}{0.567998in}}%
\pgfpathlineto{\pgfqpoint{6.090976in}{0.568745in}}%
\pgfpathlineto{\pgfqpoint{6.160274in}{0.570090in}}%
\pgfpathlineto{\pgfqpoint{6.229573in}{0.572043in}}%
\pgfpathlineto{\pgfqpoint{6.298872in}{0.574612in}}%
\pgfpathlineto{\pgfqpoint{6.368171in}{0.577806in}}%
\pgfpathlineto{\pgfqpoint{6.437470in}{0.581637in}}%
\pgfpathlineto{\pgfqpoint{6.506769in}{0.586118in}}%
\pgfpathlineto{\pgfqpoint{6.576068in}{0.591262in}}%
\pgfpathlineto{\pgfqpoint{6.645367in}{0.597085in}}%
\pgfpathlineto{\pgfqpoint{6.714666in}{0.603602in}}%
\pgfpathlineto{\pgfqpoint{6.783965in}{0.610832in}}%
\pgfpathlineto{\pgfqpoint{6.853264in}{0.618794in}}%
\pgfpathlineto{\pgfqpoint{6.922563in}{0.627507in}}%
\pgfpathlineto{\pgfqpoint{6.991862in}{0.636994in}}%
\pgfpathlineto{\pgfqpoint{7.061161in}{0.647278in}}%
\pgfpathlineto{\pgfqpoint{7.130460in}{0.658382in}}%
\pgfpathlineto{\pgfqpoint{7.199759in}{0.670333in}}%
\pgfpathlineto{\pgfqpoint{7.269058in}{0.683157in}}%
\pgfpathlineto{\pgfqpoint{7.338357in}{0.696883in}}%
\pgfpathlineto{\pgfqpoint{7.407656in}{0.711538in}}%
\pgfpathlineto{\pgfqpoint{7.476955in}{0.727154in}}%
\pgfpathlineto{\pgfqpoint{7.546254in}{0.743762in}}%
\pgfpathlineto{\pgfqpoint{7.615553in}{0.761394in}}%
\pgfpathlineto{\pgfqpoint{7.684852in}{0.780082in}}%
\pgfpathlineto{\pgfqpoint{7.754151in}{0.799861in}}%
\pgfusepath{stroke}%
\end{pgfscope}%
\begin{pgfscope}%
\pgfpathrectangle{\pgfqpoint{0.616357in}{0.515972in}}{\pgfqpoint{9.147465in}{1.141177in}}%
\pgfusepath{clip}%
\pgfsetrectcap%
\pgfsetroundjoin%
\pgfsetlinewidth{1.505625pt}%
\definecolor{currentstroke}{rgb}{0.000000,0.501961,0.000000}%
\pgfsetstrokecolor{currentstroke}%
\pgfsetdash{}{0pt}%
\pgfpathmoveto{\pgfqpoint{1.032150in}{0.809942in}}%
\pgfusepath{stroke}%
\end{pgfscope}%
\begin{pgfscope}%
\pgfsetrectcap%
\pgfsetmiterjoin%
\pgfsetlinewidth{0.803000pt}%
\definecolor{currentstroke}{rgb}{0.000000,0.000000,0.000000}%
\pgfsetstrokecolor{currentstroke}%
\pgfsetdash{}{0pt}%
\pgfpathmoveto{\pgfqpoint{0.616357in}{0.515972in}}%
\pgfpathlineto{\pgfqpoint{0.616357in}{1.657149in}}%
\pgfusepath{stroke}%
\end{pgfscope}%
\begin{pgfscope}%
\pgfsetrectcap%
\pgfsetmiterjoin%
\pgfsetlinewidth{0.803000pt}%
\definecolor{currentstroke}{rgb}{0.000000,0.000000,0.000000}%
\pgfsetstrokecolor{currentstroke}%
\pgfsetdash{}{0pt}%
\pgfpathmoveto{\pgfqpoint{9.763821in}{0.515972in}}%
\pgfpathlineto{\pgfqpoint{9.763821in}{1.657149in}}%
\pgfusepath{stroke}%
\end{pgfscope}%
\begin{pgfscope}%
\pgfsetrectcap%
\pgfsetmiterjoin%
\pgfsetlinewidth{0.803000pt}%
\definecolor{currentstroke}{rgb}{0.000000,0.000000,0.000000}%
\pgfsetstrokecolor{currentstroke}%
\pgfsetdash{}{0pt}%
\pgfpathmoveto{\pgfqpoint{0.616357in}{0.515972in}}%
\pgfpathlineto{\pgfqpoint{9.763821in}{0.515972in}}%
\pgfusepath{stroke}%
\end{pgfscope}%
\begin{pgfscope}%
\pgfsetrectcap%
\pgfsetmiterjoin%
\pgfsetlinewidth{0.803000pt}%
\definecolor{currentstroke}{rgb}{0.000000,0.000000,0.000000}%
\pgfsetstrokecolor{currentstroke}%
\pgfsetdash{}{0pt}%
\pgfpathmoveto{\pgfqpoint{0.616357in}{1.657149in}}%
\pgfpathlineto{\pgfqpoint{9.763821in}{1.657149in}}%
\pgfusepath{stroke}%
\end{pgfscope}%
\begin{pgfscope}%
\pgfsetbuttcap%
\pgfsetmiterjoin%
\definecolor{currentfill}{rgb}{1.000000,1.000000,1.000000}%
\pgfsetfillcolor{currentfill}%
\pgfsetfillopacity{0.800000}%
\pgfsetlinewidth{1.003750pt}%
\definecolor{currentstroke}{rgb}{0.800000,0.800000,0.800000}%
\pgfsetstrokecolor{currentstroke}%
\pgfsetstrokeopacity{0.800000}%
\pgfsetdash{}{0pt}%
\pgfpathmoveto{\pgfqpoint{4.123343in}{0.921038in}}%
\pgfpathlineto{\pgfqpoint{6.256834in}{0.921038in}}%
\pgfpathquadraticcurveto{\pgfqpoint{6.284612in}{0.921038in}}{\pgfqpoint{6.284612in}{0.948816in}}%
\pgfpathlineto{\pgfqpoint{6.284612in}{1.559927in}}%
\pgfpathquadraticcurveto{\pgfqpoint{6.284612in}{1.587704in}}{\pgfqpoint{6.256834in}{1.587704in}}%
\pgfpathlineto{\pgfqpoint{4.123343in}{1.587704in}}%
\pgfpathquadraticcurveto{\pgfqpoint{4.095566in}{1.587704in}}{\pgfqpoint{4.095566in}{1.559927in}}%
\pgfpathlineto{\pgfqpoint{4.095566in}{0.948816in}}%
\pgfpathquadraticcurveto{\pgfqpoint{4.095566in}{0.921038in}}{\pgfqpoint{4.123343in}{0.921038in}}%
\pgfpathlineto{\pgfqpoint{4.123343in}{0.921038in}}%
\pgfpathclose%
\pgfusepath{stroke,fill}%
\end{pgfscope}%
\begin{pgfscope}%
\pgfsetrectcap%
\pgfsetroundjoin%
\pgfsetlinewidth{1.505625pt}%
\definecolor{currentstroke}{rgb}{0.000000,0.000000,1.000000}%
\pgfsetstrokecolor{currentstroke}%
\pgfsetdash{}{0pt}%
\pgfpathmoveto{\pgfqpoint{4.151121in}{1.476593in}}%
\pgfpathlineto{\pgfqpoint{4.290010in}{1.476593in}}%
\pgfpathlineto{\pgfqpoint{4.428899in}{1.476593in}}%
\pgfusepath{stroke}%
\end{pgfscope}%
\begin{pgfscope}%
\definecolor{textcolor}{rgb}{0.000000,0.000000,0.000000}%
\pgfsetstrokecolor{textcolor}%
\pgfsetfillcolor{textcolor}%
\pgftext[x=4.540010in,y=1.427982in,left,base]{\color{textcolor}{\rmfamily\fontsize{10.000000}{12.000000}\selectfont\catcode`\^=\active\def^{\ifmmode\sp\else\^{}\fi}\catcode`\%=\active\def%{\%}Alignment Error (Positive)}}%
\end{pgfscope}%
\begin{pgfscope}%
\pgfsetrectcap%
\pgfsetroundjoin%
\pgfsetlinewidth{1.505625pt}%
\definecolor{currentstroke}{rgb}{1.000000,0.000000,0.000000}%
\pgfsetstrokecolor{currentstroke}%
\pgfsetdash{}{0pt}%
\pgfpathmoveto{\pgfqpoint{4.151121in}{1.268260in}}%
\pgfpathlineto{\pgfqpoint{4.290010in}{1.268260in}}%
\pgfpathlineto{\pgfqpoint{4.428899in}{1.268260in}}%
\pgfusepath{stroke}%
\end{pgfscope}%
\begin{pgfscope}%
\definecolor{textcolor}{rgb}{0.000000,0.000000,0.000000}%
\pgfsetstrokecolor{textcolor}%
\pgfsetfillcolor{textcolor}%
\pgftext[x=4.540010in,y=1.219649in,left,base]{\color{textcolor}{\rmfamily\fontsize{10.000000}{12.000000}\selectfont\catcode`\^=\active\def^{\ifmmode\sp\else\^{}\fi}\catcode`\%=\active\def%{\%}Alignment Error (Negative)}}%
\end{pgfscope}%
\begin{pgfscope}%
\pgfsetrectcap%
\pgfsetroundjoin%
\pgfsetlinewidth{1.505625pt}%
\definecolor{currentstroke}{rgb}{0.000000,0.501961,0.000000}%
\pgfsetstrokecolor{currentstroke}%
\pgfsetdash{}{0pt}%
\pgfpathmoveto{\pgfqpoint{4.151121in}{1.059927in}}%
\pgfpathlineto{\pgfqpoint{4.290010in}{1.059927in}}%
\pgfpathlineto{\pgfqpoint{4.428899in}{1.059927in}}%
\pgfusepath{stroke}%
\end{pgfscope}%
\begin{pgfscope}%
\definecolor{textcolor}{rgb}{0.000000,0.000000,0.000000}%
\pgfsetstrokecolor{textcolor}%
\pgfsetfillcolor{textcolor}%
\pgftext[x=4.540010in,y=1.011316in,left,base]{\color{textcolor}{\rmfamily\fontsize{10.000000}{12.000000}\selectfont\catcode`\^=\active\def^{\ifmmode\sp\else\^{}\fi}\catcode`\%=\active\def%{\%}Alignment Error (Zero)}}%
\end{pgfscope}%
\end{pgfpicture}%
\makeatother%
\endgroup%
}
	\caption{Alignment error \(\xi\) over time.}
	\label{fig:alignment-error}
\end{figure}

Figure \ref{fig:alignment-error} depicts the alignment error \( \xi \) at each planned time point.
Here, \( \xi \) is bounded within \(-45^{\circ} \leq \xi \leq 45^{\circ} \).
It is noteworthy that \( \xi \) does not reach these bounds.

\begin{figure}[h]
	\centering
	\resizebox{0.7\textwidth}{!}{%% Creator: Matplotlib, PGF backend
%%
%% To include the figure in your LaTeX document, write
%%   \input{<filename>.pgf}
%%
%% Make sure the required packages are loaded in your preamble
%%   \usepackage{pgf}
%%
%% Also ensure that all the required font packages are loaded; for instance,
%% the lmodern package is sometimes necessary when using math font.
%%   \usepackage{lmodern}
%%
%% Figures using additional raster images can only be included by \input if
%% they are in the same directory as the main LaTeX file. For loading figures
%% from other directories you can use the `import` package
%%   \usepackage{import}
%%
%% and then include the figures with
%%   \import{<path to file>}{<filename>.pgf}
%%
%% Matplotlib used the following preamble
%%   \def\mathdefault#1{#1}
%%   \everymath=\expandafter{\the\everymath\displaystyle}
%%   
%%   \ifdefined\pdftexversion\else  % non-pdftex case.
%%     \usepackage{fontspec}
%%   \fi
%%   \makeatletter\@ifpackageloaded{underscore}{}{\usepackage[strings]{underscore}}\makeatother
%%
\begingroup%
\makeatletter%
\begin{pgfpicture}%
\pgfpathrectangle{\pgfpointorigin}{\pgfqpoint{5.712043in}{4.295074in}}%
\pgfusepath{use as bounding box, clip}%
\begin{pgfscope}%
\pgfsetbuttcap%
\pgfsetmiterjoin%
\definecolor{currentfill}{rgb}{1.000000,1.000000,1.000000}%
\pgfsetfillcolor{currentfill}%
\pgfsetlinewidth{0.000000pt}%
\definecolor{currentstroke}{rgb}{1.000000,1.000000,1.000000}%
\pgfsetstrokecolor{currentstroke}%
\pgfsetdash{}{0pt}%
\pgfpathmoveto{\pgfqpoint{0.000000in}{0.000000in}}%
\pgfpathlineto{\pgfqpoint{5.712043in}{0.000000in}}%
\pgfpathlineto{\pgfqpoint{5.712043in}{4.295074in}}%
\pgfpathlineto{\pgfqpoint{0.000000in}{4.295074in}}%
\pgfpathlineto{\pgfqpoint{0.000000in}{0.000000in}}%
\pgfpathclose%
\pgfusepath{fill}%
\end{pgfscope}%
\begin{pgfscope}%
\pgfsetbuttcap%
\pgfsetmiterjoin%
\definecolor{currentfill}{rgb}{1.000000,1.000000,1.000000}%
\pgfsetfillcolor{currentfill}%
\pgfsetlinewidth{0.000000pt}%
\definecolor{currentstroke}{rgb}{0.000000,0.000000,0.000000}%
\pgfsetstrokecolor{currentstroke}%
\pgfsetstrokeopacity{0.000000}%
\pgfsetdash{}{0pt}%
\pgfpathmoveto{\pgfqpoint{0.652043in}{0.499074in}}%
\pgfpathlineto{\pgfqpoint{5.612043in}{0.499074in}}%
\pgfpathlineto{\pgfqpoint{5.612043in}{4.195074in}}%
\pgfpathlineto{\pgfqpoint{0.652043in}{4.195074in}}%
\pgfpathlineto{\pgfqpoint{0.652043in}{0.499074in}}%
\pgfpathclose%
\pgfusepath{fill}%
\end{pgfscope}%
\begin{pgfscope}%
\pgfpathrectangle{\pgfqpoint{0.652043in}{0.499074in}}{\pgfqpoint{4.960000in}{3.696000in}}%
\pgfusepath{clip}%
\pgfsetbuttcap%
\pgfsetroundjoin%
\definecolor{currentfill}{rgb}{0.000000,0.501961,0.000000}%
\pgfsetfillcolor{currentfill}%
\pgfsetfillopacity{0.200000}%
\pgfsetlinewidth{1.003750pt}%
\definecolor{currentstroke}{rgb}{0.000000,0.501961,0.000000}%
\pgfsetstrokecolor{currentstroke}%
\pgfsetstrokeopacity{0.200000}%
\pgfsetdash{}{0pt}%
\pgfsys@defobject{currentmarker}{\pgfqpoint{0.877497in}{0.667074in}}{\pgfqpoint{5.386588in}{4.027074in}}{%
\pgfpathmoveto{\pgfqpoint{0.877497in}{2.252782in}}%
\pgfpathlineto{\pgfqpoint{0.877497in}{2.252782in}}%
\pgfpathlineto{\pgfqpoint{0.915389in}{2.180270in}}%
\pgfpathlineto{\pgfqpoint{0.953281in}{2.107677in}}%
\pgfpathlineto{\pgfqpoint{0.991172in}{2.034996in}}%
\pgfpathlineto{\pgfqpoint{1.029064in}{1.962226in}}%
\pgfpathlineto{\pgfqpoint{1.066955in}{1.889367in}}%
\pgfpathlineto{\pgfqpoint{1.104847in}{1.816417in}}%
\pgfpathlineto{\pgfqpoint{1.142738in}{1.743370in}}%
\pgfpathlineto{\pgfqpoint{1.180630in}{1.670218in}}%
\pgfpathlineto{\pgfqpoint{1.218521in}{1.596952in}}%
\pgfpathlineto{\pgfqpoint{1.256413in}{1.523561in}}%
\pgfpathlineto{\pgfqpoint{1.294304in}{1.450034in}}%
\pgfpathlineto{\pgfqpoint{1.332196in}{1.376360in}}%
\pgfpathlineto{\pgfqpoint{1.370087in}{1.302528in}}%
\pgfpathlineto{\pgfqpoint{1.407979in}{1.228526in}}%
\pgfpathlineto{\pgfqpoint{1.445870in}{1.154343in}}%
\pgfpathlineto{\pgfqpoint{1.483762in}{1.079967in}}%
\pgfpathlineto{\pgfqpoint{1.521653in}{1.005389in}}%
\pgfpathlineto{\pgfqpoint{1.559545in}{0.930599in}}%
\pgfpathlineto{\pgfqpoint{1.597436in}{0.855590in}}%
\pgfpathlineto{\pgfqpoint{1.635328in}{0.780355in}}%
\pgfpathlineto{\pgfqpoint{1.673219in}{0.704892in}}%
\pgfpathlineto{\pgfqpoint{1.711111in}{0.667074in}}%
\pgfpathlineto{\pgfqpoint{1.749002in}{0.740957in}}%
\pgfpathlineto{\pgfqpoint{1.786894in}{0.813926in}}%
\pgfpathlineto{\pgfqpoint{1.824785in}{0.885998in}}%
\pgfpathlineto{\pgfqpoint{1.862677in}{0.957200in}}%
\pgfpathlineto{\pgfqpoint{1.900569in}{1.027568in}}%
\pgfpathlineto{\pgfqpoint{1.938460in}{1.097144in}}%
\pgfpathlineto{\pgfqpoint{1.976352in}{1.165979in}}%
\pgfpathlineto{\pgfqpoint{2.014243in}{1.234129in}}%
\pgfpathlineto{\pgfqpoint{2.052135in}{1.301651in}}%
\pgfpathlineto{\pgfqpoint{2.090026in}{1.368593in}}%
\pgfpathlineto{\pgfqpoint{2.127918in}{1.435002in}}%
\pgfpathlineto{\pgfqpoint{2.165809in}{1.500919in}}%
\pgfpathlineto{\pgfqpoint{2.203701in}{1.566384in}}%
\pgfpathlineto{\pgfqpoint{2.241592in}{1.631444in}}%
\pgfpathlineto{\pgfqpoint{2.279484in}{1.696154in}}%
\pgfpathlineto{\pgfqpoint{2.317375in}{1.760573in}}%
\pgfpathlineto{\pgfqpoint{2.355267in}{1.824764in}}%
\pgfpathlineto{\pgfqpoint{2.393158in}{1.888786in}}%
\pgfpathlineto{\pgfqpoint{2.431050in}{1.952703in}}%
\pgfpathlineto{\pgfqpoint{2.468941in}{2.016588in}}%
\pgfpathlineto{\pgfqpoint{2.506833in}{2.080525in}}%
\pgfpathlineto{\pgfqpoint{2.544724in}{2.144619in}}%
\pgfpathlineto{\pgfqpoint{2.582616in}{2.208995in}}%
\pgfpathlineto{\pgfqpoint{2.620507in}{2.198296in}}%
\pgfpathlineto{\pgfqpoint{2.658399in}{2.187489in}}%
\pgfpathlineto{\pgfqpoint{2.696290in}{2.176733in}}%
\pgfpathlineto{\pgfqpoint{2.734182in}{2.166149in}}%
\pgfpathlineto{\pgfqpoint{2.772073in}{2.155828in}}%
\pgfpathlineto{\pgfqpoint{2.809965in}{2.145837in}}%
\pgfpathlineto{\pgfqpoint{2.847857in}{2.136227in}}%
\pgfpathlineto{\pgfqpoint{2.885748in}{2.127036in}}%
\pgfpathlineto{\pgfqpoint{2.923640in}{2.118289in}}%
\pgfpathlineto{\pgfqpoint{2.961531in}{2.110008in}}%
\pgfpathlineto{\pgfqpoint{2.999423in}{2.102204in}}%
\pgfpathlineto{\pgfqpoint{3.037314in}{2.094887in}}%
\pgfpathlineto{\pgfqpoint{3.075206in}{2.088062in}}%
\pgfpathlineto{\pgfqpoint{3.113097in}{2.081732in}}%
\pgfpathlineto{\pgfqpoint{3.150989in}{2.075899in}}%
\pgfpathlineto{\pgfqpoint{3.188880in}{2.070563in}}%
\pgfpathlineto{\pgfqpoint{3.226772in}{2.065724in}}%
\pgfpathlineto{\pgfqpoint{3.264663in}{2.061381in}}%
\pgfpathlineto{\pgfqpoint{3.302555in}{2.057534in}}%
\pgfpathlineto{\pgfqpoint{3.340446in}{2.054181in}}%
\pgfpathlineto{\pgfqpoint{3.378338in}{2.051322in}}%
\pgfpathlineto{\pgfqpoint{3.416229in}{2.048957in}}%
\pgfpathlineto{\pgfqpoint{3.454121in}{2.047088in}}%
\pgfpathlineto{\pgfqpoint{3.492012in}{2.045716in}}%
\pgfpathlineto{\pgfqpoint{3.529904in}{2.044841in}}%
\pgfpathlineto{\pgfqpoint{3.567795in}{2.044469in}}%
\pgfpathlineto{\pgfqpoint{3.605687in}{2.044601in}}%
\pgfpathlineto{\pgfqpoint{3.643578in}{2.045244in}}%
\pgfpathlineto{\pgfqpoint{3.681470in}{2.046402in}}%
\pgfpathlineto{\pgfqpoint{3.719361in}{2.048082in}}%
\pgfpathlineto{\pgfqpoint{3.757253in}{2.050292in}}%
\pgfpathlineto{\pgfqpoint{3.795145in}{2.053041in}}%
\pgfpathlineto{\pgfqpoint{3.833036in}{2.056337in}}%
\pgfpathlineto{\pgfqpoint{3.870928in}{2.060193in}}%
\pgfpathlineto{\pgfqpoint{3.908819in}{2.064619in}}%
\pgfpathlineto{\pgfqpoint{3.946711in}{2.069629in}}%
\pgfpathlineto{\pgfqpoint{3.984602in}{2.075237in}}%
\pgfpathlineto{\pgfqpoint{4.022494in}{2.081458in}}%
\pgfpathlineto{\pgfqpoint{4.060385in}{2.088309in}}%
\pgfpathlineto{\pgfqpoint{4.098277in}{2.095806in}}%
\pgfpathlineto{\pgfqpoint{4.136168in}{2.103969in}}%
\pgfpathlineto{\pgfqpoint{4.174060in}{2.112818in}}%
\pgfpathlineto{\pgfqpoint{4.211951in}{2.122373in}}%
\pgfpathlineto{\pgfqpoint{4.249843in}{2.132656in}}%
\pgfpathlineto{\pgfqpoint{4.287734in}{2.143691in}}%
\pgfpathlineto{\pgfqpoint{4.325626in}{2.155501in}}%
\pgfpathlineto{\pgfqpoint{4.363517in}{2.168111in}}%
\pgfpathlineto{\pgfqpoint{4.401409in}{2.181548in}}%
\pgfpathlineto{\pgfqpoint{4.439300in}{2.195838in}}%
\pgfpathlineto{\pgfqpoint{4.477192in}{2.211009in}}%
\pgfpathlineto{\pgfqpoint{4.515083in}{2.227090in}}%
\pgfpathlineto{\pgfqpoint{4.552975in}{2.244108in}}%
\pgfpathlineto{\pgfqpoint{4.590866in}{2.262094in}}%
\pgfpathlineto{\pgfqpoint{4.628758in}{2.281077in}}%
\pgfpathlineto{\pgfqpoint{4.666649in}{2.301085in}}%
\pgfpathlineto{\pgfqpoint{4.704541in}{2.322149in}}%
\pgfpathlineto{\pgfqpoint{4.742433in}{2.344296in}}%
\pgfpathlineto{\pgfqpoint{4.780324in}{2.367554in}}%
\pgfpathlineto{\pgfqpoint{4.818216in}{2.391948in}}%
\pgfpathlineto{\pgfqpoint{4.856107in}{2.417502in}}%
\pgfpathlineto{\pgfqpoint{4.893999in}{2.444237in}}%
\pgfpathlineto{\pgfqpoint{4.931890in}{2.472171in}}%
\pgfpathlineto{\pgfqpoint{4.969782in}{2.501317in}}%
\pgfpathlineto{\pgfqpoint{5.007673in}{2.531682in}}%
\pgfpathlineto{\pgfqpoint{5.045565in}{2.563268in}}%
\pgfpathlineto{\pgfqpoint{5.083456in}{2.596065in}}%
\pgfpathlineto{\pgfqpoint{5.121348in}{2.630057in}}%
\pgfpathlineto{\pgfqpoint{5.159239in}{2.665213in}}%
\pgfpathlineto{\pgfqpoint{5.197131in}{2.701486in}}%
\pgfpathlineto{\pgfqpoint{5.235022in}{2.738813in}}%
\pgfpathlineto{\pgfqpoint{5.272914in}{2.777105in}}%
\pgfpathlineto{\pgfqpoint{5.310805in}{2.816250in}}%
\pgfpathlineto{\pgfqpoint{5.348697in}{2.856101in}}%
\pgfpathlineto{\pgfqpoint{5.386588in}{2.896472in}}%
\pgfpathlineto{\pgfqpoint{5.386588in}{2.896472in}}%
\pgfpathlineto{\pgfqpoint{5.386588in}{2.896472in}}%
\pgfpathlineto{\pgfqpoint{5.348697in}{2.856101in}}%
\pgfpathlineto{\pgfqpoint{5.310805in}{2.816250in}}%
\pgfpathlineto{\pgfqpoint{5.272914in}{2.777105in}}%
\pgfpathlineto{\pgfqpoint{5.235022in}{2.738813in}}%
\pgfpathlineto{\pgfqpoint{5.197131in}{2.701486in}}%
\pgfpathlineto{\pgfqpoint{5.159239in}{2.665213in}}%
\pgfpathlineto{\pgfqpoint{5.121348in}{2.630057in}}%
\pgfpathlineto{\pgfqpoint{5.083456in}{2.596065in}}%
\pgfpathlineto{\pgfqpoint{5.045565in}{2.563268in}}%
\pgfpathlineto{\pgfqpoint{5.007673in}{2.531682in}}%
\pgfpathlineto{\pgfqpoint{4.969782in}{2.501317in}}%
\pgfpathlineto{\pgfqpoint{4.931890in}{2.472171in}}%
\pgfpathlineto{\pgfqpoint{4.893999in}{2.444237in}}%
\pgfpathlineto{\pgfqpoint{4.856107in}{2.417502in}}%
\pgfpathlineto{\pgfqpoint{4.818216in}{2.391948in}}%
\pgfpathlineto{\pgfqpoint{4.780324in}{2.367554in}}%
\pgfpathlineto{\pgfqpoint{4.742433in}{2.344296in}}%
\pgfpathlineto{\pgfqpoint{4.704541in}{2.322149in}}%
\pgfpathlineto{\pgfqpoint{4.666649in}{2.301086in}}%
\pgfpathlineto{\pgfqpoint{4.628758in}{2.281077in}}%
\pgfpathlineto{\pgfqpoint{4.590866in}{2.262094in}}%
\pgfpathlineto{\pgfqpoint{4.552975in}{2.244108in}}%
\pgfpathlineto{\pgfqpoint{4.515083in}{2.227090in}}%
\pgfpathlineto{\pgfqpoint{4.477192in}{2.211009in}}%
\pgfpathlineto{\pgfqpoint{4.439300in}{2.195838in}}%
\pgfpathlineto{\pgfqpoint{4.401409in}{2.181548in}}%
\pgfpathlineto{\pgfqpoint{4.363517in}{2.168111in}}%
\pgfpathlineto{\pgfqpoint{4.325626in}{2.155501in}}%
\pgfpathlineto{\pgfqpoint{4.287734in}{2.143691in}}%
\pgfpathlineto{\pgfqpoint{4.249843in}{2.132656in}}%
\pgfpathlineto{\pgfqpoint{4.211951in}{2.122373in}}%
\pgfpathlineto{\pgfqpoint{4.174060in}{2.112818in}}%
\pgfpathlineto{\pgfqpoint{4.136168in}{2.103969in}}%
\pgfpathlineto{\pgfqpoint{4.098277in}{2.095806in}}%
\pgfpathlineto{\pgfqpoint{4.060385in}{2.088309in}}%
\pgfpathlineto{\pgfqpoint{4.022494in}{2.081458in}}%
\pgfpathlineto{\pgfqpoint{3.984602in}{2.075237in}}%
\pgfpathlineto{\pgfqpoint{3.946711in}{2.069629in}}%
\pgfpathlineto{\pgfqpoint{3.908819in}{2.064619in}}%
\pgfpathlineto{\pgfqpoint{3.870928in}{2.060193in}}%
\pgfpathlineto{\pgfqpoint{3.833036in}{2.056337in}}%
\pgfpathlineto{\pgfqpoint{3.795145in}{2.053041in}}%
\pgfpathlineto{\pgfqpoint{3.757253in}{2.050292in}}%
\pgfpathlineto{\pgfqpoint{3.719361in}{2.048082in}}%
\pgfpathlineto{\pgfqpoint{3.681470in}{2.046402in}}%
\pgfpathlineto{\pgfqpoint{3.643578in}{2.045244in}}%
\pgfpathlineto{\pgfqpoint{3.605687in}{2.044601in}}%
\pgfpathlineto{\pgfqpoint{3.567795in}{2.044469in}}%
\pgfpathlineto{\pgfqpoint{3.529904in}{2.044841in}}%
\pgfpathlineto{\pgfqpoint{3.492012in}{2.045716in}}%
\pgfpathlineto{\pgfqpoint{3.454121in}{2.047088in}}%
\pgfpathlineto{\pgfqpoint{3.416229in}{2.048957in}}%
\pgfpathlineto{\pgfqpoint{3.378338in}{2.051322in}}%
\pgfpathlineto{\pgfqpoint{3.340446in}{2.054181in}}%
\pgfpathlineto{\pgfqpoint{3.302555in}{2.057534in}}%
\pgfpathlineto{\pgfqpoint{3.264663in}{2.061381in}}%
\pgfpathlineto{\pgfqpoint{3.226772in}{2.065724in}}%
\pgfpathlineto{\pgfqpoint{3.188880in}{2.070563in}}%
\pgfpathlineto{\pgfqpoint{3.150989in}{2.075899in}}%
\pgfpathlineto{\pgfqpoint{3.113097in}{2.081732in}}%
\pgfpathlineto{\pgfqpoint{3.075206in}{2.088062in}}%
\pgfpathlineto{\pgfqpoint{3.037314in}{2.094887in}}%
\pgfpathlineto{\pgfqpoint{2.999423in}{2.102204in}}%
\pgfpathlineto{\pgfqpoint{2.961531in}{2.110008in}}%
\pgfpathlineto{\pgfqpoint{2.923640in}{2.118289in}}%
\pgfpathlineto{\pgfqpoint{2.885748in}{2.127036in}}%
\pgfpathlineto{\pgfqpoint{2.847857in}{2.136227in}}%
\pgfpathlineto{\pgfqpoint{2.809965in}{2.145837in}}%
\pgfpathlineto{\pgfqpoint{2.772073in}{2.155828in}}%
\pgfpathlineto{\pgfqpoint{2.734182in}{2.166149in}}%
\pgfpathlineto{\pgfqpoint{2.696290in}{2.176733in}}%
\pgfpathlineto{\pgfqpoint{2.658399in}{2.187489in}}%
\pgfpathlineto{\pgfqpoint{2.620507in}{2.198296in}}%
\pgfpathlineto{\pgfqpoint{2.582616in}{2.208995in}}%
\pgfpathlineto{\pgfqpoint{2.544724in}{2.295654in}}%
\pgfpathlineto{\pgfqpoint{2.506833in}{2.382595in}}%
\pgfpathlineto{\pgfqpoint{2.468941in}{2.469692in}}%
\pgfpathlineto{\pgfqpoint{2.431050in}{2.556843in}}%
\pgfpathlineto{\pgfqpoint{2.393158in}{2.643961in}}%
\pgfpathlineto{\pgfqpoint{2.355267in}{2.730973in}}%
\pgfpathlineto{\pgfqpoint{2.317375in}{2.817818in}}%
\pgfpathlineto{\pgfqpoint{2.279484in}{2.904433in}}%
\pgfpathlineto{\pgfqpoint{2.241592in}{2.990758in}}%
\pgfpathlineto{\pgfqpoint{2.203701in}{3.076733in}}%
\pgfpathlineto{\pgfqpoint{2.165809in}{3.162303in}}%
\pgfpathlineto{\pgfqpoint{2.127918in}{3.247422in}}%
\pgfpathlineto{\pgfqpoint{2.090026in}{3.332048in}}%
\pgfpathlineto{\pgfqpoint{2.052135in}{3.416140in}}%
\pgfpathlineto{\pgfqpoint{2.014243in}{3.499654in}}%
\pgfpathlineto{\pgfqpoint{1.976352in}{3.582538in}}%
\pgfpathlineto{\pgfqpoint{1.938460in}{3.664738in}}%
\pgfpathlineto{\pgfqpoint{1.900569in}{3.746197in}}%
\pgfpathlineto{\pgfqpoint{1.862677in}{3.826864in}}%
\pgfpathlineto{\pgfqpoint{1.824785in}{3.906697in}}%
\pgfpathlineto{\pgfqpoint{1.786894in}{3.985659in}}%
\pgfpathlineto{\pgfqpoint{1.749002in}{4.027074in}}%
\pgfpathlineto{\pgfqpoint{1.711111in}{3.951965in}}%
\pgfpathlineto{\pgfqpoint{1.673219in}{3.876625in}}%
\pgfpathlineto{\pgfqpoint{1.635328in}{3.801054in}}%
\pgfpathlineto{\pgfqpoint{1.597436in}{3.725254in}}%
\pgfpathlineto{\pgfqpoint{1.559545in}{3.649228in}}%
\pgfpathlineto{\pgfqpoint{1.521653in}{3.572983in}}%
\pgfpathlineto{\pgfqpoint{1.483762in}{3.496526in}}%
\pgfpathlineto{\pgfqpoint{1.445870in}{3.419867in}}%
\pgfpathlineto{\pgfqpoint{1.407979in}{3.343016in}}%
\pgfpathlineto{\pgfqpoint{1.370087in}{3.265983in}}%
\pgfpathlineto{\pgfqpoint{1.332196in}{3.188780in}}%
\pgfpathlineto{\pgfqpoint{1.294304in}{3.111418in}}%
\pgfpathlineto{\pgfqpoint{1.256413in}{3.033910in}}%
\pgfpathlineto{\pgfqpoint{1.218521in}{2.956266in}}%
\pgfpathlineto{\pgfqpoint{1.180630in}{2.878498in}}%
\pgfpathlineto{\pgfqpoint{1.142738in}{2.800615in}}%
\pgfpathlineto{\pgfqpoint{1.104847in}{2.722627in}}%
\pgfpathlineto{\pgfqpoint{1.066955in}{2.644542in}}%
\pgfpathlineto{\pgfqpoint{1.029064in}{2.566365in}}%
\pgfpathlineto{\pgfqpoint{0.991172in}{2.488100in}}%
\pgfpathlineto{\pgfqpoint{0.953281in}{2.409747in}}%
\pgfpathlineto{\pgfqpoint{0.915389in}{2.331305in}}%
\pgfpathlineto{\pgfqpoint{0.877497in}{2.252782in}}%
\pgfpathlineto{\pgfqpoint{0.877497in}{2.252782in}}%
\pgfpathclose%
\pgfusepath{stroke,fill}%
}%
\begin{pgfscope}%
\pgfsys@transformshift{0.000000in}{0.000000in}%
\pgfsys@useobject{currentmarker}{}%
\end{pgfscope}%
\end{pgfscope}%
\begin{pgfscope}%
\pgfpathrectangle{\pgfqpoint{0.652043in}{0.499074in}}{\pgfqpoint{4.960000in}{3.696000in}}%
\pgfusepath{clip}%
\pgfsetbuttcap%
\pgfsetroundjoin%
\pgfsetlinewidth{0.803000pt}%
\definecolor{currentstroke}{rgb}{0.501961,0.501961,0.501961}%
\pgfsetstrokecolor{currentstroke}%
\pgfsetstrokeopacity{0.700000}%
\pgfsetdash{{0.800000pt}{1.320000pt}}{0.000000pt}%
\pgfpathmoveto{\pgfqpoint{0.877497in}{0.499074in}}%
\pgfpathlineto{\pgfqpoint{0.877497in}{4.195074in}}%
\pgfusepath{stroke}%
\end{pgfscope}%
\begin{pgfscope}%
\pgfsetbuttcap%
\pgfsetroundjoin%
\definecolor{currentfill}{rgb}{0.000000,0.000000,0.000000}%
\pgfsetfillcolor{currentfill}%
\pgfsetlinewidth{0.803000pt}%
\definecolor{currentstroke}{rgb}{0.000000,0.000000,0.000000}%
\pgfsetstrokecolor{currentstroke}%
\pgfsetdash{}{0pt}%
\pgfsys@defobject{currentmarker}{\pgfqpoint{0.000000in}{-0.048611in}}{\pgfqpoint{0.000000in}{0.000000in}}{%
\pgfpathmoveto{\pgfqpoint{0.000000in}{0.000000in}}%
\pgfpathlineto{\pgfqpoint{0.000000in}{-0.048611in}}%
\pgfusepath{stroke,fill}%
}%
\begin{pgfscope}%
\pgfsys@transformshift{0.877497in}{0.499074in}%
\pgfsys@useobject{currentmarker}{}%
\end{pgfscope}%
\end{pgfscope}%
\begin{pgfscope}%
\definecolor{textcolor}{rgb}{0.000000,0.000000,0.000000}%
\pgfsetstrokecolor{textcolor}%
\pgfsetfillcolor{textcolor}%
\pgftext[x=0.877497in,y=0.401852in,,top]{\color{textcolor}{\rmfamily\fontsize{9.000000}{10.800000}\selectfont\catcode`\^=\active\def^{\ifmmode\sp\else\^{}\fi}\catcode`\%=\active\def%{\%}$\mathdefault{0}$}}%
\end{pgfscope}%
\begin{pgfscope}%
\pgfpathrectangle{\pgfqpoint{0.652043in}{0.499074in}}{\pgfqpoint{4.960000in}{3.696000in}}%
\pgfusepath{clip}%
\pgfsetbuttcap%
\pgfsetroundjoin%
\pgfsetlinewidth{0.803000pt}%
\definecolor{currentstroke}{rgb}{0.501961,0.501961,0.501961}%
\pgfsetstrokecolor{currentstroke}%
\pgfsetstrokeopacity{0.700000}%
\pgfsetdash{{0.800000pt}{1.320000pt}}{0.000000pt}%
\pgfpathmoveto{\pgfqpoint{1.635328in}{0.499074in}}%
\pgfpathlineto{\pgfqpoint{1.635328in}{4.195074in}}%
\pgfusepath{stroke}%
\end{pgfscope}%
\begin{pgfscope}%
\pgfsetbuttcap%
\pgfsetroundjoin%
\definecolor{currentfill}{rgb}{0.000000,0.000000,0.000000}%
\pgfsetfillcolor{currentfill}%
\pgfsetlinewidth{0.803000pt}%
\definecolor{currentstroke}{rgb}{0.000000,0.000000,0.000000}%
\pgfsetstrokecolor{currentstroke}%
\pgfsetdash{}{0pt}%
\pgfsys@defobject{currentmarker}{\pgfqpoint{0.000000in}{-0.048611in}}{\pgfqpoint{0.000000in}{0.000000in}}{%
\pgfpathmoveto{\pgfqpoint{0.000000in}{0.000000in}}%
\pgfpathlineto{\pgfqpoint{0.000000in}{-0.048611in}}%
\pgfusepath{stroke,fill}%
}%
\begin{pgfscope}%
\pgfsys@transformshift{1.635328in}{0.499074in}%
\pgfsys@useobject{currentmarker}{}%
\end{pgfscope}%
\end{pgfscope}%
\begin{pgfscope}%
\definecolor{textcolor}{rgb}{0.000000,0.000000,0.000000}%
\pgfsetstrokecolor{textcolor}%
\pgfsetfillcolor{textcolor}%
\pgftext[x=1.635328in,y=0.401852in,,top]{\color{textcolor}{\rmfamily\fontsize{9.000000}{10.800000}\selectfont\catcode`\^=\active\def^{\ifmmode\sp\else\^{}\fi}\catcode`\%=\active\def%{\%}$\mathdefault{20}$}}%
\end{pgfscope}%
\begin{pgfscope}%
\pgfpathrectangle{\pgfqpoint{0.652043in}{0.499074in}}{\pgfqpoint{4.960000in}{3.696000in}}%
\pgfusepath{clip}%
\pgfsetbuttcap%
\pgfsetroundjoin%
\pgfsetlinewidth{0.803000pt}%
\definecolor{currentstroke}{rgb}{0.501961,0.501961,0.501961}%
\pgfsetstrokecolor{currentstroke}%
\pgfsetstrokeopacity{0.700000}%
\pgfsetdash{{0.800000pt}{1.320000pt}}{0.000000pt}%
\pgfpathmoveto{\pgfqpoint{2.393158in}{0.499074in}}%
\pgfpathlineto{\pgfqpoint{2.393158in}{4.195074in}}%
\pgfusepath{stroke}%
\end{pgfscope}%
\begin{pgfscope}%
\pgfsetbuttcap%
\pgfsetroundjoin%
\definecolor{currentfill}{rgb}{0.000000,0.000000,0.000000}%
\pgfsetfillcolor{currentfill}%
\pgfsetlinewidth{0.803000pt}%
\definecolor{currentstroke}{rgb}{0.000000,0.000000,0.000000}%
\pgfsetstrokecolor{currentstroke}%
\pgfsetdash{}{0pt}%
\pgfsys@defobject{currentmarker}{\pgfqpoint{0.000000in}{-0.048611in}}{\pgfqpoint{0.000000in}{0.000000in}}{%
\pgfpathmoveto{\pgfqpoint{0.000000in}{0.000000in}}%
\pgfpathlineto{\pgfqpoint{0.000000in}{-0.048611in}}%
\pgfusepath{stroke,fill}%
}%
\begin{pgfscope}%
\pgfsys@transformshift{2.393158in}{0.499074in}%
\pgfsys@useobject{currentmarker}{}%
\end{pgfscope}%
\end{pgfscope}%
\begin{pgfscope}%
\definecolor{textcolor}{rgb}{0.000000,0.000000,0.000000}%
\pgfsetstrokecolor{textcolor}%
\pgfsetfillcolor{textcolor}%
\pgftext[x=2.393158in,y=0.401852in,,top]{\color{textcolor}{\rmfamily\fontsize{9.000000}{10.800000}\selectfont\catcode`\^=\active\def^{\ifmmode\sp\else\^{}\fi}\catcode`\%=\active\def%{\%}$\mathdefault{40}$}}%
\end{pgfscope}%
\begin{pgfscope}%
\pgfpathrectangle{\pgfqpoint{0.652043in}{0.499074in}}{\pgfqpoint{4.960000in}{3.696000in}}%
\pgfusepath{clip}%
\pgfsetbuttcap%
\pgfsetroundjoin%
\pgfsetlinewidth{0.803000pt}%
\definecolor{currentstroke}{rgb}{0.501961,0.501961,0.501961}%
\pgfsetstrokecolor{currentstroke}%
\pgfsetstrokeopacity{0.700000}%
\pgfsetdash{{0.800000pt}{1.320000pt}}{0.000000pt}%
\pgfpathmoveto{\pgfqpoint{3.150989in}{0.499074in}}%
\pgfpathlineto{\pgfqpoint{3.150989in}{4.195074in}}%
\pgfusepath{stroke}%
\end{pgfscope}%
\begin{pgfscope}%
\pgfsetbuttcap%
\pgfsetroundjoin%
\definecolor{currentfill}{rgb}{0.000000,0.000000,0.000000}%
\pgfsetfillcolor{currentfill}%
\pgfsetlinewidth{0.803000pt}%
\definecolor{currentstroke}{rgb}{0.000000,0.000000,0.000000}%
\pgfsetstrokecolor{currentstroke}%
\pgfsetdash{}{0pt}%
\pgfsys@defobject{currentmarker}{\pgfqpoint{0.000000in}{-0.048611in}}{\pgfqpoint{0.000000in}{0.000000in}}{%
\pgfpathmoveto{\pgfqpoint{0.000000in}{0.000000in}}%
\pgfpathlineto{\pgfqpoint{0.000000in}{-0.048611in}}%
\pgfusepath{stroke,fill}%
}%
\begin{pgfscope}%
\pgfsys@transformshift{3.150989in}{0.499074in}%
\pgfsys@useobject{currentmarker}{}%
\end{pgfscope}%
\end{pgfscope}%
\begin{pgfscope}%
\definecolor{textcolor}{rgb}{0.000000,0.000000,0.000000}%
\pgfsetstrokecolor{textcolor}%
\pgfsetfillcolor{textcolor}%
\pgftext[x=3.150989in,y=0.401852in,,top]{\color{textcolor}{\rmfamily\fontsize{9.000000}{10.800000}\selectfont\catcode`\^=\active\def^{\ifmmode\sp\else\^{}\fi}\catcode`\%=\active\def%{\%}$\mathdefault{60}$}}%
\end{pgfscope}%
\begin{pgfscope}%
\pgfpathrectangle{\pgfqpoint{0.652043in}{0.499074in}}{\pgfqpoint{4.960000in}{3.696000in}}%
\pgfusepath{clip}%
\pgfsetbuttcap%
\pgfsetroundjoin%
\pgfsetlinewidth{0.803000pt}%
\definecolor{currentstroke}{rgb}{0.501961,0.501961,0.501961}%
\pgfsetstrokecolor{currentstroke}%
\pgfsetstrokeopacity{0.700000}%
\pgfsetdash{{0.800000pt}{1.320000pt}}{0.000000pt}%
\pgfpathmoveto{\pgfqpoint{3.908819in}{0.499074in}}%
\pgfpathlineto{\pgfqpoint{3.908819in}{4.195074in}}%
\pgfusepath{stroke}%
\end{pgfscope}%
\begin{pgfscope}%
\pgfsetbuttcap%
\pgfsetroundjoin%
\definecolor{currentfill}{rgb}{0.000000,0.000000,0.000000}%
\pgfsetfillcolor{currentfill}%
\pgfsetlinewidth{0.803000pt}%
\definecolor{currentstroke}{rgb}{0.000000,0.000000,0.000000}%
\pgfsetstrokecolor{currentstroke}%
\pgfsetdash{}{0pt}%
\pgfsys@defobject{currentmarker}{\pgfqpoint{0.000000in}{-0.048611in}}{\pgfqpoint{0.000000in}{0.000000in}}{%
\pgfpathmoveto{\pgfqpoint{0.000000in}{0.000000in}}%
\pgfpathlineto{\pgfqpoint{0.000000in}{-0.048611in}}%
\pgfusepath{stroke,fill}%
}%
\begin{pgfscope}%
\pgfsys@transformshift{3.908819in}{0.499074in}%
\pgfsys@useobject{currentmarker}{}%
\end{pgfscope}%
\end{pgfscope}%
\begin{pgfscope}%
\definecolor{textcolor}{rgb}{0.000000,0.000000,0.000000}%
\pgfsetstrokecolor{textcolor}%
\pgfsetfillcolor{textcolor}%
\pgftext[x=3.908819in,y=0.401852in,,top]{\color{textcolor}{\rmfamily\fontsize{9.000000}{10.800000}\selectfont\catcode`\^=\active\def^{\ifmmode\sp\else\^{}\fi}\catcode`\%=\active\def%{\%}$\mathdefault{80}$}}%
\end{pgfscope}%
\begin{pgfscope}%
\pgfpathrectangle{\pgfqpoint{0.652043in}{0.499074in}}{\pgfqpoint{4.960000in}{3.696000in}}%
\pgfusepath{clip}%
\pgfsetbuttcap%
\pgfsetroundjoin%
\pgfsetlinewidth{0.803000pt}%
\definecolor{currentstroke}{rgb}{0.501961,0.501961,0.501961}%
\pgfsetstrokecolor{currentstroke}%
\pgfsetstrokeopacity{0.700000}%
\pgfsetdash{{0.800000pt}{1.320000pt}}{0.000000pt}%
\pgfpathmoveto{\pgfqpoint{4.666649in}{0.499074in}}%
\pgfpathlineto{\pgfqpoint{4.666649in}{4.195074in}}%
\pgfusepath{stroke}%
\end{pgfscope}%
\begin{pgfscope}%
\pgfsetbuttcap%
\pgfsetroundjoin%
\definecolor{currentfill}{rgb}{0.000000,0.000000,0.000000}%
\pgfsetfillcolor{currentfill}%
\pgfsetlinewidth{0.803000pt}%
\definecolor{currentstroke}{rgb}{0.000000,0.000000,0.000000}%
\pgfsetstrokecolor{currentstroke}%
\pgfsetdash{}{0pt}%
\pgfsys@defobject{currentmarker}{\pgfqpoint{0.000000in}{-0.048611in}}{\pgfqpoint{0.000000in}{0.000000in}}{%
\pgfpathmoveto{\pgfqpoint{0.000000in}{0.000000in}}%
\pgfpathlineto{\pgfqpoint{0.000000in}{-0.048611in}}%
\pgfusepath{stroke,fill}%
}%
\begin{pgfscope}%
\pgfsys@transformshift{4.666649in}{0.499074in}%
\pgfsys@useobject{currentmarker}{}%
\end{pgfscope}%
\end{pgfscope}%
\begin{pgfscope}%
\definecolor{textcolor}{rgb}{0.000000,0.000000,0.000000}%
\pgfsetstrokecolor{textcolor}%
\pgfsetfillcolor{textcolor}%
\pgftext[x=4.666649in,y=0.401852in,,top]{\color{textcolor}{\rmfamily\fontsize{9.000000}{10.800000}\selectfont\catcode`\^=\active\def^{\ifmmode\sp\else\^{}\fi}\catcode`\%=\active\def%{\%}$\mathdefault{100}$}}%
\end{pgfscope}%
\begin{pgfscope}%
\pgfpathrectangle{\pgfqpoint{0.652043in}{0.499074in}}{\pgfqpoint{4.960000in}{3.696000in}}%
\pgfusepath{clip}%
\pgfsetbuttcap%
\pgfsetroundjoin%
\pgfsetlinewidth{0.803000pt}%
\definecolor{currentstroke}{rgb}{0.501961,0.501961,0.501961}%
\pgfsetstrokecolor{currentstroke}%
\pgfsetstrokeopacity{0.700000}%
\pgfsetdash{{0.800000pt}{1.320000pt}}{0.000000pt}%
\pgfpathmoveto{\pgfqpoint{5.424480in}{0.499074in}}%
\pgfpathlineto{\pgfqpoint{5.424480in}{4.195074in}}%
\pgfusepath{stroke}%
\end{pgfscope}%
\begin{pgfscope}%
\pgfsetbuttcap%
\pgfsetroundjoin%
\definecolor{currentfill}{rgb}{0.000000,0.000000,0.000000}%
\pgfsetfillcolor{currentfill}%
\pgfsetlinewidth{0.803000pt}%
\definecolor{currentstroke}{rgb}{0.000000,0.000000,0.000000}%
\pgfsetstrokecolor{currentstroke}%
\pgfsetdash{}{0pt}%
\pgfsys@defobject{currentmarker}{\pgfqpoint{0.000000in}{-0.048611in}}{\pgfqpoint{0.000000in}{0.000000in}}{%
\pgfpathmoveto{\pgfqpoint{0.000000in}{0.000000in}}%
\pgfpathlineto{\pgfqpoint{0.000000in}{-0.048611in}}%
\pgfusepath{stroke,fill}%
}%
\begin{pgfscope}%
\pgfsys@transformshift{5.424480in}{0.499074in}%
\pgfsys@useobject{currentmarker}{}%
\end{pgfscope}%
\end{pgfscope}%
\begin{pgfscope}%
\definecolor{textcolor}{rgb}{0.000000,0.000000,0.000000}%
\pgfsetstrokecolor{textcolor}%
\pgfsetfillcolor{textcolor}%
\pgftext[x=5.424480in,y=0.401852in,,top]{\color{textcolor}{\rmfamily\fontsize{9.000000}{10.800000}\selectfont\catcode`\^=\active\def^{\ifmmode\sp\else\^{}\fi}\catcode`\%=\active\def%{\%}$\mathdefault{120}$}}%
\end{pgfscope}%
\begin{pgfscope}%
\definecolor{textcolor}{rgb}{0.000000,0.000000,0.000000}%
\pgfsetstrokecolor{textcolor}%
\pgfsetfillcolor{textcolor}%
\pgftext[x=3.132043in,y=0.235185in,,top]{\color{textcolor}{\rmfamily\fontsize{11.000000}{13.200000}\selectfont\catcode`\^=\active\def^{\ifmmode\sp\else\^{}\fi}\catcode`\%=\active\def%{\%}X-axis}}%
\end{pgfscope}%
\begin{pgfscope}%
\pgfpathrectangle{\pgfqpoint{0.652043in}{0.499074in}}{\pgfqpoint{4.960000in}{3.696000in}}%
\pgfusepath{clip}%
\pgfsetbuttcap%
\pgfsetroundjoin%
\pgfsetlinewidth{0.803000pt}%
\definecolor{currentstroke}{rgb}{0.501961,0.501961,0.501961}%
\pgfsetstrokecolor{currentstroke}%
\pgfsetstrokeopacity{0.700000}%
\pgfsetdash{{0.800000pt}{1.320000pt}}{0.000000pt}%
\pgfpathmoveto{\pgfqpoint{0.652043in}{0.810505in}}%
\pgfpathlineto{\pgfqpoint{5.612043in}{0.810505in}}%
\pgfusepath{stroke}%
\end{pgfscope}%
\begin{pgfscope}%
\pgfsetbuttcap%
\pgfsetroundjoin%
\definecolor{currentfill}{rgb}{0.000000,0.000000,0.000000}%
\pgfsetfillcolor{currentfill}%
\pgfsetlinewidth{0.803000pt}%
\definecolor{currentstroke}{rgb}{0.000000,0.000000,0.000000}%
\pgfsetstrokecolor{currentstroke}%
\pgfsetdash{}{0pt}%
\pgfsys@defobject{currentmarker}{\pgfqpoint{-0.048611in}{0.000000in}}{\pgfqpoint{-0.000000in}{0.000000in}}{%
\pgfpathmoveto{\pgfqpoint{-0.000000in}{0.000000in}}%
\pgfpathlineto{\pgfqpoint{-0.048611in}{0.000000in}}%
\pgfusepath{stroke,fill}%
}%
\begin{pgfscope}%
\pgfsys@transformshift{0.652043in}{0.810505in}%
\pgfsys@useobject{currentmarker}{}%
\end{pgfscope}%
\end{pgfscope}%
\begin{pgfscope}%
\definecolor{textcolor}{rgb}{0.000000,0.000000,0.000000}%
\pgfsetstrokecolor{textcolor}%
\pgfsetfillcolor{textcolor}%
\pgftext[x=0.290741in, y=0.767102in, left, base]{\color{textcolor}{\rmfamily\fontsize{9.000000}{10.800000}\selectfont\catcode`\^=\active\def^{\ifmmode\sp\else\^{}\fi}\catcode`\%=\active\def%{\%}$\mathdefault{\ensuremath{-}1.0}$}}%
\end{pgfscope}%
\begin{pgfscope}%
\pgfpathrectangle{\pgfqpoint{0.652043in}{0.499074in}}{\pgfqpoint{4.960000in}{3.696000in}}%
\pgfusepath{clip}%
\pgfsetbuttcap%
\pgfsetroundjoin%
\pgfsetlinewidth{0.803000pt}%
\definecolor{currentstroke}{rgb}{0.501961,0.501961,0.501961}%
\pgfsetstrokecolor{currentstroke}%
\pgfsetstrokeopacity{0.700000}%
\pgfsetdash{{0.800000pt}{1.320000pt}}{0.000000pt}%
\pgfpathmoveto{\pgfqpoint{0.652043in}{1.531644in}}%
\pgfpathlineto{\pgfqpoint{5.612043in}{1.531644in}}%
\pgfusepath{stroke}%
\end{pgfscope}%
\begin{pgfscope}%
\pgfsetbuttcap%
\pgfsetroundjoin%
\definecolor{currentfill}{rgb}{0.000000,0.000000,0.000000}%
\pgfsetfillcolor{currentfill}%
\pgfsetlinewidth{0.803000pt}%
\definecolor{currentstroke}{rgb}{0.000000,0.000000,0.000000}%
\pgfsetstrokecolor{currentstroke}%
\pgfsetdash{}{0pt}%
\pgfsys@defobject{currentmarker}{\pgfqpoint{-0.048611in}{0.000000in}}{\pgfqpoint{-0.000000in}{0.000000in}}{%
\pgfpathmoveto{\pgfqpoint{-0.000000in}{0.000000in}}%
\pgfpathlineto{\pgfqpoint{-0.048611in}{0.000000in}}%
\pgfusepath{stroke,fill}%
}%
\begin{pgfscope}%
\pgfsys@transformshift{0.652043in}{1.531644in}%
\pgfsys@useobject{currentmarker}{}%
\end{pgfscope}%
\end{pgfscope}%
\begin{pgfscope}%
\definecolor{textcolor}{rgb}{0.000000,0.000000,0.000000}%
\pgfsetstrokecolor{textcolor}%
\pgfsetfillcolor{textcolor}%
\pgftext[x=0.290741in, y=1.488241in, left, base]{\color{textcolor}{\rmfamily\fontsize{9.000000}{10.800000}\selectfont\catcode`\^=\active\def^{\ifmmode\sp\else\^{}\fi}\catcode`\%=\active\def%{\%}$\mathdefault{\ensuremath{-}0.5}$}}%
\end{pgfscope}%
\begin{pgfscope}%
\pgfpathrectangle{\pgfqpoint{0.652043in}{0.499074in}}{\pgfqpoint{4.960000in}{3.696000in}}%
\pgfusepath{clip}%
\pgfsetbuttcap%
\pgfsetroundjoin%
\pgfsetlinewidth{0.803000pt}%
\definecolor{currentstroke}{rgb}{0.501961,0.501961,0.501961}%
\pgfsetstrokecolor{currentstroke}%
\pgfsetstrokeopacity{0.700000}%
\pgfsetdash{{0.800000pt}{1.320000pt}}{0.000000pt}%
\pgfpathmoveto{\pgfqpoint{0.652043in}{2.252782in}}%
\pgfpathlineto{\pgfqpoint{5.612043in}{2.252782in}}%
\pgfusepath{stroke}%
\end{pgfscope}%
\begin{pgfscope}%
\pgfsetbuttcap%
\pgfsetroundjoin%
\definecolor{currentfill}{rgb}{0.000000,0.000000,0.000000}%
\pgfsetfillcolor{currentfill}%
\pgfsetlinewidth{0.803000pt}%
\definecolor{currentstroke}{rgb}{0.000000,0.000000,0.000000}%
\pgfsetstrokecolor{currentstroke}%
\pgfsetdash{}{0pt}%
\pgfsys@defobject{currentmarker}{\pgfqpoint{-0.048611in}{0.000000in}}{\pgfqpoint{-0.000000in}{0.000000in}}{%
\pgfpathmoveto{\pgfqpoint{-0.000000in}{0.000000in}}%
\pgfpathlineto{\pgfqpoint{-0.048611in}{0.000000in}}%
\pgfusepath{stroke,fill}%
}%
\begin{pgfscope}%
\pgfsys@transformshift{0.652043in}{2.252782in}%
\pgfsys@useobject{currentmarker}{}%
\end{pgfscope}%
\end{pgfscope}%
\begin{pgfscope}%
\definecolor{textcolor}{rgb}{0.000000,0.000000,0.000000}%
\pgfsetstrokecolor{textcolor}%
\pgfsetfillcolor{textcolor}%
\pgftext[x=0.390663in, y=2.209380in, left, base]{\color{textcolor}{\rmfamily\fontsize{9.000000}{10.800000}\selectfont\catcode`\^=\active\def^{\ifmmode\sp\else\^{}\fi}\catcode`\%=\active\def%{\%}$\mathdefault{0.0}$}}%
\end{pgfscope}%
\begin{pgfscope}%
\pgfpathrectangle{\pgfqpoint{0.652043in}{0.499074in}}{\pgfqpoint{4.960000in}{3.696000in}}%
\pgfusepath{clip}%
\pgfsetbuttcap%
\pgfsetroundjoin%
\pgfsetlinewidth{0.803000pt}%
\definecolor{currentstroke}{rgb}{0.501961,0.501961,0.501961}%
\pgfsetstrokecolor{currentstroke}%
\pgfsetstrokeopacity{0.700000}%
\pgfsetdash{{0.800000pt}{1.320000pt}}{0.000000pt}%
\pgfpathmoveto{\pgfqpoint{0.652043in}{2.973921in}}%
\pgfpathlineto{\pgfqpoint{5.612043in}{2.973921in}}%
\pgfusepath{stroke}%
\end{pgfscope}%
\begin{pgfscope}%
\pgfsetbuttcap%
\pgfsetroundjoin%
\definecolor{currentfill}{rgb}{0.000000,0.000000,0.000000}%
\pgfsetfillcolor{currentfill}%
\pgfsetlinewidth{0.803000pt}%
\definecolor{currentstroke}{rgb}{0.000000,0.000000,0.000000}%
\pgfsetstrokecolor{currentstroke}%
\pgfsetdash{}{0pt}%
\pgfsys@defobject{currentmarker}{\pgfqpoint{-0.048611in}{0.000000in}}{\pgfqpoint{-0.000000in}{0.000000in}}{%
\pgfpathmoveto{\pgfqpoint{-0.000000in}{0.000000in}}%
\pgfpathlineto{\pgfqpoint{-0.048611in}{0.000000in}}%
\pgfusepath{stroke,fill}%
}%
\begin{pgfscope}%
\pgfsys@transformshift{0.652043in}{2.973921in}%
\pgfsys@useobject{currentmarker}{}%
\end{pgfscope}%
\end{pgfscope}%
\begin{pgfscope}%
\definecolor{textcolor}{rgb}{0.000000,0.000000,0.000000}%
\pgfsetstrokecolor{textcolor}%
\pgfsetfillcolor{textcolor}%
\pgftext[x=0.390663in, y=2.930518in, left, base]{\color{textcolor}{\rmfamily\fontsize{9.000000}{10.800000}\selectfont\catcode`\^=\active\def^{\ifmmode\sp\else\^{}\fi}\catcode`\%=\active\def%{\%}$\mathdefault{0.5}$}}%
\end{pgfscope}%
\begin{pgfscope}%
\pgfpathrectangle{\pgfqpoint{0.652043in}{0.499074in}}{\pgfqpoint{4.960000in}{3.696000in}}%
\pgfusepath{clip}%
\pgfsetbuttcap%
\pgfsetroundjoin%
\pgfsetlinewidth{0.803000pt}%
\definecolor{currentstroke}{rgb}{0.501961,0.501961,0.501961}%
\pgfsetstrokecolor{currentstroke}%
\pgfsetstrokeopacity{0.700000}%
\pgfsetdash{{0.800000pt}{1.320000pt}}{0.000000pt}%
\pgfpathmoveto{\pgfqpoint{0.652043in}{3.695060in}}%
\pgfpathlineto{\pgfqpoint{5.612043in}{3.695060in}}%
\pgfusepath{stroke}%
\end{pgfscope}%
\begin{pgfscope}%
\pgfsetbuttcap%
\pgfsetroundjoin%
\definecolor{currentfill}{rgb}{0.000000,0.000000,0.000000}%
\pgfsetfillcolor{currentfill}%
\pgfsetlinewidth{0.803000pt}%
\definecolor{currentstroke}{rgb}{0.000000,0.000000,0.000000}%
\pgfsetstrokecolor{currentstroke}%
\pgfsetdash{}{0pt}%
\pgfsys@defobject{currentmarker}{\pgfqpoint{-0.048611in}{0.000000in}}{\pgfqpoint{-0.000000in}{0.000000in}}{%
\pgfpathmoveto{\pgfqpoint{-0.000000in}{0.000000in}}%
\pgfpathlineto{\pgfqpoint{-0.048611in}{0.000000in}}%
\pgfusepath{stroke,fill}%
}%
\begin{pgfscope}%
\pgfsys@transformshift{0.652043in}{3.695060in}%
\pgfsys@useobject{currentmarker}{}%
\end{pgfscope}%
\end{pgfscope}%
\begin{pgfscope}%
\definecolor{textcolor}{rgb}{0.000000,0.000000,0.000000}%
\pgfsetstrokecolor{textcolor}%
\pgfsetfillcolor{textcolor}%
\pgftext[x=0.390663in, y=3.651657in, left, base]{\color{textcolor}{\rmfamily\fontsize{9.000000}{10.800000}\selectfont\catcode`\^=\active\def^{\ifmmode\sp\else\^{}\fi}\catcode`\%=\active\def%{\%}$\mathdefault{1.0}$}}%
\end{pgfscope}%
\begin{pgfscope}%
\definecolor{textcolor}{rgb}{0.000000,0.000000,0.000000}%
\pgfsetstrokecolor{textcolor}%
\pgfsetfillcolor{textcolor}%
\pgftext[x=0.235185in,y=2.347074in,,bottom,rotate=90.000000]{\color{textcolor}{\rmfamily\fontsize{11.000000}{13.200000}\selectfont\catcode`\^=\active\def^{\ifmmode\sp\else\^{}\fi}\catcode`\%=\active\def%{\%}n State Transition Approx.}}%
\end{pgfscope}%
\begin{pgfscope}%
\pgfpathrectangle{\pgfqpoint{0.652043in}{0.499074in}}{\pgfqpoint{4.960000in}{3.696000in}}%
\pgfusepath{clip}%
\pgfsetrectcap%
\pgfsetroundjoin%
\pgfsetlinewidth{1.505625pt}%
\definecolor{currentstroke}{rgb}{0.121569,0.466667,0.705882}%
\pgfsetstrokecolor{currentstroke}%
\pgfsetdash{}{0pt}%
\pgfpathmoveto{\pgfqpoint{0.877497in}{2.252782in}}%
\pgfpathlineto{\pgfqpoint{0.915389in}{2.255010in}}%
\pgfpathlineto{\pgfqpoint{0.953281in}{2.256696in}}%
\pgfpathlineto{\pgfqpoint{0.991172in}{2.258188in}}%
\pgfpathlineto{\pgfqpoint{1.029064in}{2.259513in}}%
\pgfpathlineto{\pgfqpoint{1.066955in}{2.260673in}}%
\pgfpathlineto{\pgfqpoint{1.104847in}{2.261680in}}%
\pgfpathlineto{\pgfqpoint{1.142738in}{2.262556in}}%
\pgfpathlineto{\pgfqpoint{1.180630in}{2.263337in}}%
\pgfpathlineto{\pgfqpoint{1.218521in}{2.264069in}}%
\pgfpathlineto{\pgfqpoint{1.256413in}{2.264809in}}%
\pgfpathlineto{\pgfqpoint{1.294304in}{2.265625in}}%
\pgfpathlineto{\pgfqpoint{1.332196in}{2.266594in}}%
\pgfpathlineto{\pgfqpoint{1.370087in}{2.267804in}}%
\pgfpathlineto{\pgfqpoint{1.407979in}{2.269351in}}%
\pgfpathlineto{\pgfqpoint{1.445870in}{2.271333in}}%
\pgfpathlineto{\pgfqpoint{1.483762in}{2.273845in}}%
\pgfpathlineto{\pgfqpoint{1.521653in}{2.276968in}}%
\pgfpathlineto{\pgfqpoint{1.559545in}{2.280756in}}%
\pgfpathlineto{\pgfqpoint{1.597436in}{2.285226in}}%
\pgfpathlineto{\pgfqpoint{1.635328in}{2.290340in}}%
\pgfpathlineto{\pgfqpoint{1.673219in}{2.296001in}}%
\pgfpathlineto{\pgfqpoint{1.711111in}{2.302047in}}%
\pgfpathlineto{\pgfqpoint{1.749002in}{2.308256in}}%
\pgfpathlineto{\pgfqpoint{1.786894in}{2.314358in}}%
\pgfpathlineto{\pgfqpoint{1.824785in}{2.320058in}}%
\pgfpathlineto{\pgfqpoint{1.862677in}{2.325057in}}%
\pgfpathlineto{\pgfqpoint{1.900569in}{2.329075in}}%
\pgfpathlineto{\pgfqpoint{1.938460in}{2.331879in}}%
\pgfpathlineto{\pgfqpoint{1.976352in}{2.333313in}}%
\pgfpathlineto{\pgfqpoint{2.014243in}{2.333326in}}%
\pgfpathlineto{\pgfqpoint{2.052135in}{2.331972in}}%
\pgfpathlineto{\pgfqpoint{2.090026in}{2.329352in}}%
\pgfpathlineto{\pgfqpoint{2.127918in}{2.325544in}}%
\pgfpathlineto{\pgfqpoint{2.165809in}{2.320556in}}%
\pgfpathlineto{\pgfqpoint{2.203701in}{2.314348in}}%
\pgfpathlineto{\pgfqpoint{2.241592in}{2.306882in}}%
\pgfpathlineto{\pgfqpoint{2.279484in}{2.298197in}}%
\pgfpathlineto{\pgfqpoint{2.317375in}{2.288483in}}%
\pgfpathlineto{\pgfqpoint{2.355267in}{2.278019in}}%
\pgfpathlineto{\pgfqpoint{2.393158in}{2.266997in}}%
\pgfpathlineto{\pgfqpoint{2.431050in}{2.255523in}}%
\pgfpathlineto{\pgfqpoint{2.468941in}{2.243792in}}%
\pgfpathlineto{\pgfqpoint{2.506833in}{2.232020in}}%
\pgfpathlineto{\pgfqpoint{2.544724in}{2.220333in}}%
\pgfpathlineto{\pgfqpoint{2.582616in}{2.208995in}}%
\pgfpathlineto{\pgfqpoint{2.620507in}{2.198296in}}%
\pgfpathlineto{\pgfqpoint{2.658399in}{2.187489in}}%
\pgfpathlineto{\pgfqpoint{2.696290in}{2.176733in}}%
\pgfpathlineto{\pgfqpoint{2.734182in}{2.166149in}}%
\pgfpathlineto{\pgfqpoint{2.772073in}{2.155828in}}%
\pgfpathlineto{\pgfqpoint{2.809965in}{2.145837in}}%
\pgfpathlineto{\pgfqpoint{2.847857in}{2.136227in}}%
\pgfpathlineto{\pgfqpoint{2.885748in}{2.127036in}}%
\pgfpathlineto{\pgfqpoint{2.923640in}{2.118289in}}%
\pgfpathlineto{\pgfqpoint{2.961531in}{2.110008in}}%
\pgfpathlineto{\pgfqpoint{2.999423in}{2.102204in}}%
\pgfpathlineto{\pgfqpoint{3.037314in}{2.094887in}}%
\pgfpathlineto{\pgfqpoint{3.075206in}{2.088062in}}%
\pgfpathlineto{\pgfqpoint{3.113097in}{2.081732in}}%
\pgfpathlineto{\pgfqpoint{3.150989in}{2.075899in}}%
\pgfpathlineto{\pgfqpoint{3.188880in}{2.070563in}}%
\pgfpathlineto{\pgfqpoint{3.226772in}{2.065724in}}%
\pgfpathlineto{\pgfqpoint{3.264663in}{2.061381in}}%
\pgfpathlineto{\pgfqpoint{3.302555in}{2.057534in}}%
\pgfpathlineto{\pgfqpoint{3.340446in}{2.054181in}}%
\pgfpathlineto{\pgfqpoint{3.378338in}{2.051322in}}%
\pgfpathlineto{\pgfqpoint{3.416229in}{2.048957in}}%
\pgfpathlineto{\pgfqpoint{3.454121in}{2.047088in}}%
\pgfpathlineto{\pgfqpoint{3.492012in}{2.045716in}}%
\pgfpathlineto{\pgfqpoint{3.529904in}{2.044841in}}%
\pgfpathlineto{\pgfqpoint{3.567795in}{2.044469in}}%
\pgfpathlineto{\pgfqpoint{3.605687in}{2.044601in}}%
\pgfpathlineto{\pgfqpoint{3.643578in}{2.045244in}}%
\pgfpathlineto{\pgfqpoint{3.681470in}{2.046402in}}%
\pgfpathlineto{\pgfqpoint{3.719361in}{2.048082in}}%
\pgfpathlineto{\pgfqpoint{3.757253in}{2.050292in}}%
\pgfpathlineto{\pgfqpoint{3.795145in}{2.053041in}}%
\pgfpathlineto{\pgfqpoint{3.833036in}{2.056337in}}%
\pgfpathlineto{\pgfqpoint{3.870928in}{2.060193in}}%
\pgfpathlineto{\pgfqpoint{3.908819in}{2.064619in}}%
\pgfpathlineto{\pgfqpoint{3.946711in}{2.069629in}}%
\pgfpathlineto{\pgfqpoint{3.984602in}{2.075237in}}%
\pgfpathlineto{\pgfqpoint{4.022494in}{2.081458in}}%
\pgfpathlineto{\pgfqpoint{4.060385in}{2.088309in}}%
\pgfpathlineto{\pgfqpoint{4.098277in}{2.095806in}}%
\pgfpathlineto{\pgfqpoint{4.136168in}{2.103969in}}%
\pgfpathlineto{\pgfqpoint{4.174060in}{2.112818in}}%
\pgfpathlineto{\pgfqpoint{4.211951in}{2.122373in}}%
\pgfpathlineto{\pgfqpoint{4.249843in}{2.132656in}}%
\pgfpathlineto{\pgfqpoint{4.287734in}{2.143691in}}%
\pgfpathlineto{\pgfqpoint{4.325626in}{2.155501in}}%
\pgfpathlineto{\pgfqpoint{4.363517in}{2.168111in}}%
\pgfpathlineto{\pgfqpoint{4.401409in}{2.181548in}}%
\pgfpathlineto{\pgfqpoint{4.439300in}{2.195838in}}%
\pgfpathlineto{\pgfqpoint{4.477192in}{2.211009in}}%
\pgfpathlineto{\pgfqpoint{4.515083in}{2.227090in}}%
\pgfpathlineto{\pgfqpoint{4.552975in}{2.244108in}}%
\pgfpathlineto{\pgfqpoint{4.590866in}{2.262094in}}%
\pgfpathlineto{\pgfqpoint{4.628758in}{2.281077in}}%
\pgfpathlineto{\pgfqpoint{4.666649in}{2.301086in}}%
\pgfpathlineto{\pgfqpoint{4.704541in}{2.322149in}}%
\pgfpathlineto{\pgfqpoint{4.742433in}{2.344296in}}%
\pgfpathlineto{\pgfqpoint{4.780324in}{2.367554in}}%
\pgfpathlineto{\pgfqpoint{4.818216in}{2.391948in}}%
\pgfpathlineto{\pgfqpoint{4.856107in}{2.417502in}}%
\pgfpathlineto{\pgfqpoint{4.893999in}{2.444237in}}%
\pgfpathlineto{\pgfqpoint{4.931890in}{2.472171in}}%
\pgfpathlineto{\pgfqpoint{4.969782in}{2.501317in}}%
\pgfpathlineto{\pgfqpoint{5.007673in}{2.531682in}}%
\pgfpathlineto{\pgfqpoint{5.045565in}{2.563268in}}%
\pgfpathlineto{\pgfqpoint{5.083456in}{2.596065in}}%
\pgfpathlineto{\pgfqpoint{5.121348in}{2.630057in}}%
\pgfpathlineto{\pgfqpoint{5.159239in}{2.665213in}}%
\pgfpathlineto{\pgfqpoint{5.197131in}{2.701486in}}%
\pgfpathlineto{\pgfqpoint{5.235022in}{2.738813in}}%
\pgfpathlineto{\pgfqpoint{5.272914in}{2.777105in}}%
\pgfpathlineto{\pgfqpoint{5.310805in}{2.816250in}}%
\pgfpathlineto{\pgfqpoint{5.348697in}{2.856101in}}%
\pgfpathlineto{\pgfqpoint{5.386588in}{2.896472in}}%
\pgfusepath{stroke}%
\end{pgfscope}%
\begin{pgfscope}%
\pgfpathrectangle{\pgfqpoint{0.652043in}{0.499074in}}{\pgfqpoint{4.960000in}{3.696000in}}%
\pgfusepath{clip}%
\pgfsetrectcap%
\pgfsetroundjoin%
\pgfsetlinewidth{1.505625pt}%
\definecolor{currentstroke}{rgb}{1.000000,0.498039,0.054902}%
\pgfsetstrokecolor{currentstroke}%
\pgfsetdash{}{0pt}%
\pgfpathmoveto{\pgfqpoint{0.877497in}{2.252782in}}%
\pgfpathlineto{\pgfqpoint{0.915389in}{2.255987in}}%
\pgfpathlineto{\pgfqpoint{0.953281in}{2.259503in}}%
\pgfpathlineto{\pgfqpoint{0.991172in}{2.263301in}}%
\pgfpathlineto{\pgfqpoint{1.029064in}{2.267366in}}%
\pgfpathlineto{\pgfqpoint{1.066955in}{2.271679in}}%
\pgfpathlineto{\pgfqpoint{1.104847in}{2.276218in}}%
\pgfpathlineto{\pgfqpoint{1.142738in}{2.280957in}}%
\pgfpathlineto{\pgfqpoint{1.180630in}{2.285865in}}%
\pgfpathlineto{\pgfqpoint{1.218521in}{2.290905in}}%
\pgfpathlineto{\pgfqpoint{1.256413in}{2.296037in}}%
\pgfpathlineto{\pgfqpoint{1.294304in}{2.301218in}}%
\pgfpathlineto{\pgfqpoint{1.332196in}{2.306400in}}%
\pgfpathlineto{\pgfqpoint{1.370087in}{2.311532in}}%
\pgfpathlineto{\pgfqpoint{1.407979in}{2.316560in}}%
\pgfpathlineto{\pgfqpoint{1.445870in}{2.321428in}}%
\pgfpathlineto{\pgfqpoint{1.483762in}{2.326075in}}%
\pgfpathlineto{\pgfqpoint{1.521653in}{2.330443in}}%
\pgfpathlineto{\pgfqpoint{1.559545in}{2.334471in}}%
\pgfpathlineto{\pgfqpoint{1.597436in}{2.338098in}}%
\pgfpathlineto{\pgfqpoint{1.635328in}{2.341268in}}%
\pgfpathlineto{\pgfqpoint{1.673219in}{2.343925in}}%
\pgfpathlineto{\pgfqpoint{1.711111in}{2.346018in}}%
\pgfpathlineto{\pgfqpoint{1.749002in}{2.347503in}}%
\pgfpathlineto{\pgfqpoint{1.786894in}{2.348339in}}%
\pgfpathlineto{\pgfqpoint{1.824785in}{2.348492in}}%
\pgfpathlineto{\pgfqpoint{1.862677in}{2.347936in}}%
\pgfpathlineto{\pgfqpoint{1.900569in}{2.346652in}}%
\pgfpathlineto{\pgfqpoint{1.938460in}{2.344629in}}%
\pgfpathlineto{\pgfqpoint{1.976352in}{2.341865in}}%
\pgfpathlineto{\pgfqpoint{2.014243in}{2.338364in}}%
\pgfpathlineto{\pgfqpoint{2.052135in}{2.334136in}}%
\pgfpathlineto{\pgfqpoint{2.090026in}{2.329187in}}%
\pgfpathlineto{\pgfqpoint{2.127918in}{2.323526in}}%
\pgfpathlineto{\pgfqpoint{2.165809in}{2.317159in}}%
\pgfpathlineto{\pgfqpoint{2.203701in}{2.310096in}}%
\pgfpathlineto{\pgfqpoint{2.241592in}{2.302353in}}%
\pgfpathlineto{\pgfqpoint{2.279484in}{2.293959in}}%
\pgfpathlineto{\pgfqpoint{2.317375in}{2.284947in}}%
\pgfpathlineto{\pgfqpoint{2.355267in}{2.275360in}}%
\pgfpathlineto{\pgfqpoint{2.393158in}{2.265241in}}%
\pgfpathlineto{\pgfqpoint{2.431050in}{2.254640in}}%
\pgfpathlineto{\pgfqpoint{2.468941in}{2.243622in}}%
\pgfpathlineto{\pgfqpoint{2.506833in}{2.232268in}}%
\pgfpathlineto{\pgfqpoint{2.544724in}{2.220681in}}%
\pgfpathlineto{\pgfqpoint{2.582616in}{2.208995in}}%
\pgfpathlineto{\pgfqpoint{2.620507in}{2.198296in}}%
\pgfpathlineto{\pgfqpoint{2.658399in}{2.187489in}}%
\pgfpathlineto{\pgfqpoint{2.696290in}{2.176733in}}%
\pgfpathlineto{\pgfqpoint{2.734182in}{2.166149in}}%
\pgfpathlineto{\pgfqpoint{2.772073in}{2.155828in}}%
\pgfpathlineto{\pgfqpoint{2.809965in}{2.145837in}}%
\pgfpathlineto{\pgfqpoint{2.847857in}{2.136227in}}%
\pgfpathlineto{\pgfqpoint{2.885748in}{2.127036in}}%
\pgfpathlineto{\pgfqpoint{2.923640in}{2.118289in}}%
\pgfpathlineto{\pgfqpoint{2.961531in}{2.110008in}}%
\pgfpathlineto{\pgfqpoint{2.999423in}{2.102204in}}%
\pgfpathlineto{\pgfqpoint{3.037314in}{2.094887in}}%
\pgfpathlineto{\pgfqpoint{3.075206in}{2.088062in}}%
\pgfpathlineto{\pgfqpoint{3.113097in}{2.081732in}}%
\pgfpathlineto{\pgfqpoint{3.150989in}{2.075899in}}%
\pgfpathlineto{\pgfqpoint{3.188880in}{2.070563in}}%
\pgfpathlineto{\pgfqpoint{3.226772in}{2.065724in}}%
\pgfpathlineto{\pgfqpoint{3.264663in}{2.061381in}}%
\pgfpathlineto{\pgfqpoint{3.302555in}{2.057534in}}%
\pgfpathlineto{\pgfqpoint{3.340446in}{2.054181in}}%
\pgfpathlineto{\pgfqpoint{3.378338in}{2.051322in}}%
\pgfpathlineto{\pgfqpoint{3.416229in}{2.048957in}}%
\pgfpathlineto{\pgfqpoint{3.454121in}{2.047088in}}%
\pgfpathlineto{\pgfqpoint{3.492012in}{2.045716in}}%
\pgfpathlineto{\pgfqpoint{3.529904in}{2.044841in}}%
\pgfpathlineto{\pgfqpoint{3.567795in}{2.044469in}}%
\pgfpathlineto{\pgfqpoint{3.605687in}{2.044601in}}%
\pgfpathlineto{\pgfqpoint{3.643578in}{2.045244in}}%
\pgfpathlineto{\pgfqpoint{3.681470in}{2.046402in}}%
\pgfpathlineto{\pgfqpoint{3.719361in}{2.048082in}}%
\pgfpathlineto{\pgfqpoint{3.757253in}{2.050292in}}%
\pgfpathlineto{\pgfqpoint{3.795145in}{2.053041in}}%
\pgfpathlineto{\pgfqpoint{3.833036in}{2.056337in}}%
\pgfpathlineto{\pgfqpoint{3.870928in}{2.060193in}}%
\pgfpathlineto{\pgfqpoint{3.908819in}{2.064619in}}%
\pgfpathlineto{\pgfqpoint{3.946711in}{2.069629in}}%
\pgfpathlineto{\pgfqpoint{3.984602in}{2.075237in}}%
\pgfpathlineto{\pgfqpoint{4.022494in}{2.081458in}}%
\pgfpathlineto{\pgfqpoint{4.060385in}{2.088309in}}%
\pgfpathlineto{\pgfqpoint{4.098277in}{2.095806in}}%
\pgfpathlineto{\pgfqpoint{4.136168in}{2.103969in}}%
\pgfpathlineto{\pgfqpoint{4.174060in}{2.112818in}}%
\pgfpathlineto{\pgfqpoint{4.211951in}{2.122373in}}%
\pgfpathlineto{\pgfqpoint{4.249843in}{2.132656in}}%
\pgfpathlineto{\pgfqpoint{4.287734in}{2.143691in}}%
\pgfpathlineto{\pgfqpoint{4.325626in}{2.155501in}}%
\pgfpathlineto{\pgfqpoint{4.363517in}{2.168111in}}%
\pgfpathlineto{\pgfqpoint{4.401409in}{2.181548in}}%
\pgfpathlineto{\pgfqpoint{4.439300in}{2.195838in}}%
\pgfpathlineto{\pgfqpoint{4.477192in}{2.211009in}}%
\pgfpathlineto{\pgfqpoint{4.515083in}{2.227090in}}%
\pgfpathlineto{\pgfqpoint{4.552975in}{2.244108in}}%
\pgfpathlineto{\pgfqpoint{4.590866in}{2.262094in}}%
\pgfpathlineto{\pgfqpoint{4.628758in}{2.281077in}}%
\pgfpathlineto{\pgfqpoint{4.666649in}{2.301086in}}%
\pgfpathlineto{\pgfqpoint{4.704541in}{2.322149in}}%
\pgfpathlineto{\pgfqpoint{4.742433in}{2.344296in}}%
\pgfpathlineto{\pgfqpoint{4.780324in}{2.367554in}}%
\pgfpathlineto{\pgfqpoint{4.818216in}{2.391948in}}%
\pgfpathlineto{\pgfqpoint{4.856107in}{2.417502in}}%
\pgfpathlineto{\pgfqpoint{4.893999in}{2.444237in}}%
\pgfpathlineto{\pgfqpoint{4.931890in}{2.472171in}}%
\pgfpathlineto{\pgfqpoint{4.969782in}{2.501317in}}%
\pgfpathlineto{\pgfqpoint{5.007673in}{2.531682in}}%
\pgfpathlineto{\pgfqpoint{5.045565in}{2.563268in}}%
\pgfpathlineto{\pgfqpoint{5.083456in}{2.596065in}}%
\pgfpathlineto{\pgfqpoint{5.121348in}{2.630057in}}%
\pgfpathlineto{\pgfqpoint{5.159239in}{2.665213in}}%
\pgfpathlineto{\pgfqpoint{5.197131in}{2.701486in}}%
\pgfpathlineto{\pgfqpoint{5.235022in}{2.738813in}}%
\pgfpathlineto{\pgfqpoint{5.272914in}{2.777105in}}%
\pgfpathlineto{\pgfqpoint{5.310805in}{2.816250in}}%
\pgfpathlineto{\pgfqpoint{5.348697in}{2.856101in}}%
\pgfpathlineto{\pgfqpoint{5.386588in}{2.896472in}}%
\pgfusepath{stroke}%
\end{pgfscope}%
\begin{pgfscope}%
\pgfpathrectangle{\pgfqpoint{0.652043in}{0.499074in}}{\pgfqpoint{4.960000in}{3.696000in}}%
\pgfusepath{clip}%
\pgfsetbuttcap%
\pgfsetroundjoin%
\pgfsetlinewidth{1.505625pt}%
\definecolor{currentstroke}{rgb}{1.000000,0.000000,0.000000}%
\pgfsetstrokecolor{currentstroke}%
\pgfsetdash{{1.500000pt}{2.475000pt}}{0.000000pt}%
\pgfpathmoveto{\pgfqpoint{0.877497in}{2.252782in}}%
\pgfpathlineto{\pgfqpoint{0.915389in}{2.180270in}}%
\pgfpathlineto{\pgfqpoint{0.953281in}{2.107677in}}%
\pgfpathlineto{\pgfqpoint{0.991172in}{2.034996in}}%
\pgfpathlineto{\pgfqpoint{1.029064in}{1.962226in}}%
\pgfpathlineto{\pgfqpoint{1.066955in}{1.889367in}}%
\pgfpathlineto{\pgfqpoint{1.104847in}{1.816417in}}%
\pgfpathlineto{\pgfqpoint{1.142738in}{1.743370in}}%
\pgfpathlineto{\pgfqpoint{1.180630in}{1.670218in}}%
\pgfpathlineto{\pgfqpoint{1.218521in}{1.596952in}}%
\pgfpathlineto{\pgfqpoint{1.256413in}{1.523561in}}%
\pgfpathlineto{\pgfqpoint{1.294304in}{1.450034in}}%
\pgfpathlineto{\pgfqpoint{1.332196in}{1.376360in}}%
\pgfpathlineto{\pgfqpoint{1.370087in}{1.302528in}}%
\pgfpathlineto{\pgfqpoint{1.407979in}{1.228526in}}%
\pgfpathlineto{\pgfqpoint{1.445870in}{1.154343in}}%
\pgfpathlineto{\pgfqpoint{1.483762in}{1.079967in}}%
\pgfpathlineto{\pgfqpoint{1.521653in}{1.005389in}}%
\pgfpathlineto{\pgfqpoint{1.559545in}{0.930599in}}%
\pgfpathlineto{\pgfqpoint{1.597436in}{0.855590in}}%
\pgfpathlineto{\pgfqpoint{1.635328in}{0.780355in}}%
\pgfpathlineto{\pgfqpoint{1.673219in}{0.704892in}}%
\pgfpathlineto{\pgfqpoint{1.711111in}{0.667074in}}%
\pgfpathlineto{\pgfqpoint{1.749002in}{0.740957in}}%
\pgfpathlineto{\pgfqpoint{1.786894in}{0.813926in}}%
\pgfpathlineto{\pgfqpoint{1.824785in}{0.885998in}}%
\pgfpathlineto{\pgfqpoint{1.862677in}{0.957200in}}%
\pgfpathlineto{\pgfqpoint{1.900569in}{1.027568in}}%
\pgfpathlineto{\pgfqpoint{1.938460in}{1.097144in}}%
\pgfpathlineto{\pgfqpoint{1.976352in}{1.165979in}}%
\pgfpathlineto{\pgfqpoint{2.014243in}{1.234129in}}%
\pgfpathlineto{\pgfqpoint{2.052135in}{1.301651in}}%
\pgfpathlineto{\pgfqpoint{2.090026in}{1.368593in}}%
\pgfpathlineto{\pgfqpoint{2.127918in}{1.435002in}}%
\pgfpathlineto{\pgfqpoint{2.165809in}{1.500919in}}%
\pgfpathlineto{\pgfqpoint{2.203701in}{1.566384in}}%
\pgfpathlineto{\pgfqpoint{2.241592in}{1.631444in}}%
\pgfpathlineto{\pgfqpoint{2.279484in}{1.696154in}}%
\pgfpathlineto{\pgfqpoint{2.317375in}{1.760573in}}%
\pgfpathlineto{\pgfqpoint{2.355267in}{1.824764in}}%
\pgfpathlineto{\pgfqpoint{2.393158in}{1.888786in}}%
\pgfpathlineto{\pgfqpoint{2.431050in}{1.952703in}}%
\pgfpathlineto{\pgfqpoint{2.468941in}{2.016588in}}%
\pgfpathlineto{\pgfqpoint{2.506833in}{2.080525in}}%
\pgfpathlineto{\pgfqpoint{2.544724in}{2.144619in}}%
\pgfpathlineto{\pgfqpoint{2.582616in}{2.208995in}}%
\pgfpathlineto{\pgfqpoint{2.620507in}{2.198296in}}%
\pgfpathlineto{\pgfqpoint{2.658399in}{2.187489in}}%
\pgfpathlineto{\pgfqpoint{2.696290in}{2.176733in}}%
\pgfpathlineto{\pgfqpoint{2.734182in}{2.166149in}}%
\pgfpathlineto{\pgfqpoint{2.772073in}{2.155828in}}%
\pgfpathlineto{\pgfqpoint{2.809965in}{2.145837in}}%
\pgfpathlineto{\pgfqpoint{2.847857in}{2.136227in}}%
\pgfpathlineto{\pgfqpoint{2.885748in}{2.127036in}}%
\pgfpathlineto{\pgfqpoint{2.923640in}{2.118289in}}%
\pgfpathlineto{\pgfqpoint{2.961531in}{2.110008in}}%
\pgfpathlineto{\pgfqpoint{2.999423in}{2.102204in}}%
\pgfpathlineto{\pgfqpoint{3.037314in}{2.094887in}}%
\pgfpathlineto{\pgfqpoint{3.075206in}{2.088062in}}%
\pgfpathlineto{\pgfqpoint{3.113097in}{2.081732in}}%
\pgfpathlineto{\pgfqpoint{3.150989in}{2.075899in}}%
\pgfpathlineto{\pgfqpoint{3.188880in}{2.070563in}}%
\pgfpathlineto{\pgfqpoint{3.226772in}{2.065724in}}%
\pgfpathlineto{\pgfqpoint{3.264663in}{2.061381in}}%
\pgfpathlineto{\pgfqpoint{3.302555in}{2.057534in}}%
\pgfpathlineto{\pgfqpoint{3.340446in}{2.054181in}}%
\pgfpathlineto{\pgfqpoint{3.378338in}{2.051322in}}%
\pgfpathlineto{\pgfqpoint{3.416229in}{2.048957in}}%
\pgfpathlineto{\pgfqpoint{3.454121in}{2.047088in}}%
\pgfpathlineto{\pgfqpoint{3.492012in}{2.045716in}}%
\pgfpathlineto{\pgfqpoint{3.529904in}{2.044841in}}%
\pgfpathlineto{\pgfqpoint{3.567795in}{2.044469in}}%
\pgfpathlineto{\pgfqpoint{3.605687in}{2.044601in}}%
\pgfpathlineto{\pgfqpoint{3.643578in}{2.045244in}}%
\pgfpathlineto{\pgfqpoint{3.681470in}{2.046402in}}%
\pgfpathlineto{\pgfqpoint{3.719361in}{2.048082in}}%
\pgfpathlineto{\pgfqpoint{3.757253in}{2.050292in}}%
\pgfpathlineto{\pgfqpoint{3.795145in}{2.053041in}}%
\pgfpathlineto{\pgfqpoint{3.833036in}{2.056337in}}%
\pgfpathlineto{\pgfqpoint{3.870928in}{2.060193in}}%
\pgfpathlineto{\pgfqpoint{3.908819in}{2.064619in}}%
\pgfpathlineto{\pgfqpoint{3.946711in}{2.069629in}}%
\pgfpathlineto{\pgfqpoint{3.984602in}{2.075237in}}%
\pgfpathlineto{\pgfqpoint{4.022494in}{2.081458in}}%
\pgfpathlineto{\pgfqpoint{4.060385in}{2.088309in}}%
\pgfpathlineto{\pgfqpoint{4.098277in}{2.095806in}}%
\pgfpathlineto{\pgfqpoint{4.136168in}{2.103969in}}%
\pgfpathlineto{\pgfqpoint{4.174060in}{2.112818in}}%
\pgfpathlineto{\pgfqpoint{4.211951in}{2.122373in}}%
\pgfpathlineto{\pgfqpoint{4.249843in}{2.132656in}}%
\pgfpathlineto{\pgfqpoint{4.287734in}{2.143691in}}%
\pgfpathlineto{\pgfqpoint{4.325626in}{2.155501in}}%
\pgfpathlineto{\pgfqpoint{4.363517in}{2.168111in}}%
\pgfpathlineto{\pgfqpoint{4.401409in}{2.181548in}}%
\pgfpathlineto{\pgfqpoint{4.439300in}{2.195838in}}%
\pgfpathlineto{\pgfqpoint{4.477192in}{2.211009in}}%
\pgfpathlineto{\pgfqpoint{4.515083in}{2.227090in}}%
\pgfpathlineto{\pgfqpoint{4.552975in}{2.244108in}}%
\pgfpathlineto{\pgfqpoint{4.590866in}{2.262094in}}%
\pgfpathlineto{\pgfqpoint{4.628758in}{2.281077in}}%
\pgfpathlineto{\pgfqpoint{4.666649in}{2.301085in}}%
\pgfpathlineto{\pgfqpoint{4.704541in}{2.322149in}}%
\pgfpathlineto{\pgfqpoint{4.742433in}{2.344296in}}%
\pgfpathlineto{\pgfqpoint{4.780324in}{2.367554in}}%
\pgfpathlineto{\pgfqpoint{4.818216in}{2.391948in}}%
\pgfpathlineto{\pgfqpoint{4.856107in}{2.417502in}}%
\pgfpathlineto{\pgfqpoint{4.893999in}{2.444237in}}%
\pgfpathlineto{\pgfqpoint{4.931890in}{2.472171in}}%
\pgfpathlineto{\pgfqpoint{4.969782in}{2.501317in}}%
\pgfpathlineto{\pgfqpoint{5.007673in}{2.531682in}}%
\pgfpathlineto{\pgfqpoint{5.045565in}{2.563268in}}%
\pgfpathlineto{\pgfqpoint{5.083456in}{2.596065in}}%
\pgfpathlineto{\pgfqpoint{5.121348in}{2.630057in}}%
\pgfpathlineto{\pgfqpoint{5.159239in}{2.665213in}}%
\pgfpathlineto{\pgfqpoint{5.197131in}{2.701486in}}%
\pgfpathlineto{\pgfqpoint{5.235022in}{2.738813in}}%
\pgfpathlineto{\pgfqpoint{5.272914in}{2.777105in}}%
\pgfpathlineto{\pgfqpoint{5.310805in}{2.816250in}}%
\pgfpathlineto{\pgfqpoint{5.348697in}{2.856101in}}%
\pgfpathlineto{\pgfqpoint{5.386588in}{2.896472in}}%
\pgfusepath{stroke}%
\end{pgfscope}%
\begin{pgfscope}%
\pgfpathrectangle{\pgfqpoint{0.652043in}{0.499074in}}{\pgfqpoint{4.960000in}{3.696000in}}%
\pgfusepath{clip}%
\pgfsetbuttcap%
\pgfsetroundjoin%
\pgfsetlinewidth{1.505625pt}%
\definecolor{currentstroke}{rgb}{1.000000,0.000000,0.000000}%
\pgfsetstrokecolor{currentstroke}%
\pgfsetdash{{5.550000pt}{2.400000pt}}{0.000000pt}%
\pgfpathmoveto{\pgfqpoint{0.877497in}{2.252782in}}%
\pgfpathlineto{\pgfqpoint{0.915389in}{2.331305in}}%
\pgfpathlineto{\pgfqpoint{0.953281in}{2.409747in}}%
\pgfpathlineto{\pgfqpoint{0.991172in}{2.488100in}}%
\pgfpathlineto{\pgfqpoint{1.029064in}{2.566365in}}%
\pgfpathlineto{\pgfqpoint{1.066955in}{2.644542in}}%
\pgfpathlineto{\pgfqpoint{1.104847in}{2.722627in}}%
\pgfpathlineto{\pgfqpoint{1.142738in}{2.800615in}}%
\pgfpathlineto{\pgfqpoint{1.180630in}{2.878498in}}%
\pgfpathlineto{\pgfqpoint{1.218521in}{2.956266in}}%
\pgfpathlineto{\pgfqpoint{1.256413in}{3.033910in}}%
\pgfpathlineto{\pgfqpoint{1.294304in}{3.111418in}}%
\pgfpathlineto{\pgfqpoint{1.332196in}{3.188780in}}%
\pgfpathlineto{\pgfqpoint{1.370087in}{3.265983in}}%
\pgfpathlineto{\pgfqpoint{1.407979in}{3.343016in}}%
\pgfpathlineto{\pgfqpoint{1.445870in}{3.419867in}}%
\pgfpathlineto{\pgfqpoint{1.483762in}{3.496526in}}%
\pgfpathlineto{\pgfqpoint{1.521653in}{3.572983in}}%
\pgfpathlineto{\pgfqpoint{1.559545in}{3.649228in}}%
\pgfpathlineto{\pgfqpoint{1.597436in}{3.725254in}}%
\pgfpathlineto{\pgfqpoint{1.635328in}{3.801054in}}%
\pgfpathlineto{\pgfqpoint{1.673219in}{3.876625in}}%
\pgfpathlineto{\pgfqpoint{1.711111in}{3.951965in}}%
\pgfpathlineto{\pgfqpoint{1.749002in}{4.027074in}}%
\pgfpathlineto{\pgfqpoint{1.786894in}{3.985659in}}%
\pgfpathlineto{\pgfqpoint{1.824785in}{3.906697in}}%
\pgfpathlineto{\pgfqpoint{1.862677in}{3.826864in}}%
\pgfpathlineto{\pgfqpoint{1.900569in}{3.746197in}}%
\pgfpathlineto{\pgfqpoint{1.938460in}{3.664738in}}%
\pgfpathlineto{\pgfqpoint{1.976352in}{3.582538in}}%
\pgfpathlineto{\pgfqpoint{2.014243in}{3.499654in}}%
\pgfpathlineto{\pgfqpoint{2.052135in}{3.416140in}}%
\pgfpathlineto{\pgfqpoint{2.090026in}{3.332048in}}%
\pgfpathlineto{\pgfqpoint{2.127918in}{3.247422in}}%
\pgfpathlineto{\pgfqpoint{2.165809in}{3.162303in}}%
\pgfpathlineto{\pgfqpoint{2.203701in}{3.076733in}}%
\pgfpathlineto{\pgfqpoint{2.241592in}{2.990758in}}%
\pgfpathlineto{\pgfqpoint{2.279484in}{2.904433in}}%
\pgfpathlineto{\pgfqpoint{2.317375in}{2.817818in}}%
\pgfpathlineto{\pgfqpoint{2.355267in}{2.730973in}}%
\pgfpathlineto{\pgfqpoint{2.393158in}{2.643961in}}%
\pgfpathlineto{\pgfqpoint{2.431050in}{2.556843in}}%
\pgfpathlineto{\pgfqpoint{2.468941in}{2.469692in}}%
\pgfpathlineto{\pgfqpoint{2.506833in}{2.382595in}}%
\pgfpathlineto{\pgfqpoint{2.544724in}{2.295654in}}%
\pgfpathlineto{\pgfqpoint{2.582616in}{2.208995in}}%
\pgfpathlineto{\pgfqpoint{2.620507in}{2.198296in}}%
\pgfpathlineto{\pgfqpoint{2.658399in}{2.187489in}}%
\pgfpathlineto{\pgfqpoint{2.696290in}{2.176733in}}%
\pgfpathlineto{\pgfqpoint{2.734182in}{2.166149in}}%
\pgfpathlineto{\pgfqpoint{2.772073in}{2.155828in}}%
\pgfpathlineto{\pgfqpoint{2.809965in}{2.145837in}}%
\pgfpathlineto{\pgfqpoint{2.847857in}{2.136227in}}%
\pgfpathlineto{\pgfqpoint{2.885748in}{2.127036in}}%
\pgfpathlineto{\pgfqpoint{2.923640in}{2.118289in}}%
\pgfpathlineto{\pgfqpoint{2.961531in}{2.110008in}}%
\pgfpathlineto{\pgfqpoint{2.999423in}{2.102204in}}%
\pgfpathlineto{\pgfqpoint{3.037314in}{2.094887in}}%
\pgfpathlineto{\pgfqpoint{3.075206in}{2.088062in}}%
\pgfpathlineto{\pgfqpoint{3.113097in}{2.081732in}}%
\pgfpathlineto{\pgfqpoint{3.150989in}{2.075899in}}%
\pgfpathlineto{\pgfqpoint{3.188880in}{2.070563in}}%
\pgfpathlineto{\pgfqpoint{3.226772in}{2.065724in}}%
\pgfpathlineto{\pgfqpoint{3.264663in}{2.061381in}}%
\pgfpathlineto{\pgfqpoint{3.302555in}{2.057534in}}%
\pgfpathlineto{\pgfqpoint{3.340446in}{2.054181in}}%
\pgfpathlineto{\pgfqpoint{3.378338in}{2.051322in}}%
\pgfpathlineto{\pgfqpoint{3.416229in}{2.048957in}}%
\pgfpathlineto{\pgfqpoint{3.454121in}{2.047088in}}%
\pgfpathlineto{\pgfqpoint{3.492012in}{2.045716in}}%
\pgfpathlineto{\pgfqpoint{3.529904in}{2.044841in}}%
\pgfpathlineto{\pgfqpoint{3.567795in}{2.044469in}}%
\pgfpathlineto{\pgfqpoint{3.605687in}{2.044601in}}%
\pgfpathlineto{\pgfqpoint{3.643578in}{2.045244in}}%
\pgfpathlineto{\pgfqpoint{3.681470in}{2.046402in}}%
\pgfpathlineto{\pgfqpoint{3.719361in}{2.048082in}}%
\pgfpathlineto{\pgfqpoint{3.757253in}{2.050292in}}%
\pgfpathlineto{\pgfqpoint{3.795145in}{2.053041in}}%
\pgfpathlineto{\pgfqpoint{3.833036in}{2.056337in}}%
\pgfpathlineto{\pgfqpoint{3.870928in}{2.060193in}}%
\pgfpathlineto{\pgfqpoint{3.908819in}{2.064619in}}%
\pgfpathlineto{\pgfqpoint{3.946711in}{2.069629in}}%
\pgfpathlineto{\pgfqpoint{3.984602in}{2.075237in}}%
\pgfpathlineto{\pgfqpoint{4.022494in}{2.081458in}}%
\pgfpathlineto{\pgfqpoint{4.060385in}{2.088309in}}%
\pgfpathlineto{\pgfqpoint{4.098277in}{2.095806in}}%
\pgfpathlineto{\pgfqpoint{4.136168in}{2.103969in}}%
\pgfpathlineto{\pgfqpoint{4.174060in}{2.112818in}}%
\pgfpathlineto{\pgfqpoint{4.211951in}{2.122373in}}%
\pgfpathlineto{\pgfqpoint{4.249843in}{2.132656in}}%
\pgfpathlineto{\pgfqpoint{4.287734in}{2.143691in}}%
\pgfpathlineto{\pgfqpoint{4.325626in}{2.155501in}}%
\pgfpathlineto{\pgfqpoint{4.363517in}{2.168111in}}%
\pgfpathlineto{\pgfqpoint{4.401409in}{2.181548in}}%
\pgfpathlineto{\pgfqpoint{4.439300in}{2.195838in}}%
\pgfpathlineto{\pgfqpoint{4.477192in}{2.211009in}}%
\pgfpathlineto{\pgfqpoint{4.515083in}{2.227090in}}%
\pgfpathlineto{\pgfqpoint{4.552975in}{2.244108in}}%
\pgfpathlineto{\pgfqpoint{4.590866in}{2.262094in}}%
\pgfpathlineto{\pgfqpoint{4.628758in}{2.281077in}}%
\pgfpathlineto{\pgfqpoint{4.666649in}{2.301086in}}%
\pgfpathlineto{\pgfqpoint{4.704541in}{2.322149in}}%
\pgfpathlineto{\pgfqpoint{4.742433in}{2.344296in}}%
\pgfpathlineto{\pgfqpoint{4.780324in}{2.367554in}}%
\pgfpathlineto{\pgfqpoint{4.818216in}{2.391948in}}%
\pgfpathlineto{\pgfqpoint{4.856107in}{2.417502in}}%
\pgfpathlineto{\pgfqpoint{4.893999in}{2.444237in}}%
\pgfpathlineto{\pgfqpoint{4.931890in}{2.472171in}}%
\pgfpathlineto{\pgfqpoint{4.969782in}{2.501317in}}%
\pgfpathlineto{\pgfqpoint{5.007673in}{2.531682in}}%
\pgfpathlineto{\pgfqpoint{5.045565in}{2.563268in}}%
\pgfpathlineto{\pgfqpoint{5.083456in}{2.596065in}}%
\pgfpathlineto{\pgfqpoint{5.121348in}{2.630057in}}%
\pgfpathlineto{\pgfqpoint{5.159239in}{2.665213in}}%
\pgfpathlineto{\pgfqpoint{5.197131in}{2.701486in}}%
\pgfpathlineto{\pgfqpoint{5.235022in}{2.738813in}}%
\pgfpathlineto{\pgfqpoint{5.272914in}{2.777105in}}%
\pgfpathlineto{\pgfqpoint{5.310805in}{2.816250in}}%
\pgfpathlineto{\pgfqpoint{5.348697in}{2.856101in}}%
\pgfpathlineto{\pgfqpoint{5.386588in}{2.896472in}}%
\pgfusepath{stroke}%
\end{pgfscope}%
\begin{pgfscope}%
\pgfsetrectcap%
\pgfsetmiterjoin%
\pgfsetlinewidth{0.803000pt}%
\definecolor{currentstroke}{rgb}{0.000000,0.000000,0.000000}%
\pgfsetstrokecolor{currentstroke}%
\pgfsetdash{}{0pt}%
\pgfpathmoveto{\pgfqpoint{0.652043in}{0.499074in}}%
\pgfpathlineto{\pgfqpoint{0.652043in}{4.195074in}}%
\pgfusepath{stroke}%
\end{pgfscope}%
\begin{pgfscope}%
\pgfsetrectcap%
\pgfsetmiterjoin%
\pgfsetlinewidth{0.803000pt}%
\definecolor{currentstroke}{rgb}{0.000000,0.000000,0.000000}%
\pgfsetstrokecolor{currentstroke}%
\pgfsetdash{}{0pt}%
\pgfpathmoveto{\pgfqpoint{5.612043in}{0.499074in}}%
\pgfpathlineto{\pgfqpoint{5.612043in}{4.195074in}}%
\pgfusepath{stroke}%
\end{pgfscope}%
\begin{pgfscope}%
\pgfsetrectcap%
\pgfsetmiterjoin%
\pgfsetlinewidth{0.803000pt}%
\definecolor{currentstroke}{rgb}{0.000000,0.000000,0.000000}%
\pgfsetstrokecolor{currentstroke}%
\pgfsetdash{}{0pt}%
\pgfpathmoveto{\pgfqpoint{0.652043in}{0.499074in}}%
\pgfpathlineto{\pgfqpoint{5.612043in}{0.499074in}}%
\pgfusepath{stroke}%
\end{pgfscope}%
\begin{pgfscope}%
\pgfsetrectcap%
\pgfsetmiterjoin%
\pgfsetlinewidth{0.803000pt}%
\definecolor{currentstroke}{rgb}{0.000000,0.000000,0.000000}%
\pgfsetstrokecolor{currentstroke}%
\pgfsetdash{}{0pt}%
\pgfpathmoveto{\pgfqpoint{0.652043in}{4.195074in}}%
\pgfpathlineto{\pgfqpoint{5.612043in}{4.195074in}}%
\pgfusepath{stroke}%
\end{pgfscope}%
\begin{pgfscope}%
\pgfsetbuttcap%
\pgfsetmiterjoin%
\definecolor{currentfill}{rgb}{1.000000,1.000000,1.000000}%
\pgfsetfillcolor{currentfill}%
\pgfsetfillopacity{0.800000}%
\pgfsetlinewidth{1.003750pt}%
\definecolor{currentstroke}{rgb}{0.800000,0.800000,0.800000}%
\pgfsetstrokecolor{currentstroke}%
\pgfsetstrokeopacity{0.800000}%
\pgfsetdash{}{0pt}%
\pgfpathmoveto{\pgfqpoint{3.482410in}{3.115599in}}%
\pgfpathlineto{\pgfqpoint{5.514821in}{3.115599in}}%
\pgfpathquadraticcurveto{\pgfqpoint{5.542598in}{3.115599in}}{\pgfqpoint{5.542598in}{3.143377in}}%
\pgfpathlineto{\pgfqpoint{5.542598in}{4.097852in}}%
\pgfpathquadraticcurveto{\pgfqpoint{5.542598in}{4.125630in}}{\pgfqpoint{5.514821in}{4.125630in}}%
\pgfpathlineto{\pgfqpoint{3.482410in}{4.125630in}}%
\pgfpathquadraticcurveto{\pgfqpoint{3.454632in}{4.125630in}}{\pgfqpoint{3.454632in}{4.097852in}}%
\pgfpathlineto{\pgfqpoint{3.454632in}{3.143377in}}%
\pgfpathquadraticcurveto{\pgfqpoint{3.454632in}{3.115599in}}{\pgfqpoint{3.482410in}{3.115599in}}%
\pgfpathlineto{\pgfqpoint{3.482410in}{3.115599in}}%
\pgfpathclose%
\pgfusepath{stroke,fill}%
\end{pgfscope}%
\begin{pgfscope}%
\pgfsetrectcap%
\pgfsetroundjoin%
\pgfsetlinewidth{1.505625pt}%
\definecolor{currentstroke}{rgb}{0.121569,0.466667,0.705882}%
\pgfsetstrokecolor{currentstroke}%
\pgfsetdash{}{0pt}%
\pgfpathmoveto{\pgfqpoint{3.510188in}{4.021463in}}%
\pgfpathlineto{\pgfqpoint{3.649077in}{4.021463in}}%
\pgfpathlineto{\pgfqpoint{3.787966in}{4.021463in}}%
\pgfusepath{stroke}%
\end{pgfscope}%
\begin{pgfscope}%
\definecolor{textcolor}{rgb}{0.000000,0.000000,0.000000}%
\pgfsetstrokecolor{textcolor}%
\pgfsetfillcolor{textcolor}%
\pgftext[x=3.899077in,y=3.972852in,left,base]{\color{textcolor}{\rmfamily\fontsize{10.000000}{12.000000}\selectfont\catcode`\^=\active\def^{\ifmmode\sp\else\^{}\fi}\catcode`\%=\active\def%{\%}Relaxation Variable Value}}%
\end{pgfscope}%
\begin{pgfscope}%
\pgfsetrectcap%
\pgfsetroundjoin%
\pgfsetlinewidth{1.505625pt}%
\definecolor{currentstroke}{rgb}{1.000000,0.498039,0.054902}%
\pgfsetstrokecolor{currentstroke}%
\pgfsetdash{}{0pt}%
\pgfpathmoveto{\pgfqpoint{3.510188in}{3.827790in}}%
\pgfpathlineto{\pgfqpoint{3.649077in}{3.827790in}}%
\pgfpathlineto{\pgfqpoint{3.787966in}{3.827790in}}%
\pgfusepath{stroke}%
\end{pgfscope}%
\begin{pgfscope}%
\definecolor{textcolor}{rgb}{0.000000,0.000000,0.000000}%
\pgfsetstrokecolor{textcolor}%
\pgfsetfillcolor{textcolor}%
\pgftext[x=3.899077in,y=3.779179in,left,base]{\color{textcolor}{\rmfamily\fontsize{10.000000}{12.000000}\selectfont\catcode`\^=\active\def^{\ifmmode\sp\else\^{}\fi}\catcode`\%=\active\def%{\%}Actual Bilinear Value}}%
\end{pgfscope}%
\begin{pgfscope}%
\pgfsetbuttcap%
\pgfsetmiterjoin%
\definecolor{currentfill}{rgb}{0.000000,0.501961,0.000000}%
\pgfsetfillcolor{currentfill}%
\pgfsetfillopacity{0.200000}%
\pgfsetlinewidth{1.003750pt}%
\definecolor{currentstroke}{rgb}{0.000000,0.501961,0.000000}%
\pgfsetstrokecolor{currentstroke}%
\pgfsetstrokeopacity{0.200000}%
\pgfsetdash{}{0pt}%
\pgfpathmoveto{\pgfqpoint{3.510188in}{3.585506in}}%
\pgfpathlineto{\pgfqpoint{3.787966in}{3.585506in}}%
\pgfpathlineto{\pgfqpoint{3.787966in}{3.682728in}}%
\pgfpathlineto{\pgfqpoint{3.510188in}{3.682728in}}%
\pgfpathlineto{\pgfqpoint{3.510188in}{3.585506in}}%
\pgfpathclose%
\pgfusepath{stroke,fill}%
\end{pgfscope}%
\begin{pgfscope}%
\definecolor{textcolor}{rgb}{0.000000,0.000000,0.000000}%
\pgfsetstrokecolor{textcolor}%
\pgfsetfillcolor{textcolor}%
\pgftext[x=3.899077in,y=3.585506in,left,base]{\color{textcolor}{\rmfamily\fontsize{10.000000}{12.000000}\selectfont\catcode`\^=\active\def^{\ifmmode\sp\else\^{}\fi}\catcode`\%=\active\def%{\%}Bounds}}%
\end{pgfscope}%
\begin{pgfscope}%
\pgfsetbuttcap%
\pgfsetroundjoin%
\pgfsetlinewidth{1.505625pt}%
\definecolor{currentstroke}{rgb}{1.000000,0.000000,0.000000}%
\pgfsetstrokecolor{currentstroke}%
\pgfsetdash{{1.500000pt}{2.475000pt}}{0.000000pt}%
\pgfpathmoveto{\pgfqpoint{3.510188in}{3.440445in}}%
\pgfpathlineto{\pgfqpoint{3.649077in}{3.440445in}}%
\pgfpathlineto{\pgfqpoint{3.787966in}{3.440445in}}%
\pgfusepath{stroke}%
\end{pgfscope}%
\begin{pgfscope}%
\definecolor{textcolor}{rgb}{0.000000,0.000000,0.000000}%
\pgfsetstrokecolor{textcolor}%
\pgfsetfillcolor{textcolor}%
\pgftext[x=3.899077in,y=3.391833in,left,base]{\color{textcolor}{\rmfamily\fontsize{10.000000}{12.000000}\selectfont\catcode`\^=\active\def^{\ifmmode\sp\else\^{}\fi}\catcode`\%=\active\def%{\%}Lower Bound}}%
\end{pgfscope}%
\begin{pgfscope}%
\pgfsetbuttcap%
\pgfsetroundjoin%
\pgfsetlinewidth{1.505625pt}%
\definecolor{currentstroke}{rgb}{1.000000,0.000000,0.000000}%
\pgfsetstrokecolor{currentstroke}%
\pgfsetdash{{5.550000pt}{2.400000pt}}{0.000000pt}%
\pgfpathmoveto{\pgfqpoint{3.510188in}{3.246772in}}%
\pgfpathlineto{\pgfqpoint{3.649077in}{3.246772in}}%
\pgfpathlineto{\pgfqpoint{3.787966in}{3.246772in}}%
\pgfusepath{stroke}%
\end{pgfscope}%
\begin{pgfscope}%
\definecolor{textcolor}{rgb}{0.000000,0.000000,0.000000}%
\pgfsetstrokecolor{textcolor}%
\pgfsetfillcolor{textcolor}%
\pgftext[x=3.899077in,y=3.198161in,left,base]{\color{textcolor}{\rmfamily\fontsize{10.000000}{12.000000}\selectfont\catcode`\^=\active\def^{\ifmmode\sp\else\^{}\fi}\catcode`\%=\active\def%{\%}Upper Bound}}%
\end{pgfscope}%
\end{pgfpicture}%
\makeatother%
\endgroup%
}
	\caption{State transition approximation for \( n \) using bilinear term relaxation.}
	\label{fig:dn-term-approx}
\end{figure}

Figure \ref{fig:dn-term-approx} compares the actual bilinear value \( v\xi \) with the relaxation variable \( w \) introduced via McCormick
envelopes.
This comparison highlights the accuracy of the relaxation approach in approximating the bilinear interaction and its effect on the state transition
of \( n \).
Once the velocity reaches its limit, the approximation becomes increasingly accurate.

Note: The x-axis does not represent time directly but instead shows discrete time points.
Here, 30 time points per second were chosen, resulting in a range from 0 to 120 for this figure, whereas the other figures range from 0 to 4.

