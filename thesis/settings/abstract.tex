\chapter{\abstractname}
Motion planning is a critical component of robotics and autonomous systems, requiring algorithms that efficiently generate safe and feasible
trajectories.
This thesis investigates motion planning for autonomous vehicles by framing it as an optimization problem that balances safety, efficiency, and
computational feasibility.
A central challenge in the optimal control of autonomous driving is the vehicle dynamics model, where a trade-off exists between model accuracy and
computational speed.
This trade-off becomes especially significant in real-time planning for dynamic environments, where vehicles must navigate while accounting for
constraints such as road boundaries and vehicle dynamics.

To address these challenges, we develop optimization-based motion planning approaches using both the double integrator and bicycle models.
Our methodology reformulates the motion planning problem within a convex optimization framework, enabling the efficient computation of feasible
trajectories while preserving computational tractability.
Initially, the approach employs a double integrator model augmented with additional constraints to capture vehicle dynamics without sacrificing
convexity.
By formulating the problem in the Frenet coordinate frame, we exploit a structured representation that simplifies trajectory generation.
System linearization, constraint reformulation, and convex relaxation techniques are applied to manage nonlinearities, ensuring both solution
robustness and computational efficiency.

To enhance model fidelity while retaining computational efficiency, the methodology is further extended to incorporate a bicycle model.
This extension allows for a more accurate representation of vehicle dynamics, particularly in scenarios demanding realistic steering and acceleration
constraints.
The bicycle model is approximated to maintain convexity, utilizing techniques such as bilinear term relaxation and auxiliary variables to manage
nonconvex components.
The proposed approach is evaluated through simulations to assess its performance.

The results demonstrate that the framework can generate feasible and efficient trajectories while maintaining computational efficiency.
The evaluation confirms its applicability across a variety of planning scenarios, underscoring its potential for use in autonomous driving
applications.
